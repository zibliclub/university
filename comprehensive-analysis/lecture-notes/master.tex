\documentclass[a4paper]{report}
\usepackage[utf8]{inputenc}
\usepackage[T2A]{fontenc}
\usepackage[russian]{babel}

\usepackage{amsmath}
\usepackage{amssymb}
\usepackage{amsthm}
\usepackage{float}
\usepackage{tikz}
\usepackage{booktabs} % tabular \toprule, ...
\usepackage{gensymb} % \degree


% for title
\author{
  Основано на лекциях \lecturer \\
  \small Конспект написан Заблоцким Данилом
}
\date{\term\ \year}
\title{\course}

\makeatletter

\let\@real@maketitle\maketitle
\renewcommand{\maketitle}{
  {\let\newpage\relax\@real@maketitle}
  \begin{center}
    \begin{minipage}[c]{0.9\textwidth}
      \centering\footnotesize Эти записи не одобряются лекторами, и я вношу в них изменения (часто существенно) после лекций. Они далеко не точно отражают то, что на самом деле читалось, и, в частности, все ошибки почти наверняка мои.
    \end{minipage}
  \end{center}
}

\let\@real@tableofcontents\tableofcontents
\renewcommand{\tableofcontents}{\@real@tableofcontents\newpage}

\makeatother


% hyperref
\usepackage{url}

\usepackage{hyperref}
\hypersetup{
  colorlinks,
  linkcolor={black},
  citecolor={black},
  urlcolor={blue!80!black}
}


% horizontal rule
\newcommand\hr{
  \noindent\rule[0.5ex]{\linewidth}{0.5pt}
}


% theorems
\usepackage{thmtools}
\usepackage[framemethod=TikZ]{mdframed}
\mdfsetup{skipabove=1em,skipbelow=0em, innertopmargin=5pt, innerbottommargin=6pt}

\theoremstyle{definition}

\declaretheoremstyle[
  headfont=\bfseries\sffamily,
  bodyfont=\normalfont,
  mdframed={ nobreak }
]{thmbox}

\declaretheoremstyle[
  headfont=\bfseries\sffamily,
  bodyfont=\normalfont
]{thmunbox}

\declaretheoremstyle[
  headfont=\bfseries\sffamily,
  bodyfont=\normalfont,
  numbered=no,
  mdframed={ rightline=false, topline=false, bottomline=false, },
  qed=\qedsymbol
]{thmproofline}

\declaretheorem[style=thmbox, name=Определение]{definition}
\declaretheorem[style=thmbox, name=Следствие]{corollary}
\declaretheorem[style=thmbox, name=Предложение]{prop}
\declaretheorem[style=thmbox, name=Теорема]{theorem}
\declaretheorem[style=thmbox, name=Лемма]{lemma}

\declaretheorem[numbered=no, style=thmproofline, name=Доказательство]{replacementproof}
\declaretheorem[style=thmunbox, numbered=no, name=Упражнение]{ex}
\declaretheorem[style=thmunbox, numbered=no, name=Пример]{eg}
\declaretheorem[style=thmunbox, numbered=no, name=Замечание]{remark}
\declaretheorem[style=thmunbox, numbered=no, name=Примечание]{note}

\renewenvironment{proof}[1][\proofname]{\begin{replacementproof}}{\end{replacementproof}}

\AtEndEnvironment{eg}{\null\hfill$\diamond$}

\newtheorem*{notation}{Обозначение}
\newtheorem*{previouslyseen}{Как было замечено ранее}
\newtheorem*{problem}{Проблема}
\newtheorem*{observe}{Наблюдение}
\newtheorem*{property}{Свойство}
\newtheorem*{intuition}{Предположение}

\usepackage{etoolbox}
\AtEndEnvironment{vb}{\null\hfill$\diamond$}
\AtEndEnvironment{intermezzo}{\null\hfill$\diamond$}


% lesson
\usepackage{xifthen}

\def\testdateparts#1{\dateparts#1\relax}
\def\dateparts#1 #2 #3 #4 #5\relax{
  \marginpar{\small\textsf{\mbox{#1 #2 #3 #5}}}
}

\def\@lesson{}
\newcommand{\lesson}[3]{
  \ifthenelse{\isempty{#3}}{
    \def\@lesson{Лекция #1}
  }{
    \def\@lesson{Лекция #1: #3}
  }
  \subsection*{\@lesson}
  \testdateparts{#2}
}


% fancy headers
\usepackage{fancyhdr}
\pagestyle{fancy}

\fancyhead[RO,LE]{\course}
\fancyhead[RE,LO]{\@lesson}
\fancyfoot[LE,RO]{\thepage}
\fancyfoot[C]{\leftmark}
\renewcommand{\headrulewidth}{0.4pt}


% incfig
\usepackage{import}
\usepackage{pdfpages}
\usepackage{transparent}
\usepackage{xcolor}

\newcommand{\incfig}[2][1]{%
  \def\svgwidth{#1\columnwidth}
  \import{./figures/}{#2.pdf_tex}
}

\pdfsuppresswarningpagegroup=1


% custom commands
\let\epsilon\varepsilon

\newcommand\N{\ensuremath{\mathbb{N}}}
\newcommand\R{\ensuremath{\mathbb{R}}}
\newcommand\Z{\ensuremath{\mathbb{Z}}}
\newcommand\Q{\ensuremath{\mathbb{Q}}}
\renewcommand\C{\ensuremath{\mathbb{C}}}
\newcommand{\diff}{\ensuremath{\operatorname{d}\!}}

\newcommand{\abs}[1]{\left\lvert #1\right\rvert}
\newcommand{\verteq}[0]{\rotatebox{90}{$=$}}
\newcommand{\vertneq}[0]{\rotatebox{90}{$\ne$}}
\newcommand{\equalto}[2]{\underset{\scriptstyle\overset{\mkern4mu\verteq}{#2}}{#1}}
\newcommand{\nequalto}[2]{\underset{\scriptstyle\overset{\mkern4mu\vertneq}{#2}}{#1}}
\newcommand{\RomanNumeralCaps}[1]{\MakeUppercase{\romannumeral #1}}

\newcommand*\circled[1]{
  \tikz[baseline=(char.base)]{
    \node[shape=circle,draw,inner sep=1pt] (char) {#1};
  }
}


% for \xrightrightarrows
\makeatletter

\newcommand*{\relrelbarsep}{.450ex}
\newcommand*{\relrelbar}{
  \mathrel{
    \mathpalette\@relrelbar\relrelbarsep
  }
}
\newcommand*{\@relrelbar}[2]{
  \raise#2\hbox to 0pt{$\m@th#1\relbar$\hss}
  \lower#2\hbox{$\m@th#1\relbar$}
}

\providecommand*{\rightrightarrowsfill@}{
  \arrowfill@\relrelbar\relrelbar\rightrightarrows
}
\providecommand*{\leftleftarrowsfill@}{
  \arrowfill@\leftleftarrows\relrelbar\relrelbar
}
\providecommand*{\xrightrightarrows}[2][]{
  \ext@arrow 0359\rightrightarrowsfill@{#1}{#2}
}
\providecommand*{\xleftleftarrows}[2][]{
  \ext@arrow 3095\leftleftarrowsfill@{#1}{#2}
}

\makeatother


\title{Комплексный анализ}

\begin{document}

\maketitle
\tableofcontents
\newpage

\chapter{Голоморфные функции}

% start lessons
\lesson{1}{от 13 фев 2024 12:45}{Начало}


\begin{note}[История]
    ?

    $ \begin{array}{ll}
            \boxed{1936\text{г.}} & \text{алгоритм Евклида (>2 тыс.)}             \\
                                  & \text{алгоритм сложения, умножения (>1 тыс.)} \\
                                  & \text{метод Гаусса}                           \\
                                  & \vdots
        \end{array} $

    $ \begin{array}{ll}
            \boxed{1900\text{г.}} & \text{Гильберт}
        \end{array} $

    \begin{center}
        \underline{аксиомы} $ \longrightarrow $ теоремы
    \end{center}

    $ \begin{array}{ll}
            \boxed{1931\text{г.}} & \text{Гёдель}
        \end{array} $

    $ \begin{array}{ll}
            \boxed{1936\text{г.}} & \text{формализация понятия алгоритма: модели вычислений} \\
                                  & \text{Чёрч -- }\lambda\text{-исчисление}                 \\
                                  & \text{Тьюринг -- машина Тьюринга}                        \\
                                  & \text{Пост -- машина Поста}                              \\
                                  & \text{Марков -- алгорифмы Маркова}                       \\
                                  & \text{Клини -- рекурсивные функции}
        \end{array} $

    \[
        \begin{array}{ccccc}
                                                                       &             & \boxed{\begin{array}{c}
                                                                                                      \text{поиск}         \\
                                                                                                      \text{неразрешенных} \\
                                                                                                      \text{проблем}
                                                                                                  \end{array}}                                                \\
                                                                       & \nearrow    &                           &                      &                         \\
            \underset{\text{понятие алгоритма}}{\boxed{1936\text{г.}}} & \rightarrow & \boxed{\begin{array}{c}
                                                                                                      \text{абстрактная} \\
                                                                                                      \text{теория}      \\
                                                                                                      \text{алгоритмов}
                                                                                                  \end{array}  }   &                      &                           \\
                                                                       & \searrow    &                           &                      & \boxed{\begin{array}{c}
                                                                                                                                                         \text{сложность} \\
                                                                                                                                                         \text{вычисления}
                                                                                                                                                     \end{array}} \\
                                                                       &             & \boxed{\text{компьютеры}} & \begin{matrix}
                                                                                                                       \nearrow \\ \searrow
                                                                                                                   \end{matrix} &                            \\
                                                                       &             &                           &                      & \boxed{\begin{array}{c}
                                                                                                                                                         \text{сложность} \\
                                                                                                                                                         \text{вычисления}
                                                                                                                                                     \end{array}} \\
        \end{array}
    \]
\end{note}

\begin{note}
    \[
        A = \{a_1,\ldots,a_m\} \text{ -- рабочий алфавит}
    \]
\end{note}

\begin{definition}[Машина Тьюринга]
    \emph{Машина Тьюринга} (МТ) над алфавитом $ A $ состоит из:
    \begin{enumerate}
        \item Бесконечной в обе стороны ленты, разбитой на ячейки. Ячейка может быть пустой (записан $ \square $), или содержать символ из $ A $.
              \begin{figure}[H]
                  \centering
                  \incfig{fig_01}
                  \label{fig:fig_01}
              \end{figure}
        \item Каретка, которая двигается над лентой, читает и пишет символы в ячейки.
        \item Внутренние состояния:
              \[
                  q_0,q_1,q_2,\ldots,q_n \quad \Bigg| \ \begin{array}{l}
                      q_0 \text{ -- конечное} \\
                      q_1 \text{ -- начальное}
                  \end{array}
              \]
        \item Программа -- набор правил вида:
              \[
                  (q_i,a)\longrightarrow (q_j,b,S),
              \]
              где $ \begin{array}{ll}
                      q_i \text{ -- любое состояние }\ne q_0       \\
                      a,b \text{ -- символы из }A \cup \{\square\} \\
                      q_j \text{ -- набор состояний }              \\
                      S \text{ -- сдвиг }R \text{ и }L
                  \end{array} $,
              по одному правилу: $ \forall $ комбинации $ (q_i,a) $
              \[
                  q_i \ne q_0, \quad a \in A \cup \{\square\}.
              \]
    \end{enumerate}
\end{definition}

\begin{definition}[Работа МТ]
    \emph{Работа МТ} $ M $ на слове $ w \in A^* $:
    \begin{enumerate}
        \item (на рисунке)
              \begin{figure}[H]
                  \centering
                  \incfig{fig_02}
                  \label{fig:fig_02}
              \end{figure}
        \item Согласно программе $ M $ работает.
        \item $ M $ останавливается, если она нападает в $ q_0 $
              \[
                  \bigl(M(w)\downarrow\bigr)
              \]
              и результат работы $ M(w) $ -- это слово, которое остается записанным. Иначе $ M $ не останавливается на $ w $
              \[
                  \bigl(M(w)\uparrow\bigr).
              \]
    \end{enumerate}
\end{definition}

\begin{definition}[Вычисление фукнкции МТ]
    МТ $ M $ \emph{вычисляет функцию} $ f_M: A^* \longrightarrow A^* $, если $ \forall w \in A^* $ если $ f_M(w) $ определена, то $ M(w)\downarrow $ и $ M(w) = f_M(w) $, а если $ f_M(w) $ не определена, то $ M(W)\uparrow $.
\end{definition}

\begin{note}[Тезис Тьюринга]
    Если $ f: A^* \longrightarrow A^* $ вычислима интуитивно, то $ \exists $ МТ $ M $, которая ее вычисляет.
\end{note}

\begin{example}
    \[
        f(w) = \left\{\begin{array}{ll}
            1, & \text{если }| w |\emph{ -- четная длина }w \\
            0, & \text{иначе}
        \end{array}\right.
    \]
    \begin{figure}[H]
        \centering
        \incfig{fig_03}
        \label{fig:fig_03}
    \end{figure}
    \[
        \begin{array}{l}
            q_1 \implies \text{четная} \\
            q_2 \implies \text{нечетная}
        \end{array}
    \]
    \[
        \begin{array}{ll}
            (q_1,0) \longrightarrow (q_2,\square,R) & (q_2,0) \longrightarrow (q_1,\square,R) \\
            (q_1,1) \longrightarrow (q_2,\square,R) & (q_2,1) \longrightarrow (q_1,\square,R) \\
            (q_1,\square) \longrightarrow (q_0,1,L) & (q_2,\square) \longrightarrow (q_0,0,R) \\
        \end{array}
    \]
\end{example}
\lesson{2}{от 15 окт 2023 08:47}{Продолжение}


\section{Особые решения}

\begin{note}
    \[
        \left\{\begin{array}{rl}
            y'     & = f(y), \ f \in C(D), \ D \subset\mathbb{R}, \\
            y(x_0) & = y_0
        \end{array}\right.
    \]
    \[
        \left[\begin{array}{rl}
            f(y) = 0                           & \implies y = c\text{ -- ?}       \\
            \left\{\begin{array}{rl}
                       f(y)                & \ne 0,    \\
                       \int\frac{dy}{f(y)} & = \int dx
                   \end{array}\right. & \implies \left[\begin{array}{rl}
                                                           y           & = \phi(x,C), \\
                                                           \psi(y,x,C) & = 0
                                                       \end{array}\right.
        \end{array}\right.
    \]

    Для $\forall$ точки $x \in \{y = C\} \ \exists$ точка $(x_1,y_1)$ и интегральная кривая, проходящая через точку $(x_1,y_1)$, которая пересекает прямую $y = C$ в точке $x \ \big(x \equiv (x,C)\big)$.

    Проинтегрируем на отрезке $[x_1;x]$:
    \[
        \int_{y_1}^{C}\frac{dy}{f(y)} = \int_{x_1}^{x}dx \iff \int_{y_1}^{C}\frac{dy}{f(y)} = x-x_1 \iff \underbrace{x}_{\text{конечная}} = x_1 + \underbrace{\int_{y_1}^{C}\frac{dy}{f(y)}}_{\text{конечный}},
    \]
    \begin{center}
        (несобственный интеграл сходится)
    \end{center}
\end{note}

\begin{note}[Критерий]
    Решение $y = C$ дифференциального уравнения $y' = f(y), \ f\in C(D)$ такое, что $f(C) = 0$ называется \emph{особым} $\iff$
    \[
        \iff \int_{y_1}^{C}\frac{dy}{f(y)} < \infty \quad \text{(несобственный интеграл сходится)}
    \]
\end{note}

\begin{example}
    $y'=3y^{\frac{2}{3}}$
    \begin{enumerate}
        \item Непрерывно.
        \item $f_y' = 2y^{-\frac{1}{3}}$ -- разрывна в точке $0$ (условие Липшица не выполнено?).
    \end{enumerate}
    \[
        \left[\begin{array}{rl}
            y                             & = 0 \ ?                                \\
            \int\frac{dy}{3y^\frac{2}{3}} & = \int dx \implies y^\frac{1}{3} = x+C
        \end{array}\right.
    \]
    \[
        \int_{y_1}^{0}\frac{dy}{3y^\frac{2}{3}} = y^\frac{1}{3}\Big|_{y_1}^0 = 0 - y_1^\frac{1}{3} < + \infty \overset{\text{по критерию}}{\implies} y = 0\text{ -- особое}
    \]
\end{example}

\begin{example}
    Найти особое решение $y' = \left\{\begin{array}{rl}
            y\cdot\ln y, & y > 0 \\
            0,           & y = 0
        \end{array}\right., \ D = [0;+\infty)$:
    \[
        f(y) = \left\{\begin{array}{rl}
            y\cdot\ln y, & y > 0 \\
            0,           & y = 0
        \end{array}\right.
    \]
    \begin{enumerate}
        \item Непрерывно.
              \[
                  \underset{y\rightarrow+0}{\lim}(y\cdot \ln y) = \underset{y\rightarrow+0}{\lim}\frac{\ln y}{\frac{1}{y}} = \underset{y\rightarrow+0}{\lim}\frac{\frac{1}{y}}{\frac{1}{y^2}} = -\underset{y\rightarrow+0}{\lim} = 0
              \]
        \item Условие Липшица:
              \[
                  \big|f(y_1) - f(y_2)\big| \leqslant L \cdot |y_1 - y_2|, \quad \begin{array}{l}
                      y_1 \in (0;+\infty), \\
                      y_2 = 0
                  \end{array}
              \]
              \[
                  \big|f(y_1) - f(y_2)\big| = |y_1\cdot \ln y_1 - 0| \leqslant |y_1|\cdot|\ln y_1| \leqslant |y_1| \cdot L,
              \]
              то есть $|\ln y_1| \leqslant L$.

              Для $\forall L > 0 \ \exists y_1^*$ близкий к $0$ и такой, что $|\ln y_1^*| > L$.
        \item $f(y) = 0 \implies \left[\begin{array}{l}
                      y = 0 \\
                      y = 1
                  \end{array}\right.$
              \begin{enumerate}
                  \item $y = 0$:
                        \[
                            \int_{y_1}^{0}\frac{dy}{y \cdot \ln y} = \int_{y_1}^{0}\frac{d(\ln y)}{\ln y} = \ln |\ln y| \Big|_{y_1}^0 = \infty - \ln|\ln y_1| = \infty \implies
                        \]
                        $\implies y =0$ не является особым.
                  \item $y = 1$:
                        \[
                            \int_{y_1}^{1}\frac{dy}{y\cdot \ln y} = \ln|\ln y| \Big|_{y_1}^1 = -\infty - \ln|\ln y_1| = - \infty \implies
                        \]
                        $\implies y =1$ не является особым.
              \end{enumerate}
    \end{enumerate}
\end{example}

\begin{example}
    Найти особое решение $y' = \left\{\begin{array}{rl}
            y\cdot\ln^2 y, & y > 0 \\
            0,             & y = 0
        \end{array}\right., \ D = [0;+\infty)$:
    \[
        f(y) = \left\{\begin{array}{rl}
            y\cdot\ln^2 y, & y > 0 \\
            0,             & y = 0
        \end{array}\right.
    \]
    \begin{enumerate}
        \item Непрерывно (аналогично).
        \item Условие Липшица (аналогично).
        \item $f(y) = 0 \implies y\cdot\ln^2y=0 \left[\begin{array}{l}
                      y = 0 \\
                      y = 1
                  \end{array}\right.$
              \begin{enumerate}
                  \item $y = 0$:
                        \[
                            \int_{y_1}^{0}\frac{dy}{y \cdot \ln^2 y} = \int_{y_1}^{0}\frac{d(\ln y)}{\ln^2 y} = \frac{1}{\ln y} \Big|_{y_1}^0 = 0 + \frac{1}{\ln y_1} \implies
                        \]
                        $\implies y =0$ -- особое.
                  \item $y = 1$:
                        \[
                            \int_{y_1}^{1}\frac{dy}{y\cdot \ln^2 y} = -\frac{1}{\ln y} \Big|_{y_1}^1 = -\infty + \frac{1}{\ln y_1}\text{ -- расходится } \implies
                        \]
                        $\implies y =1$ не является особым.
              \end{enumerate}
    \end{enumerate}
\end{example}

\chapter{Методы интегрирования дифференциальных уравнений $1$-го порядка}

\section{Однородные уравнения}

\begin{definition}[Однородное уравнение первого порядка]
    \emph{Однородным уравнением первого порядка} называется уравнение вида:
    \begin{equation}\label{eq6}
        y' = f\left(\frac{y}{x}\right),\quad f\in C(D)
    \end{equation}
    или:
    \begin{equation}\label{eq7}
        y'=\frac{P(x,y)}{Q(x,y)},
    \end{equation}
    где $P(x,y)$ и $Q(x,y)$ являются однородными функциями одного и того же порядка.
\end{definition}

\begin{definition}[Однородная функция порядка $k$]
    \emph{Однородной функцией порядка $k$} называется функция:
    \[
        P(\lambda x,\lambda y) = \lambda^k \cdot P(x,y),\quad \lambda \in \mathbb{R}, \ \lambda \ne 0
    \]
\end{definition}

\begin{example}
    $P(x,y) = x^2 - 2xy + 7y^2$
    \[
        P(\lambda x,\lambda y) = (\lambda x)^2 - 2(\lambda x)(\lambda y) + 7(\lambda y)^2 = \lambda^2(x^2 - 2xy + 7y^2)
    \]
\end{example}

\begin{example}
    $x(x^2 + y^2)dy = y(y^2 - xy + x^2)dx$
    \[
        \underbrace{x^3 + xy^2}_{Q(x,y)},\quad \underbrace{y^3 - xy^2 + yx^2}_{P(x,y)}
    \]
\end{example}

\begin{note}[Замена переменной]
    $t = \frac{y}{x} \implies y = t\cdot x$
    \begin{align*}
         & y = t(x)\cdot x \implies y' = t'\cdot x + t\text{ -- подставим в \ref{eq6}}: \\
         & t'\cdot x = f(t) - t\text{ -- уравнение с разделяющей переменной (РП)}
    \end{align*}
    \[
        \frac{dt}{dx}x = f(t) - t
    \]
    \[
        \left[\begin{array}{l}
            f(t) - t = 0 \\
            \left\{\begin{array}{rl}
                       f(t) - t                 & \ne 0               \\
                       \int \frac{dt}{f(t) - t} & = \int \frac{dx}{x}
                   \end{array}\right. \iff \left[\begin{array}{rl}
                                                     t = \phi(x,C)   & \implies y = x \cdot \phi(x,C)     \\
                                                     \psi(t,x,C) = 0 & \implies \psi(\frac{y}{x},x,C) = 0
                                                 \end{array}\right.
        \end{array}\right.
    \]
    $t'x = 0 \implies t'=0 \implies t = C \implies y = Cx$ -- решение при $C: \ f(C) - C = 0$.

    \[
        \text{Изоклины: }\begin{array}{l}
            f(\frac{y}{x}) = const                \\
            \frac{y}{x} = C \implies f(C) = const \\
            y = Cx\text{ -- изоклины уравнения \ref{eq6}}
        \end{array}
    \]

    \begin{enumerate}
        \item \begin{enumerate}
                  \item Уравнение вида $y' = f\left(\frac{a_1x + b_1y + c_1}{a_2x + b_2y + c_2}\right)$ сводится к однородному с помощью замены:
                        \[
                            \left\{\begin{array}{l}
                                x = \xi + \alpha \\
                                y = \eta + \beta
                            \end{array}\right.,\quad (\alpha;\beta)\text{ -- решение системы }\left\{\begin{array}{l}
                                a_1x + b_1y + c_1 = 0 \\
                                a_2x + b_2y + c_2 = 0
                            \end{array}\right.
                        \]
                  \item Прямые параллельны, то есть $a_1x + b_1y = k(a_2x + b_2y)$.

                        Замена переменной $t = a_2x + b_2 y$ и привести к уравнению с РП.
              \end{enumerate}
        \item Замена переменной $y = t^m$. Подставить эту замену в уравнение и из условия однородности выбрать $m$.
    \end{enumerate}
\end{note}

\begin{example}
    $ydx + x(2xy + 1)dy = 0, \quad ydx + (2x^2 + x)dy = 0$
    \[
        y = t^m \implies t^mdx + (2x^2t^m + x)\cdot mt^{m-1}dt = 0
    \]
    \begin{align*}
        m = 2m + 1 = m \\
        m = -1 \implies y = t^{-1} = \frac{1}{t}
    \end{align*}
    \begin{multline*}
        \frac{dx}{t} + \left(\frac{2x^2}{t} + x\right)\left(-\frac{1}{t^2}\right)dt = Q \implies \\
        \implies dx - \frac{x}{t}\left(\frac{2x}{t} + 1\right)dt = 0\text{ -- однородное уравнение}\implies \\
        \implies \frac{dx}{dt}=\frac{x}{t}\left(\frac{2x}{t} + 1\right)\equiv f\left(\frac{x}{t}\right)
    \end{multline*}
    Замена переменной: $u = \frac{x}{t}\implies x = ut \implies dx = udt + tdu$
    \begin{align*}
        udt + tdu - u(2u + t)dt = 0               \\
        tdu - 2u^2dt = 0, \quad t\ne0, \ u^2\ne 0 \\
        \int \frac{du}{2u^2} = \int \frac{dt}{t}  \\
        -\frac{1}{2u} = \ln|t| + C
    \end{align*}
    \[
        \left\{\begin{array}{l}
            u = \frac{x}{t} \\
            t=\frac{1}{y}
        \end{array}\right. \implies \left\{\begin{array}{l}
            u = xy \\ t = \frac{1}{y}
        \end{array}\right., \quad \begin{array}{l}
            -\frac{1}{xy} = \ln \left(\frac{1}{y}\right)^2 + C \\
            -\frac{1}{xy} = \ln y^2 + C                        \\
        \end{array}
    \]
    \[
        \ln y^2 - \frac{1}{xy} = C
    \]
    \[
        u = 0 \implies x\cdot y = 0 \implies \left[\begin{array}{l}
            x = 0\text{ -- решение} \\
            y = 0\text{ -- решение}
        \end{array}\right.
    \]

    Ответ: $\ln y^2 - \frac{1}{xy} = C, \quad x=0, \ y=0$
\end{example}

\section{Линейные уравнения $1$-го порядка}

\begin{definition}[Линейное уравнение $1$-го порядка]
    \emph{Линейным уравнением $1$-го порядка} называется уравнение вида:
    \begin{equation}\label{eq8}
        a_0(x)y' + a_1(x)y = b(x),
    \end{equation}
    \begin{equation}\label{eq9}
        a_0(x)y' + a_1(x)y = 0,
    \end{equation}
    где $b(x), a_0(x),a_1 \in C(\alpha;\beta), \ a_0(x) \ne 0, \quad -\infty \leqslant \alpha \leqslant\beta\leqslant+\infty$.
\end{definition}

\begin{note}[Задача Коши]
    $y(x_0) = y_0$
\end{note}

\begin{theorem}[$\exists$ и $!$]
    \[
        y' = \underbrace{\frac{b(x)}{a_0(x)} - \frac{a_1(x)}{a_0(x)}y}_{f(x,y)}
    \]
    \begin{enumerate}
        \item $f\in C\big((\alpha;\beta) \times (-\infty;+\infty)\big)$.
        \item $f'_y = -\frac{a_1(x)}{a_0(x)}$
    \end{enumerate}

    Однородное уравнение \ref{eq9}.
    \[
        y = 0\text{ -- решение: }y = c\cdot e^{-\int \frac{a_1(x)}{a_0(x)}dx}
    \]
\end{theorem}

\begin{definition}[Линейное уравнение $1$-го порядка]
    \emph{Линейным уравнением $1$-го порядка} называется уравнение вида:
    \begin{equation}\label{eq10}
        y'+p(x)\cdot y = q(x),
    \end{equation}
    \begin{equation}\label{eq11}
        y' + p(x)\cdot y = 0,
    \end{equation}
    где $p(x),q(x) \in C(\alpha;\beta)$.
\end{definition}

\begin{note}[Свойства \ref{eq11}]\leavevmode
    \begin{enumerate}
        \item Пусть $y_1(x)$ -- решение \ref{eq11} $\implies k\cdot y_1$ -- решение \ref{eq11}, $k \in \mathbb{R}(\mathbb{C})$.
        \item Если $y_1,y_2$ -- решения \ref{eq11} $\implies y_1 + y_2$ -- решение \ref{eq11} ($k_1y_1 + k_2y_2$ -- решение).
        \item $y=0$ -- решение \ref{eq11}.
              \begin{statement}
                  Решения \ref{eq11} образуют линейное пространство:
                  \[
                      y' + p(x)y = 0
                  \]
                  \begin{align*}
                       & \frac{dy}{y} = -p(x)dx         \\
                       & \ln|y| = -\int p(x)dx + \ln|C|
                  \end{align*}
                  \[
                      (\star) \ y = C \cdot e^{-\int p(x)dx}
                  \]
                  \begin{align*}
                       & \left\{\begin{array}{rl}
                                    y' + p(x)y & = 0   \\
                                    y(x_0)     & = y_0
                                \end{array}\right., \quad [x_0;x]                   \\
                       & \int_{x_0}^{x}\frac{y'(s)}{y(s)}ds = -\int_{x_0}^{x}p(s)dx
                  \end{align*}
                  \[
                      \int_{y_0}^{y}\frac{dy}{y} = -\int_{x_0}^{x}p(s)dx \implies \ln|y| - \ln|y_0| = - \int_{x_0}^{x}p(x)dx \implies
                  \]
                  \[
                      (\star\star) \ \implies y = y_0\cdot e^{-\int_{x_0}^{x}p(x)dx},
                  \]
                  при $C = y_0$.
              \end{statement}
        \item Если $y$ -- частное решение, то $C \cdot y$ -- общее решение \ref{eq11}.
    \end{enumerate}
\end{note}
\lesson{3}{от 29 фев 2024 12:45}{Продолжение}


\begin{definition}[Связное множество]
    $ A \subset \overline{\Comp} $ называется \emph{связным}, если $ \nexists U,V \in O_P \overline{\Comp}: \ U \cap A \ne \O, \ U \cap V = \O $.
    \[
        O_P \overline{\Comp} \text{ -- совокупность всех открытых множеств}
    \]
\end{definition}

\begin{example}
    $ A = \big\{(0,y): \ -1 \leqslant y \leqslant 1\big\} \cup \left\{\left(x,\sin\frac{1}{x}\right): \ 0 < x \leqslant 1\right\} $ -- связное.
\end{example}

\begin{definition}[Линейно связное множество]
    Множество называется \emph{линейно связным}, если любые его точки можно соединить путем, значения которого лежат в этом множестве.
\end{definition}

\begin{remark}
    В пространстве $ \R^n $, и в частности $ \overline{\Comp} $, любое открытое множество связно $ \iff $ оно линейно связно.
\end{remark}

\begin{definition}[Область]
    \emph{Областью} в $ \overline{\Comp} $ называется любое непустое открытое связное множество.
\end{definition}

\begin{definition}[Замкнутая область]
    \emph{Замкнутой областью} будем называть замыкание области.
\end{definition}

\section{Функции комплексного переменного}

\subsection{Структура функции комплексного переменного}

\begin{note}
    $ f : \Comp \longrightarrow \Comp $
    \[
        \begin{array}{l}
            dom f \text{ -- область определения функции} \\
            im f \text{ -- область значения функции}
        \end{array}
    \]
\end{note}

\begin{definition}[Предел отображения]
    $ D \subset dom f, \ z_0 \in \overline{\Comp} $ -- предельная точка $ D $. Тогда $ w_0 \in \overline{\Comp} $ называется \emph{пределом отображения} $ f $,
    \[
        w_0 \coloneqq \underset{D \circ z \rightarrow z_0}{\lim}f(z) \text{, если }\forall V \in O_{w_0} \ \exists U \in O_{z_0}: \ f(\mathring{U}\cap D)\subset V,
    \]
    \[
        U \in O_{z_0}, \quad \mathring{U} = U\setminus\{z_0\}.
    \]
\end{definition}

\begin{note}
    В случае, когда $ z_0,w_0 \in\Comp $ следует, что $ \forall \epsilon > 0 \ \exists \delta > 0 : \ \forall z \in D $
    \[
        0 < | z - z_0 | < \delta \implies \big| f(z) - w_0 \big| < \epsilon.
    \]
\end{note}

\begin{definition}[Непрерывная функция в точке]
    Функция $ f $ называется \emph{непрерывной в точке} $ z_0 \in \Comp $, если:
    \begin{enumerate}
        \item $ z_0 \in dom f $.
        \item $ \forall \epsilon > 0 \ \exists \delta > 0: \ \forall z \in D $
              \[
                  0 < | z - z_0 | < \delta \implies | f(z) - w_0 | < \epsilon.
              \]
    \end{enumerate}
\end{definition}

\newpage

\begin{definition}[Непрерывная функция на множестве]
    Функция $ f : \Comp \longrightarrow \Comp $ непрерывна на $ D \subset \Comp $, если
    \begin{enumerate}
        \item $ D \subset dom f $.
        \item $ \forall z_0 \in D \ \forall \epsilon > 0 \ \exists \delta > 0 \ \forall z \in D $
              \[
                  | z - z_0 | < \delta \implies \big|f(z) - f(z_0)\big| < \epsilon.
              \]
    \end{enumerate}
\end{definition}

\begin{note}[Функция Дирихле]
    $ D(x) = \left\{\begin{array}{ll}
            1, & x \in \Q              \\
            0, & x \in \R \setminus \Q
        \end{array}\right. $, непрерывна на $ \Q $, непрерывна на $ \R \setminus \Q $.
\end{note}

\begin{remark}
    Если множество является открытым или совпадает с областью определения функции, то непрерывность функции на этом множестве равносильно ее непрерывности в каждой точке.
    \[
        f_n: \Comp \rightarrow \Comp (n \in \N), \quad D \coloneqq \underset{n \in \N}{\bigcap} dom f_n.
    \]
\end{remark}

\begin{definition}
    $ A \subset D, \ f : A \rightarrow \Comp, \ f_n \rightrightarrows f $ на $ A $, если $ \forall \epsilon > 0 \ \exists n_0 \in \N : \ \forall z \in A \ \forall n \geqslant n_0 $
    \[
        \big|f_n(z) - f(z)\big| < \epsilon.
    \]

    ($ \forall \epsilon > 0 \ \exists n_0 \in \N:  \forall n \geqslant n_0 \quad \underset{z \in A}{\sup} \big| f_n(z) - f(z) \big| < \epsilon, \ | z - z_0 | < \delta \implies \big| f(z) - f(z_0) \big| < \epsilon $).
\end{definition}

\begin{theorem}[Вейерштрасса]
    Если $ \{f_n\}_{n\in\N} \subset C(A), \ f_n \rightrightarrows f $, то $ f \in C(A) $.
\end{theorem}

\begin{definition}[Функциональный ряд]
    \emph{Функциональным рядом} называется формальная сумма членов последовательности функции.
    \[
        \text{Обозначение: } \sum_{n=1}^{\infty}f_n.
    \]
\end{definition}

\begin{definition}[Числовой ряд]
    $ \forall z \in D \ \sum_{n=1}^{\infty}f_n(z) $ называется \emph{числовым рядом} $ \{f_n(z)\}_{n \in \N} $.
    \[
        S_n \coloneqq \sum_{k=1}^{n}f_k \text{ -- частичная сумма}.
    \]
\end{definition}

\begin{theorem}[Признак Вейерштрасса]
    $ \sum_{n=1}^{\infty}f_n $ таков, что $ \forall n \in \N \ \forall z \in A \ | f_n | \leqslant c_n $, причем $ \sum_{n=1}^{\infty} c_n $ сходится. Тогда ряд $ \sum_{n=1}^{\infty} f_n $ равномерно абсолютно сходится на $ A $.
\end{theorem}

\begin{theorem}[Критерий Коши (равномерная сходимость)]
    $ \{f_n\}_{n\in\N} $ равномерно сходится на $ A \iff \forall \epsilon > 0 \ \exists n_0 \in \N: \forall n,m \geqslant n_0 $
    \[
        \underset{z \in A}{\sup}\big|f_n(z) - f_n(z_0)\big| < \epsilon.
    \]
\end{theorem}

\begin{definition}[Линейная функция]
    Функция $ f : \Comp \longrightarrow \Comp $ называется \emph{линейной}, если $ \forall \alpha, \beta \in \Comp \ \forall z_1,z_2 \in \Comp $
    \[
        f(\alpha z_1 + \beta z_2) = \alpha f(z_1) + \beta f(z_2).
    \]
\end{definition}

\begin{remark}
    Функция $ f: \Comp \longrightarrow \Comp $ является линейной $ \iff \exists a \in \Comp : \forall z \in \Comp $
    \[
        f(z) = az
    \]
\end{remark}
\lesson{4}{от 7 мар 2024 12:45}{Продолжение}

\lesson{5}{от 5 дек 2023 10:29}{Продолжение}


\begin{theorem}
    Пусть $y_1(x), \ldots, y_n(x)$ -- система линейно независимых на $(\alpha;\beta)$ решений уравнения $Ly = 0$. Тогда $W(x) \ne 0$ ни в какой точке интервала $(\alpha;\beta)$.
\end{theorem}

\begin{proof}
    От противного. Предположим, что $\exists x_0 \in (\alpha;\beta)$. $W(x_0) = 0$,
    \[
        W(x_0) = \left|\begin{array}{ccc}
            y_1(x_0)         & \cdots & y_n(x_0)         \\
            y_1'(x_0)        & \cdots & y_n'(x_0)        \\
            \vdots           & \ddots & \vdots           \\
            y_1^{(n-1)}(x_0) & \cdots & y_n^{(n-1)}(x_0) \\
        \end{array}\right| = 0, \quad \left(\begin{array}{c}
                c_1    \\
                c_2    \\
                \vdots \\
                c_n
            \end{array}\right)
    \]

    \begin{equation}\label{eq35}
        \left\{\begin{array}{l}
            c_1y_1(x_0) + \ldots + c_ny_n(x_0) = 0                 \\
            c_1y_1'(x_0) + \ldots + c_ny_n'(x_0) = 0               \\
            \vdots                                                 \\
            c_1y_1^{(n-1)}(x_0) + \ldots + c_ny_n^{(n-1)}(x_0) = 0 \\
        \end{array}\right.
    \end{equation}

    Однородная система линейно алгебраических уравнений, $\det = W(x_0) = 0, \implies$ система \ref{eq35} имеет нетривиальное решение: $\overrightarrow{c^0} - (c_1^0,c_2^0,\ldots,c_n^0)$, $y_1(x),\ldots,y_n(x)$ -- линейно зависимая?
    \[
        y(x) = c_1^0 \cdot y_1(x) + \ldots + c_n^0 \cdot y_n(x)
    \]
    \begin{enumerate}
        \item $y(x)$ -- решение $Ly = 0$;
        \item $\left\{\begin{array}{l}
                      y(x_0) = 0                                                        \\
                      y'(x_0) = c_1^0 y_1'(x) + \ldots + c_n^0y_n'(x)\bigg|_{x=x_0} = 0 \\
                      \vdots                                                            \\
                      y^{(n-1)}(x_0) = c_1^0 y_1^{(n-1)}(x) + \ldots + c_n^0y_n^{(n-1)}(x)\bigg|_{x=x_0} = 0
                  \end{array}\right.$
    \end{enumerate}
    \begin{enumerate}
        \item $y\equiv0 \implies Ly = 0 \implies$ из теоремы существования и единственности $\implies y(x) = \sum_{k=1}^{n}c_k^0 y_k(x) \equiv 0 \implies y_1,\ldots,y_n$ -- линейно зависимые $\implies$ противоречие.
    \end{enumerate}
\end{proof}

\begin{theorem}[Лиувилля-Остроградского $\big(W(x), \ W(x_0)\big)$]
    Пусть задано уравнение:
    \[
        a_0(x)\cdot y^{(n)} + a_1(x)\cdot y^{(n-1)} + \ldots + a_{n-1}(x)\cdot y' + a_n(c)\cdot y = 0,
    \]
    $a_0(x) \ne 0, \ a_j(x)\in C(\alpha;\beta), \ j = \overline{0,n}, \ -\infty\leqslant\alpha<\beta\leqslant+\infty$. Тогда:
    \[
        W(x) = W(x_0) \cdot e^{-\int_{x_0}^{x}\frac{a_1(s)}{a_0(s)}ds}, \quad x_0 \in (\alpha;\beta)
    \]
\end{theorem}

\begin{corollary}
    Если $\exists x_0 \in (\alpha;\beta): \ W(x_0) = 0 \implies W(x) = 0 \ \forall x \in (\alpha;\beta)$
\end{corollary}

\section{Построение общего решения уравнения \\ $Ly = 0$}

\begin{definition}[Решение $Ly=0$]
    Функция $y = \phi(x,C_1,C_2,\ldots,C_n)$ называется \emph{решением $Ly = 0$}, если для $\forall$ набора $C_1,C_2,\ldots,C_n$ она является решением $Ly = 0$ и для $\forall$ задачи Коши $y(x_0) = y_0^\circ, \ y'(x_0) = y_1^\circ, \ \ldots, \ y^{(n-1)}(x_0) = y_{n-1}^\circ \ \exists$ набор $C_1^\circ,C_2^\circ,\ldots,C_n^\circ: \ y = \phi(x,C_1^\circ,\ldots,C_n^\circ)$ является решением $Ly = 0$.
\end{definition}

\begin{theorem}[Структура решения однородного уравнения]
    Пусть \\ $y_1(x),\ldots,y_n(x)$ -- линейно независимые решения $Ly=0 \ n$-го порядка. Тогда:
    \[
        y_{\text{ОО}} = C_1y_1(x) + \ldots + C_ny_n(x),
    \]
    где $C_1,\ldots,C_n$ -- произвольные константы.
\end{theorem}

\begin{definition}[Фундаментальная система решений (ФСР)]
    Любые $n$ линейно независимых решений задачи Коши уравнения $Ly=0$ называются \emph{фундаментальной системой решений (ФСР)}.
\end{definition}

\begin{theorem}
    ФСР уравнения $Ly=0$ -- существует.
\end{theorem}

\begin{theorem}
    Любые $(n+1)$ решения задачи Коши для $Ly = 0 n$-го порядка линейно зависимы, то есть $\exists \alpha_1,\ldots,\alpha_{n+1}$:
    \[
        \alpha_1^2 + \ldots + \alpha_{n+1}^2 \ne 0, \quad \alpha_1y_1(x) + \alpha_2y_2(x) + \ldots + \alpha_{n+1}y_{n+1}(x) = 0
    \]
\end{theorem}

\section{Линейные уравнения с переменными коэффициентами}

\begin{note}
    Решения линейного однородного уравнения $n$-го порядка с переменными коэффициентами -- линейное пространство размерности $n$ с базисом ФСР.

    \emph{Нормированная ФСР} -- это задача Коши с начальными условиями:
    \begin{eqnarray*}
        (1,0,\ldots,0),(0,1,\ldots,0),\ldots,(0,0,\ldots,1)
    \end{eqnarray*}

    Если имеем $y_1,y_2,\ldots,y_n$ -- решений $Ly = 0$ и $\exists x_0 \in (\alpha;\beta): \ W(x_0) \ne 0$, то пытаемся восстановить дифференциальное уравнение.
\end{note}

\begin{example}
    $n = 2, \ \{\sin x, \cos x\}, \ w(x) = \left|\begin{array}{cc}
            \sin x & \cos x   \\
            \cos x & - \sin x
        \end{array}\right| = -1 \ne 0$
    \[
        a_0(x)y'' + a_1(x)y' + a_2(x)y = 0, \quad a_0(x) \ne 0
    \]
    \[
        y'' + p(x)y' + q(x) \cdot y = 0
    \]

    \[
        \left\{\begin{array}{l}
            -\sin x + p(x) \cdot \cos x + q(x)\cdot \sin x = 0 \\
            -\cos x - p(x) \cdot \sin x + q(x)\cdot \cos x = 0
        \end{array}\right.
    \]

    \[
        \begin{pmatrix}
            \cos x & \sin x \\ -\sin x & \cos x
        \end{pmatrix} \cdot \begin{pmatrix}
            p(x) \\ q(x)
        \end{pmatrix} = \begin{pmatrix}
            \sin x \\ \cos x
        \end{pmatrix}
    \]
    \[
        \begin{array}{l}
            \Delta = \left|\begin{array}{cc}
                               \cos x & \sin x \\ -\sin x & \cos x
                           \end{array}\right| = \cos^2 x + \sin^2 x = 1 \ne 0 \\
            \Delta_1 = \left|\begin{array}{cc}
                                 \sin x & \sin x \\ \cos x & \cos x
                             \end{array}\right| = 0                \\
            \Delta_2 = \left|\begin{array}{cc}
                                 \cos x & \sin x \\ -\sin x & \cos x
                             \end{array}\right| = \cos^2 x + \sin^2x = 1
        \end{array}
    \]

    \[
        \begin{array}{l}
            p(x) = \frac{\Delta_1}{\Delta} = 0 \\
            q(x) = \frac{\Delta_2}{\Delta} = 1
        \end{array} \implies y'' + y = 0
    \]
\end{example}

\begin{note}[Способы восстановления дифференциального уравнения]\leavevmode
    \begin{enumerate}
        \item Способ первый:
              \[
                  \left\{\begin{array}{l}
                      y^{(n)} + p_{n-1}(x)\cdot y^{(n-1)} + \ldots + p_1(x)\cdot y' + p_0(x)\cdot y = 0         \\
                      y^{(n)}_1 + p_{n-1}(x)\cdot y^{(n-1)}_1 + \ldots + p_1(x)\cdot y'_1 + p_0(x)\cdot y_1 = 0 \\
                      y^{(n)}_2 + p_{n-1}(x)\cdot y^{(n-1)}_2 + \ldots + p_1(x)\cdot y'_2 + p_0(x)\cdot y_2 = 0 \\
                      \vdots                                                                                    \\
                      y^{(n)}_n + p_{n-1}(x)\cdot y^{(n-1)}_n + \ldots + p_1(x)\cdot y'_n + p_0(x)\cdot y_n = 0 \\
                  \end{array}\right.
              \]
              \begin{multline*}
                  \Delta = \left|\begin{array}{cccc}
                      y_1    & y_1'   & \ldots & y_1^{(n-1)} \\
                      y_2    & y_2'   & \ldots & y_2^{(n-1)} \\
                      \vdots & \vdots & \ddots & \vdots      \\
                      y_n    & y_n'   & \ldots & y_n^{(n-1)}
                  \end{array}\right|\begin{array}{l}
                      p_0 \\ p_1 \\ \vdots \\ p_n
                  \end{array} = \\
                  = \left|\begin{array}{cccc}
                      y_1         & y_2         & \ldots & y_n         \\
                      y_1'        & y_2'        & \ldots & y_n'        \\
                      \vdots      & \vdots      & \ddots & \vdots      \\
                      y_1^{(n-1)} & y_2^{(n-1)} & \ldots & y_n^{(n-1)}
                  \end{array}\right| = W(x) \ne 0
              \end{multline*}
              $(\implies y_1,y_2,\ldots,y_n\text{ -- ЛНЗ}) \implies$ система имеет $!$ решение \\ $p_0(x),p_1(x),\ldots,p_{n-1}(x)$, которое выражается через $y_1,y_2,\ldots,y_n$ и их производные.

        \item Способ второй: потерян
    \end{enumerate}
\end{note}

\begin{example}
    По второму способу:

    $y_1 = x, \ y_2 = x^2, \ W(x) = \left|\begin{array}{cc}
            y_1 & y_2 \\ y_1' & y_2'
        \end{array}\right| = \left|\begin{array}{cc}
            x & x^2 \\ 1 & 2x
        \end{array}\right| = 2x^2 - x^2 = x^2 \ne 0, \ \text{при } x\ne 0$.
    \begin{multline*}
        \left|\begin{array}{ccc}
            y_1'' & y_1' & y_1 \\
            y_2'' & y_2' & y_2 \\
            y_1'' & y_1' & y_1 \\
        \end{array}\right| = 0 \iff \left|\begin{array}{ccc}
            0   & 1  & x   \\
            2   & 2x & x^2 \\
            y'' & y' & y
        \end{array}\right| = 0 \iff \\
        \iff y'' \cdot \left|\begin{array}{cc}
            2x & x^2 \\ y' & y
        \end{array}\right| - y' \cdot \left|\begin{array}{cc}
            0 & x \\ 2 & x^2
        \end{array}\right| + y \cdot \left|\begin{array}{cc}
            0 & 1 \\ 2 & 2x
        \end{array}\right| = 0
    \end{multline*}
    \[
        x^2 \cdot y'' - 2x \cdot y' + 2y = 0
    \]
\end{example}

\section{Структура общего решения линейного неоднородного уравнения $Ly = f$}

\begin{note}
    \begin{equation}\label{eq36}
        Ly = a_0(x) \cdot y^{(n)} + a_1(x) \cdot y^{(n-1)} + \ldots + a_{n-1}(x)\cdot y' + a_n(x) \cdot y = f(x),
    \end{equation}
    где $a_0(x)\ne0, \ a_j(x), \ f(x) \in C(\alpha;\beta), \ j = \overline{0,n}, \ -\infty \leqslant \alpha <\beta \leqslant +\infty$
\end{note}

\begin{theorem}
    Все решения уравнения вида \ref{eq36} даются формулой:
    \begin{equation}\label{eq37}
        y_{\text{ОН}} = y_{\text{ОО}} + y_{2\text{ЧН}}
    \end{equation}
\end{theorem}

\begin{proof}
    Пусть $y_{2\text{ЧН}}$ -- произвольное частное решение \ref{eq36}, то есть
    \[
        L(y_{2\text{Н}}) = f(x)
    \]
    \begin{enumerate}
        \item Покажем, что решение \ref{eq37} удовлетворяет \ref{eq36}:
              \[
                  L(y_{\text{ОН}}) = L(y_{\text{ОО}} + y_{2\text{ЧН}}) = L(y_{\text{ОО}}) + L(y_{2\text{ЧН}}) = 0 + f(x) = f(x)
              \]

        \item Покажем, что формула \ref{eq37} покрывает все решения \ref{eq36}:

              $\widetilde{y}$ -- частное решение \ref{eq36}, $L(\widetilde{y}) = f(x)$
              \[
                  \widetilde{y} = (\widetilde{y} - y_{\text{ЧН}}) + y_{\text{ЧН}}
              \]
              \begin{multline*}
                  L(\widetilde{y} - y_{\text{ЧН}}) = L(\widetilde{y}) - L(y_{\text{ЧН}}) = f(x) - f(x) = 0 \implies \\
                  \implies \widetilde{y} = (\widetilde{y} - y_{\text{ЧН}}) + y_{\text{ЧН}} = y_{\text{ОО}} + y_{\text{ЧН}}
              \end{multline*}
    \end{enumerate}
\end{proof}

\subsection*{Построение общего решения неоднородного уравнения \ref{eq36}}

\begin{note}[Метод вариации произвольных постоянных]\leavevmode
    \begin{enumerate}
        \item $y_1,y_2,\ldots,y_n$ -- ФСР уравнения \ref{eq36} $\implies y_{\text{ОО}} = C_1y_1(x) + \ldots + C_ny_n(x)$.
        \item $y = y_{\text{ОН}} = C_1(x)y_1(x) + \ldots + C_n(x)y_n(x)$. Найдем производные до $n$-го порядка:
              \[
                  \begin{array}{rl}
                      a_{n}(x):   & C_1(x)y_1(x) + \ldots + C_n(x)y_n(x)                                                                                                    \\
                      a_{n-1}(x): & C_1(x)y'_1(x) + \ldots + C_n(x)y'_n(x) + \equalto{\underbrace{C_1'(x)y_1(x) + \ldots + C_n'(x)y_n(x)}}{0}                               \\
                      a_{n-2}(x): & C_1(x)y''_1(x) + \ldots + C_n(x)y''_n(x) + \equalto{\underbrace{C_1'(x)y_1'(x) + \ldots + C_n'(x)y_n'(x)}}{0}                           \\
                      \vdots      & \vdots                                                                                                                                  \\
                      a_1(x):     & C_1(x)y^{(n-1)}_1(x) + \ldots + C_n(x)y^{(n-1)}_n(x) + \equalto{\underbrace{C_1'(x)y_1^{(n-2)}(x) + \ldots + C_n'(x)y_n^{(n-2)}(x)}}{0} \\
                      a_0(x):     & C_1(x)y^{(n)}_1(x) + \ldots + C_n(x)y^{(n)}_n(x) + \equalto{\underbrace{C_1'(x)y_1^{(n-1)}(x) + \ldots + C_n'(x)y_n^{(n-1)}(x)}}{0}     \\
                  \end{array}
              \]
              \[
                  C_1(x)\equalto{\underbrace{\big(a_0(x)y_1^{(n)}(x) + a_1(x)y_1^{(n-1)}(x) + \ldots + a_{n_1}(x)y_1' + a_n(x)y_1(x)\big)}}{0}
              \]

              Система $n$ уравенений и $n$ неизвестных $C_1'(x), \ldots, C_n'(x)$, определитель $\Delta = ?$
              \[
                  \Delta = \left|\begin{matrix}
                      y_1         & \cdots & y_n         \\
                      y_1'        & \cdots & y_n'        \\
                      \vdots      & \ddots & \vdots      \\
                      y_1^{(n-1)} & \cdots & y_n^{(n-1)}
                  \end{matrix}\right| = W(x) \ne 0
              \]
    \end{enumerate}
\end{note}

\begin{note}[Метод вариации произвольных постоянных (продолжение?)]
    \[
        y_{\text{ОН}} = C_1(x) \cdot y_1 + \ldots = C_n(x)y_n
    \]
    \[
        \left\{\begin{array}{l}
            C_1'(x)y_1 + \ldots + C_n'(x)y_n = 0                 \\
            C_1'(x)y_1' + \ldots + C_n'(x)y_n' = 0               \\
            \vdots                                               \\
            C_1'(x)y_1^{(n-2)} + \ldots + C_n'(x)y_n^{(n-2)} = 0 \\
            C_1'(x)y_1^{(n-1)} + \ldots + C_n'(x)y_n^{(n-1)} = \frac{f(x)}{a_0(x)}
        \end{array}\right.
    \]
    $\Delta = W(y_1,y_2,\ldots,y_n) \ne 0$, так как $y_1,y_2,\ldots,y_n$ -- ФСР $\implies$ система имеет $!$ решение. Найти это решение по формулам Крамера:
    \begin{equation}\label{eq38}
        C_k'(x) = \frac{W_k(x)}{W(x)}dx + C_k, \quad k = \overline{1,n},
    \end{equation}
    где:
    \[
        W(x) = \left|\begin{matrix}
            y_1         & \cdots & y_n         \\
            y_1'        & \cdots & y_n'        \\
            \vdots      & \ddots & \vdots      \\
            y_1^{(n-1)} & \cdots & y_n^{(n-1)}
        \end{matrix}\right|,
    \]
    \[
        W_k(x) = \left|\begin{matrix}
            y_1         & \cdots & y_n         & 0      & y_{k+1}         & \cdots & y_n         \\
            y_1'        & \cdots & y_n'        & 0      & y_{k+1}'        & \cdots & y_n'        \\
            \vdots      & \ddots & \vdots      & \vdots & \vdots          & \ddots & \vdots      \\
            y_1^{(n-1)} & \cdots & y_n^{(n-1)} & 0      & y_{k+1}^{(n-1)} & \cdots & y_n^{(n-1)}
        \end{matrix}\right|
    \]

    Проинтегрируем \ref{eq38}:

    \begin{equation}
        C_k(x) = \int \frac{W_k(x)}{W(x)}dx + C_k, \quad k = \overline{1,n},
    \end{equation}

    \[
        y(x) = y_{\text{ОН}} = \sum_{k=1}^{n}\left(\frac{W_k(x)}{W(x)}dx + C_k\right)y_k = \equalto{\underbrace{\sum_{k=1}^{n}C_ky_k}}{y_{\text{ОО}}} + \equalto{\underbrace{\sum_{k=1}^{n}y_k\int \frac{W_k(x)}{W(x)}dx}}{y_{\text{ЧН}}}
    \]
\end{note}
% end lessons

\addcontentsline{toc}{section}{Список используемой литературы}

\begin{thebibliography}{}
    \bibitem{litlink1}  Шабат  --  «Введение в комплексный анализ, 1976» (том 1)
    \bibitem{litlink2}  Привалов  --  «Введение в ТФКП, 1967»
    \bibitem{litlink3}  Бицадзе  --  «Основы теории аналитических функций комплексного переменного, 1984»
    \bibitem{litlink4}  Волковыский, Лунц, Араманович  --  «Сборник задач по ТФКП», 1975»
    \bibitem{litlink5}  Гилев В.М.  --  «Основы комплексного анализа. Ч.1», 2000»
    \bibitem{litlink6}  Исапенко К.А.  --  «Комплексный анализ в примерах и упражнениях (Ч.1, 2017, Ч.2, 2018)»
    \bibitem{litlink7}  Мещеряков Е.А., Чемеркин А.А.  --  «Комплексный анализ. Практикум»
    \bibitem{litlink8}  Боярчук А.К.  --  «Справочное пособие по высшей математике» (том 4)
\end{thebibliography}

\end{document}