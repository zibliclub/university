\lesson{1}{от 15 фев 2024 12:45}{Начало}


\section{Комплексная плоскость}

\subsection{Комплексные числа}

\begin{note}
    $ \R^2 \coloneqq \R \times \R $
    \[
        \begin{array}{l}
            (x_1,y_1) + (x_2,y_2) \coloneqq (x_1 + x_2, y_1 + y_2)          \\
            (x_1,y_1)(x_2,y_2) \coloneqq (x_1x_2 - y_1y_2, x_1y_2 - x_2y_1) \\
        \end{array}
    \]
    \begin{figure}[H]
        \centering
        \incfig{fig_02}
        \caption{$ x = r\cos\phi, \ y = r\sin\phi $}
        \label{fig:fig_02}
    \end{figure}
    \[
        \begin{array}{ll}
            z = (x,y) = x+iy \\
            \overline{z} = x - iy
        \end{array}, \quad x,y \in \R
    \]
    \[
        (1,0) \eqqcolon 1, \quad (0,1) \eqqcolon i, \quad (0,0) \eqqcolon 0
    \]
    \[
        x \eqqcolon Re z, \quad y \eqqcolon Im z
    \]
    \[
        r = \sqrt{x^2 + y^2} \eqqcolon | z |
    \]
    \[
        \phi = arg z, \underbrace{0 \leqslant arg z < 2\pi}_{\text{главное значение аргумента}}
    \]
    \[
        Argz \coloneqq argz + 2\pi k, \ k \in \Z
    \]
    \[
        \boxed{e^{i\phi} = \cos\phi + i \cdot \sin \phi, \ \forall \phi \in \R} \text{ -- формула Эйлера}
    \]
    \[
        \boxed{z = | z | (\cos argz + i\cdot \sin argz)} \text{ -- тригонометрическая форма записи}
    \]
    \[
        \boxed{z = | z | e^{i arg z}} \text{ -- показательная форма записи}
    \]
    \[
        e^z = e^{x + iy} = e^x \cdot e^{iy}, \quad e^{z_1 + z_2} = e^{z_1}\cdot e^{z_2}
    \]
    \[
        z^n = | z |^n e^{in arg z}, \quad z = re^{ir}
    \]
    \[
        \boxed{z^n = r^n(\cos n\phi + i\sin n \phi)} \text{ -- формула Муавра}
    \]
    \[
        z^n = z_0, \quad \sqrt[n]{z_0} = \sqrt[n]{| z_0 |}\cdot e^{i\frac{arg z_0 + 2\pi k}{n}}, \ 0 \leqslant k \leqslant n-1
    \]
\end{note}

\begin{theorem}[Свойства комплексных чисел]
    $ \forall z,z_1,z_2 \in \Comp $ справедливы равенства:
    \begin{multicols}{2}
        \begin{enumerate}
            \item $ z \cdot \overline{z} = | z |^2  $.
            \item $ \overline{(z_1 + z_2)} = \overline{z_1} + \overline{z_2} $.
            \item $ \overline{z_1 \cdot z_2} = \overline{z_1} \cdot \overline{z_2} $.
            \item $ \overline{\overline{z}} = z $.
            \item $ \overline{z} = z \iff z \in \R $.
            \item $ | z_1 \cdot z_2 | = | z_1 | \cdot | z_2 | $.
            \item $ | z_1 + z_2 | \leqslant | z_1 | + | z_2 | $.
            \item $ \big|| z_1 | - | z_2 |\big| \leqslant | z_1 - z_2 | $.
            \item $ \underset{(mod \ 2\pi)}{arg(z_1 \cdot z_2) = argz_1 + argz_2} $.
            \item $ \underset{(mod \ 2\pi)}{arg\left(\frac{z_1}{z_2}\right) = argz_1 - argz_2} $.
        \end{enumerate}
    \end{multicols}
\end{theorem}

\newpage

\begin{note}
    \[
        \xi = \frac{x}{1 + | z |^2 }, \quad \eta = \frac{y}{1 + | z |^2 }, \quad \zeta = \frac{| z |^2 }{1 + | z |^2 },
    \]
    \[
        \xi^2 + \eta^2 + \zeta ^2 - \zeta = 0.
    \]
    \begin{figure}[H]
        \centering
        \incfig{fig_01}
        \caption{Сфера Римана $ S $}
        \label{fig:fig_01}
    \end{figure}
    \[
        P: \Comp \xrightarrow[]{\text{на}} S \setminus \{N\}, \quad P(z) = (\xi, \eta, \zeta)
    \]
    \[
        A(x^2 + y^2) + Bx + Cy + D = 0, \quad A,B,C,D \in \R,
    \]
    \begin{center}
        $ \gamma  $ -- окружность на $ \Comp, \quad P(\Upsilon) $ -- окружность на $ S $.
    \end{center}
    \[
        | z |^2 = x^2 + y^2 = \frac{\zeta}{1 - \zeta}, \quad \left\{\begin{array}{l}
            x = \frac{\xi}{1 - \zeta} \\
            y = \frac{\eta}{1 - \zeta}
        \end{array}\right.
    \]
    \[
        A\zeta + B\xi + C\eta + D(1-\zeta) = 0,
    \]
    \[
        \overline{\Comp} \coloneqq \Comp \cup \{\infty\}, \quad P(\infty) \coloneqq N.
    \]
\end{note}

\newpage

\subsection{Топология комплексной плоскости}

\begin{note}
    $ M_1,M_2 \in \R^3 $,
    \[
        dist(M_1,M_2)\coloneqq \sqrt{(\xi_1 - \xi_2)^2 + (\eta_1 - \eta_2)^2 + (\zeta_1 - \zeta_2)^2},
    \]
    $ d(z_1,z_2) \coloneqq | z_1-z_2 |, \ z_1,z_2 \in \Comp $ -- расстояние на комплексной плоскости,
    \[
        \rho(z_1,z_2) \coloneqq dist\bigl(P(z_1),P(z_2)\bigr).
    \]
    \begin{figure}[H]
        \centering
        \incfig{fig_03}
        % \caption{Сфера Римана $ S $}
        \label{fig:fig_03}
    \end{figure}
    \[
       B_\epsilon(z_0)\coloneqq \{z \in \Comp : \ | z - z_0 | < \epsilon \}
    \]
\end{note}

\begin{definition}[Окрестность]
    Множество называется \emph{окрестностью} точки, если оно содержит шар с центром в этой точке.
    \[
       \text{Обозначение:} \quad O_z, \ z \in \overline{\Comp}.
    \]
\end{definition}

\begin{note}
    \[
        \forall z \in \Comp \quad d(z,\infty) \coloneqq +\infty,
    \]
    \[
        \begin{array}{c}
            d: \ \Comp^2 \longrightarrow \R, \\
            d: \ \Comp^2 \longrightarrow \overline{\R}, \\
            \rho: \overline{\Comp^2} \longrightarrow \R, \quad \rho(z,\infty) \in \R.
        \end{array}
    \]
\end{note}