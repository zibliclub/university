\lesson{1}{от 15 фев 2024 12:45}{Начало}


\section{Комплексная плоскость}

\subsection{Комплексные числа}

\begin{note}
    $ \R^2 \coloneqq \R \times \R $

    \[
        \begin{array}{l}
            (x_1,y_1) + (x_2,y_2) \coloneqq (x_1 + x_2, y_1 + y_2)          \\
            (x_1,y_1)(x_2,y_2) \coloneqq (x_1x_2 - y_1y_2, x_1y_2 - x_2y_1) \\
        \end{array}
    \]

    $ z = (x,y) = x+iy, \  x,y \in \R $

    $ | z | = r = \sqrt{x^2 + y^2}, \ \phi = arg z $

    \[
        Argz \coloneqq argz + 2\pi k, \ k \in \Z
    \]

    $ z = | z | (\cos argz + i\cdot \sin argz) $

    $ e^{i\phi} = i \cdot \sin\phi + \cos\phi, \ e^z = e^{x + iy} = e^xe^{iy} $

    \[
        z = | z | e^{i\cdot argz} \implies z^n = | z |^n e^{i\cdot n arg z} = | z |^n\bigl(\cos(nargz) + i\cdot\sin(nargz)\bigr)
    \]
\end{note}

\begin{theorem}[Свойства комплексных чисел]\leavevmode
    \begin{multicols}{2}
        \begin{enumerate}
            \item $ z \cdot \overline{z} = | z |^2  $.
            \item $ \overline{(z_1 + z_2)} = \overline{z_1} + \overline{z_2} $.
            \item $ \overline{z_1 \cdot z_2} = \overline{z_1} \cdot \overline{z_2} $.
            \item $ \overline{\overline{z}} = z $.
            \item $ \overline{z} = z \iff z \in \R $.
            \item $ | z_1 \cdot z_2 | = | z_1 | \cdot | z_2 | $.
            \item $ | z_1 + z_2 | \leqslant | z_1 | + | z_2 | $.
            \item $ \big|| z_1 | - | z_2 |\big| \leqslant | z_1 - z_2 | $.
            \item $ \underset{(mod \ 2\pi)}{arg(z_1 \cdot z_2) = argz_1 + argz_2} $.
            \item $ \underset{(mod \ 2\pi)}{arg\left(\frac{z_1}{z_2}\right) = argz_1 - argz_2} $.
        \end{enumerate}
    \end{multicols}
\end{theorem}

\begin{note}
    \[
        \xi = \frac{x}{1 + | z |^2 }, \quad \eta = \frac{y}{1 + | z |^2 }, \quad \zeta = \frac{| z |^2 }{1 + | z |^2 },
    \]
    \[
        \xi^2 + \eta^2 + \zeta ^2 - \zeta = 0.
    \]
    \begin{figure}[H]
        \centering
        \incfig{fig_01}
        \caption{Сфера Римана $ S $}
        \label{fig:fig_01}
    \end{figure}

    \[
        A(x^2 + y^2) + Bx + Cy + D = 0, \quad A,B,C,D \in \R,
    \]
    \begin{center}
        $ \gamma  $ -- окружность на $ \Comp, \ P(\Upsilon) $ -- окружность на $ S $.
    \end{center}

    \[
        A\xi + B\xi + C\eta + D(1-\zeta) = 0,
    \]
    \[
        \overline{\Comp} \coloneqq = \Comp \cup \{\infty\}, \quad P(\infty) \coloneqq N.
    \]
\end{note}

\subsection{Топология комплексной плоскости}

\begin{note}
    $ M_1,M_2 \in \R^3 $,
    \[
        dist(M_1,M_2)\coloneqq \sqrt{(\zeta_1 - \zeta_2)^2 + (\eta_1 - \eta_2)^2 + (\xi_1 - \xi_2)^2},
    \]
    $ d(z_1,z_2) = | z_1-z_2 |, \ z_1,z_2 \in \Comp $ -- расстояние на комплексной плоскости,
    \[
        \rho(z_1,z_2) = dist\bigl(\rho(z_1),\rho(z_2)\bigr).
    \]
\end{note}

\begin{definition}[Окрестность]
    Множество называется \emph{окрестностью}, если оно содержит шар с центром в точке $ x_0 $.
\end{definition}