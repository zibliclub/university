\section{Деревья. Первая теорема о деревьях.}

\begin{definition}[Ациклический граф, дерево]
    Граф называется \emph{ациклическим}, если в нем нет цикла. Связный ациклический граф называется \emph{деревом}.
\end{definition}

\begin{theorem}[Первая теорема о деревьях]
    Для $ (n,m) $-графа $ G $ следующие утверждения эквивалентны:
    \begin{description}
        \item[$ \quad 1. $] $ G $ -- дерево, то есть связный ациклический граф.
        \item[$ \quad 2. $] $ G $ -- связен и $ m = n - 1 $.
        \item[$ \quad 3. $] $ G $ -- ациклический и $ m=n-1 $.
    \end{description}
\end{theorem}

\section{Деревья. Вторая теорема о деревьях.}

\begin{theorem}[Вторая теорема о деревьях]
    Для $ (n,m) $-графа $ G $ следующие утверждения эквивалентны:
    \begin{description}
        \item[$ \quad 1. $] $ G $ -- дерево, то есть связный ациклический граф.
        \item[$ \quad 4. $] $ G $ -- ациклический и если любую пару его несмежных вершин соединить ребром, то полученный граф будет содержать ровно один цикл.
        \item[$ \quad 5. $] Любые две вершины графа $ G $ соединены единственной простой цепью.
    \end{description}
\end{theorem}

\section{Теорема Кэли о числе помеченных $n$-вершинных деревьев (с леммой).}

\begin{lemma}\label{lemma:1}
    При $ n \geqslant 2 $ существует взаимнооднозначное соответствие между множеством всех помеченных $ n $-вершинных деревьев с метками $ 1,2,\ldots,n $ и множеством всех слов длины $ n-2 $ в алфавите $ \{1,2,\ldots,n\} $.
\end{lemma}

\begin{theorem}[А. Кэли, 1889]
    Число различных помеченных деревьев с $ n $ вершинами равно
    \[
        t_n = n^{n-2}.
    \]
\end{theorem}

\section{Центр дерева. Центральные и бицентральные деревья. Теорема Жордана.}

\begin{note}
    $ d(u,v) $ -- \emph{длина} самой короткой простой $ (u,v) $-цепи (длина -- число ребер).
\end{note}

\begin{definition}[Эксцентриситет]
    \emph{Эксцентриситет} вершины $ v $ -- расстояние до самой удаленной от $ v $ вершины графа:
    \[
        \epsilon(v) = \underset{u \in V}{\max}d(v,u).
    \]
\end{definition}

\begin{definition}[Радиус]
    \emph{Радиус} связного графа -- это наименьший из эксцентриситетов его вершин:
    \[
        \tau(G) = \underset{v \in V}{\min}\epsilon(v).
    \]
\end{definition}

\begin{definition}[Центральная вершина]
    Вершина называется \emph{центральной}, если ее эксцентриситет равен радиусу графа.
\end{definition}

\begin{definition}[Центр графа]
    Множество центральных вершин графа называется его \emph{центром}.
\end{definition}

\begin{example}
    Центр графа:
    \begin{figure}[H]
        \centering
        \incfig{fig_25}
        \label{fig:fig_25}
    \end{figure}
\end{example}

\begin{definition}[Центральное, бицентральное дерево]
    Дерево, центр которого состоит из одной вершины, называется \emph{центральным}, а дерево, центр которого состоит из двух смежных вершин -- \emph{бицентральным}.
\end{definition}

\begin{theorem}[Жордан]
    Центр любого дерева состоит из одной или двух смежных вершин.
\end{theorem}

\section{Изоморфизм деревьев. Процедура кортежирования (на примере). Теорема Эдмондса.}

\begin{note}[Процедура кортежирования дерева]\leavevmode
    \begin{description}
        \item[Вход:] $ n $-вершинное дерево $ T = (V,E) $.
        \item[Выход:] Список натуральных чисел, представляющий кортеж $ T $.
    \end{description}
    \begin{figure}[H]
        \centering
        \incfig{fig_26}
        \label{fig:fig_26}
    \end{figure}
\end{note}

\begin{theorem}[Эдмондс]
    Для изоморфизма деревьев необходимо и достаточно, чтобы совпадали их центральные кортежи.
\end{theorem}

\section{Вершинная и реберная связность графа. Основное неравенство связности.}

\begin{definition}[Вершинная связность (связность)]
    \emph{Вершинной связностью (связностью)} обыкновенного нетривиального графа $ G $ называется наименьшее число вершин, в результате удаления которых получается несвязный или тривиальный граф:
    \[
        \mathcal{X}(G).
    \]
\end{definition}

\begin{note}
    Для тривиального графа по определению полагаем
    \[
        \mathcal{X}(O_1) = 0.
    \]
\end{note}

\begin{example}
    Для $ C_5,K_5 $ и $ C_3 $
    \begin{figure}[H]
        \centering
        \incfig{fig_27}
        \label{fig:fig_27}
    \end{figure}
\end{example}

\begin{definition}[Реберная связность]
    \emph{Реберной связностью} нетривиального графа называется наименьшее число ребер, в результате удаления которых получается несвязный граф:
    \[
        \lambda(G).
    \]
\end{definition}

\begin{example}
    $ \lambda(O_1) = 0 $,
    \begin{figure}[H]
        \centering
        \incfig{fig_28}
        \label{fig:fig_28}
    \end{figure}
\end{example}

\begin{theorem}[Основное неравенство связности]
    Для любого графа $ G $
    \[
        \mathcal{X}(G) \leqslant \lambda(G).
    \]
\end{theorem}

\section{Отделимость и соединимость. Теорема Менгера.}

\begin{definition}[Разделение вершин]
    Пусть $ G = (V,E) $ -- связный граф, $ s $ и $ t $ -- две несмежные вершины. Говорят, что множество вершин $ \Omega \subset V $ \emph{разделяет} $ s $ и $ t $, если эти вершины принадлежат разным компонентам связности графа $ G - \Omega $.
\end{definition}

\begin{definition}[$ k $-отделимые вершины]
    Несмежные вершины $ s $ и $ t $ называются \emph{$ k $-отделимыми}, если $ k $ равно наименьшему числу вершин, разделяющих $ s $ и $ t $.
\end{definition}

\begin{definition}[Вершинно-независимые цепи]
    Две простые цепи, соединяющие $ s $ и $ t $, называются \emph{вершинно-независимыми}, если они не имеют общих вершин, отличных от $ s $ и $ t $.
\end{definition}

\begin{definition}[$ l $-соединимые вершины]
    Вершины $ s $ и $ t $ называются \emph{$ l $-соединимыми}, если $ l $ равно наибольшему числу вершинно-независимых цепей.
\end{definition}

\begin{theorem}[Менгер]
    В связном графе любые две несмежные вершины $ k $-отделимы $ \iff $ они $ k $-соединимы.
\end{theorem}

\section{Реберный вариант теоремы Менгера.}

\begin{definition}[Разделение вершин]
    Пусть $ G = (V,E) $ -- связный граф, $ s $ и $ t $ -- две его произвольные вершины. Говорят, что множество ребер $ R \subset E $ \emph{разделяет} $ s $ и $ t $, если эти вершины принадлежат разным компонентам связности графа $ G - R $.
\end{definition}

\begin{definition}[$ k $-реберно-отделимые вершины]
    Вершины $ s $ и $ t $ называются \emph{$ k $-реберно-отделимыми}, если $ k $ равно наименьшему числу ребер, разделяющих $ s $ и $ t $.
\end{definition}

\begin{definition}[Вершинно-независимые цепи]
    Две простые цепи, соединяющие $ s $ и $ t $, называются \emph{вершинно-независимыми}, если они не имеют общих вершин, отличных от $ s $ и $ t $.
\end{definition}

\begin{definition}[Реберно-независимые цепи]
    Две простые цепи, соединяющие $ s $ и $ t $, называются \emph{реберно-независимыми}, если они не имеют общих ребер.
\end{definition}

\begin{definition}[$ l $-реберно-соединимые вершины]
    Вершины $ s $ и $ t $ называются \emph{$ l $-реберно-соединимыми}, если наибольшее число \\ реберно-независимых $ (s,t) $-цепей равно $ l $.
\end{definition}

\begin{theorem}[Реберный аналог теоремы Менгера]
    В связном графе любые две вершины $ k $-реберно-отделимы $ \iff $ они $ k $-реберно-соединимы.
\end{theorem}