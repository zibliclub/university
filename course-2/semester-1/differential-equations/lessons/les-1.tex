\lesson{1}{от 5 сен 2023 10:28}{Начало}


\section{Уравнение $1$-го порядка}

\begin{definition}[Дифференциальное уравнение $n$-го порядка]
    \emph{Дифференциальным уравнением $n$-го порядка} называется уравнение вида:
    \begin{equation}\label{eq1}
        F(x,y,y',\ldots,y^{(n)}) = 0, \quad x \in \underset{[a,b), \ [a,b], \ (a,b]}{(a,b)} \subset \mathbb{R},
    \end{equation}
    где $-\infty \leqslant a < b \leqslant +\infty$.
\end{definition}

\begin{definition}[Дифференциальное уравнение, разрешенное относительно старшей производной]
    \emph{Дифференциальным уравнением, разрешенным относительно старшей производной} называется уравнение вида:
    \begin{equation}\label{eq2}
        y^{(n)} = f(x,y,y',\ldots,y^{(n-1)}), \quad x \in (a;b)
    \end{equation}
\end{definition}

\begin{definition}[Решение дифференциального уравнения]
    \emph{Решением дифференциального уравнения} $\ref{eq1}$ или $\ref{eq2}$ называется $n$ раз дифференцируемая функция $y = \phi(x)$ на интервале $(a,b)$, если при подстановке она обращает уравнение в тождество на этом интервале.
\end{definition}

\begin{example}
    \[
        y = \frac{1}{x+1}, \quad (-\infty; -1) \cup (-1; -\infty),
    \]
    где $(-\infty; -1)$ -- первое решение, a $(-1; -\infty)$ -- второе решение.
\end{example}

\newpage

\begin{note}[Предмет дифференциального уравнения]\leavevmode
    \begin{enumerate}
        \item Решение дифференциального уравнения.
        \item Существует ли решение на $(a;b)$?
        \item Единственность, $y(x_0)=y_0$ (задача Коши).
        \item О продолжении.
        \item Свойства решения: \begin{itemize}
                  \item ограниченность
                  \item монотонность
                  \item поведение решения вблизи границ ($x \rightarrow +\infty$)
                  \item нули функции на $(a;b)$
              \end{itemize}
    \end{enumerate}
\end{note}

\begin{definition}[Дифференциальное уравнение $1$-го порядка]
    \emph{Дифференциальным уравнением $1$-го порядка} называется уравнение вида:
    \begin{equation}\label{eq3}
        F(x,y,y')=0, \quad x \in (a;b)
    \end{equation}
    \begin{center}
        (неразрешенное относительно $y'$)
    \end{center}
\end{definition}

\begin{definition}[Дифференциальное уравнение, разрешеноое относительно первой производной]
    \emph{Дифференциальным уравнением $1$-го порядка, разрешенным относительно первой производной}, называется уравнение вида:
    \begin{equation}\label{eq4}
        y'=f(x,y), \quad x \in (a;b)
    \end{equation}
\end{definition}

\begin{definition}[Решение дифференциального уравнения $\ref{eq3}$ и $\ref{eq4}$]
    \emph{Решением дифференциального уравнения} $\ref{eq3}$ и $\ref{eq4}$ называется дифференцируемая функция $y = \phi(x)$, обращающая уравнение в тождество на этом интервале.
\end{definition}

\begin{example}
    $y' = - \frac{x}{y}$ имеет решение $x^2 + y^2 = c$, где $c$ - произвольная константа, $c > 0$.
\end{example}

\begin{definition}[Поле направлений]
    Сопоставим любой точке $(x_0,y_0) \rightarrow y'(x_0) = f(x_0,y_0) = \tan\alpha$ направления $l$. Семейство (совокупность) направлений $l$ дает \emph{поле направлений}.
\end{definition}

\begin{definition}[Интегральная кривая]
    Кривая, касающаяся в каждой своей точке поля направлений, называется \emph{интегральной кривой}:
    \[
        y = \phi(x,c)\text{ -- интергральная кривая }\equiv\text{ график решения}
    \]
\end{definition}

\begin{definition}[Изоклины]
    Кривые, вдоль которых поле направлений постоянно, называется \emph{изоклинами}.
\end{definition}

\begin{example}
    $y' = y-x^2$

    Напишем уравнение изоклин: $y-x^2 = c$ (заменяем $y'$ на $c$)
    \begin{enumerate}
        \item $c=0 \implies y-x^2 = 0 \implies y=x^2$ (уравнение изоклины)

              $\tan \alpha = 0 \implies \alpha = 0; \quad y \ const$.
        \item $c=1 \implies y-x^2 = 1 \implies y=x^2 + 1$

              $\tan \alpha = 1 \implies \alpha = 45^{\circ}; \quad y\nearrow$
        \item $c=2 \implies y-x^2 = 2 \implies y=x^2 + 2$

              $\tan \alpha = 2 \implies \alpha = \arctan 2; \quad y\nearrow$
        \item $c=-1 \implies y-x^2 = -1 \implies y=x^2 - 1$

              $\tan \alpha = -1 \implies \alpha = -45^{\circ}; \quad y\searrow$
        \item $c=-2 \implies y-x^2 = -2 \implies y=x^2 - 2$

              $\tan \alpha = -2 \implies \alpha = -\arctan 2; \quad y\searrow$
    \end{enumerate}
    \[
        y' = 0
    \]
    \[
        y' > 0, \quad y > x^2
    \]
    \[
        y' < 0, \quad y < x^2
    \]
\end{example}

\begin{definition}[Общее решение]
    \emph{Общее решение} -- совокупность функций, которая содержит все решения уравнения.

    Если решение задается функцией $y = \phi(x,c)$ или $\psi(x,y,c) = 0$, то общее решение должно удовлетворять условиям:
    \begin{enumerate}
        \item При любом $c$ формула дает решение уравнение.
        \item Любое решение уравнения находится по формуле при некотором $c = c_0$.
    \end{enumerate}
\end{definition}

\begin{definition}[Частное решение]
    \emph{Частное решение} определяется из общего при некотором $c = c_0$.
\end{definition}

\begin{example}
    $y'=x \implies y = \frac{x^2}{2}+c$ -- общее решение,
    \[
        \left\{\begin{array}{ll}
            c=0: & y = \frac{x^2}{2},  \\
            c=1: & y=\frac{x^2}{2} + 1
        \end{array}\right.\text{ -- частное решение}.
    \]
\end{example}

\section{Уравнения с разделяющимися переменными}

\begin{definition}[Уравнения с разделяющимися переменными]
    \emph{Уравнениями с разделяющимися переменными} называются уравнения вида:
    \[
        y'=f(x)\cdot g(y)\text{ или }f_1(x) \cdot g_1(y)\cdot dx + f_2(x) \cdot g_2(y)\cdot dy = 0,
    \]
    \[
        \text{где }\begin{array}{l}
            f, \ f_1, \ f_2\text{ зависят от }x, \\
            g, \ g_1, \ g_2\text{ зависят от }y
        \end{array}
    \]
\end{definition}

\begin{note}[Алгоритм]
    \[
        \left[\begin{array}{rl}
            g(y) = 0 & \implies y = c                                                            \\
            \left\{
            \begin{array}{rl}
                g(y)            & \ne 0  \\
                \frac{y'}{g(y)} & = f(x)
            \end{array}
            \right.  & \implies \int \frac{y'dx}{g(y)} = \int f(x)dx \overset{dy=y'dx}{\implies}
        \end{array}\right.
    \]
    \[
        \overset{dy=y'dx}{\implies}\int \frac{dy}{g(y)} = \int f(x)dx \implies
    \]
    \[
        \implies\left[ \begin{array}{rl}
            y            & = \phi(x,c) \\
            \psi (x,y,c) & = 0
        \end{array}\right. \iff \left[\begin{array}{l}
            y = c_1 \\
            \left[\begin{array}{l}
                      y = \phi(y,c_2) \\
                      \psi(x,y,c_2) = 0
                  \end{array}
            \right.
        \end{array}\right.
    \]
\end{note}

\begin{example}
    $y' = xy^2$
    \[
        \left[\begin{array}{l}
            y = 0 \\
            \left\{\begin{array}{rl}
                       \frac{dy}{y^2} & = xdx \\
                       y              & \ne 0
                   \end{array}\right.
        \end{array}\right. \iff \int \frac{dy}{y^2} = \int xdx \implies -\frac{1}{y} = \frac{x^2}{2} + C \implies
    \]
    \[
        \implies\left[\begin{array}{l}
            y = -\frac{2}{x^2 + 2C}, \ C \in \mathbb{R} \\
            y = 0
        \end{array}\right.
    \]
\end{example}

\begin{theorem}[Задача Коши]
    \begin{equation}\label{eq5}
        \left\{\begin{array}{rl}
            y'     & =f(x,y) \\
            y(x_0) & = y_0
        \end{array}\right.
    \end{equation}
    \[
        f(x,y) \in C(D), \quad (x_0, y_0) \in D
    \]
\end{theorem}

\begin{example}
    $y' = \sqrt{y}$
    \[
        \left[\begin{array}{l}
            y = 0 \\
            \left\{\begin{array}{rl}
                       \frac{dy}{\sqrt{y}} & = \int dx \\
                       y                   & \ne 0
                   \end{array}\right.
        \end{array}\right. \iff 2\sqrt{y} = x + C \implies
    \]
    \[
        \implies y = \left(\frac{x + c}{2}\right)^2\text{ при }x + c \geqslant 0
    \]

    \begin{enumerate}
        \item $y = 0 \ \cup$ парабола $AB_1D_1$;
        \item $x_0$ на кривой $y = 0 \left[\begin{array}{l}
                      y = 0   \\
                      ABD     \\
                      AB_1D_1 \\
                      AB_2D_2
                  \end{array}\right.$
    \end{enumerate}

    Ответ: $\left[\begin{array}{l}
            y = 0 \\
            y = \left(\frac{x + c}{2}\right)^2, \quad x + c \geqslant 0
        \end{array}\right.$
\end{example}

\begin{definition}[Точка единственности, неединственности решения, особое решение]
    Точка $(x_0, y_0)$ называется \emph{точкой единственности решения} $y = \phi(x)$, если через нее не проходит другое решение, не совпадающее с решением $y = \phi(x)$ ни в какой окрестности этой точки.

    Остальные точки называются \emph{точками неединственности}.

    Решение, которое содержит точки неединственности, называется \emph{особым решением}.
\end{definition}

\begin{theorem}[$\exists$ и $!$-ть решения задачи Коши]
    Пусть $f(x,y)$ в \ref{eq5}:
    \begin{enumerate}
        \item Определена и непрерывна в прямоугольнике в прямоугольнике:
              \[
                  \Pi = \big\{(x,y): \ |x - x_0| \leqslant a, \ |y - y_0| \leqslant b\big\}
              \]
        \item Удовлетворяет условию Липшица по $y$ в $\Pi$:
              \begin{center}
                  \big($f_y'(x,y)$ непрерывна в $\Pi$\big)
              \end{center}
    \end{enumerate}

    Тогда $\exists !$ решение задачи \ref{eq5} в окрестности точки $x_0$:
    \[
        (x_0 - h; x_0 + h),
    \]
    где $h = \min\left(a;\frac{b}{M}\right), \ M = \max|f(x,y)|, \ (x,y) \in \Pi$.
\end{theorem}

\begin{definition}[Функция, удовлетворяющая условию Липшица]
    $f(x,y)$ \emph{удовлетворяет условию Липшица} по переменной $y$, если $\exists L > 0$ такая, что $\forall (x,y_1)$ и $(x,y_2)$ имеет место соотношение:
    \[
        \big|f(x,y_1) - f(x,y_2)\big| \leqslant L \cdot|y_1 - y_2|
    \]

    Если $f_y'(x,y)$ -- непрерывна в $\Pi$, то выполняется условие Липшица: $\forall (x,y_1), \ (x,y_2) \in \Pi, \ \exists \widetilde{y} \in [y_1;y_2]$:
    \[
        \big|f(x,y_1) - f(x,y_2)\big| \leqslant \big|f_y'(x,\widetilde{y}) \cdot (y_1 - y_2)\big| \leqslant \underbrace{\big|f_y'(x,\widetilde{y})\big|}_{\leqslant L}\cdot|y_1 - y_2| = L\cdot|y_1 - y_2|
    \]
\end{definition}

\begin{example}
    $y' = \frac{1}{y^2}, \ f(x,y) = \frac{1}{y^2}, \ f_y'=\frac{2}{y^3}$
    \begin{multline*}
        \int y^2dy = \int ydx \implies \frac{y^3}{3} = x + c \implies \\
        \implies \left\{\begin{array}{rl}
            y      & = \sqrt[3]{3(x + c)} \\
            y(x_0) & = y_0
        \end{array}\right. \implies y = \sqrt[3]{3(x-x_0)} + y_0^3
    \end{multline*}
\end{example}

\begin{example}
    $y'=sign x = \left\{\begin{array}{rl}
            1,  & x > 0 \\
            0,  & x = 0 \\
            -1, & x < 0
        \end{array}\right., \ y = |x|$
\end{example}

\begin{example}
    $y' = y^2 - 2y + 1 = (y-1)^2$
    \[
        \left[\begin{array}{l}
            y = 1 \\
            \left\{\begin{array}{rl}
                       y                    & \ne 1     \\
                       \frac{dy}{(y - 1)^2} & = \int dx
                   \end{array}\right.
        \end{array}\right. \iff -\frac{1}{y-1} = x + C \implies y = 1 - \frac{1}{x + C}
    \]
    \[
        \left[\begin{array}{l}
            y = 1,                                       \\
            y = 1 - \frac{1}{x + C_1}, \ (-\infty;-C_1), \\
            y = 1 - \frac{1}{x + C_2}, \ (-C_2;+\infty)
        \end{array}\right.
    \]
\end{example}