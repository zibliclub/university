\lesson{8}{от 5 окт 2023 8:50}{Продолжение}


\begin{theorem}[Признак Раббе]
    Пусть ряд $(A)$ -- положительный. Если $ \underset{n\rightarrow\infty}{\lim}n \cdot \left(\frac{a_n}{a_{n+1}} - 1\right) = r, $ то:
    \begin{enumerate}
        \item При $r>1$ ряд $(A)$ сходится.
        \item При $r<1$ ряд $(A)$ расходится.
        \item При $r=1$ ряд $(A)$ может как сходиться, так и расходиться.
    \end{enumerate}
\end{theorem}

\begin{proof}\leavevmode
    \begin{enumerate}
        \item Пусть $r>1$. Возьмем $p$ и $q$: $ 1 < p < q < r $. Так как $ \underset{n\rightarrow\infty}{\lim}n\cdot\left(\frac{a_n}{a_{n+1}} - 1\right) = r, $ то $\exists N_1: \ \forall n > N_1 \ n\cdot\left(\frac{a_n}{a_{n+1}} - 1\right) > q$, то есть:
              \begin{equation}\label{eq:6.4}
                  \frac{a_n}{a_{n+1}} > 1 + \frac{q}{n}.
              \end{equation}

              Далее, рассмотрим:
              \[
                  \underset{n\rightarrow\infty}{\lim}\frac{(1 + \frac{1}{n})^p - 1}{\frac{1}{n}} \overset{\begin{array}{c}
                          \text{формула} \\
                          \text{Тейлора}
                      \end{array}}{=} \underset{n\rightarrow\infty}{\lim}\frac{1 + \frac{p}{n} + o(\frac{1}{n}) - 1}{\frac{1}{n}} = p < q \implies
              \]
              $\implies \exists N_2: \ \forall n > N_2$:
              \begin{equation}\label{eq:6.5}
                  \frac{(1 + \frac{1}{n})^p - 1}{\frac{1}{n}} < q \implies \left(1 + \frac{1}{n}\right)^p < 1 + \frac{q}{n}.
              \end{equation}

              Сравниваем неравенства \ref{eq:6.4} и \ref{eq:6.5}, получим, что при \\
              $n > \max(N_1,N_2)$:
              \begin{multline*}
                  \left(1 + \frac{1}{n}\right)^p < 1 + \frac{q}{n} < \frac{a_n}{a_{n+1}} \implies \\
                  \implies \frac{a_n}{a_{n+1}} > \left(1 + \frac{1}{n}^p\right) = \frac{(n+1)^p}{n^p} = \frac{\frac{1}{n^p}}{\frac{1}{(n+1)^p}}.
              \end{multline*}

              Ряд $\sum_{n=1}^{\infty}\frac{1}{n^p}$ сходится при $p > 1$:
              \[
                  \frac{a_n}{a_{n+1}} > \frac{\frac{1}{n^p}}{\frac{1}{(n+1)^p}} \implies a_n \cdot \frac{1}{(n+1)^p} > \frac{1}{n^p} \cdot a_{n+1} \implies \frac{a_{n+1}}{a_n} < \frac{\frac{1}{(n+1)^p}}{\frac{1}{n^p}}.
              \]

              По 3-му признаку сравнения, ряд $(A)$ сходится при $p > 1 \implies$ при $r > 1$.

        \item Пусть $r < 1$. Тогда $\exists N: \ \forall n > N$:
              \begin{multline*}
                  n \cdot \left(\frac{a_n}{a_{n+1}} - 1\right) < 1 \implies \\
                  \implies \frac{a_n}{a_{n+1}} < 1 + \frac{1}{n} = \frac{n+1}{n} = \frac{\frac{1}{n}}{\frac{1}{n+1}} \implies \\
                  \implies \frac{a_{n+1}}{a_n} > \frac{\frac{1}{n+1}}{\frac{1}{n}}.
              \end{multline*}

              Ряд $\sum_{n=1}^{\infty}\frac{1}{n}$ -- гармонический, расходящийся $\implies$ по 3-му признаку сравнения ряд $(A)$ расходится.

        \item \underline{Упражнение:} привести 2 примера рядов (сходящийся, расходящийся), но $r=1$ в обоих случаях.
    \end{enumerate}
\end{proof}

\begin{theorem}[Признак Кумера]
    Пусть дан ряд $(A)$ -- положительный. Пусть числа $c_1,c_2,\ldots,c_n,\ldots: \ \forall n > N \ c_n > 0$ и ряд $\sum_{n=1}^{\infty}c_n$ -- расходится. Если
    \[
        \underset{n\rightarrow\infty}{\lim}\left(c_n \cdot \frac{a_n}{a_{n+1}} - c_{n+1}\right) = k,
    \]
    то
    \begin{enumerate}
        \item При $k > 0$ ряд $(A)$ сходится.
        \item При $k < 0$ ряд $(A)$ расходится.
        \item При $k = 0$ может как сходиться, так и расходиться.
    \end{enumerate}
\end{theorem}

\begin{proof}\leavevmode
    \begin{enumerate}
        \item Пусть $k > 0$. Возьмем $0 < p < k$. Тогда $\exists N : \ \forall n > N$:
              \begin{multline*}
                  c_n \cdot \frac{a_n}{a_{n+1}} - c_{n+1}>p \implies \\
                  \implies c_n \cdot a_n - c_{n+1} \cdot a_{n+1} > p \cdot a_{n+1} > 0 \implies \\
                  \implies c_n \cdot a_n > c_{n+1} \cdot a_{n+1}, \quad \forall n > N
              \end{multline*}

              Тогда последовательность $\{c_n\cdot a_n\}$ убывает и ограничена снизу $\implies$ последовательность сходится.

              Пусть $c=\underset{n\rightarrow\infty}{\lim}c_n\cdot a_n$. Рассмотрим ряд:
              \begin{multline*}
                  \sum_{m=1}^{n}(c_m\cdot a_m - c_{m+1}\cdot a_{m+1}) = \\
                  = (c_1 \cdot a_1 - c_2 \cdot a_2) + (c_2 \cdot a_2 - c_3 \cdot a_3) + \ldots + (c_n \cdot a_n - c_{n+1}\cdot a_{n+1}) = \\
                  = c_1 \cdot a_1 - c_{n+1} \cdot a_{n+1},
              \end{multline*}
              \begin{multline*}
                  \underset{n\rightarrow\infty}{\lim}\sum_{m=1}^{n}(c_m \cdot a_m - c_{m+1} \cdot a_{n+1}) = \\
                  = \underset{n\rightarrow\infty}{\lim}(c_1 \cdot a_1 - c_{n+1} \cdot a_{n+1}) = c_1 \cdot a_1 - c \implies
              \end{multline*}
              $\implies$ сходится ряд $\sum_{n=1}^{\infty}(c_n \cdot a_n - c_{n+1} \cdot a_{n+1}) \implies$ из того, что $c_n \cdot a_n - c_{n+1} \cdot a_{n+1} > p \cdot a_{n+1} > 0$ и 1-го признака сравнения $\implies$ ряд $\sum_{n=1}^{\infty}p\cdot a_{n+1}$ сходится $\implies$ ряд $(A)$ сходится.

        \item Пусть $k < 0 \implies \exists N: \ \forall n > N$
              \begin{multline*}
                  c_n \cdot \frac{a_n}{a_{n+1}} - c_{n+1} < 0 \implies \\
                  \implies \frac{a_n}{a_{n+1}} < \frac{c_{n+1}}{c_n} = \frac{\frac{1}{c_n}}{\frac{1}{c+{n+1}}} \implies \frac{a_{n+1}}{a_{n}} > \frac{\frac{1}{c_{n+1}}}{\frac{1}{c_n}}.
              \end{multline*}

              $\sum_{n=1}^{\infty}\frac{1}{c_n}$ расходится $\implies$ по 3-му признаку сравнения ряд $(A)$ расходится.

        \item Придумать 2 примера когда $k=0$ и ряды сходятся/расходятся.
    \end{enumerate}
\end{proof}

\newpage

\begin{theorem}[Признак Бертрана]
    Пусть ряд $(A)$ -- положительный. Если
    \[
        \underset{n\rightarrow\infty}{\lim} \ln n \cdot \left[n \cdot (\frac{a_n}{a_{n+1}} - 1)\right] = B,
    \]
    то
    \begin{enumerate}
        \item При $B > 1$ ряд $(A)$ сходится.
        \item При $B < 1$ ряд $(A)$ расходится.
        \item При $B = 1$ ряд $(A)$ может как сходиться, так и расходиться.
    \end{enumerate}
\end{theorem}

\begin{proof}
    Рассмотрим ряд $\sum_{n=2}^{\infty} \frac{1}{n\cdot \ln n}$ -- расходится. Составим последовательность Кумера:
    \begin{multline*}
        k_n = \underbrace{n \cdot \ln n}_{c_n} \cdot \frac{a_n}{a_{n+1}} - \underbrace{(n+1) \cdot \ln(n+1)}_{c_{n+1}} = \\
        = \left| \ln(n+1) = \ln\left(n\cdot \frac{n+1}{n}\right) = \ln n + \ln\left(1 + \frac{1}{n}\right) \right| = \\
        = n \cdot \ln n \cdot \frac{a_n}{a_{n+1}} - (n+1)\cdot \left(\ln n + \ln\left(1 + \frac{1}{n}\right)\right) = \\
        = n \cdot \ln n \cdot \frac{a_n}{a_{n+1}} - n\cdot \ln n - \ln n - \ln\left(1 + \frac{1}{n}\right)^{n+1} = \\
        = \ln n \left(n \cdot \frac{a_n}{a_{n+1}} - n - 1\right) - \ln \left(1 + \frac{1}{n}\right)^{n+1} = \\
        = \ln n \cdot \left(n\left(\frac{a_n}{a_{n+1}} - 1\right) - 1\right) - \ln\left(1+\frac{1}{n}\right)^{n+1};
    \end{multline*}
    \begin{multline*}
        \underset{n\rightarrow\infty}{\lim} k_n = \\
        = \underset{n\rightarrow\infty}{\lim}\left[\underbrace{\ln n \cdot \left(n\left(\frac{a_n}{a_{n+1}} - 1\right) - 1\right)}_{B} - \ln\underbrace{\left(1 + \frac{1}{n}^n\right)}_{e} - \ln\left(1 + \frac{1}{n}\right)\right] = \\
        = B - 1,
    \end{multline*}
    по признаку Кумера, при $B-1 > 0$ ряд $(A)$ сходится, при $B-1 < 0$ ряд $(A)$ расходится, при $B=1$ ряд $(A)$ может как сходиться, так и расходиться.
\end{proof}

\begin{theorem}[Признак Гаусса]
    Ряд $(A), \ a_n > 0, \ \forall n \in \N, \ \lambda, \mu \in \R$. Если
    \[
        \frac{a_n}{a_{n+1}} = \left(\lambda + \frac{\mu}{n}\right) + O\left(\frac{1}{n^2}\right),
    \]
    то
    \begin{enumerate}
        \item При $\lambda > 1$, ряд $(A)$ сходится.
        \item При $\lambda < 1$, ряд $(A)$ расходится.
        \item При $\lambda = 1$ и \begin{enumerate}
                  \item $\mu > 1 \implies$ ряд $(A)$ сходится.
                  \item $\mu \leqslant 1 \implies$ ряд $(A)$ расходится.
              \end{enumerate}
    \end{enumerate}
\end{theorem}

\begin{proof}\leavevmode
    \begin{enumerate}
        \item Если $\lambda < 1$, то
              \begin{multline*}
                  \underset{n\rightarrow\infty}{\lim} \frac{a_{n+1}}{a_n} = \left[\underset{n\rightarrow\infty}{\lim}\left(\lambda + \frac{\mu}{n} + O\left(\frac{1}{n^2}\right)\right)\right]^{-1} = \\
                  = \left[\underset{n\rightarrow\infty}{\lim}\left(\lambda + \underbrace{\frac{\mu}{n}}_{\rightarrow 0} + \underbrace{\frac{1}{n^2}}_{\rightarrow 0} \cdot \Omega\left(\frac{1}{n^2}\right)\right)\right]^{-1} = \frac{1}{\lambda},
              \end{multline*}
              по признаку Даламбера, если $\frac{1}{\lambda} < 1$, то есть $\lambda > 1$, ряд $(A)$ сходится.

        \item $ \implies $ из 1.

        \item Если $\lambda = 1$, то
              \[
                  \frac{a_n}{a_{n+1}} = 1 + \frac{\mu}{n} + O\left(\frac{1}{n^2}\right),
              \]
              \[
                  n\left(\frac{a_n}{a_{n+1}} - 1\right) = \mu + n \cdot O\left(\frac{1}{n^2}\right),
              \]
              \[
                  \underset{n\rightarrow\infty}{\lim}\left(n\cdot \frac{a_n}{a_{n+1}} - 1\right) = \underset{n\rightarrow\infty}{\lim}\left(\mu + \underbrace{n \cdot \frac{1}{n^2} \cdot \Omega (\frac{1}{n^2})}_{\rightarrow 0}\right) = \mu \implies
              \]
              $\implies$ по признаку Раббе $\implies \left[\begin{array}{l}
                      \mu > 1 \implies (A) \text{ сходится.} \\
                      \mu < 1 \implies (A) \text{ расходится.}
                  \end{array} \right.$

              Пусть $\mu = 1$, тогда
              \begin{multline*}
                  \underset{n\rightarrow\infty}{\lim}\ln n \cdot \left(n \cdot \left(\frac{a_n}{a_{n+1}} - 1\right) - 1\right) = \\
                  = \underset{n\rightarrow\infty}{\lim}\ln n \cdot \left(n \cdot \left(1 + \frac{1}{n} + O\left(\frac{1}{n^2}\right) - 1\right) - 1\right) = \\
                  = \underset{n\rightarrow\infty}{\lim}\ln n \cdot \left(1 + n \cdot O\left(\frac{1}{n^2}\right) - 1\right) = \\
                  = \underset{n\rightarrow\infty}{\lim}\ln n \cdot n \cdot O\left(\frac{1}{n^2}\right) = \\
                  = \underset{n\rightarrow\infty}{\lim}\left(\ln n \cdot n \cdot \frac{1}{n^2} \cdot \Omega\left(\frac{1}{n^2}\right)\right) = \\
                  = \underset{n\rightarrow\infty}{\lim}\frac{\ln n}{n} \cdot \Omega\left(\frac{1}{n^2}\right) = 0.
              \end{multline*}
              В самом деле,
              \[
                  \underset{n\rightarrow\infty}{\lim}\frac{\ln n}{n} = \underset{n\rightarrow\infty}{\lim} \frac{1}{n} \cdot \ln n = \underset{n\rightarrow\infty}{\lim}\ln n^{\frac{1}{n}} = \underset{n\rightarrow\infty}{\lim} \ln \sqrt[n]{n} = 0 \implies
              \]
              $\implies$ по прихнаку Бертрана ряд $(A)$ расходится.
    \end{enumerate}
\end{proof}

\section{Сходимость знакопеременных рядов}

\begin{note}
    Пусть дан ряд $(A)$. Если $\exists N: \ \forall n > N \ a_n$ не меняет знак, то исследование сходимости такого ряда сводится к исследованию сходимости положительных рядов. Будем считать, что "$+$" и "$-$" бесконечно много. Такие ряды будем называть \emph{знакопеременными}.
\end{note}

\begin{definition}[Абсолютно сходящийся ряд]
    Ряд $(A)$ называется \emph{абсолютно сходящимся}, если сходится ряд
    \[
        (A^*) \ \sum_{n=1}^{\infty}|a_n|.
    \]
\end{definition}

\begin{statement}
    Если ряд $(A)$ абсолютно сходящийся, то он сходящийся.
\end{statement}

\begin{proof}
    Пусть ряд $(A)$ абсолютно сходящийся, то есть сходится ряд $(A^*) \implies$ по критерию Коши $\forall \epsilon > 0 \ \exists N: \ \forall n > N \ \forall p > 0$
    \[
        |a_{n+1}| + |a_{n+1}| + \ldots + |a_{n+1}| < \epsilon.
    \]

    Пусть $\epsilon > 0$ задано. Рассмотрим:
    \[
        |A_{n+p} - A_n| = |a_{n+1} + \ldots + a_{n+p}| \leqslant |a_{n+1}| + \ldots + |a_{n+p}| < \epsilon \implies
    \]
    $\implies$ ряд $(A)$ сходится.
\end{proof}

\begin{definition}[Условно сходящийся ряд]
    Если ряд $(A)$ сходится, а ряд $(A^*)$ расходится, то ряд $(A)$ называется \emph{условно сходящимся}.
\end{definition}

\begin{definition}[Знакочередующийся ряд]
    Ряд $(A)$ называется \emph{знакочередующимся}, если $\forall n \in \N \quad a_n \cdot a_{n+1} < 0$. Обозначим знакочередующийся ряд:
    \[
        (\overline{A}) \ \sum_{n=1}^{\infty}(-1)^{n-1}\cdot a_n, \quad a_n > 0 \ \forall n \in\N.
    \]
\end{definition}

\begin{theorem}[признак Лейбница]
    Пусть ряд $(\overline{A}), \ a_n > 0 \ \forall n$ удовлетворяет условиям:
    \begin{enumerate}
        \item $a_1 \geqslant a_2 \geqslant a_3 \geqslant \ldots \geqslant a_n \geqslant \ldots$.
        \item $\underset{n\rightarrow\infty}{\lim} a_n = 0$.
    \end{enumerate}

    Тогда ряд $(\overline{A})$ сходится и его сумма $S: \ 0 < S \leqslant a_1$.
\end{theorem}

\begin{proof}
    Рассмотрим:
    \begin{multline*}
        S_{2n} = a_1 - a_2 + a_3 - \ldots + a_{2n - 1} - a_{2n} = \\
        = (a_1 - a_2) + (a_3 - a_4) + \ldots + (a_{2n-1} - a_{2n}),
    \end{multline*}
    тогда $\forall i: \ a_i - a_{i+1} \geqslant 0 \implies S_{2n}\geqslant 0 \ \forall n \implies$ последовательность $S_{2n} \nearrow$.

    С другой стороны,
    \[
        S_{2n} = a_1 - \underbrace{(a_2 - a_3)}_{\geqslant 0} - \underbrace{(a_4 - a_5)}_{\geqslant 0} - \ldots - \underbrace{(a_{2n-2} - a_{2n-1})}_{\geqslant0} - a_{2n} \implies
    \]
    $\implies S_{2n} \leqslant a_1 \ \forall n$.

    Таким образом, $S_{2n}$ не убывает и ограничена сверху $\implies$ по теореме Вейерштрасса $\implies \exists \underset{n\rightarrow\infty}{\lim} S_{2n} = S$.

    Далее,
    \[
        \underset{n\rightarrow\infty}{\lim}S_{2n+1} = \underset{n\rightarrow\infty}{\lim}(S_{2n} + a_{2n+1}) = \underset{n\rightarrow\infty}{\lim}S_{2n} + \underset{n\rightarrow\infty}{\lim} a_{2n+1} = S + 0 = S.
    \]

    Таким образом, $\underset{n\rightarrow\infty}{\lim}S_n = S$.

    Так как $0 < S_n \leqslant a_1$ (если $S_n = 0$, то $a_1$ может быть $=0$, что невозможно, так как $a_n > 0$) $\implies$ (берем пределы от неравенства) $0 < S \leqslant a_1$.
\end{proof}

\begin{corollary}
    Если знакочередующийся ряд $(\overline{A})$ сходящийся, то сумма его $n$-го остатка имеет знак $(n+1)$-го члена ряда и не больше его по модулю.
\end{corollary}

\begin{example}
    Рассмотрим ряд
    \begin{equation}\label{eq:H}
        \sum_{n=1}^{\infty}(-1)^{n-1}\frac{1}{n} = 1 - \frac{1}{2} + \frac{1}{3} - \frac{1}{4} + \ldots + (-1)^{n-1}\frac{1}{n} + \ldots,
    \end{equation}
    по признаку Лейбница:
    \begin{enumerate}
        \item $1 > \frac{1}{2} > \frac{1}{3} > \ldots > \frac{1}{n}$;
        \item $\underset{n\rightarrow\infty}{\lim}\frac{1}{n} = 0$
    \end{enumerate}
    $\implies $ \ref{eq:H} сходится, $0 < S \leqslant 1$;
\end{example}

\begin{example}
    Рассмотрим
    \[
        \sum_{n=1}^{\infty}\left|(-1)^{n-1}\frac{1}{n}\right| = \sum_{n=1}^{\infty}\frac{1}{n} \text{ -- расходящийся} \implies
    \] $\implies$ ряд \ref{eq:H} -- условно сходящийся.
\end{example}

\begin{lemma}
    Если
    \begin{enumerate}
        \item Числа $a_1,a_2,\ldots,a_n$ либо не возрастают, либо не убывают.
        \item Суммы $B_1 = b_1, \ B_2 = b_1 + b_2, \ \ldots, \ B_n = b_1 + b_2 + \ldots + b_n: \ \forall k = 1,\ldots,n \quad |B_k| \leqslant L$.
    \end{enumerate}

    Тогда
    \begin{equation}\label{eq:6.6}
        \bigg|\sum_{k=1}^{n} a_k \cdot b_k \bigg| \leqslant L\cdot (|a_1| + |a_n|)
    \end{equation}
\end{lemma}

\begin{proof}
    Рассмотрим
    \begin{multline*}
        a_1 \cdot b_1 + a_2 \cdot b_2 + \ldots + a_n \cdot b_n = \\
        = a_1 \cdot B_1 + a_2 \cdot (B_2 - B_1) + a_3 \cdot (B_3 - B_2) + \ldots + a_n \cdot (B_n - B_{n-1}) = \\
        = a_1 \cdot B_1 + a_2 \cdot B_2 - a_2 \cdot B_1 + a_3\cdot B_3 - a_3 \cdot B_2 + \ldots + a_n \cdot B_n - a_n \cdot B_{n-1} = \\
        = B_1\cdot (a_1 - a_2) + B_2\cdot (a_2 - a_3) + B_3 \cdot (a_3 - a_4) + \ldots + B_{n-1} \cdot (a_{n-1} - a_n) + a_n \cdot B_n = \\
        = \sum_{k=1}^{n-1} B_k \cdot (a_k - a_{k-1}) + a_n \cdot B_n.
    \end{multline*}

    Таким образом,
    \begin{multline*}
        \bigg|\sum_{k=1}^{n}a_k\cdot b_k \bigg| = \bigg|\sum_{k=1}^{n-1} B_k \cdot (a_k - a_{k+1}) + a_n \cdot B_n\bigg| \leqslant \\
        \leqslant \sum_{k=1}^{n-1}|B_k| \cdot |a_k - a_{k+1} + |a_n| \cdot |B_n| \leqslant L\cdot \bigg(\sum_{k=1}^{n-1}|a_k - a_{k+1}| + |a_n|\bigg) = \\
        = L\cdot (|a_1| + |a_n| + |a_n|) = L\cdot (|a_1| + 2\cdot |a_n|).
    \end{multline*}
\end{proof}

\begin{theorem}[Признак Абеля и Дирихле]\leavevmode
    \begin{enumerate}
        \item \textbf{Абеля.} Если \begin{itemize}
                  \item последовательность $\{a_n\}$ монотонна и ограничена,
                  \item ряд $\sum_{n=1}^{\infty} b_n$ сходится,
              \end{itemize}
              то ряд $\sum_{n=1}^{\infty}a_n \cdot b_n$ сходится.

        \item \textbf{Дирихле.} Если \begin{itemize}
                  \item последовательность $\{a_n\}$ монотонна и $\underset{n\rightarrow\infty}{\lim}a_n = 0$,
                  \item частичные суммы ряда $(B)$ ограничены, то есть $\exists k > 0: $ \\ $ \forall n \ \left|\sum_{m=1}^{n} b_m\right|< k$,
              \end{itemize}
              то $\sum_{n=1}^{\infty}a_n \cdot b_n$ сходится.
    \end{enumerate}
\end{theorem}

\begin{proof}\leavevmode
    \begin{enumerate}
        \item Пусть выполнены условия признака Абеля. Тогда $\exists M > 0: \ |a_n| \leqslant M$. Пусть $\epsilon > 0$ задано. Возьмем номер $N: \ \forall n > N, \ \forall p > 0$
              \[
                  \bigg|\sum_{k=n+1}^{n+p}b_k\bigg| < \epsilon^* = \frac{\epsilon}{3\cdot M}.
              \]

              Частичные суммы ряда $\sum_{n=1}^{\infty}a_n\cdot b_n$ имеют вид $S_n = a_1\cdot b_1 + \ldots + a_n \cdot b_n$. По критерию Коши найдем $N_1: \ \forall n > N_1, \forall p > 0$
              \[
                  |S_{n+p} - S_n| < \epsilon,
              \]
              \begin{multline*}
                  |a_{n+1} \cdot b_{n+1} + a_{n+2} \cdot b_{n+2} + \ldots + a_{n+p} \cdot b_{n+p}| \leqslant \\
                  \leqslant \epsilon^* \cdot (|a_{n+1}| + 2 \cdot |a_{n+p}|) \leqslant \epsilon^* \cdot 3 \cdot M = \frac{\epsilon}{3 \cdot M} = \epsilon \implies
              \end{multline*}
              $\implies$ по критерию Коши ряд $\sum_{n=1}^{\infty}a_n \cdot b_n$ сходится.

        \item Пусть выполнены условия признака Дирихле. Так как $\underset{n\rightarrow\infty}{\lim}a_n = 0$, то $\exists N: \ \forall n > N \quad (\epsilon > 0$ задано):
              \[
                  |a_n| < \frac{\epsilon}{3 \cdot k}, \quad \bigg|\sum_{k=1}^{n}b_k\bigg| \leqslant k.
              \]

              По критерию Коши:
              \begin{multline*}
                  |S_{n+p} - S_n| = |a_{n+1} \cdot b_{n+1} + \ldots + a_{n+p} \cdot b_{n+p}| \overset{\text{по лемме}}{\leqslant} \\
                  \leqslant k\cdot(|a_{n+1}| + 2\cdot |a_{n+p}|) < k\cdot \frac{3\cdot \epsilon}{3 \cdot k} = \epsilon.
              \end{multline*}
    \end{enumerate}
\end{proof}

\begin{example}
    $\sum_{n=1}^{\infty} \frac{\sin (n\cdot x)}{n} = \sum_{n=1}^{\infty} \frac{1}{n} \cdot \sin (n\cdot x)$ \\

    $a_n = \frac{1}{n}\rightarrow 0$ при $n\rightarrow\infty$. Оценим частичную сумму $\sum_{n=1}^{\infty}\sin (n\cdot x)$:
    \begin{multline*}
        \sin x + \sin (2\cdot x) + \sin (3\cdot x) + \ldots + \sin (n\cdot x) = \\
        = \frac{1}{\sin \frac{x}{2}} \cdot \bigg(\sin x \cdot \sin \frac{x}{2} + \sin (2\cdot x) \cdot \sin\frac{x}{2}\bigg) = \\
        = \frac{1}{2} \cdot \frac{1}{\sin\frac{x}{2}} \cdot \bigg(\cos \frac{x}{2} - \cos \frac{3\cdot x}{2} + \cdot \frac{3\cdot x}{2} - \cos \frac{5\cdot x}{2} + \ldots \\
        \ldots + \cos \frac{(2\cdot n-1)\cdot x}{2} - \cos \frac{(2\cdot n+1) \cdot x}{2}\bigg) = \\
        = \frac{1}{2} \cdot \frac{1}{\sin \frac{x}{2}} \cdot \bigg(\cos \frac{x}{2} - \cos \frac{(2\cdot n + 1) \cdot x}{2}\bigg) = \\
        = \frac{2\cdot \sin \frac{(n+1)\cdot x}{2} \cdot \sin \frac{n\cdot x}{2}}{2\cdot \sin \frac{x}{2}}.
    \end{multline*}

    Тогда
    \[
        \bigg|\sum_{k=1}^{n} \sin (k\cdot x)\bigg| = \bigg|\frac{\sin\frac{(n+1)\cdot x}{2}\cdot \sin \frac{n \cdot x}{2}}{\sin \frac{x}{2}}\bigg| \leqslant \frac{1}{\sin\frac{x}{2}},
    \]
    $\frac{x}{2} \ne \pi \cdot k, \ k \in \mathbb{Z} \implies x \ne 2 \cdot \pi \cdot k, \ k \in \mathbb{Z}$. \\

    По признаку Дирихле ряд $\sum_{n=1}^{\infty}\frac{\sin (n\cdot x)}{n}$ сходится.
\end{example}

\newpage