\lesson{11}{от 17 окт 2023 10:28}{Продолжение}


\section{Поточечная и равномерная сходимость семейства функций}

\begin{definition}[Семейство функций, параметры]
    \emph{Семейство функций} -- это произвольное множество функций.

    Пусть $f:X\times T \rightarrow Y$. Если по каким-либо соображениям элементам множества $T$ уделяется особое внимание, то будем их называть \emph{параметрами}.

    То есть $\forall t \in T$ можно рассмотреть функцию
    \[
        f_t(x) = f(x,t).
    \]

    В этом случае будем говорить, что задано семейство функций, зависящих от параметра $t$.
\end{definition}

\begin{example}
    $T = \N$, тогда $f_n(x) = x^n$.
\end{example}

\begin{note}
    Пусть задано семейство отображений $f_t: \ X\rightarrow Y_\rho, \ Y$ -- метрическое пространство с заданной метрикой $\rho, \ t \in T$.

    Пусть $\mathfrak{B}$ -- база на $T$.
\end{note}

\begin{definition}[Сходимость в точке]
    Будем говорить, что \emph{семейство $\{f_t\}$ сходится в точке $x \in X$}, если $f_t(x)$ как функция аргумента $t$ имеет предел по базе $\mathfrak{B}$, то есть $\exists y_x \in Y_\rho: \ \forall \epsilon > 0 \ \exists B \in \mathfrak{B}: \ \forall t \in B$
    \[
        \rho\big(f_t(x),y_x\big) < \epsilon.
    \]
\end{definition}

\begin{definition}[Область сходимости, предельная функция]
    Множество $E = \big\{x \in X : \ \{f_t\}$ сходится в точке $x\big\}$ называется \emph{областью сходимости} семейства $\{f_t\}$ по базе $\mathfrak{B}$.

    Далее, на $E$ введем функцию, положив
    \[
        f(x) = \underset{\mathfrak{B}}{\lim}f_t(x).
    \]

    Функция $f(x)$ называется \emph{предельной}.
\end{definition}

\begin{definition}[Поточечная сходимость по базе]
    Пусть дано семейство $f_t: X \rightarrow Y_u$ и $f: X \rightarrow Y$. Будем говорить, что $f_t$ сходится по базе $\mathfrak{B}$ \emph{поточечно} к $f$ на $X$, если $\forall x \in X \ \forall \epsilon > 0 \ \exists B_x \in \mathfrak{B}: \ \forall t \in B_x$
    \[
        \rho\big(f_t(x),f(x)\big) < \epsilon.
    \]

    Обозначение:
    \[
        f_t \xrightarrow[\mathfrak{B}]{} f \ (\text{на } X)
    \]
\end{definition}

\begin{definition}[Равномерная сходимость по базе]
    Семейство $\{f_t\}$ сходится \emph{равномерно} по базе $\mathfrak{B}$ к $f$ на $X$, если $\forall \epsilon > 0 \ \exists B \in \mathfrak{B}: \ \forall t \in B$ и $\forall x \in X$
    \[
        \rho\big(f_t(x),f(x)\big) < \epsilon.
    \]

    Обозначение:
    \[
        f_t \xrightrightarrows[\mathfrak{B}]{} f \ (\text{на } X)
    \]
\end{definition}

\begin{definition}[Поточечная сходимость]
    Пусть $f_n: X \rightarrow \R$ -- последовательность функций и $f: X \rightarrow \R$. Семейство $\{f_n\}$ \emph{сходится поточечно} к $f$ на $X$, если $\forall x \in X \ \exists f(x) = \underset{n\rightarrow\infty}{\lim}f_n(x), \ \forall \epsilon > 0 \ \exists N: \ \forall n > N$
    \[
        \big|f_n(x) - f(x)\big| < \epsilon.
    \]

    Обозначение:
    \[
        f_n \xrightarrow[n\rightarrow\infty]{} f \ (\text{на } X)
    \]
\end{definition}

\begin{definition}[Равномерная сходимость]
    Последовательность $\{f_n\}$ \emph{равномерно сходится} к $f$ на $X$ при $n\rightarrow\infty$, если $\forall \epsilon > 0 \ \exists N \in \N: \ \forall n > N \ \forall x \in X$
    \[
        \big|f_n(x) - f(x)\big| < \epsilon.
    \]

    Обозначение:
    \[
        f_n \xrightrightarrows[n\rightarrow\infty]{} f \ (\text{на } X)
    \]
\end{definition}

\begin{example}
    $f_n: \R\rightarrow\R, \quad f_n(x) = x^n$

    Имеем при фиксирвоанном $x$:
    \[
        \underset{n\rightarrow\infty}{\lim}f_n(x) = \left\{\begin{array}{rl}
            0,        & -1 < x < 1     \\
            1,        & x = 1          \\
            + \infty, & x > 1          \\
            \nexists, & x \leqslant -1
        \end{array}\right.
    \]

    Таким образом область сходимости этой последовательности $E = (-1;1]$. На множестве $E$ определим предельую функцию
    \[
        f(x) = \underset{n\rightarrow\infty}{\lim}f_n(x) = \underset{n\rightarrow\infty}{\lim}x^n = \left\{\begin{array}{rl}
            0, & x \in (-1;1) \\
            1, & x = 1
        \end{array}\right.
    \]

    Покажем, что $f_n$ сходится к $f$ на $E$ неравномерно, то есть $\exists \epsilon > 0 \ \forall N \in \N: \ \exists n > N \ \exists x \in X:$
    \[
        \big|f_n(x) - f(x)\big| \geqslant \epsilon.
    \]

    Возьмем $\epsilon = \frac{1}{2}$. Пусть $N$ задано произвольно. Возьмем $n = N + 1$ и $x: \ x^n = \frac{3}{4}$, то есть $x = \sqrt[n]{\frac{3}{4}}$. Тогда:
    \[
        \big|f_n(x) - f(x)\big| = \left|\left(\sqrt[n]{\frac{3}{4}}\right)^n - 0\right| = \frac{3}{4} > \frac{1}{2}
    \]
\end{example}

\begin{example}
    $f_n(x) = \frac{x}{1 + n^2x^2}$

    $\forall x \in X$:
    \[
        \underset{n\rightarrow\infty}{\lim}f_n(x) = \underset{n\rightarrow\infty}{\lim}\frac{x}{1 + n^2x^2} = 0.
    \]

    Таким образом, $f(x) = 0 \ \forall x \in \R$. Покажем, что $f_n \xrightrightarrows[n\rightarrow\infty]{} f$ на $\R$.

    Имеем:
    \begin{multline*}
        \big|f_n(x) - 0\big| = \left|\frac{x}{1 + n^2x^2}\right| = \frac{1}{2n} \cdot \left|\frac{2nx}{1 + n^2x^2}\right| \leqslant \\
        \leqslant\left|\begin{array}{c}
            0 \leqslant (1 - nx)^2 = 1 + n^2x^2 - 2nx \implies \\
            \implies 2nx \leqslant 1 + n^2x^2
        \end{array}\right| \leqslant \frac{1}{2n}\cdot 1 = \frac{1}{2n}.
    \end{multline*}

    Пусть $\epsilon > 0$ задано. Возьмем $N: \ \forall n > N \ \frac{1}{2n} < \epsilon, \ N = \left[\frac{1}{2\epsilon}\right]$. Таким образом, $\forall n > N \ \forall x \in \R$
    \[
        \big|f_n(x)\big| \leqslant\frac{1}{2n} < \frac{1}{2N} = \epsilon \implies
    \]
    $\implies f_n(x) \xrightrightarrows[n\rightarrow\infty]{} f(x)$ на $\R^\infty$
\end{example}

\begin{example}
    $f_n(x) = \frac{n\cdot x}{1 + n^2 x^2}$

    $\forall x \in \R$
    \[
        f(x) = \underset{n\rightarrow\infty}{\lim}f_n(x) = \underset{n\rightarrow\infty}{\lim}\frac{n\cdot x}{1 + n^2 x^2} = 0 \implies
    \] $ \forall x \in \R \ f(x) = 0 $ (имеется поточечная сходимость).

    Покажем, что данное семейство не имеет равномерной сходимости к $f$. Рассмотрим $f_n(x) - f(x) = f_n(x) = \frac{n\cdot x}{1 + n^2 x^2}$:
    \[
        f_n'(x) = \frac{n\cdot (1 + n^2 x^2) - n \cdot x \cdot (2xn^2)}{(1 + n^2 x^2)^2} = \frac{n - n^3x^2}{(1 + n^2x^2)^2} = 0, \quad x = \pm\frac{1}{n}
    \]

    Далее, $f_n\left(\frac{1}{n}\right) = \frac{1}{2}$. Возьмем $\epsilon = \frac{1}{4}$.

    Тогда если $N$ задано, то выберем $n = N + 1$ и $x = \frac{1}{n}$.

    Тогда
    \[
        \big|f_n(x) - f(x)\big| \Big|_{x = \frac{1}{n}} = \frac{1}{2} > \frac{1}{4} \implies
    \]$\implies f_n(x) \xnrightrightarrows[n\rightarrow\infty]{}f(x)$ (пока что придется обозначать как $ \xnrightrightarrows[n\rightarrow\infty]{} $, так как нормально я не научился).
\end{example}

\begin{theorem}[Критерий Коши сходимости семейства функций]
    Пусть $Y$ -- полное метрическое пространство, $f_t:X \rightarrow Y, \ t \in T$ -- семейство $\{f_t\}$ равномерно сходится на $X$ по базе $\mathfrak{B} \iff \forall \epsilon > 0 \ \exists B \in \mathfrak{B}: \ \forall t_1,t_2 \in B$ и $\forall x \in X$
    \[
        \rho\big(f_{t_1}(x);f_{t_2}(x)\big) < \epsilon.
    \]
\end{theorem}

\begin{definition}[Равномерная сходимость семейства функций по базе]
    Будем говорить, что семейство функций $f_t: X \rightarrow Y$ \emph{равномерно сходится на $X$ по базе} $\mathfrak{B}$, если:
    \begin{enumerate}
        \item $\exists f: X \rightarrow Y:$
              \[
                  \underset{\mathfrak{B}}{\lim}f_t(x) = f(x), \quad \forall x \in X.
              \]
        \item $f_t$ сходится равномерно к $f$ на $X$ по базе $\mathfrak{B}$.
    \end{enumerate}
\end{definition}

\begin{theorem}[Формулировка критерия Коши для послед. $f_n(x)$]
    Последовательность $f_n(x)$ равномерно сходится на $X \iff \forall \epsilon > 0 \ \exists N \in \N: \ \forall n > N \ \forall p > 0 \ \forall x \in X$
    \[
        \big|f_n(x) - f_{n+p}(x)\big| < \epsilon.
    \]
\end{theorem}

\begin{proof}\leavevmode
    \begin{itemize}
        \item $ |\Rightarrow| $ Проведем доказательство для $Y = \R$.

              Пусть семейство $f_t$ сходится равномерно на $X$ по базе $\mathfrak{B}$, то есть $\exists f(x): X \rightarrow \R:$
              \[
                  f_t(x) \xrightrightarrows[\mathfrak{B}]{} f(x).
              \]

              Покажем, что выполнено условие Коши.

              Пусть $\epsilon > 0$ задано. Выберем $B \in \mathfrak{B}: \forall t \in B \ \forall x \in X$
              \[
                  \big|f_t(x) - f(x)\big| < \frac{\epsilon}{2}.
              \]

              Тогда $\forall t_1, t_2 \in B \ \forall x \in X$
              \begin{multline*}
                  \big|f_{t_1}(x) - f_{t_2}(x)\big| = \big|f_{t_1}(x) - f(x) + f(x) - f_{t_2}(x)\big| \leqslant \\
                  \leqslant \big|f_{t_1}(x) - f(x)\big| + \big|f_{t_2}(x) - f(x)\big| < \frac{\epsilon}{2} + \frac{\epsilon}{2} = \epsilon.
              \end{multline*}

        \item $ |\Leftarrow| $ Пусть $\forall \epsilon > 0 \ \exists B \in \mathfrak{B}: $
              \begin{equation}\label{eq:for_proof2}
                  \forall t_1,t_2 \in B \text{ и }\forall x \in X \quad \big|f_{t_1}(x) - f_{t_2}(x)\big| < \epsilon
              \end{equation}

              Зафиксируем $x \in X$. Тогда выражение \ref{eq:for_proof2} есть точная формулировка критерия Коши существования предела функции $f_t(x)$ по базе $\mathfrak{B} \implies \forall x \in X \ \exists \underset{\mathfrak{B}}{\lim}f_t(x) = f(x)$.

              Покажем, что $f_t(x) \xrightrightarrows[\mathfrak{B}]{} f(x)$ на $X$.

              В \ref{eq:for_proof2} перейдем к пределу по базе $\mathfrak{B}$ по переменной $t_1$. Получим, что
              \[
                  \big|f(x) - f_{t_2}(x)\big| < \epsilon.
              \]

              Таким образом получаем равномерную сходимость семейства $f_{t_2}(x)$ к $f$ на $X$ по базе $\mathfrak{B}$, то есть $\forall \epsilon > 0 \ \exists B \in \mathfrak{B} \ \forall t_2 \in B$ и $\forall x \in X$
              \[
                  \big|f_{t_2}(x) - f(x)\big| < \epsilon.
              \]
    \end{itemize}
\end{proof}

\begin{corollary}
    Пусть $X,Y$ -- метрические пространства, $E \subset X, \ x_0 \in E$ -- предельная точка для $E$. Семейство $f_t: X \rightarrow Y$:
    \begin{enumerate}
        \item $f_t$ сходится на $E$ по базе $\mathfrak{B}$.
        \item $f_t$ расходится в точке $x_0$ по базе $\mathfrak{B}$.
        \item $\forall t \ f_t$ непрерывно в точке $x_0$.
    \end{enumerate}

    Тогда на $E$ семейство $f_t$ сходится неравномерно.
\end{corollary}

\begin{proof}
    Применим критерий Коши, покажем, что $\exists \epsilon > 0: \ \forall B \in \mathfrak{B} \ \exists t_1, t_2 \in B$ и $\exists x \in E$:
    \[
        \rho_Y\big(f_{t_1}(x), f_{t_2}(x)\big) \geqslant \epsilon.
    \]

    Таким образом $f_t$ расходится в точке $x_0$, тогда $\exists \epsilon > 0: \forall B \in \mathfrak{B} \ \exists t_1,t_2 \in B$:
    \[
        \rho_Y\big(f_{t_1}(x_0), f_{t_2}(x_0)\big) \geqslant \epsilon.
    \]

    Так как $f_{t_1}$ и $f_{t_2}$ непрерывны, тогда $\exists U(x_0)\subset X: \ \forall x \in U(x_0)$
    \[
        \rho_Y\big(f_{t_1}(x), f_{t_2}(x)\big) \geqslant \epsilon.
    \]

    Возьмем $\forall x \in U(x_0) \cap E \implies$ тогда в $x$ будет выполняться неравенство
    \[
        \rho_Y\big(f_{t_1}(x), f_{t_2}(x)\big)\geqslant \epsilon \implies
    \]
    $\implies f_t$ на $E$ сходится неравномерно.
\end{proof}

\begin{corollary}[Из следствия выше]
    Если $f_t:(a;b] \rightarrow D, \ D$ -- область в $Y$:
    \begin{enumerate}
        \item $\forall t \ f_t$ непрерывно в точке $b$.
        \item $f_t$ сходится на $(a;b)$ по $\mathfrak{B}$.
        \item $f_t$ расходится в точке $b$.
    \end{enumerate}

    Тогда на $(a;b) \ f_t$ сходится неравномерно.
\end{corollary}

\section{Равномерная сходимость функциональных рядов}

\begin{definition}[Функциональный ряд]
    Пусть $f_n: X \rightarrow\R, \ X$ -- произвольное множество.

    \emph{Функциональным рядом} называется выражение вида
    \begin{equation}\label{eq:6.14}
        \sum_{n=1}^{\infty}f_n(x)
    \end{equation}

    Говорят, что ряд \ref{eq:6.14} сходится на $X$ поточечно, если на $X$ сходится поточечно последовательность его частичных сумм. Ряд \ref{eq:6.14} равномерно сходится на $X$, если на $X$ равномерно сходится последовательность его частичных сумм.
\end{definition}

\begin{theorem}[Критерий Коши равномерной сходимости функциональных рядов]
    Ряд \ref{eq:6.14} равномерно сходится на $X \iff \forall \epsilon > 0 \ \exists N: \ \forall n > N \ \forall p > 0 \ \forall x \in X$
    \[
        \big|f_{n+1}(x) + \ldots + f_{n+p}(x)\big| < \epsilon.
    \]
\end{theorem}

\begin{proof}
    Самостоятельно.
\end{proof}

\begin{corollary}
    Если:
    \begin{enumerate}
        \item Ряд \ref{eq:6.14} сходится на $(a;b)$.
        \item Расходится в точке $b$.
        \item $\forall n \ f_n(x)$ непрерывно в точке $b$.
    \end{enumerate}

    Тогда ряд \ref{eq:6.14} сходится на $(a;b)$ неравномерно.
\end{corollary}

\begin{proof}
    Следует из предыдущих следствий.
\end{proof}