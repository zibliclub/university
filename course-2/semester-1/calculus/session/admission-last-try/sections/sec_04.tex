\section{Теория рядов}

\setcounter{subsection}{20}

\subsection{Сходимость числового ряда}

\begin{definition}
    Ряд $(A)$ \emph{сходится}, если
    \[
        \exists \underset{n\rightarrow\infty}{\lim}A_n = A.
    \]

    Тогда сумма бесконечного ряда $(A)$ полагается равной
    \[
        A = \sum_{n=1}^{\infty}a_n.
    \]
\end{definition}

\subsection{Критерий Коши сходимости числовых рядов}

\begin{theorem}
    Ряд $(A)$ сходится $\iff \forall \epsilon > 0 \ \exists N \in \mathbb{N}: \ \forall n > N, \ \forall p > 0$
    \[
        |a_{n+1} + \ldots + a_{n+p}| < \epsilon.
    \]
\end{theorem}

\subsection{Необходимое условие сходимости числового ряда}

\begin{theorem}
    Ряд $(A)$ сходится $ \implies $
    \[
        \underset{n\rightarrow\infty}{\lim}a_n = 0.
    \]
\end{theorem}

\subsection{Основная теорема о сходимости положительных рядов}

\begin{theorem}\label{theorem:2.1}
    Положительный ряд $ (A) $ сходится $ \iff \exists M > 0: \ \forall n \ A_n < M $.
\end{theorem}

\subsection{Первый признак сравнения}

\begin{theorem}\label{theorem:6.2}
    Даны $ (A),(B), \ \forall n \ a_n, b_n > 0, \ \exists N \in \mathbb{N}: \ \forall n > N \ a_n \leqslant b_n \implies $
    \begin{enumerate}
        \item Из сходимости $(B) \implies$ сходимость $(A)$.
        \item Из расходимости $(A) \implies$ расходимость $(B)$.
    \end{enumerate}
\end{theorem}

\newpage

\subsection{Второй признак сравнения}

\begin{theorem}
    Даны $ (A),(B), \ \forall n \ a_n, b_n > 0, \ \underset{n\rightarrow\infty}{\lim}\frac{a_n}{b_n}=k \in [0;\infty] \implies $
    \begin{enumerate}
        \item При $k=\infty$ из сходимости $(A) \implies$ сходимость $(B)$.
        \item При $k=0$ из сходимости $(B) \implies$ сходимость $(A)$.
        \item Иначе $(A)$ и $(B)$ ведут себя одинаково.
    \end{enumerate}
\end{theorem}

\subsection{Третий признак сравнения}

\begin{theorem}
    Даны $ (A),(B), \ \forall n \ a_n, b_n > 0, \ \exists N \in \N \cup \{0\}: \forall n > N \ \frac{a_{n+1}}{a_n}\leqslant\frac{b_{n+1}}{b_n} \implies$
    \begin{enumerate}
        \item Из сходимости $(B) \implies$ сходимость $(A)$.
        \item Из расходимости $(A) \implies$ расходимость $(B)$.
    \end{enumerate}
\end{theorem}

\subsection{Интегральный признак сходимости Коши-Маклорена}

\begin{theorem}
    $ (A), \ \forall n \ a_n > 0, \ f(x): \ [1;+\infty) \rightarrow \R $ -- непрерывна и монотонна, $ \forall n \in \N \ f(x) = a_n \implies (A)$ и $\int_{1}^{\infty}f(x)dx$ ведут себя одинаково.
\end{theorem}

\subsection{Радикальный признак Коши}

\begin{theorem}
    Дан положительный ряд $(A), \ \underset{n\rightarrow\infty}{\overline{\lim}}\sqrt[n]{a_n} = q\implies $
    \begin{enumerate}
        \item $q < 1: (A)$ сходится.
        \item $q > 1: (A)$ расходится.
        \item $q = 1: (A)$ может как сходиться, так и расходиться.
    \end{enumerate}
\end{theorem}

\newpage

\subsection{Признак Даламбера}

\begin{theorem}
    Дан положительный ряд $(A), \ \underset{n\rightarrow\infty}{\lim}\frac{a_{n+1}}{a_n} = d \implies $
    \begin{enumerate}
        \item $d < 1:(A)$ сходится.
        \item $d > 1:(A)$ расходится.
        \item $d = 1:(A)$ может как сходиться, так и расходиться.
    \end{enumerate}
\end{theorem}

\subsection{Признак Раббе}

\begin{theorem}
    Дан положительный ряд $(A), \ \underset{n\rightarrow\infty}{\lim}n \cdot \left(\frac{a_n}{a_{n+1}} - 1\right) = r \implies $
    \begin{enumerate}
        \item $r>1: (A)$ сходится.
        \item $r<1: (A)$ расходится.
        \item $r=1: (A)$ может как сходиться, так и расходиться.
    \end{enumerate}
\end{theorem}

\subsection{Признак Кумера}

\begin{theorem}
    $ (A),(C), \ \forall n \ a_n,c_n > 0 $ и ряд $\sum_{n=1}^{\infty}c_n$ расходится. Если
    \[
        \underset{n\rightarrow\infty}{\lim}\left(c_n \cdot \frac{a_n}{a_{n+1}} - c_{n+1}\right) = k \implies
    \]
    \begin{enumerate}
        \item $k > 0: (A)$ сходится.
        \item $k < 0: (A)$ расходится.
        \item $k = 0: (A)$ может как сходиться, так и расходиться.
    \end{enumerate}
\end{theorem}

\subsection{Признак Бертрана}

\begin{theorem}
    $(A), \ \forall n \ a_n > 0$. Если
    \[
        \underset{n\rightarrow\infty}{\lim} \ln n \cdot \left[n \cdot \left(\frac{a_n}{a_{n+1}} - 1\right)\right] = B \implies
    \]
    \begin{enumerate}
        \item $B > 1: (A)$ сходится.
        \item $B < 1: (A)$ расходится.
        \item $B = 1: (A)$ может как сходиться, так и расходиться.
    \end{enumerate}
\end{theorem}

\newpage

\subsection{Признак Гаусса}

\begin{theorem}
    Ряд $(A), \ a_n > 0, \ \forall n \in \N, \ \lambda, \mu \in \R$. Если
    \[
        \frac{a_n}{a_{n+1}} = \left(\lambda + \frac{\mu}{n}\right) + O\left(\frac{1}{n^2}\right),
    \]
    то
    \begin{enumerate}
        \item $\lambda > 1: (A)$ сходится.
        \item $\lambda < 1: (A)$ расходится.
        \item $\lambda = 1$ и \begin{enumerate}
                  \item $\mu > 1 \implies (A)$ сходится.
                  \item $\mu \leqslant 1 \implies (A)$ расходится.
              \end{enumerate}
    \end{enumerate}
\end{theorem}

\subsection{Знакопеременные ряды}

\begin{note}
    Дан ряд $(A)$. Если $\exists N: \ \forall n > N \ a_n$ не меняет знак, то исследование сходимости такого ряда сводится к исследованию сходимости положительных рядов. Будем считать, что «$+$» и «$-$» бесконечно много. Такие ряды будем называть \emph{знакопеременными}.
\end{note}

\subsection{Абсолютно сходящийся ряд}

\begin{definition}
    Ряд $(A)$ называется \emph{абсолютно сходящимся}, если сходится ряд
    \[
        (A^*) \ \sum_{n=1}^{\infty}|a_n|.
    \]
\end{definition}

\subsection{Условно сходящийся ряд}

\begin{definition}
    Если ряд $(A)$ сходится, а ряд $(A^*)$ расходится, то ряд $(A)$ называется \emph{условно сходящимся}.
\end{definition}

\subsection{Знакочередующиеся ряды}

\begin{definition}
    Ряд $(A)$ называется \emph{знакочередующимся}, если \\ $\forall n \in \N  \ a_n \cdot a_{n+1} < 0$. Обозначим знакочередующийся ряд:
    \[
        (\overline{A}) \ \sum_{n=1}^{\infty}(-1)^{n-1}\cdot a_n, \quad a_n > 0 \ \forall n \in\N.
    \]
\end{definition}

\newpage

\subsection{Признак Лейбница}

\begin{theorem}
    $(\overline{A}), \ \forall n \ a_n > 0, $ 
    \begin{enumerate}
        \item $a_1 \geqslant \ldots \geqslant a_n \geqslant \ldots$.
        \item $\underset{n\rightarrow\infty}{\lim} a_n = 0$.
    \end{enumerate}

    $\implies (\overline{A})$ сходится, сумма $S: \ 0 < S \leqslant a_1$.
\end{theorem}

\subsection{Признак Абеля}

\begin{theorem}
    Если \begin{itemize}
        \item $\{a_n\}$ монотонна и ограничена,
        \item ряд $\sum_{n=1}^{\infty} b_n$ сходится,
    \end{itemize}
    то ряд $\sum_{n=1}^{\infty}a_n \cdot b_n$ сходится.
\end{theorem}

\subsection{Признак Дирихле}

\begin{theorem}
    Если \begin{itemize}
        \item $\{a_n\}$ монотонна и $\underset{n\rightarrow\infty}{\lim}a_n = 0$,
        \item частичные суммы ряда $(B)$ ограничены,
    \end{itemize}
    то $\sum_{n=1}^{\infty}a_n \cdot b_n$ сходится.
\end{theorem}

\subsection{Теорема Римана о перестановке членов условно сходящегося ряда}

\begin{theorem}
    $(A)$ условно сходится $\implies \forall B \in \R \ \exists$ перестановка ряда $(A)$ такая, что полученный ряд сходится и имеет сумму $B$. Более того, $\exists$ перестановка ряда $(A)$ такая, что частичные суммы полученного ряда не стремятся ни к конечному, ни к бесконечному пределу.
\end{theorem}

\newpage
