\section{Функции многих переменных}

\subsection{Градиент функции}

\begin{definition}
    $ D \subset \R^n, \ f:D\rightarrow \R $ диф-ма в $x_0 \in D, \ \vec{a} \in \R^n: \ df(x_0)\cdot h = \vec{a} \cdot h, \ h \in \mathbb{R}$ называется \emph{градиентом функции $f$ в} $x_0$ и обозначается
    \[
        gradf(x_0).
    \]
\end{definition}

\begin{note}
    Если в $\R^n$ зафиксировать ортонормированный базис, то
    \[
        gradf(x_0) = \left\{\frac{\partial f}{\partial x_1}(x_0),\frac{\partial f}{\partial x_2}(x_0),\ldots,\frac{\partial f}{\partial x_n}(x_0)\right\}.
    \]
\end{note}

\subsection{Производная по направлению вектора}

\begin{definition}
    $ \vec{v}\in T\R_{x_0}^n, $ $ |\vec{v}| = 1 \implies \frac{\partial f}{\partial \vec{v}}(x) $ называется \emph{производной по направлению вектора} $ \vec{v} $.
\end{definition}
