\section{Поточечная и равномерная сходимость семейства функций}

\setcounter{subsection}{74}

\subsection{Критерий Коши сходимости семейства функций}

\begin{theorem}[Критерий Коши сходимости семейства функций]
    Пусть $Y$ -- полное метрическое пространство, $f_t:X \rightarrow Y, \ t \in T$ -- семейство $\{f_t\}$ равномерно сходится на $X$ по базе $\mathfrak{B} \iff \forall \epsilon > 0 \ \exists B \in \mathfrak{B}: \ \forall t_1,t_2 \in B$ и $\forall x \in X$
    \[
        \rho\big(f_{t_1}(x);f_{t_2}(x)\big) < \epsilon.
    \]
\end{theorem}

\begin{proof}
   А где
\end{proof}

\subsection{Следствие из критерия Коши сходимости семейства функций}

\begin{corollary}
    Пусть $X,Y$ -- метрические пространства, $E \subset X, \ x_0 \in E$ -- предельная точка для $E$. Семейство $f_t: X \rightarrow Y$:
    \begin{enumerate}
        \item $f_t$ сходится на $E$ по базе $\mathfrak{B}$.
        \item $f_t$ расходится в точке $x_0$ по базе $\mathfrak{B}$.
        \item $\forall t \ f_t$ непрерывно в точке $x_0$.
    \end{enumerate}

    Тогда на $E$ семейство $f_t$ сходится неравномерно.
\end{corollary}

\begin{proof}
    Применим критерий Коши, покажем, что $\exists \epsilon > 0: \ \forall B \in \mathfrak{B} \ \exists t_1, t_2 \in B$ и $\exists x \in E$:
    \[
        \rho_Y\big(f_{t_1}(x), f_{t_2}(x)\big) \geqslant \epsilon.
    \]

    Таким образом $f_t$ расходится в точке $x_0$, тогда $\exists \epsilon > 0: \forall B \in \mathfrak{B} \ \exists t_1,t_2 \in B$:
    \[
        \rho_Y\big(f_{t_1}(x_0), f_{t_2}(x_0)\big) \geqslant \epsilon.
    \]

    Так как $f_{t_1}$ и $f_{t_2}$ непрерывны, тогда $\exists U(x_0)\subset X: \ \forall x \in U(x_0)$
    \[
        \rho_Y\big(f_{t_1}(x), f_{t_2}(x)\big) \geqslant \epsilon.
    \]

    Возьмем $\forall x \in U(x_0) \cap E \implies$ тогда в $x$ будет выполняться неравенство
    \[
        \rho_Y\big(f_{t_1}(x), f_{t_2}(x)\big)\geqslant \epsilon \implies
    \]
    $\implies f_t$ на $E$ сходится неравномерно.
\end{proof}