\section{Приложение теоремы о неявной функции}

\setcounter{subsection}{17}

\subsection{Теорема о неявной функции}

\begin{theorem}[О неявной функции, общий случай]
    Пусть $ F:U(x_0,y_0) \rightarrow \R^n $, где $ U(x_0,y_0) \subset \R^{m+n} $ -- окрестность точки $ (x_0,y_0) $ такая, что
    \begin{enumerate}
        \item $ F\in C^{(p)}(U,\R^n), \ p \geqslant 1 $.
        \item $ F(x_0,y_0) = 0 $.
        \item $ F_y'(x_0,y_0) $ -- обратная матрица.
    \end{enumerate}

    Тогда $ \exists (n+m) $-мерный промежуток $ I = I_x^m \times I_y^n \subset U(x_0;y_0) $, где
    \begin{align*}
        I_x^m & = \big\{x\in \R^m \ \big| \ |x - x_0| < \alpha \big\}, \\
        I_x^n & = \big\{y\in \R^n \ \big| \ |y - y_0| < \beta \big\},
    \end{align*} то есть $ f:I_x^m \rightarrow I_y^n $:
    \begin{itemize}
        \item $ \forall (x,y) \in I_x^m \times I_y^n \ F(x,y) = 0 \iff y = f(x) $.
        \item $ f'(x) = -\big[F_y'(x,y)\big]^{-1} \cdot F_x'(x,y) $.
    \end{itemize}
\end{theorem}

\begin{proof}
    Например, можно посмотреть в Зориче (надо найти).
\end{proof}

\setcounter{subsection}{21}

\subsection{Теорема о структуре касательного пространства}

\begin{theorem}[О структуре касательного пространства]\label{theorem:2}
    Пусть $S$ -- $k$-мерная поверхность в $\R^n, \ x_0 \in S$. Тогда касательное пространство $TS_{x_0}$ в точке $x_0$ состоит из направляющих векторов касательных к гладким кривым на поверхности $S$, проходящих через точку $x_0$.
\end{theorem}

\begin{proof}
    Пусть $x = x(t)$ -- гладкая кривая в $\R^n$, то есть
    \[
        \left\{\begin{array}{l}
            x^1 = x^1(t) \\
            \vdots       \\
            x^n = x^n(t)
        \end{array}\right., \ t \in \R.
    \]

    $ x_0 = x(t_0) $. Касательный вектор в точке $x_0$ к кривой имеет вид:
    \[
        \left(\begin{matrix}
                \frac{dx^1}{dt}(t_0) \\
                \vdots               \\
                \frac{dx^n}{dt}(t_0)
            \end{matrix}\right) = \left(\begin{matrix}
                x^{1'}(t_0) \\
                \vdots      \\
                x^{n'}(t_0)
            \end{matrix}\right).
    \]
    \begin{enumerate}
        \item Пусть $S$ -- $k$-мерная поверхность, задана системой уравнений $F(x) = 0$ и пусть $x = x(t)$ -- гладкая кривая на $S$. Покажем, что вектор $x'(t_0) = \left(\begin{matrix}
                          \frac{dx^1}{dt}(t_0) \\
                          \vdots               \\
                          \frac{dx^n}{dt}(t_0)
                      \end{matrix}\right): \ x'(t_0) \in TS_{x_0}, \ x_0 = x(t_0)$, то есть покажем, что $x'(t_0)$ удовлетворяет уравнению $F_x'(t_0)\cdot \xi=0$.

              Так как кривая $x = x(t)$ лежит на $S$, то $F\big(x(t)\big) = 0$ -- верно. Продифференцируем $F\big(x(t)\big) = 0$ по $t$ в точке $x_0$:
              \[
                  F_x'(x_0) \cdot x'(t_0) = 0,
              \]
              это и есть уравнение касательного пространства, то есть $x'(t_0)$ удовлетворяет уравнению касательного пространства $F_x'(x_0)\cdot \xi = 0$.

        \item Пусть $\xi = (\xi^1,\xi^2,\ldots,\xi^n) \in TS_{x_0}$, то есть $\xi$ удовлетворяет уравнению $F_x'(x_0)\cdot \xi = 0$

              Покажем, что $\exists$ гладкая кривая $l$ на поверхности $S$:
              \begin{itemize}
                  \item $x_0 \in l$
                  \item $\xi$ ялвяется направляющим вектором касательной к $l$ в точке $x_0$
              \end{itemize}

              Поверхность $S$ задана системой уравнений:
              \begin{equation}\label{eq:12}
                  \left\{\begin{array}{l}
                      F^1(x) = 0 \\
                      \vdots     \\
                      F^{n-k}(x) = 0
                  \end{array}\right.
              \end{equation}

              Пусть
              \begin{equation*}
                  \left|\begin{matrix}
                      \frac{\partial F^1}{\partial x^{k+1}}     & \cdots & \frac{\partial F^1}{\partial x^n}     \\
                      \vdots                                & \ddots & \vdots                            \\
                      \frac{\partial F^{n-k}}{\partial x^{k+1}} & \cdots & \frac{\partial F^{n-k}}{\partial x^n}
                  \end{matrix}\right| (x_0) \ne 0.
              \end{equation*}

              По теореме о неявной функции, система \ref{eq:12} эквивалентна системе:
              \begin{equation}\label{eq:13}
                  \left\{\begin{array}{l}
                      x^{k+1} = f^1(x^1,\ldots,x^k) \\
                      \vdots                        \\
                      x^n = f^{n-k}(x^1,\ldots,x^k)
                  \end{array}\right.
              \end{equation}

              Обозначим $u = (x^1,\ldots,x^k), \ v = (x^{k+1},\ldots,x^n)$, тогда \ref{eq:13} имеет вид:
              \[
                  v = f(u).
              \]

              Тогда по утверждению касательное пространство задается уравнениями:
              \begin{equation}\label{eq:14}
                  \left\{\begin{array}{l}
                      x^{k+1} = x_0^{k + 1} + \frac{\partial f^1}{\partial x^1}(x_0)\cdot(x^1-x_0^1) + \ldots + \frac{\partial f^1}{\partial x^k}(x_0)\cdot(x^k - x_0^k) \\
                      \vdots                                                                                                                                     \\
                      x^n = x_0^n + \frac{\partial f^{n-k}}{\partial x^1}(x_0)\cdot(x^1 - x_0^1) + \ldots + \frac{\partial f^{n-k}}{\partial x^k}(x_0)\cdot(x^k - x_0^k)
                  \end{array}\right.
              \end{equation}

              Пусть
              \[
                  \eta = \left(\begin{matrix}
                          \eta'      \\
                          \vdots     \\
                          \eta^k     \\
                          \eta^{k+1} \\
                          \vdots     \\
                          \eta^n
                      \end{matrix}\right) = \left(\begin{matrix}
                          x^1 - x_0^1         \\
                          \vdots              \\
                          x^k - x_0^k         \\
                          x^{k+1} - x^{k+1}_0 \\
                          \vdots              \\
                          x^n - x_0^n
                      \end{matrix}\right).
              \]

              Тогда система \ref{eq:14} примет вид:
              \begin{equation}\label{eq:15}
                  \left\{\begin{array}{l}
                      \eta^{k+1} = \frac{\partial f^1}{\partial x^1}(x_0) \cdot \eta^1 + \ldots + \frac{\partial f^1}{\partial x^k}(x_0)\cdot \eta^k \\
                      \vdots                                                                                                                 \\
                      \eta^{n} = \frac{\partial f^{n-k}}{\partial x^1}(x_0) \cdot \eta^1 + \ldots + \frac{\partial f^{n-k}}{\partial x^k}(x_0)\cdot \eta^k
                  \end{array}\right.
              \end{equation}

              Таким образом, если вектор $\xi \in TS_{x_0}$, то он полностью определяется своими первыми $k$ координатами, а остальные можно волучить с помощью системы \ref{eq:15}.

              Построим кривую в $\R^n$, то есть зададим ее уравнением $x = x(t)$:
              \begin{equation}\label{eq:16}
                  l: \ \left\{\begin{array}{l}
                      x^1 = x_0^1 + \xi^1t  \\
                      \vdots                \\
                      x^k = x_0^k + \xi^k t \\
                      \left.\begin{array}{l}
                                x^{k+1} = f^1(x_0^1 + \xi^1 t, \ldots, x_0^k + \xi^k t) \\
                                \vdots                                                  \\
                                x^{n} = f^{n-k}(x_0^1 + \xi^1 t, \ldots, x_0^k + \xi^k t)
                            \end{array}\right\} \ v = f(u)
                  \end{array}\right.
              \end{equation}

              Пусть точка $x_0$ соответствует параметру $t = 0$:
              \[
                  x(0) = \left\{\begin{array}{l}
                      x^1 = x_0^1                       \\
                      \vdots                            \\
                      x^k = x_0^k                       \\
                      x^{k+1} = f^1(x_0^1,\ldots,x_0^k) \\
                      \vdots                            \\
                      x^n = f^{n-k}(x_0^1, \ldots, x_0^k)
                  \end{array}\right.,
              \]
              то есть кривая проходит через точку $x_0$.

              Далее, функция $f$ удовлетворяет условию $v = f(u) \iff F(u,v) = 0$. Тогда $F(u,f(u)) = 0\implies l$, заданная системой \ref{eq:16}, $l \subset S$.
              \[
                  (\text{\ref{eq:16}})_t': \ x_t'(0) = \left(\begin{matrix}
                          \xi^1                                                                                                  \\
                          \vdots                                                                                                 \\
                          \xi^k                                                                                                  \\
                          \frac{\partial f^1}{\partial x^1}(x_0)\cdot \xi^1 + \ldots + \frac{\partial f^1}{\partial x^k}(x_0)\cdot \xi^k \\
                          \vdots                                                                                                 \\
                          \frac{\partial f^{n-k}}{\partial x^1}(x_0)\cdot \xi^1 + \ldots + \frac{\partial f^{n-k}}{\partial x^k}(x_0)\cdot \xi^k
                      \end{matrix}\right) = \left(\begin{matrix}
                          \xi^1 \\ \vdots \\ \xi^k \\ \xi^{k+1} \\ \vdots \\ \xi^n
                      \end{matrix}\right).
              \]

              Таким образом построили гладкий путь, лежащий на поверхности $S$, проходящий через точку $x_0 \in S$, вектор $x'(t_0)$ -- его касательный вектор $\in TS_{x_0}$.
    \end{enumerate}
\end{proof}

\setcounter{subsection}{24}

\subsection{Необходимое условие условного локального экстремума}

\begin{theorem}[Необходимое условие условного локального экстремума]
    Пусть система уровнений
    \begin{equation}\label{eq:20}
        \left\{\begin{array}{l}
            F^1(x^1,\ldots,x^n) = 0 \\
            \vdots                  \\
            F^{n-k}(x^1,\ldots,x^n) = 0
        \end{array}\right.
    \end{equation}
    задает $(n-k)$-мерную гладкую поверхность $S$ в $D \subset \R^n, \ D$ -- область. Функция $f:D\rightarrow\R$ -- гладкая. Если $x_0 \in S$ является точкой условного локального экстремума для функции $f$, то существует такой набор чисел $\lambda_1,\lambda_2,\ldots,\lambda_{n-k} \in \R:$
    \[
        grad f(x_0) = \sum_{i = 1}^{n-k}\lambda_i \cdot grad F^i(x_0).
    \]
\end{theorem}

\begin{lemma}\label{lemma:1}
    Если $x_0$ -- точка условного локального экстремума для функции $f$ и $x_0$ не является критической для функции $f$ (то есть $df(x_0)\ne0$), то касательное пространство $TS_{x_0}\subset TN_{x_0}$, где
    \[
        N_{x_0} = \big\{x \in D \ \big| \ f(x) = f(x_0)\big\},
    \] -- поверхность уровня, проходящая через $x_0$.
\end{lemma}

\begin{proof}
    Касательное пространство $TS_{x_0}$ задается системой уравнений:
    \begin{equation}\label{eq:21}
        \left\{\begin{array}{l}
            \frac{\partial F^1}{\partial x^1}(x_0)\cdot \xi^1 + \ldots + \frac{\partial F^1}{\partial x^n}(x_0) \cdot \xi^n = 0 \\
            \vdots                                                                                                      \\
            \frac{\partial F^{n-k}}{\partial x^1}(x_0)\cdot \xi^1 + \ldots + \frac{\partial F^{n-k}}{\partial x^n}(x_0) \cdot \xi^n = 0
        \end{array}\right.,
    \end{equation}
    но $\forall i = \overline{1,n-k}:$
    \[
        \left\{\frac{\partial F^i}{\partial x^1}\cdot (x_0),\ldots,\frac{\partial F^i}{\partial x^n}\cdot (x_0)\right\} = grad F^i(x_0).
    \]

    Перепишем \ref{eq:21} в виде:
    \begin{equation}\label{eq:22}
        \left\{\begin{array}{l}
            \big(grad F^1(x_0),\xi\big) = 0 \\
            \vdots                          \\
            \big(grad F^{n-k}(x_0,),\xi\big) = 0
        \end{array}\right.
    \end{equation}

    Касательное пространство $TN_{x_0}$ к $N_{x_0} = \big\{x \in D \ \big| \ f(x) = f(x_0)\big\}$ задается уравнением: $f'(x_0)\cdot\xi = 0$. Заметим, что:
    \[
        f'(x_0) = grad f(x_0) = \\ \left\{\frac{\partial f(x_0)}{\partial x^1},\ldots,\frac{\partial f(x_0)}{\partial x^n}\right\} \implies
    \]
    \begin{equation}\label{eq:23}
        \implies f'(x_0)\cdot \xi = 0 \iff \big(grad f(x_0),\xi\big) = 0
    \end{equation}

    Таким образом из леммы \ref{lemma:1} следует, что $\forall \xi$, удовлетворяющего системе уравнений \ref{eq:22}, так же удовлетворяет уравнению \ref{eq:23}, то есть из того, что $\forall i \in \overline{1,n-k}$
    \[
        \xi \perp grad F^i(x_0) \implies \xi \perp grad f(x_0) \implies
    \]
    $ \implies \exists \lambda_1,\ldots,\lambda_{n-k} \in \R $:
    \[
        grad f(x_0) = \sum_{i = 1}^{n-k} \lambda_i\cdot grad F^i(x_0).
    \]
\end{proof}

\setcounter{subsection}{27}

\subsection{Достаточное условие условного локального экстремума}

\begin{theorem}[Достаточное условие условного экстремума]
    Если при введенных выше условиях квадратичная форма
    \[
        Q(\xi) = \sum_{i,j=1}^{n}\frac{\partial^2 L}{\partial x^i \partial x^j}(x_0)\cdot\xi^i\xi^j,\ \big(\xi=(\xi^1,\ldots,\xi^n)\big)
    \]
    \begin{enumerate}
        \item Знакоопределена на $TS_{x_0}$:
              \begin{itemize}
                  \item если $Q$ знакоположительна, то точка $x_0$ -- точка условного локального $\min$
                  \item если $Q$ знакоотрицательна, то точка $x_0$ -- точка условного локального $\max$
              \end{itemize}
        \item Если $Q$ может принимать значения разных знаков, то в точке $x_0$ условного экстремума не наблюдается.
    \end{enumerate}
\end{theorem}

\begin{proof}
    Заметим, что $f\big|_S$ и $L\big|_S$ совпадают. В самом деле, если $x\in S$, то:
    \[
        L(x,\lambda) = f(\equalto{x}{(x^1,\ldots,x^n)}) + \sum_{i=1}^{k}\lambda_i\cdot \equalto{F^i(x)}{0} = f(x).
    \]

    Поэтому покажем, что условие знакопостоянства $Q$ является достаточным для экстремума функции $L\big|_s$.

    Имеем, что
    \[
        \left\{\begin{array}{l}
            \frac{\partial L}{\partial x^1}(x_0) = 0 \\
            \vdots                               \\
            \frac{\partial L}{\partial x^n}(x_0) = 0
        \end{array}\right..
    \]

    По формуле Тейлора:
    \begin{equation}\label{eq:25}
        L\big|_S(x) - L(x_0) = \sum_{i,j=1}^{n}\frac{\partial^2 L(x_0)}{\partial x^i \partial x^j}\cdot(x^i - x_0^i)(x^j - x_0^j) + o\big(\|x-x_0\|^2\big)
    \end{equation}

    Так как $S$ -- $m=(n-k)$-мерная поверхность, то существует гладкое отображение $x(t):\R^m\rightarrow\R^n: \ x = x(t) \subset S \ \forall t \in \R^m, \ x(0) = x_0$. Отображение $x(t)$ биективно отображает $\R^m$ на $U_S(x_0) = U(x_0)\cap S$.

    Если $x\in S$, то условие дифференцируемости $x(t)$:
    \[
        x-x_0 = x(\underset{\in\R^m}{t}) - x(0) = x'(0)\cdot t + o\big(\|t\|\big)
    \]
    \begin{center}
        или
    \end{center}
    \[
        \left\{\begin{array}{l}
            x^1 - x^1_0 = \frac{\partial x^1}{\partial t^1}(0)\cdot t^1 + \ldots + \frac{\partial x^1}{\partial t^m}(0)\cdot t^m + o\big(\|t\|\big) \\
            \vdots                                                                                                                          \\
            x^n - x^n_0 = \frac{\partial x^n}{\partial t^1}(0)\cdot t^1 + \ldots + \frac{\partial x^n}{\partial t^m}(0)\cdot t^m + o\big(\|t\|\big) \\
        \end{array}\right.
    \]
    \begin{center}
        или кратко
    \end{center}
    \begin{equation}\label{eq:26}
        \left\{\begin{array}{l}
            x^1 - x^1_0 = \sum_{i=1}^{m}\frac{\partial x^1}{\partial t^i}(0)\cdot t^i + o\big(\|t\|\big) \\
            \vdots                                                                                   \\
            x^n - x^n_0 = \sum_{i=1}^{m}\frac{\partial x^n}{\partial t^i}(0)\cdot t^i + o\big(\|t\|\big) \\
        \end{array}\right.
    \end{equation}

    Подставим \ref{eq:26} в \ref{eq:25}:
    \begin{multline*}
        L\big|_S(x) - L(x_0) = \frac{1}{2}\cdot\sum_{i,j=1}^{n}\frac{\partial^2L(x_0)}{\partial x^i \partial x^j} \cdot \underbrace{\left(\sum_{\alpha=1}^{m}\frac{\partial x^i}{\partial t^\alpha}(0)\cdot t^\alpha + o\big(\|t\|\big)\right)}_{x^i - x_0^i}\cdot \\
        \cdot \underbrace{\left(\sum_{\beta=1}^{m}\frac{\partial x^j}{\partial t^\beta}(0)\cdot t^\beta + o\big(\|t\|\big)\right)}_{x^j - x_0^j} + o\big(\|x-x_0\|^2\big) = \\
        = \frac{1}{2}\sum_{i,j=1}^{n}\frac{\partial^2 L(x_0)}{\partial x^i \partial x^j}\cdot\Bigg[\left(\sum_{\alpha=1}^{m}\frac{\partial x^i}{\partial t^\alpha}(0)\cdot t^\alpha\right)\cdot\left(\sum_{\beta=1}^{m}\frac{\partial x^i}{\partial t^\beta}(0)\cdot t^\beta\right) + \\
            + \left(\sum_{\alpha=1}^{m}\frac{\partial x^i}{\partial t^\alpha}(0)\cdot t^\alpha\right)\cdot o\big(\|t\|\big) + \left(\sum_{\beta=1}^{m}\frac{\partial x^i}{\partial t^\beta}(0)\cdot t^\beta\right)\cdot \\
            \cdot o\big(\|t\|\big) + o\big(\|t\|\big)\Bigg] + o\big(\|x-x_0\|^2\big) \overset{(\heartsuit)}{=} \frac{1}{2}\sum_{i,j=1}^{n}\frac{\partial^2 L(x_0)}{\partial x^i \partial x^j} \cdot \\
        \cdot \sum_{\alpha,\beta = 1}^{m}\frac{\partial x^i}{\partial t^\alpha} \cdot \frac{x^i}{\partial t^\beta} \cdot t^\alpha \cdot t^\beta + o\big(\|t\|^2\big) = \frac{\|t\|^2}{2} \cdot \sum_{i,j=1}^{n}\frac{\partial^2 L(x_0)}{\partial x^i\partial x^j} \cdot \\
        \cdot \underbrace{\sum_{\alpha,\beta=1}^{m}\frac{\partial x^i}{\partial t^\alpha}\cdot \frac{\partial x^j}{\partial t^\beta} \cdot \frac{t^\alpha}{\|t\|} \cdot \frac{t^\beta}{\|t\|}}_{\xi^i,\xi^j\text{ -- координаты вектора }\xi\in TS_{x_0}} + o\big(\|t\|^2\big) = \frac{\|t\|^2}{2}Q(\xi) + o\big(\|t\|^2\big).
    \end{multline*}

    Таким образом получаем, что
    \[
        L\big|_S(x) - L(x_0) = \frac{\|t\|^2}{2} \cdot Q(\xi) + o\big(\|t\|^2\big), \ \xi \in TS_{x_0}.
    \]

    Тогда, если $Q> 0$, то
    \[
        L\big|_S(x) - L(x_0)> 0 \implies x_0 \text{ -- } \min \text{ для } L\big|_S(x) \implies x_0 \text{ -- } \min \text{ для } f\big|_S.
    \]

    Если $Q < 0$, то
    \[
        L\big|_S(x) - L(x_0) < 0 \implies x_0 \text{ -- локальный } \max \text{ для } L\big|_S(x) \implies
    \]
    \[
        \implies x_0 \text{ -- локальный } \max \text{ для } f\big|_S \ (\forall x \in U_S(x_0))
    \]

    Если $Q$ -- знакопеременна, то не для всех $x \in U_S(x_0)$ разность $L\big|_S(x) - L(x_0)$ имеет постоянный знак $\implies$ в этом случае в точке $x_0$ нет экстремума.

    Докажем $(\heartsuit)$, то есть покажем, что
    \[
        o\big(\|t\|\big) \cdot \sum_{\alpha = 1}^{m} \frac{\partial x^i}{\partial t^\alpha}\cdot t^\alpha = o\big(\|t\|^2\big)
    \]
    \begin{center}
        и
    \end{center}
    \[
        o\big(\|x-x_0\|^2\big) = o\big(\|t\|^2\big), \ x \in S.
    \]

    В самом деле,
    \[
        \left| \sum_{\alpha=1}^{m}\frac{\partial x^i}{\partial t^\alpha}(0)\cdot t^\alpha\right| \leqslant \sum_{\alpha=1}^{m}\left|\frac{\partial x^i}{\partial t^\alpha}(0)\right| \cdot \big|t^\alpha \big| \leqslant \|t\| \cdot \equalto{\underbrace{\sum_{\alpha = 1}^{m} \left|\frac{\partial x^i}{\partial t^\alpha}(0)\right|}}{const>0} = \equalto{\underbrace{\overset{>0}{A}\cdot \|t\|}}{O\big(\|t\|\big)}.
    \]

    Таким образом,
    \begin{multline*}
        o\big(\|t\|\big) \cdot \bigg|\sum_{\alpha = 1}^{m}\frac{\partial x^i(0)}{\partial t^\alpha}\cdot t^\alpha \bigg| \leqslant o\big(\|t\|\big)\cdot O\big(\|t\|\big) = \\
        = \omega(t)\cdot \|t\| \cdot \gamma(t) \cdot \|t\| = \left|\begin{array}{c}
            \text{где }\omega(t)\rightarrow0 \text{ при } t\rightarrow 0, \\
            \gamma(t)\text{ -- ограниченная функция}
        \end{array}\right| = \\
        = \equalto{\alpha(t)}{\omega(t)\gamma(t)} \cdot \|t\|^2 = o\big(\|t\|^2\big), \begin{array}{l}
            \alpha(t)\rightarrow 0, \\
            t\rightarrow 0
        \end{array}.
    \end{multline*}

    Далее, если $x\in S$, то
    \begin{multline*}
        \|x-x_0\|^2 = \left\|\left(\begin{matrix}
                x^1 - x_0^1 \\
                \vdots      \\
                x^n - x^n_0
            \end{matrix}\right)\right\|^2 \overset{\text{\ref{eq:26}}}{=} \left\|\left(\begin{matrix}
                \sum_{\alpha=1}^{m}\frac{\partial x^1}{\partial t^\alpha}\cdot t^\alpha + o\big(\|t\|\big) \\
                \vdots                                                                                 \\
                \sum_{\alpha=1}^{m}\frac{\partial x^n}{\partial t^\alpha}\cdot t^\alpha + o\big(\|t\|\big)
            \end{matrix}\right)\right\|^2 = \\
        = \left(\sum_{\alpha = 1}^{m}\frac{\partial x^1}{\partial t^\alpha}\cdot t^\alpha + o\big(\|t\|\big)\right)^2 + \ldots + \left(\sum_{\alpha=1}^{m}\frac{\partial x^n}{\partial t^\alpha} + o\big(\|t\|\big)\right)^2 =\\
        = \left(\sum_{\alpha = 1}^{m}\frac{\partial x^1}{\partial t^\alpha}\cdot t^\alpha \right)^2 + \ldots + \left(\sum_{\alpha=1}^{m}\frac{\partial x^n}{\partial t^\alpha} \right)^2 + o\big(\|t\|^2\big) \leqslant \\
        \leqslant \left( \underset{\alpha}{\max}\frac{\partial x^1}{\partial t^\alpha}\right)^2 \cdot \left(\sum_{\alpha=1}^{m} t^\alpha\right)^2 + \ldots + \left(\underset{\alpha}{\max}\frac{\partial x^n}{\partial t^\alpha}\right)^2\cdot\left(\sum_{\alpha=1}^{m}t^\alpha\right)^2 \leqslant \\
        \leqslant \|t\|^2 \cdot \left(\left(\underset{\alpha}{\max}\frac{\partial x^1}{\partial t^\alpha}\right)^2 + \left(\max \frac{\partial x^n}{\partial t^\alpha}\right)^2\right) \leqslant \\
        \leqslant \|t\|^2 \cdot \equalto{\underbrace{\left(\underset{i}{\max}\left(\underset{\alpha}{\max}\frac{\partial x^i}{\partial t^\alpha}(0)\right)\right)^2 \cdot n}}{const>0} = B\|t\|^2 = o\big(\|t\|^2\big).
    \end{multline*}

    Поэтому
    \begin{multline*}
        o\big(\|x-x_0\|^2\big) = \\
        = \beta(x-x_0)\cdot \|x-x_0\|^2 = \beta(t)\cdot \|x-x_0\|^2 \leqslant \beta(t) \cdot B \cdot \|t\|^2 = \\
        = o\big(\|t\|^2\big)
    \end{multline*}
    \[
        \big(\beta(x-x_0)\rightarrow0\text{ при }x \rightarrow x_0 \iff t \rightarrow 0\big)
    \]
\end{proof}