\section{Основные теоремы дифференциального исчисления функций многих переменных}

\setcounter{subsection}{4}

\subsection{Теорема о среднем (аналог теоремы Лагранжа)}

\begin{theorem}[О среднем]
    Пусть $ D $ -- область в $ \R^n, \ x \in D, \ x + h \in D, \ [x, x+h]\subset D, \ f: D \rightarrow\R $ -- дифференцируемо на $ (x,x+h) $ и непрерывно на $ [x,x+h] $. Тогда $ \exists \xi \in (x,x+h): $
    \[
        f(x+h)-f(x) = f'(\xi)\cdot h = \frac{\partial f}{\partial x_1}(\xi)\cdot h^1 + \frac{\partial f}{\partial x_2}(\xi)\cdot h^2 + \ldots + \frac{\partial f}{\partial x_n}(\xi)\cdot h^n,
    \] где $ \{1,2,\ldots,n\} $ над $ h $ -- индексы.
\end{theorem}

\begin{proof}
    Рассмотрим отображение $ \gamma:[0;1]\rightarrow D $, определенное:
    \[
        \gamma(t) = x + t\cdot h, \quad \gamma(t) = \left\{\begin{array}{l}
            x_1(t) = x_1 + t\cdot h^1 \\
            x_2(t) = x_2 + t\cdot h^2 \\
            \vdots                    \\
            x_n(t) = x_n + t\cdot h^n
        \end{array}\right.,
    \]
    \[
        x = (x_1,x_2,\ldots,x_n),\quad h = \{h^1,h^2,\ldots,h^n\}, \ t\in [0;1],
    \]
    \[
        \begin{array}{l}
            \gamma(0) = x, \\
            \gamma(1) = x+h
        \end{array}, \quad [0;1]\overset{\gamma}{\longrightarrow}[x;x+h].
    \]

    Заметим, что $ gamma(t) $ дифференцируемо на $ (0;1) $, непрерывно на $ [0;1] $, причем $ \big(x_i(t)\big)' = h^i $.

    Рассмотрим функцию $ F(t) = f\big(\gamma(t)\big), \ F:[0;1]\rightarrow\R $. Имеем:
    \begin{enumerate}
        \item $ F $ -- дифференцируема на $ (0;1) $ (как композиция двух дифференцируемых).
        \item $ F $ -- непрерывна на $ [0;1] $ (как композиция двух непрерывных).
    \end{enumerate}

    Следовательно, по теореме Лагранжа:
    \[
        \begin{array}{cccl}
            F(1)    & - & F(0)    & = F'(\tau)\cdot(1-0), \ \tau \in (0;1)       \\
            \verteq &   & \verteq &                                              \\
            f(x+h)  & - & f(x)    & = \Big(f\big(\gamma(\tau)\big)\Big)' \cdot 1
        \end{array},
    \]
    \[
        \Big(f\big(\gamma(\tau)\big)\Big)' \cdot 1 = f'\big(\gamma(\tau)\big) \cdot \gamma'(\tau) = \equalto{\frac{\partial f}{\partial x_1} \cdot h'}{x_1'(t)} + \frac{\partial f}{\partial x_2}\cdot h_2 + \ldots + \frac{\partial f}{\partial x_n} \cdot h^n.
    \]

    Пусть $ \gamma(\tau) = \xi \in D $, тогда:
    \[
        f(x+h) - f(x) = \left(\begin{matrix}
                \frac{\partial f}{\partial x_1}(\xi) & \cdots & \frac{\partial f}{\partial x_n}(\xi)
            \end{matrix}\right) \cdot \left(\begin{matrix}
                h^1 \\ \vdots \\ h^n
            \end{matrix}\right) = f'(\xi)\cdot h.
    \]
\end{proof}

\subsection{Следствие теоремы о среднем}

\begin{corollary}
    Пусть $ D $ -- область в $ \R^n, \ f:D \rightarrow \R $ -- дифференцируема на $ D $ и $ \forall x \in D \ d(fx) = 0 $ (то есть $ \forall i \ \frac{\partial f}{\partial x_i} = 0 $). Тогда $ f(x) = const $.
\end{corollary}

\begin{proof}
    Пусть $ x_0 \in D $ и $ B(x_0,\rho)\subset D $ -- шар $ \exists $, так как $ D $ -- область. Тогда $ \forall x \in B(x_0,\rho) \quad [x_0;x]\subset B(x_0,\rho)\subset D $. Следовательно:
    \[
        f(x) - f(x_0) = \equalto{f'(\xi)}{\left\{\frac{\partial f}{\partial x_1}(\xi),\ldots,\frac{\partial f}{\partial x_n}(\xi)\right\}}\cdot (x-x_0) = 0.
    \]

    Таким образом, $ \forall x \in B(x_0,\rho) \ f(x) = f(x_0) $.

    Построим путь из точки $ x_0 $ к некоторой точке $ x \in D $:
    \[
        \gamma:[0;1]\rightarrow D, \quad \begin{array}{l}
            \gamma(0) = x_0 \\
            \gamma(1) = x
        \end{array}.
    \]

    По определению пути, $ \gamma $ -- непрерывно. Тогда $ \exists\delta: \ \forall 0 \leqslant t \leqslant\delta$
    \[
        \gamma(t) \in B(x_0,\rho) \implies f\big(\gamma(t)\big) = f(x_0), \ t \in [0;\delta],
    \] где $ t $ -- точка из $ B(x_0,\rho) $.

    Пусть $ \Delta = \sup\delta \implies f\big(\gamma(\Delta)\big) = f(x_0) $. Покажем, что $\Delta = 1$.

    Пусть $ \Delta < 1 \ (\Delta \ne 1) $. Построим шар $ B\big(\gamma(\Delta),\rho_\Delta\big) $. Тогда $ \exists \epsilon > 0: \ \Delta - \epsilon < t < \Delta + \epsilon $.

    Но тогда $ f\big(\gamma(\Delta + \epsilon)\big) = f(x_0) $ (так как точка $ \gamma(\Delta + \epsilon) \in B\big(\gamma(\Delta),\rho_\Delta\big) $) -- противоречие с тем, что $ \Delta = \sup\delta \implies \Delta = 1 $.

    $ \gamma(1) = x $ и $ f(x) = f(x_0) \implies $ так как $ x \in D $ -- произведение точек, то имеем, что $ \forall x \in D \ f(x) = f(x_0) \implies f(x)\text{ -- }const $.
\end{proof}

\subsection{Достаточное условие дифференцируемости функции}

\begin{theorem}[Достаточное условие дифференцируемости функции]
    Пусть $ D $ -- область в $ \R^n, \ f:D \rightarrow\R, \ f $ имеет непрерывные частные производные в каждой окрестности точки $ x\in D $. Тогда $ f $ -- дифференцируема в точке $ x $.
\end{theorem}

\begin{proof}
    Без ограничения общности, будем считать, что окрестность точки $ x_0\in D $ является шаром $ B(x_0,\rho)\subset D $.

    Пусть $ h:x_0+h \in B(x_0,\rho) $. Здесь
    \[
        \begin{array}{l}
            x_0 = (x^1,x^2,\ldots,x^n) \\
            x_0 + h = (x^1 + h^1,x^2 + h^2,\ldots,x^n+h^n)
        \end{array}.
    \]

    Заметим, что точки
    \[
        \begin{array}{l}
            x_1 = (x^1,x^2+h^2,\ldots,x^n+h^n)       \\
            x_2 = (x^1,x^2,x^3 + h^3,\ldots,x^n+h^n) \\
            \vdots                                   \\
            x_{n-1} = (x^1,x^2,x^3,\ldots,x^{n-1},x^n+h^n)
        \end{array} \in B(x_0,\rho).
    \]
    \begin{multline*}
        f(x_0 + h) - f(x_0) = \\
        = f(x_0 + h) - f(x_1) + f(x_1) - f(x_2) + f(x_2) - \ldots\\
        \ldots - f(x_{n-1}) + f(x_{n-1}) - f(x_0) = \\
        = f(x^1 + h^1, \ldots,  x^n + h^n) - f(x^1, x^2 + h^2,  \ldots,  x^n + h^n) + \\
        + f(x^1,  x^2 + h^2,  \ldots,  x^n + h^n) - f(x^1,  x^2,  \ldots,  x^n + h^n) + \\
        + f(x^1,  x^2,  \ldots,  x^n + h^n) - \ldots - f(x^1,  x^2,  \ldots,  x^{n-1},  x^n) + \\
        + f(x^1,  x^2,  \ldots,  x^{n-1},  x^n + h^n) - f(x^1,  x^2,  \ldots,  x^n) = \\
        = \left|\begin{array}{c}
            \text{Теорема Лагранжа для} \\
            \text{функции одной переменной}
        \end{array}\right| = \\
        = \frac{\partial f}{\partial x_1}(x^1 + \theta^1 h^1,  x^2 + h^2,  \ldots,  x^n + h^n) \cdot h^1 + \\
        + \frac{\partial f}{\partial x^2}(x^1,  x^2 + \theta^2 h^2,  \ldots,  x^n + h^n) \cdot h^2 + \ldots \\
        \ldots + \frac{\partial f}{\partial x^n}(x^1,  x^2,  \ldots,  x^n + \theta^n h^n) \cdot h^n.
    \end{multline*}

    Используя непрерывность частных производных, запишем:
    \begin{multline*}
        f(x_0 + h) - f(x_0) = \\
        = \frac{\partial f}{\partial x^1}(x^1, x^2, \ldots, x^n) \cdot h^1 + \alpha^1(h^1) + \ldots \\
        \ldots + \frac{\partial f}{\partial x^n}(x^1, x^2, \ldots, x^n) \cdot h^n + \alpha^n(h^n),
    \end{multline*}
    где $\alpha^1,\alpha^2,\ldots,\alpha^n$ стремятся к нулю при $\vec{h}\rightarrow0$.

    Это означает, что:
    \[
        \begin{array}{c}
            f(x_0 + h) - f(x_0) = L(x_0)\cdot h + \underset{h\rightarrow0}{o}(h) \\
            \left(\text{где} \ L(x_0) = \frac{\partial f}{\partial x^1}(x_0)\cdot h^1 + \ldots + \frac{\partial f}{\partial x^n}(x_0) \cdot h^n = df(x_0)\right)
        \end{array} \implies
    \]$\implies$ по определению $f(x)$ дифференцируема в точке $x_0$.
\end{proof}

\setcounter{subsection}{8}

\subsection{Теорема о смешанных производных}

\begin{theorem}[О смешанных производных]
    Пусть $ D $ -- область в $ \R^n, \ f: D \rightarrow\R, \ x \in D, \ f $ имеет в $ D $ непрерывные смешанные производные (второго порядка). Тогда эти производные не зависят от порядка дифференцирования.
\end{theorem}

\begin{proof}
    Пусть $ \frac{\partial^2f}{\partial x^i\partial x^j} $ и $ \frac{\partial^2f}{\partial x^j\partial x^i} $ -- непрерывны в точке $ x\in D $.

    Так как остальные переменные фиксированы, то можно считать, что $ f $ зависит только от двух переменных.

    Тогда $ D\subset\R^2,\ f:D \rightarrow\R $ и $ \frac{\partial^2f}{\partial x\partial y} $ и $ \frac{\partial^2f}{\partial y\partial x} $ -- непрерывны в точке $ x_0 = (x,y) \in D $.

    Покажем, что $ \frac{\partial^2f}{\partial x\partial y} = \frac{\partial^2f}{\partial y\partial x} $.

    Рассмотрим функции:
    \[
        \begin{array}{l}
            \phi(t) = f(x+t\cdot\Delta x,y + \Delta y) - f(x + t\cdot \Delta x, y) \\
            \psi(t) = f(x + \Delta x, y + t\cdot \Delta y) - f(x,y+t\cdot\Delta y)
        \end{array}, \quad t \in [0;1].
    \]

    Имеем:
    \begin{multline*}
        \phi(1) - \phi(0) = \\
        = f(x + \Delta x, y + \Delta y) - f(x + \Delta x, y) - f(x,y+ \Delta y) + f(x,y)
    \end{multline*}
    \begin{multline*}
        \psi(1) - \psi(0) = \\
        = f(x + \Delta x,y + \Delta y) - f(x,y+\Delta y) - f(x+\Delta x,y) + f(x,y)
    \end{multline*}

    Тогда:
    \begin{equation}\label{eq:1}
        \phi(1) - \phi(0) =\psi(1) - \psi(0)
    \end{equation}
    \begin{multline*}
        \phi(1) - \phi(0) = \phi'(\xi) \cdot (1-0) = \\
        = \frac{\partial f}{\partial x}(x + \xi \cdot \Delta x, y + \Delta y)\cdot \Delta x + \frac{\partial f}{\partial y}(x + \xi \cdot \Delta x,y + \Delta y) - \\
        - \frac{\partial f}{\partial x}(x + \xi \cdot \Delta x,y) \cdot \Delta x - \frac{\partial f}{\partial y}(x + \xi \cdot \Delta x, y) \cdot 0 = \\
        = \left(\frac{\partial f}{\partial x}(x + \xi \cdot \Delta x, y + \Delta y) - \frac{\partial f}{\partial x}(x + \xi \cdot \Delta x, y) \right) = \\
        = \left|\begin{array}{c}
            \text{по теореме Лагранжа для} \\
            \text{функции 1-ой переменной}
        \end{array}\right| = \\
        = \frac{\partial^2 f}{\partial x \partial y}(x + \xi \cdot \Delta x, y + \eta \cdot \Delta y)\Delta x\Delta y.
    \end{multline*}

    Положим $ (x + \xi \Delta x, y + \eta\cdot \Delta y) = P \in \Pi $.

    Аналогично:
    \begin{multline*}
        \psi(1) - \psi(0) = \psi'(\xi) \cdot (1-0) = \\
        = \frac{\partial f}{\partial x}(x + \Delta x, y + \xi \cdot \Delta y) \cdot 0 + \frac{\partial f}{\partial y}(x + \Delta x, y + \xi \cdot \Delta y) \cdot \Delta y - \\
        - \frac{\partial f}{\partial x}(x, y + \xi \cdot \Delta y) \cdot 0 - \frac{\partial f}{\partial y}(x, y + \xi \cdot \Delta y) \cdot \Delta y = \\
        = \left(\frac{\partial f}{\partial y}(x + \Delta x,y + \xi \cdot \Delta y) - \frac{\partial f}{\partial y}(x, y + \xi \cdot \Delta y)\right) \Delta y = \\
        = \left|\begin{array}{c}
            \text{по теореме Лагранжа для} \\
            \text{функции 1-ой переменной}
        \end{array}\right| = \\
        = \frac{\partial^2f}{\partial y\partial x}(x + \tau \cdot \Delta x, y + \xi \cdot \Delta y)\Delta y \Delta x
    \end{multline*}

    Положим, что $ (x + \tau\cdot \Delta x, y + \xi \cdot \Delta y) = Q $.

    Тогда из \ref{eq:1} следует, что:
    \[
        \begin{array}{ccc}
            \frac{\partial^2f}{\partial x\partial y}(P)\Delta x\Delta y                             & = & \frac{\partial^2f}{\partial y\partial x}(Q)\Delta x\Delta y                           \\
            \verteq                                                                           &   & \verteq                                                                         \\
            \frac{\partial^2f}{\partial x\partial y}(x + \xi \cdot \Delta x, y + \eta\cdot\Delta y) & = & \frac{\partial^2f}{\partial y\partial x}(x + \tau \cdot \Delta x, y+\xi\cdot\Delta y)
        \end{array}.
    \]

    Используя непрерывность частных производных при $ \Delta x \rightarrow0 $ и $ \Delta y \rightarrow0 \implies $
    \[
        x + \xi \cdot \Delta x \rightarrow x, \quad y + \eta \cdot \Delta y \rightarrow y.
    \]

    Таким образом,
    \[
        \frac{\partial^2f}{\partial x\partial y} = \frac{\partial^2f}{\partial y\partial x}.
    \]
\end{proof}

\subsection{Формула Тейлора}

\begin{theorem}[Формула Тейлора]
    Пусть $D$ -- область в $\R^n, \ f:D\rightarrow\R, \ f\in C^{(k)}(D,\R), \ x \in D, \ x + h \in D, \ [x;x+h] \subset D$. Тогда:
    \[
        f(x + h) = f(x) + \sum_{i=1}^{k-1}\frac{1}{i!}\left(\frac{\partial}{\partial x^1}\cdot h^1 + \ldots + \frac{\partial}{\partial x^n}\cdot h^n\right)^i \cdot f(x) + R^k,
    \]
    где $R^k$ -- остаточный член,
    \[
        R^k = \frac{1}{k!}\left(\frac{\partial}{\partial x^1}\cdot h^1 + \ldots + \frac{\partial}{\partial x^n}\cdot h^n\right)^k \cdot f(x + \xi \cdot h),
    \]
    \[
        x = (x^1,\ldots,x^n), \quad h = (h^1,\ldots,h^n).
    \]
\end{theorem}

\begin{proof}
    Рассмотрим функцию:
    \[
        \phi(t) = f(x + t\cdot h), \ t \in [0;1]
    \]

    Применим формулу Тейлора к $\phi(t)$:
    \begin{multline}\label{eq:2}
        \phi(1) = \phi(0) + \frac{1}{1!} \cdot \phi'(0) \cdot (1-0) + \frac{1}{2!} \cdot \phi''(0) \cdot (1-0)^2 + \\
        + \frac{1}{3!} \cdot \phi'''(0) \cdot (1-0)^3 + \ldots + \frac{1}{k!} \cdot \phi^{(k)} \cdot (1-0)^k.
    \end{multline}

    \[
        \phi(1) = f(x + h), \quad \phi(0) = f(x).
    \]

    \begin{multline*}
        \phi'(0) = f'(x + th) \cdot (x + t\cdot h)_k'\Big|_{t = 0} = \\
        = \left(\begin{matrix}
                \frac{\partial f(x + t\cdot h)}{\partial x^1} & \frac{\partial f(x + t\cdot h)}{\partial x^2} & \cdots & \frac{\partial f(x + t\cdot h)}{\partial x^n}
            \end{matrix}\right) \cdot \left(\begin{matrix}
                h^1 \\ h^2 \\ \vdots \\ h^n
            \end{matrix}\right)\Bigg|_{t=0} = \\
        = \left(\frac{\partial f(x + t\cdot h)}{\partial x^1} \cdot h^1 + \frac{\partial f(x+t\cdot h)}{\partial x^2} \cdot h^2 + \ldots + \frac{\partial f(x + t\cdot h)}{\partial x^n}\cdot h^n\right) \Bigg|_{t=0} = \\
        = \frac{\partial f}{\partial x^1}(x)\cdot h^1 + \frac{\partial f}{\partial x^2}(x)\cdot h^2 + \ldots + \frac{\partial f}{\partial x^n}(x) \cdot h^n = \\
        = \left(\frac{\partial}{\partial x^1} \cdot h^1 + \ldots + \frac{\partial}{\partial x^n}\cdot h^n\right)\cdot f(x)
    \end{multline*}
    \begin{multline*}
        \phi''(0) = \left(\sum_{i = 1}^{n} \frac{\partial f(x + t\cdot h)}{\partial x^i}\cdot h^i\right)_t' \Bigg|_{t = 0} = \\
        = \left(\sum_{i = 1}^{n} \sum_{j = 1}^{n} \frac{\partial^2 f(x + t\cdot h)}{\partial x^i \partial x^j}\cdot h^i h^j\right) \Bigg|_{t = 0} = \sum_{i = 1}^{n}\sum_{j = 1}^{n}\frac{\partial^2 f(x)}{\partial x^i \partial x^j}\cdot h^i h^j = \\
        = \left(\frac{\partial}{\partial x^1}\cdot h^1 + \ldots + \frac{\partial}{\partial x^n}\cdot h^n\right)^2 \cdot f(x)
    \end{multline*}

    И так далее. Подставим получившиеся выражения в \ref{eq:2} и получим искомое.
\end{proof}

\setcounter{subsection}{11}

\subsection{Необходимое условие локального экстремума}

\begin{theorem}[Необходимое условие локального экстремума]
    Пусть $D$ -- область в $ \R^n, \ f:D \rightarrow\R, \ x_0 \in D $ -- точка локального экстремума, тогда в точке $ x_0 \ \forall i = \overline{1,n}$
    \[
        \frac{\partial(x_0)}{\partial x^i} = 0.
    \]
\end{theorem}

\begin{proof}
    Фиксируем все переменные за исключением $ x^i $, тогда можно рассматривать функцию $ f(x^1,\ldots,x^i,\ldots,x^n) $ как функцию одной переменной, для которой $ x_0 $ -- точка локального экстремума, следовательно $ \frac{\partial f}{\partial x^i}(x_0) = 0 $,
    \[
        i \text{ -- произвольная }\implies \forall i \text{ выполняется}.
    \]
\end{proof}

\setcounter{subsection}{13}

\subsection{Достаточное условие локального экстремума}

\begin{theorem}[Достаточное условие локального экстремума]
    Пусть $D$ -- область в $\R^n, \ f: D \rightarrow \R$ дифференцируема в точке $x \in D, \ x$ -- критическая точка для $f, \ f \in C^n(D,\R), \ n = 2$. Тогда, если:
    \begin{enumerate}
        \item $Q(h)$ -- знакоположительна, то в точке $x$ -- локальный минимум.
        \item $Q(h)$ -- знакоотрицательна, то в точке $x$ -- локальный максимум.
        \item $Q(h)$ может принимать различные значения ($>0, < 0$), тогда в точке $x$ нет экстремума.
    \end{enumerate}
\end{theorem}

\begin{proof}
    По формуле Тейлора:
    \begin{multline*}
        f(x+h) - f(x) = \frac{1}{2}\cdot \sum_{i,j=1}^{n}\frac{\partial^2f(x)}{\partial x^i \partial x^j}\cdot h^ih^j + o\big(\|h\|^2\big) = \\
        = \frac{\|h\|^2}{2}\cdot\left(\sum_{i,j=1}^{n}\frac{\partial^2f(x)}{\partial x^i\partial x^j}\cdot \frac{h^i}{\|h\|}\frac{h^j}{\|h\|} + \alpha(h)\right) = \left|\begin{array}{c}
            \text{где }\alpha(h)\rightarrow 0\text{ при} \\
            h \rightarrow 0
        \end{array}\right| = \\
        = \frac{\|h\|^2}{2}\cdot\left(Q\left(\frac{h}{\|h\|}\right) + \alpha(h)\right).
    \end{multline*}

    Вектор $\frac{h}{\|h\|} < S^{(n-1)}$ -- единичная $(n-1)$-мерная сфера. Сфера $S^{(n-1)}$ -- компактное множество $\implies$ по теореме Больцано - Вейерштраса, $\exists e_1,e_2 \in S^{(n-1)}:$
    \[
        Q_1(e_1) = \max Q(h) = M, \quad Q_2(e_2) = \min Q(h) = m
    \]

    \begin{enumerate}
        \item Если $Q(h)$ -- знакоположительна $\implies m > 0$. Следовательно, $ \exists \partial > 0: \forall h \ \|h\| < \partial, \ |\alpha(h)| < m $
              \[
                  Q\left(\frac{h}{\|h\|}\right) + \alpha(h) > 0,
              \] следовательно, $ \forall h: \ \|h\| < \delta $
              \[
                  f(x+h) - f(x) > 0,
              \] по определению, $x$ -- точка локального минимума (здесь $\|h\| < \delta$ -- аналог понятия окрестности точки $x$).
        \item Если $Q(h)$ -- знакоотрицательна, то $M < 0$. Тогда $ \exists \delta > 0: \forall h \ \|h\| < \delta \ |\alpha(h)| < -M$
              \[
                  Q\left(\frac{h}{\|h\|}\right) + \alpha(h) < 0,
              \] следовательно, $ \forall h: \ \|h\| < \delta $
              \[
                  f(x+h) - f(x) < 0,
              \] тогда $ x $ -- точка локального максимума.
        \item Если $Q(h)$ -- знакопеременна, то $m < 0 < M, \ \forall t > 0$
              \[
                  Q(t\cdot e_2) < 0, \quad Q(t \cdot e_1) > 0,
              \] тогда в точке $x$ нет экстремума.
    \end{enumerate}
\end{proof}

\setcounter{subsection}{15}

\subsection{Теорема о неявной функции}

\begin{theorem}[О неявной функции]\label{theorem:1}
    Пусть $ F(x,y) $ отображает окрестность $ U(x_0;y_0) \subset \R^2 $ в $ \R, \ F:U(x_0,y_0)\rightarrow\R $.

    Пусть $ F $ имеет следующие свойства:
    \begin{enumerate}
        \item $ F(x_0,y_0) = 0 $.
        \item $ F(x,y) \in C^P(U,\R), \ p \geqslant 1 $.
        \item $ \frac{\partial F}{\partial y}(x_0,y_0)\ne 0 $.
    \end{enumerate}

    Тогда $ \exists $ открезки $ I_x,I_y: \ f:I_x \rightarrow I_y $:
    \begin{enumerate}
        \item $ I_x \times I_y \subset U(x_0,y_0) $.
        \item $ \forall x \in I_x \ y = f(x) \iff F(x,y) = 0 $.
        \item $ f \in C^P(I_x,I_y) $.
        \item $ \forall x \in I_x \ f'(x) = -\frac{F_x'(x,y)}{F_y'(x,y)} $.
    \end{enumerate}
\end{theorem}

\begin{proof}
    Будем считать, что окрестность $ U(x_0,y_0) $ -- круг с центром в точке $ (x_0,y_0) $. Для определенности будем считать, что \\ $ F_y'(x_0,y_0) > 0 $.

    В силу непрерывности $ F_y' \ \exists $ окрестность $ V(x_0,y_0) \subset U(x_0,y_0): \ \forall(x,y) \in V \ F_y'(x,y) > 0 $. Если посмотрим на функцию $ F(x,y) $ при фиксированной $ x $ как на функцию по переменной $ y $, то $ F(\overline{x},y) $ будет монотонной (в силу того, что $ F_y'(\overline{x},y) > 0 $). Тогда для $ \beta = \frac{1}{2}\tau $, где $ \tau $ -- радиус круга $ U(x_0,y_0) $.
    \[
        F(x_0,y_0 - \beta) < F(x_0,y_0) < F(x_0, y_0 + \beta).
    \]

    Так как $ F(x,y) $ непрерывна, то $ \exists\delta > 0: \ \forall x \in [x_0 - \delta, x_0 + \delta] $
    \[
        F(x,y_0 - \beta) < 0, \quad F(x,y_0 + \beta) > 0.
    \]

    При фиусированном $ x $ функция $ f(\overline{x},y) $ непрерывно монотонна, на концах отрезка $ [y_0 - \beta;y_0 + \beta] $ имеет разные знаки, тогда $ \exists!y_x \in [y_0-\beta;y_0+\beta]: \ F(\overline{x},y_x) = 0 $. В силу непрерывности $ F(x,y) $ по $ x, \ \exists\delta>0: \ \forall x \in [x_0 - \delta;x_0 + \delta] \ F(x,y_x = 0) $.

    Определим функцию $ f:[x_0-\delta;x_0+\delta]\rightarrow[y_0-\beta;y_0 + \beta] $ положив, что $ y=f(x) \iff F(x,y) = 0 $, то есть $ y_x = f(x) $.

    Положим $ f\in C^{(P)}(I_x,I_y) $.
    \begin{enumerate}
        \item Покажем, что $ f $ -- непрерывна.

              Для начала покажем, что $ f $ непрерывна в точке $ x_0 $.

              Пусть $ \epsilon > 0 $ задано. Покажем, что $ \exists \delta > 0: \ \forall x \in (x_0 - \delta;x_0 + \delta) \implies f(x) \in (y_0-\epsilon;y_0+\epsilon) $.

              Будем считать, что $ \epsilon < \beta \implies [y_0-\epsilon;y_0+\epsilon]\subset [y_0-\beta;y_0 + \beta] \implies $ найдется отрезок $ [x_0-\delta;x_0+\delta] $ и функция
              \[
                  \hat{f}(x): [x_0-\delta;x_0 + \delta]\rightarrow[y_0 -\epsilon;y_0+\epsilon],
              \]
              \[
                  \hat{f}(x) = y \iff F(x,y) = 0.
              \]

              Но на $ [x_0 - \delta;x_0 + \delta] \ \hat{f}(x) \equiv f(x) \implies f\big([x_0-\delta;x_0+\delta]\big)\subset[y_0-\epsilon;y_0+\epsilon] \implies f(x) $ непрерывна в точке $ x_0 $.

              Теперь, пусть $ x \in I_x = [x_0 -\delta;x_0+\delta] $.

              Для точки $ (x,y_x) $ выполнены все условия теоремы $ \implies \exists $ отрезок $ [x-\alpha;x+\alpha] = \widehat{I}_x $ и $ [y_x - \gamma;y_x + \gamma] = \widehat{I}_y $ и функция $ g:\widehat{I}_x \rightarrow \widehat{I}_y: \ g(\overline{x}) = y \iff F(\overline{x},y) = 0 \ \forall \overline{x} \in \widehat{I}_x $.

              Но на отрезке $ [x-\alpha;x + \alpha] $ функция $ g(x)\equiv f(x) $.

              По построению $ g(x) $ непрерывна в точке $ x $, следовательно и $ f(x) $ непрерывна в точке $ x $.

        \item Покажем, что $ f(x) $ дифференцируема на $ I_x $.

              Пусть $ x \in I_x, \ x + \Delta x \in I_x, \ y = f(x), \ y + \Delta y = f(x + \Delta x) $. Тогда
              \begin{multline*}
                  0 = \equalto{F(x+\Delta x, y+\Delta y)}{0} - \equalto{F(x,y)}{0} = \\
                  = \left|\begin{array}{c}
                      \text{Теорема} \\ \text{о среднем}
                  \end{array}\right| = F_x'(x+\theta\cdot \Delta x, y + \theta \cdot \Delta y)\Delta x + \\
                  + F_y'(x + \theta \cdot \Delta x, y + \theta \cdot \Delta y)\Delta y, \ 0 < \theta < 1 \implies
              \end{multline*}

              \[
                  \implies \frac{\Delta y}{\Delta x} = \frac{- F_x'(x + \theta \cdot \Delta x, y + \theta \cdot \Delta y)}{F_y'(x + \theta \cdot \Delta x, y + \theta \cdot \Delta y)}
              \]

              Поскольку $f$ -- непрерывная функция, то при $\Delta x \rightarrow 0: \ \Delta y \rightarrow 0 \ \big(f(x + \Delta x) - f(x) = \Delta y \rightarrow 0\big)$. Тогда:
              \begin{equation}\label{eq:3}
                  f'(x) = \underset{\Delta x \rightarrow 0}{\lim}\frac{\Delta y}{\Delta x} = \frac{-F_x'(x,y)}{F_y'(x,y)}
              \end{equation}

              Из теоремы о непрерывности композиции непрерывной функции $\implies f'(x)$ -- непрерывна в точке $x \implies f \in C^{(1)}(I_x,I_y)$.

              Если $F \in C^{(p)}(U,\R), \ p > 1$, то:
              \begin{multline}\label{eq:4}
                  \quad f''(x) = \left(-\frac{F_x'(x,y)}{F_y'(x,y)}\right)' = \frac{(-F_x')' \cdot F_y' + F_x' \cdot (F_y')'}{(F_y')^2} = \\
                  = \frac{-\big(F_{xx}'' + F_{xy}'' \cdot \equalto{f'(x)}{y'(x)}\big)\cdot F_y' + F_x'\cdot \big(F_{yx}'' + F_{yy}'' \cdot f'(x)\big)}{(F_y')^2},
              \end{multline}
              где $F_{xx}'', F_{xy}'', F_{yy}''$ вычисляются в точке $\big(x,f(x)\big)\implies f(x) \in C^{(2)}(I_x,I_y)$, если $F(x,y) \in C^{(2)}(U,\R)$.

              Заметим, что в левой части выражения \ref{eq:4} производная функции $f$ имеет порядок на $1$ больше, чем производная функции $f$ в правой части. Тогда по индукции можно показать, что $f \in C^{(p)}(I_x,I_y)$, если $F(x,y) \in C^{(p)}(U,\R)$.
    \end{enumerate}
\end{proof}