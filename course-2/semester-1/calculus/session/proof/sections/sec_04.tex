\section{Теория рядов}

\setcounter{subsection}{30}

\subsection{Критерий Коши сходимости числовых рядов}

\begin{definition}[Ряд]
    \emph{Рядом} называется выражение:
    \[
        a_1 + a_2 + \ldots + a_n + \ldots, \quad a_i \in \R.
    \]

    Числа $ a_i $ называются \emph{членами ряда}, $ a_n $ -- \emph{$ n $-ым членом ряда}.

    \begin{equation}\label{eq:6.1}
        \sum_{n=1}^{\infty}a_n
    \end{equation}

    Рассмотрим числа:
    \begin{align*}
         & A_1 = a_1,                      \\
         & A_2 = a_1 + a_2,                \\
         & \vdots                          \\
         & A_n = a_1 + a_2 + \ldots + a_n.
    \end{align*}

    Числа $ A_1,A_2,\ldots,A_n $ называются \emph{частичными суммами ряда} \ref{eq:6.1}.
\end{definition}

\begin{theorem}[Критерий Коши]
    Ряд \ref{eq:6.1} сходится тогда и только тогда, когда $\forall \epsilon > 0 \ \exists N \in \mathbb{N}: \ \forall n > N, \ \forall p > 0$
    \[
        |a_{n+1} + \ldots + a_{n+p}| < \epsilon.
    \]
\end{theorem}

\begin{proof}
    Ряд \ref{eq:6.1} сходится $\underset{\text{по определению}}{\iff} \exists \underset{n\rightarrow\infty}{\lim}A_n \iff A_n$ -- фундаментальная последовательность: $\forall \epsilon > 0 \ \exists N \in \mathbb{N}: \ \forall n > N$ и $\forall p > 0$
    \[
        |A_n - A_{n+p}| < \epsilon, \quad \left(\begin{array}{c}
                \text{критерий Коши сходимости} \\
                \text{последовательности}
            \end{array}\right).
    \]
    Имеем
    \begin{multline*}
        |A_n - A_{n+p}| = \\
        =\big|a_1 + a_2 + \ldots + a_n - (a_1 + a_2 + \ldots + a_n + \ldots + a_{n+p})\big| = \\
        = |a_{n+1} + \ldots + a_{n+p}| < \epsilon.
    \end{multline*}
\end{proof}

\subsection{Необходимое условие сходимости числового ряда}

\begin{theorem}[Необходимое условие сходимости ряда]
    Если ряд \ref{eq:6.1} сходится, тогда:
    \[
        \underset{n\rightarrow\infty}{\lim}a_n = 0.
    \]
\end{theorem}

\begin{proof}
    Пусть ряд \ref{eq:6.1} сходится, тогда $\exists \underset{n\rightarrow\infty}{\lim}A_n$:
    \begin{eqnarray*}
        \underset{n\rightarrow\infty}{\lim}a_n &=& \underset{n\rightarrow\infty}{\lim}(A_n - A_{n-1}) = \\
        = \underset{n\rightarrow\infty}{\lim}A_n - \underset{n\rightarrow\infty}{\lim}A_{n-1} &=& 0
    \end{eqnarray*}
\end{proof}

\subsection{Теорема об остатке ряда}

\begin{definition}[$ m $-ый остатный ряд]
    Пусть дан ряд \ref{eq:6.1}. Ряд вида
    \begin{equation}\label{eq:6.3}
        \sum_{n=m+1}^{\infty}a_n
    \end{equation}
    называется \emph{$ m $-ым остатным ряда} \ref{eq:6.1}.
\end{definition}

\begin{theorem}[Об остатке ряда]
    Следующие условия эквивалентны:
    \begin{enumerate}
        \item Ряд \ref{eq:6.1} сходится.
        \item Любой его остаток сходится.
        \item Некоторый его остаток \ref{eq:6.3} сходится.
    \end{enumerate}
\end{theorem}

\begin{proof}\leavevmode
    \begin{itemize}
        \item Докажем, что из 1. $ \implies $ 2.

              Пусть ряд \ref{eq:6.1} сходится и его сумма равна $ A $.

              Пусть $ A_k^* = \sum_{n=m+1}^{m+k}a_n $ -- $ k $-тая частичная сумма ряда \ref{eq:6.3}.

              Ряд \ref{eq:6.3} сходится, если $ \exists \underset{k \rightarrow\infty}{\lim} A_k^*$.
              \[
                  A_k^* = \underbrace{A_{m+k}}_{\begin{array}{c}
                          \text{частичная сумма} \\
                          \text{ряда \ref{eq:6.1}}
                      \end{array}} - A_m.
              \]
              \begin{eqnarray*}
                  \underset{k \rightarrow\infty}{\lim}A_k^* & = & \underset{k \rightarrow\infty}{\lim}(A_{m+k} - A_m) = \\
                  = \underset{k \rightarrow\infty}{\lim}A_{m+k} - \equalto{\underset{k \rightarrow\infty}{\lim}A_m}{const = A_m} &=& A - A_m.
              \end{eqnarray*}

        \item Доказательство того, что из 2. $ \implies $ 3. -- очевидно.

        \item Докажем, что из 3. $ \implies $ 1.

              Пусть ряд \ref{eq:6.3} -- сходится.

              Тогда при $ n > m $:
              \[
                  A_n = A_m + \overbrace{\equalto{A_{n-m}^*}{\sum_{k=m+1}^{m+(n-m)}a_k}}^{\begin{array}{c}
                          m\text{-тая частичная} \\
                          \text{сумма ряда \ref{eq:6.3}}
                      \end{array}}
              \]
              \[
                  A_n = \underbrace{a_1 + a_2 + \ldots + a_m}_{A_m} + \overbrace{\underbrace{a_{m+1} + \ldots + a_n}_{A_{n-m}^*}}^{k \text{ штук, }k=n-m}.
              \]

              Ряд \ref{eq:6.1} сходится $ \underset{\text{по опр.}}{\iff} \exists \underset{n \rightarrow\infty}{\lim}A_n $.

              Рассмотрим:
              \[
                  \underset{n \rightarrow\infty}{\lim}A_n = \underset{n \rightarrow\infty}{\lim}(\equalto{A_m}{const} + A_{n-m}^*) = \equalto{\underset{n \rightarrow\infty}{\lim}A_m}{A_m = const} + \underbrace{\underset{n \rightarrow\infty}{\lim}A_{n-m}^*}_{\begin{array}{c}
                      \exists \text{, так как ряд} \\
                      \text{\ref{eq:6.3} сходится}
                  \end{array}} \implies
              \]
              $ \implies \exists \underset{n \rightarrow\infty}{\lim}A_n \implies $ \ref{eq:6.1} сходится.
    \end{itemize}
\end{proof}

\subsection{Теорема о сумме рядов и умножении ряда на число}

\begin{theorem}
    Если ряды $ (A),(B) $ сходятся, то:
    \begin{enumerate}
        \item $ \forall a \in \R $ ряд $ \sum_{n=1}^{\infty}\alpha a_n $ -- сходится и его сумма равна $ \alpha \cdot A $, где $ A = \sum_{n=1}^{\infty}a_n $.

        \item Ряд $ (A+B) $ сходится и его сумма равна $ A^* + B^* $, где $ A^* = \sum_{n=1}^{\infty}a_n,$ $ B^* = \sum_{n=1}^{\infty}b_n $.
    \end{enumerate}
\end{theorem}

\begin{proof}
    \begin{enumerate}
        \item Пусть ряд $ (A) $ сходится.

              Рассмотрим ряд $ \sum_{n=1}^{\infty}\alpha \cdot a_n $:
              \[
                  A_n' = \sum_{k=1}^{n}\alpha \cdot a_k,
              \]
              \[
                  \underset{n \rightarrow\infty}{\lim}A_n' = \underset{n \rightarrow\infty}{\lim}\sum_{k=1}^{n} \alpha \cdot a_k = \alpha \cdot \underset{n \rightarrow\infty}{\lim}\sum_{k=1}^{n}a_k = \alpha \cdot A
              \]

        \item Самостоятельно.
    \end{enumerate}
\end{proof}

\setcounter{subsection}{35}

\subsection{Основная теорема о сходимости положительных рядов}

\begin{theorem}\label{theorem:2.1}
    Положительный ряд $ (A) $ сходится $ \iff $ его частичные суммы ограничены, то есть $ \exists M > 0: \ \forall n \ A_n < M $.
\end{theorem}

\begin{proof}
    Заметим, что последовательность частичных сумм $ A_n $ возрастает, то есть $ \forall n \ A_{n+1} > A_n $.

    По теореме Вейерштрасса, возрастающая последовательность $ A_n $ имеет предел $ \iff $ она ограничена, то есть $ \exists M>0: \ \forall n \ A_n < M $.
\end{proof}

\subsection{Первый признак сравнения}

\begin{theorem}[1-ый признак сравнения]\label{theorem:6.2}
    Пусть даны ряды $ (A),(B) $, причем $a_n > 0, \ b_n > 0 \ \forall n$.

    Если $\exists N \in \mathbb{N}: \ \forall n > N \ a_n \leqslant b_n$, то:
    \begin{enumerate}
        \item Из сходимости ряда $(B) \implies$ сходимость ряда $(A)$.
        \item Из расходимости ряда $(A) \implies$ расходимость ряда $(B)$.
    \end{enumerate}
\end{theorem}

\begin{proof}\leavevmode
    \begin{enumerate}
        \item Пусть ряд $(B)$ -- сходится $\implies$ по теореме \ref{theorem:2.1} его частичные суммы ограничены $\implies$ по неравенству $a_n\leqslant b_n$ частичные суммы ряда $(A)$ также ограничены $\implies$ по \ref{theorem:2.1} ряд $(A)$ сходится.
        \item Аналогично.
    \end{enumerate}
\end{proof}

\subsection{Второй признак сравнения}

\begin{theorem}[2-ой признак сравнения]
    Пусть даны ряды $ (A),(B) $, причем $a_n > 0, \ b_n > 0 \ \forall n$.

    Если $\underset{n\rightarrow\infty}{\lim}\frac{a_n}{b_n}=k, \ k\in [0;\infty]$, то:
    \begin{enumerate}
        \item При $k=\infty$ из сходимости $(A) \implies$ сходимость ряда $(B)$.
        \item При $k=0$ из сходимости ряда $(B) \implies$ сходимость ряда $(A)$.
        \item При $0<\equalto{k}{const \ne 0}<\infty$ ряды $(A)$ и $(B)$ ведут себя одинаково.
    \end{enumerate}
\end{theorem}

\begin{proof}
    Переписать доказательство для несобственных интегралов, заменив слово "интеграл" на слово "ряд".
\end{proof}

\subsection{Третий признак сравнения}

\begin{theorem}[3-й признак сравнения]
    Пусть даны ряды $ (A),(B) $, причем $a_n > 0, \ b_n > 0 \ \forall n$.

    Если $\exists N \in \N \cup \{0\}: \ \forall n > N \ \frac{a_{n+1}}{a_n}\leqslant\frac{b_{n+1}}{b_n} $, то:
    \begin{enumerate}
        \item Из сходимости ряда $(B) \implies$ сходимость ряда $(A)$.
        \item Из расходимости ряда $(A) \implies$ расходимость ряда $(B)$.
    \end{enumerate}
\end{theorem}

\begin{proof}
    Можно считать, что $N = 0$. Тогда $\forall n > N$ имеем:
    \[
        \frac{a_2}{a_1}\leqslant\frac{b_2}{b_1}; \quad \frac{a_3}{a_2}\leqslant\frac{b_3}{b_2}; \quad \frac{a_4}{a_3}\leqslant\frac{b_4}{b_3}; \quad \ldots; \quad \frac{a_{n+1}}{a_n} \leqslant \frac{b_{n+1}}{b_n}.
    \]

    Перемножим левые и правые части:
    \[
        \frac{a_2 \cdot a_3 \cdot a_4 \cdot \ldots \cdot a_{n+1}}{a_1 \cdot a_2 \cdot a_3 \cdot \ldots \cdot a_n} \leqslant \frac{b_2 \cdot b_3 \cdot b_4 \cdot \ldots \cdot b_{n+1}}{b_1 \cdot b_2 \cdot b_3 \cdot \ldots \cdot b_n},
    \]
    \[
        \frac{a_{n+1}}{a_1} \leqslant \frac{b_{n+1}}{b_n} \implies a_{n+1} \leqslant \frac{a_1}{b_1}\cdot b_{n+1}\text{ (по теореме \ref{theorem:6.2})}.
    \]
    \begin{enumerate}
        \item Если ряд $(B)$ сходится $\implies$ сходится ряд $\sum_{n=1}^{\infty}\frac{a_1}{b_1}\cdot b_{n+1} \implies$ сходится ряд $\sum_{n=1}^{\infty}a_{n+1} \implies $ сходится $ (A) $.
        \item Аналогично.
    \end{enumerate}
\end{proof}

\subsection{Интегральный признак сходимости Коши-Маклорена}

\begin{theorem}[Интегральный признак Коши-Маклорена]
    Пусть дан положительный ряд $ (A) $.

    Пусть функция $f(x)$ удовлетворяет следующим условиям:
    \begin{enumerate}
        \item $f(x): \ [1;+\infty) \rightarrow\R$.
        \item $f(x)$ -- непрерывна.
        \item $f(x)$ -- монотонна.
        \item $f(x) = a_n, \ \forall n \in \N$.
    \end{enumerate}

    Тогда ряд $(A)$ и интеграл $\int_{1}^{\infty}f(x)dx$ ведут себя одинаково.
\end{theorem}

\begin{proof}
    Ограничимся случаем, когда $f(x)$ монотонно убывает.

    Рассмотрим функцию $\phi(x) = a_n$ при $n \leqslant x < n+1$ и $\psi(x) = a_{n+1}$ при $n\leqslant x <n + 1$. Тогда $\forall x \in [1;+\infty)$:
    \[
        \psi(x) \leqslant f(x) \leqslant\phi(x).
    \]

    Отсюда
    \begin{multline*}
        \int_{1}^{N}\psi(x)dx \leqslant \int_{1}^{N}f(x)dx \leqslant\int_{1}^{N}\phi(x)dx \implies \\
        \implies \underbrace{\sum_{n=1}^{N}a_{n+1}}_{\begin{array}{c}
            \text{частичная сумма} \\
            \text{ряда }(A)
        \end{array}} \leqslant \int_{1}^{N}f(x)dx \leqslant \underbrace{\sum_{n=1}^{N}a_n}_{\begin{array}{c}
                \text{частичная сумма} \\
                \text{ряда }(A)
            \end{array}}
    \end{multline*}
    \begin{itemize}
        \item Если интеграл сходится, то частичная сумма $\sum_{n=1}^{N}a_{n+1}$ ограничена $\implies$ ряд $(A)$ сходится.

        \item Если интеграл расходится, то частичная сумма $\sum_{n=1}^{N}a_n$ непрерывна $\implies$ ряд $(A)$ -- расходится.

        \item Если ряд $(A)$ сходится, то $\sum_{n=1}^{N}a_n$ -- ограничена $\implies$ $\int_{1}^{N}f(x)dx$ -- ограничен $\implies \int_{1}^{\infty}f(x)dx$ -- сходится.

        \item Если ряд $(A)$ расходится $\implies$ частичная сумма $\sum_{n=1}^{N}a_{n+1}$ неограничена $\implies$ интеграл расходится.
    \end{itemize}
\end{proof}

\subsection{Радикальный признак Коши}

\begin{theorem}[Радикальный признак Коши]
    Пусть ряд $(A)$ положительный и $\underset{n\rightarrow\infty}{\overline{\lim}}\sqrt[n]{a_n} = q$. Тогда:
    \begin{enumerate}
        \item При $q < 1$ ряд $(A)$ сходится.
        \item При $q > 1$ ряд $(A)$ расходится.
        \item При $q = 1$ может как сходиться, так и расходиться.
    \end{enumerate}
\end{theorem}

\begin{proof}\leavevmode
    \begin{enumerate}
        \item Пусть $q < 1$. Возьмем число $r: \ q < r < 1$. Тогда $\exists N: \ \forall n > N$
              \[
                  \sqrt[n]{a_n} < r \implies a_n < r^n.
              \]

              $0 < r < 1 \implies \sum_{n=1}^{\infty}r^n$ -- сходится $\implies$ по 1-му признаку сравнения сходится ряд $(A)$.

        \item Пусть $q > 1$, тогда существует подпоследовательность $\sqrt[n_i]{a_{n_i}} \rightarrow q$ при $i\rightarrow\infty \implies a_{n_i}\rightarrow q^{n_i} > 1 \implies a_n \nrightarrow  0 \implies $ необходимое условие сходимости не выполняется $ \implies $ ряд $(A)$ расходится.

        \item Рассмотрим ряды $\underset{\text{расходится}}{\sum_{n=1}^{\infty}\frac{1}{n}}$ и $\underset{\text{сходится}}{\sum_{n=1}^{\infty}\frac{1}{n^2}}$:
              \[
                  \underset{n\rightarrow\infty}{\lim}\sqrt[n]{\frac{1}{n}} = \underset{n\rightarrow\infty}{\lim}\sqrt[n]{\frac{1}{n^2}} = 1.
              \]
    \end{enumerate}
\end{proof}

\subsection{Признак Даламбера}

\begin{theorem}[Признак Даламбера]
    Пусть ряд $(A)$ положительный и  \\ $ \underset{n\rightarrow\infty}{\lim}\frac{a_{n+1}}{a_n} = d $. Тогда:
    \begin{enumerate}
        \item При $d < 1$ ряд $(A)$ сходится.
        \item При $d > 1$ ряд $(A)$ расходится.
        \item При $d = 1$ может как сходиться, так и расходиться.
    \end{enumerate}
\end{theorem}

\begin{proof}\leavevmode
    \begin{enumerate}
        \item Пусть $d < 1$. Возьмем $d < r < 1 \implies \exists N: \ \forall n > N \ \frac{a_{n+1}}{a_n}<r $,
              \[
                  b_1 = \frac{a_2}{a_1}; \quad b_2 = \frac{a_3}{a_2}; \quad b_3 = \frac{a_4}{a_3}; \quad \ldots; \quad b_n = \frac{a_{n+1}}{a_n}; \quad \ldots.
              \]
              Можно считать, что $N=0$, тогда $\forall n > N$:
              \[
                  \begin{array}{l}
                      a_2 < r \cdot a_1                 \\
                      a_3 < r \cdot a_2 < r^2 \cdot a_1 \\
                      a_4 < r \cdot a_3 < r^3 \cdot a_1 \\
                      \vdots                            \\
                      a_{n+1} < r^n \cdot a_1
                  \end{array}.
              \]

              Так как $0 < r < 1$, то $\sum_{n=1}^{\infty} r^n \cdot a_1$ сходится $\implies$ сходится ряд $(A)$ по 1 признаку сравнения.
        \item Самостоятельно.
        \item $\sum_{n=1}^{\infty}\frac{1}{n}, \quad \sum_{n=1}^{\infty}\frac{1}{n^2}$.
              \[
                  \underset{n\rightarrow\infty}{\lim}\frac{a_{n+1}}{a_n} = \underset{n\rightarrow\infty}{\lim}\frac{\frac{1}{n+1}}{\frac{1}{n}} = \underset{n\rightarrow\infty}{\lim}\frac{n}{n+1} = 1,
              \]
              \[
                  \underset{n\rightarrow\infty}{\lim}\frac{\frac{1}{(n+1)^2}}{\frac{1}{n^2}} = \underset{n\rightarrow\infty}{\lim}\frac{n^2}{(n+1)^2} = 1.
              \]
    \end{enumerate}
\end{proof}

\subsection{Признак Раббе}

\begin{theorem}[Признак Раббе]
    Пусть ряд $(A)$ -- положительный. Если $ \underset{n\rightarrow\infty}{\lim}n \cdot \left(\frac{a_n}{a_{n+1}} - 1\right) = r, $ то:
    \begin{enumerate}
        \item При $r>1$ ряд $(A)$ сходится.
        \item При $r<1$ ряд $(A)$ расходится.
        \item При $r=1$ ряд $(A)$ может как сходиться, так и расходиться.
    \end{enumerate}
\end{theorem}

\begin{proof}\leavevmode
    \begin{enumerate}
        \item Пусть $r>1$. Возьмем $p$ и $q$: $ 1 < p < q < r $. Так как $ \underset{n\rightarrow\infty}{\lim}n\cdot\left(\frac{a_n}{a_{n+1}} - 1\right) = r, $ то $\exists N_1: \ \forall n > N_1 \ n\cdot\left(\frac{a_n}{a_{n+1}} - 1\right) > q$, то есть:
              \begin{equation}\label{eq:6.4}
                  \frac{a_n}{a_{n+1}} > 1 + \frac{q}{n}.
              \end{equation}

              Далее, рассмотрим:
              \[
                  \underset{n\rightarrow\infty}{\lim}\frac{(1 + \frac{1}{n})^p - 1}{\frac{1}{n}} \overset{\begin{array}{c}
                          \text{формула} \\
                          \text{Тейлора}
                      \end{array}}{=} \underset{n\rightarrow\infty}{\lim}\frac{1 + \frac{p}{n} + o(\frac{1}{n}) - 1}{\frac{1}{n}} = p < q \implies
              \]
              $\implies \exists N_2: \ \forall n > N_2$:
              \begin{equation}\label{eq:6.5}
                  \frac{(1 + \frac{1}{n})^p - 1}{\frac{1}{n}} < q \implies \left(1 + \frac{1}{n}\right)^p < 1 + \frac{q}{n}.
              \end{equation}

              Сравниваем неравенства \ref{eq:6.4} и \ref{eq:6.5}, получим, что при \\
              $n > \max(N_1,N_2)$:
              \begin{multline*}
                  \left(1 + \frac{1}{n}\right)^p < 1 + \frac{q}{n} < \frac{a_n}{a_{n+1}} \implies \\
                  \implies \frac{a_n}{a_{n+1}} > \left(1 + \frac{1}{n}^p\right) = \frac{(n+1)^p}{n^p} = \frac{\frac{1}{n^p}}{\frac{1}{(n+1)^p}}.
              \end{multline*}

              Ряд $\sum_{n=1}^{\infty}\frac{1}{n^p}$ сходится при $p > 1$:
              \[
                  \frac{a_n}{a_{n+1}} > \frac{\frac{1}{n^p}}{\frac{1}{(n+1)^p}} \implies a_n \cdot \frac{1}{(n+1)^p} > \frac{1}{n^p} \cdot a_{n+1} \implies \frac{a_{n+1}}{a_n} < \frac{\frac{1}{(n+1)^p}}{\frac{1}{n^p}}.
              \]

              По 3-му признаку сравнения, ряд $(A)$ сходится при $p > 1 \implies$ при $r > 1$.

        \item Пусть $r < 1$. Тогда $\exists N: \ \forall n > N$:
              \begin{multline*}
                  n \cdot \left(\frac{a_n}{a_{n+1}} - 1\right) < 1 \implies \\
                  \implies \frac{a_n}{a_{n+1}} < 1 + \frac{1}{n} = \frac{n+1}{n} = \frac{\frac{1}{n}}{\frac{1}{n+1}} \implies \\
                  \implies \frac{a_{n+1}}{a_n} > \frac{\frac{1}{n+1}}{\frac{1}{n}}.
              \end{multline*}

              Ряд $\sum_{n=1}^{\infty}\frac{1}{n}$ -- гармонический, расходящийся $\implies$ по 3-му признаку сравнения ряд $(A)$ расходится.

        \item \underline{Упражнение:} привести 2 примера рядов (сходящийся, расходящийся), но $r=1$ в обоих случаях.
    \end{enumerate}
\end{proof}

\subsection{Признак Кумера}

\begin{theorem}[Признак Кумера]
    Пусть дан ряд $(A)$ -- положительный. Пусть числа $c_1,c_2,\ldots,c_n,\ldots: \ \forall n > N \ c_n > 0$ и ряд $\sum_{n=1}^{\infty}c_n$ -- расходится. Если
    \[
        \underset{n\rightarrow\infty}{\lim}\left(c_n \cdot \frac{a_n}{a_{n+1}} - c_{n+1}\right) = k,
    \]
    то
    \begin{enumerate}
        \item При $k > 0$ ряд $(A)$ сходится.
        \item При $k < 0$ ряд $(A)$ расходится.
        \item При $k = 0$ может как сходиться, так и расходиться.
    \end{enumerate}
\end{theorem}

\begin{proof}\leavevmode
    \begin{enumerate}
        \item Пусть $k > 0$. Возьмем $0 < p < k$. Тогда $\exists N : \ \forall n > N$:
              \begin{multline*}
                  c_n \cdot \frac{a_n}{a_{n+1}} - c_{n+1}>p \implies \\
                  \implies c_n \cdot a_n - c_{n+1} \cdot a_{n+1} > p \cdot a_{n+1} > 0 \implies \\
                  \implies c_n \cdot a_n > c_{n+1} \cdot a_{n+1}, \quad \forall n > N
              \end{multline*}

              Тогда последовательность $\{c_n\cdot a_n\}$ убывает и ограничена снизу $\implies$ последовательность сходится.

              Пусть $c=\underset{n\rightarrow\infty}{\lim}c_n\cdot a_n$. Рассмотрим ряд:
              \begin{multline*}
                  \sum_{m=1}^{n}(c_m\cdot a_m - c_{m+1}\cdot a_{m+1}) = \\
                  = (c_1 \cdot a_1 - c_2 \cdot a_2) + (c_2 \cdot a_2 - c_3 \cdot a_3) + \ldots + (c_n \cdot a_n - c_{n+1}\cdot a_{n+1}) = \\
                  = c_1 \cdot a_1 - c_{n+1} \cdot a_{n+1},
              \end{multline*}
              \begin{multline*}
                  \underset{n\rightarrow\infty}{\lim}\sum_{m=1}^{n}(c_m \cdot a_m - c_{m+1} \cdot a_{n+1}) = \\
                  = \underset{n\rightarrow\infty}{\lim}(c_1 \cdot a_1 - c_{n+1} \cdot a_{n+1}) = c_1 \cdot a_1 - c \implies
              \end{multline*}
              $\implies$ сходится ряд $\sum_{n=1}^{\infty}(c_n \cdot a_n - c_{n+1} \cdot a_{n+1}) \implies$ из того, что $c_n \cdot a_n - c_{n+1} \cdot a_{n+1} > p \cdot a_{n+1} > 0$ и 1-го признака сравнения $\implies$ ряд $\sum_{n=1}^{\infty}p\cdot a_{n+1}$ сходится $\implies$ ряд $(A)$ сходится.

        \item Пусть $k < 0 \implies \exists N: \ \forall n > N$
              \begin{multline*}
                  c_n \cdot \frac{a_n}{a_{n+1}} - c_{n+1} < 0 \implies \\
                  \implies \frac{a_n}{a_{n+1}} < \frac{c_{n+1}}{c_n} = \frac{\frac{1}{c_n}}{\frac{1}{c+{n+1}}} \implies \frac{a_{n+1}}{a_{n}} > \frac{\frac{1}{c_{n+1}}}{\frac{1}{c_n}}.
              \end{multline*}

              $\sum_{n=1}^{\infty}\frac{1}{c_n}$ расходится $\implies$ по 3-му признаку сравнения ряд $(A)$ расходится.

        \item Придумать 2 примера когда $k=0$ и ряды сходятся/расходятся.
    \end{enumerate}
\end{proof}

\subsection{Признак Бертрана}

\begin{theorem}[Признак Бертрана]
    Пусть ряд $(A)$ -- положительный. Если
    \[
        \underset{n\rightarrow\infty}{\lim} \ln n \cdot \left[n \cdot (\frac{a_n}{a_{n+1}} - 1)\right] = B,
    \]
    то
    \begin{enumerate}
        \item При $B > 1$ ряд $(A)$ сходится.
        \item При $B < 1$ ряд $(A)$ расходится.
        \item При $B = 1$ ряд $(A)$ может как сходиться, так и расходиться.
    \end{enumerate}
\end{theorem}

\begin{proof}
    Рассмотрим ряд $\sum_{n=2}^{\infty} \frac{1}{n\cdot \ln n}$ -- расходится. Составим последовательность Кумера:
    \begin{multline*}
        k_n = \underbrace{n \cdot \ln n}_{c_n} \cdot \frac{a_n}{a_{n+1}} - \underbrace{(n+1) \cdot \ln(n+1)}_{c_{n+1}} = \\
        = \left| \ln(n+1) = \ln\left(n\cdot \frac{n+1}{n}\right) = \ln n + \ln\left(1 + \frac{1}{n}\right) \right| = \\
        = n \cdot \ln n \cdot \frac{a_n}{a_{n+1}} - (n+1)\cdot \left(\ln n + \ln\left(1 + \frac{1}{n}\right)\right) = \\
        = n \cdot \ln n \cdot \frac{a_n}{a_{n+1}} - n\cdot \ln n - \ln n - \ln\left(1 + \frac{1}{n}\right)^{n+1} = \\
        = \ln n \left(n \cdot \frac{a_n}{a_{n+1}} - n - 1\right) - \ln \left(1 + \frac{1}{n}\right)^{n+1} = \\
        = \ln n \cdot \left(n\left(\frac{a_n}{a_{n+1}} - 1\right) - 1\right) - \ln\left(1+\frac{1}{n}\right)^{n+1};
    \end{multline*}
    \begin{multline*}
        \underset{n\rightarrow\infty}{\lim} k_n = \\
        = \underset{n\rightarrow\infty}{\lim}\left[\underbrace{\ln n \cdot \left(n\left(\frac{a_n}{a_{n+1}} - 1\right) - 1\right)}_{B} - \ln\underbrace{\left(1 + \frac{1}{n}^n\right)}_{e} - \ln\left(1 + \frac{1}{n}\right)\right] = \\
        = B - 1,
    \end{multline*}
    по признаку Кумера, при $B-1 > 0$ ряд $(A)$ сходится, при $B-1 < 0$ ряд $(A)$ расходится, при $B=1$ ряд $(A)$ может как сходиться, так и расходиться.
\end{proof}

\subsection{Признак Гаусса}

\begin{theorem}[Признак Гаусса]
    Ряд $(A), \ a_n > 0, \ \forall n \in \N, \ \lambda, \mu \in \R$. Если
    \[
        \frac{a_n}{a_{n+1}} = \left(\lambda + \frac{\mu}{n}\right) + O\left(\frac{1}{n^2}\right),
    \]
    то
    \begin{enumerate}
        \item При $\lambda > 1$, ряд $(A)$ сходится.
        \item При $\lambda < 1$, ряд $(A)$ расходится.
        \item При $\lambda = 1$ и \begin{enumerate}
                  \item $\mu > 1 \implies$ ряд $(A)$ сходится.
                  \item $\mu \leqslant 1 \implies$ ряд $(A)$ расходится.
              \end{enumerate}
    \end{enumerate}
\end{theorem}

\begin{proof}\leavevmode
    \begin{enumerate}
        \item Если $\lambda < 1$, то
              \begin{multline*}
                  \underset{n\rightarrow\infty}{\lim} \frac{a_{n+1}}{a_n} = \left[\underset{n\rightarrow\infty}{\lim}\left(\lambda + \frac{\mu}{n} + O\left(\frac{1}{n^2}\right)\right)\right]^{-1} = \\
                  = \left[\underset{n\rightarrow\infty}{\lim}\left(\lambda + \underbrace{\frac{\mu}{n}}_{\rightarrow 0} + \underbrace{\frac{1}{n^2}}_{\rightarrow 0} \cdot \Omega\left(\frac{1}{n^2}\right)\right)\right]^{-1} = \frac{1}{\lambda},
              \end{multline*}
              по признаку Даламбера, если $\frac{1}{\lambda} < 1$, то есть $\lambda > 1$, ряд $(A)$ сходится.

        \item $ \implies $ из 1.

        \item Если $\lambda = 1$, то
              \[
                  \frac{a_n}{a_{n+1}} = 1 + \frac{\mu}{n} + O\left(\frac{1}{n^2}\right),
              \]
              \[
                  n\left(\frac{a_n}{a_{n+1}} - 1\right) = \mu + n \cdot O\left(\frac{1}{n^2}\right),
              \]
              \[
                  \underset{n\rightarrow\infty}{\lim}\left(n\cdot \frac{a_n}{a_{n+1}} - 1\right) = \underset{n\rightarrow\infty}{\lim}\left(\mu + \underbrace{n \cdot \frac{1}{n^2} \cdot \Omega (\frac{1}{n^2})}_{\rightarrow 0}\right) = \mu \implies
              \]
              $\implies$ по признаку Раббе $\implies \left[\begin{array}{l}
                      \mu > 1 \implies (A) \text{ сходится.} \\
                      \mu < 1 \implies (A) \text{ расходится.}
                  \end{array} \right.$

              Пусть $\mu = 1$, тогда
              \begin{multline*}
                  \underset{n\rightarrow\infty}{\lim}\ln n \cdot \left(n \cdot \left(\frac{a_n}{a_{n+1}} - 1\right) - 1\right) = \\
                  = \underset{n\rightarrow\infty}{\lim}\ln n \cdot \left(n \cdot \left(1 + \frac{1}{n} + O\left(\frac{1}{n^2}\right) - 1\right) - 1\right) = \\
                  = \underset{n\rightarrow\infty}{\lim}\ln n \cdot \left(1 + n \cdot O\left(\frac{1}{n^2}\right) - 1\right) = \\
                  = \underset{n\rightarrow\infty}{\lim}\ln n \cdot n \cdot O\left(\frac{1}{n^2}\right) = \\
                  = \underset{n\rightarrow\infty}{\lim}\left(\ln n \cdot n \cdot \frac{1}{n^2} \cdot \Omega\left(\frac{1}{n^2}\right)\right) = \\
                  = \underset{n\rightarrow\infty}{\lim}\frac{\ln n}{n} \cdot \Omega\left(\frac{1}{n^2}\right) = 0.
              \end{multline*}
              В самом деле,
              \[
                  \underset{n\rightarrow\infty}{\lim}\frac{\ln n}{n} = \underset{n\rightarrow\infty}{\lim} \frac{1}{n} \cdot \ln n = \underset{n\rightarrow\infty}{\lim}\ln n^{\frac{1}{n}} = \underset{n\rightarrow\infty}{\lim} \ln \sqrt[n]{n} = 0 \implies
              \]
              $\implies$ по прихнаку Бертрана ряд $(A)$ расходится.
    \end{enumerate}
\end{proof}

\setcounter{subsection}{49}

\subsection{Следствие абсолютной сходимости ряда}

\begin{statement}
    Если ряд $(A)$ абсолютно сходящийся, то он сходящийся.
\end{statement}

\begin{proof}
    Пусть ряд $(A)$ абсолютно сходящийся, то есть сходится ряд $(A^*) \implies$ по критерию Коши $\forall \epsilon > 0 \ \exists N: \ \forall n > N \ \forall p > 0$
    \[
        |a_{n+1}| + |a_{n+1}| + \ldots + |a_{n+1}| < \epsilon.
    \]

    Пусть $\epsilon > 0$ задано. Рассмотрим:
    \[
        |A_{n+p} - A_n| = |a_{n+1} + \ldots + a_{n+p}| \leqslant |a_{n+1}| + \ldots + |a_{n+p}| < \epsilon \implies
    \]
    $\implies$ ряд $(A)$ сходится.
\end{proof}

\setcounter{subsection}{51}

\subsection{Признак Лейбница}

\begin{theorem}[признак Лейбница]
    Пусть ряд $(\overline{A}), \ a_n > 0 \ \forall n$ удовлетворяет условиям:
    \begin{enumerate}
        \item $a_1 \geqslant a_2 \geqslant a_3 \geqslant \ldots \geqslant a_n \geqslant \ldots$.
        \item $\underset{n\rightarrow\infty}{\lim} a_n = 0$.
    \end{enumerate}

    Тогда ряд $(\overline{A})$ сходится и его сумма $S: \ 0 < S \leqslant a_1$.
\end{theorem}

\begin{proof}
    Рассмотрим:
    \begin{multline*}
        S_{2n} = a_1 - a_2 + a_3 - \ldots + a_{2n - 1} - a_{2n} = \\
        = (a_1 - a_2) + (a_3 - a_4) + \ldots + (a_{2n-1} - a_{2n}),
    \end{multline*}
    тогда $\forall i: \ a_i - a_{i+1} \geqslant 0 \implies S_{2n}\geqslant 0 \ \forall n \implies$ последовательность $S_{2n} \nearrow$.

    С другой стороны,
    \[
        S_{2n} = a_1 - \underbrace{(a_2 - a_3)}_{\geqslant 0} - \underbrace{(a_4 - a_5)}_{\geqslant 0} - \ldots - \underbrace{(a_{2n-2} - a_{2n-1})}_{\geqslant0} - a_{2n} \implies
    \]
    $\implies S_{2n} \leqslant a_1 \ \forall n$.

    Таким образом, $S_{2n}$ не убывает и ограничена сверху $\implies$ по теореме Вейерштрасса $\implies \exists \underset{n\rightarrow\infty}{\lim} S_{2n} = S$.

    Далее,
    \[
        \underset{n\rightarrow\infty}{\lim}S_{2n+1} = \underset{n\rightarrow\infty}{\lim}(S_{2n} + a_{2n+1}) = \underset{n\rightarrow\infty}{\lim}S_{2n} + \underset{n\rightarrow\infty}{\lim} a_{2n+1} = S + 0 = S.
    \]

    Таким образом, $\underset{n\rightarrow\infty}{\lim}S_n = S$.

    Так как $0 < S_n \leqslant a_1$ (если $S_n = 0$, то $a_1$ может быть $=0$, что невозможно, так как $a_n > 0$) $\implies$ (берем пределы от неравенства) $0 < S \leqslant a_1$.
\end{proof}

\subsection{Признак Абеля}

\begin{theorem}[Признак Абеля]
    Если \begin{itemize}
        \item последовательность $\{a_n\}$ монотонна и ограничена,
        \item ряд $\sum_{n=1}^{\infty} b_n$ сходится,
    \end{itemize}
    то ряд $\sum_{n=1}^{\infty}a_n \cdot b_n$ сходится.
\end{theorem}

\begin{proof}
    Пусть выполнены условия признака Абеля. Тогда $\exists M > 0: \ |a_n| \leqslant M$. Пусть $\epsilon > 0$ задано. Возьмем номер $N: \ \forall n > N, \ \forall p > 0$
    \[
        \bigg|\sum_{k=n+1}^{n+p}b_k\bigg| < \epsilon^* = \frac{\epsilon}{3\cdot M}.
    \]

    Частичные суммы ряда $\sum_{n=1}^{\infty}a_n\cdot b_n$ имеют вид $S_n = a_1\cdot b_1 + \ldots + a_n \cdot b_n$. По критерию Коши найдем $N_1: \ \forall n > N_1, \forall p > 0$
    \[
        |S_{n+p} - S_n| < \epsilon,
    \]
    \begin{multline*}
        |a_{n+1} \cdot b_{n+1} + a_{n+2} \cdot b_{n+2} + \ldots + a_{n+p} \cdot b_{n+p}| \leqslant \\
        \leqslant \epsilon^* \cdot (|a_{n+1}| + 2 \cdot |a_{n+p}|) \leqslant \epsilon^* \cdot 3 \cdot M = \frac{\epsilon}{3 \cdot M} = \epsilon \implies
    \end{multline*}
    $\implies$ по критерию Коши ряд $\sum_{n=1}^{\infty}a_n \cdot b_n$ сходится.
\end{proof}

\subsection{Признак Дирихле}

\begin{theorem}[Признак Дирихле]
    Если \begin{itemize}
        \item последовательность $\{a_n\}$ монотонна и $\underset{n\rightarrow\infty}{\lim}a_n = 0$,
        \item частичные суммы ряда $(B)$ ограничены, то есть $\exists k > 0: $ \\ $ \forall n \ \left|\sum_{m=1}^{n} b_m\right|< k$,
    \end{itemize}
    то $\sum_{n=1}^{\infty}a_n \cdot b_n$ сходится.
\end{theorem}

\begin{proof}
    Пусть выполнены условия признака Дирихле. Так как $\underset{n\rightarrow\infty}{\lim}a_n = 0$, то $\exists N: \ \forall n > N \quad (\epsilon > 0$ задано):
    \[
        |a_n| < \frac{\epsilon}{3 \cdot k}, \quad \bigg|\sum_{k=1}^{n}b_k\bigg| \leqslant k.
    \]

    По критерию Коши:
    \begin{multline*}
        |S_{n+p} - S_n| = |a_{n+1} \cdot b_{n+1} + \ldots + a_{n+p} \cdot b_{n+p}| \overset{\text{по лемме}}{\leqslant} \\
        \leqslant k\cdot(|a_{n+1}| + 2\cdot |a_{n+p}|) < k\cdot \frac{3\cdot \epsilon}{3 \cdot k} = \epsilon.
    \end{multline*}
\end{proof}

\subsection{Сочетательное свойство сходящихся рядов}

\begin{theorem}[Сочетательное свойство сходящихся рядов]\leavevmode
    \begin{enumerate}
        \item Если ряд $(A)$ сходится, то для любой возрастающей последовательности $n_k$ ряд $(\widetilde{A})$ сходится и их суммы совпадают ($A = \widetilde{A}$).
        \item Если ряд $(\widetilde{A})$ сходится и внутри каждой  скобки знак не меняется, то ряд $(A)$ сходится и их суммы совпадают, то есть $\widetilde{A} = A$.
    \end{enumerate}
\end{theorem}

\begin{proof}\leavevmode
    \begin{enumerate}
        \item Пусть ряд $(A)$ сходится, $\widetilde{A}_k$ -- частичные суммы ряда $(\widetilde{A})$:
              \[
                  \begin{array}{l}
                      \widetilde{A}_1 = \widetilde{a}_1 = \sum_{k=1}^{n_1}a_k = A_{n_1}                         \\
                      \widetilde{A}_2 = \widetilde{a}_1 + \widetilde{a}_2 = \sum_{k=n_1 + 1}^{n_2}a_k = A_{n_1} \\
                      \vdots                                                                                    \\
                      \widetilde{A}_k = A_{n_k}
                  \end{array}.
              \]

              Так как ряд $(A)$ сходится, то $ \exists \underset{k\rightarrow\infty}{\lim}A_{n_k} = A $, следовательно:
              \begin{eqnarray*}
                  A &=& \underset{k\rightarrow\infty}{\lim}A_{n_k} = \\
                  = \underset{n\rightarrow\infty}{\lim}\widetilde{A}_k &=& \widetilde{A}
              \end{eqnarray*}

              \newpage

        \item Пусть ряд $(\widetilde{A})$ сходится. Имеем:

              \[
                  \text{при: }\begin{array}{l}
                      a_1 > 0: \quad A_1 < A_2 < \ldots < A_{n_1} \\
                      a_1 < 0: \quad A_1 > A_2 > \ldots > A_{n_1}
                  \end{array}.
              \]

              \begin{itemize}
                  \item Далее, если $a_{n_1 + 1} > 0$, тогда:

                        при $ a_1 > 0: \ A_{n_1 + 1} < A_{n_1 + 2} < \ldots < A_{n_2}$
                        \[
                            A_{n_1} = \widetilde{A}_1 < A_{n_2} = \widetilde{A}_2,
                        \]
                        при $a_1 < 0: \ A_{n_1} < 0$ и $A_{n_1} < A_{n_2} $
                        \[
                            \widetilde{A}_1 < \widetilde{A}_2.
                        \]

                  \item Если же $a_{n_1 + 1} < 0$, тогда:
                        \[
                            \text{при: }\begin{array}{l}
                                a_1 < 0: \quad A_{n_1} = \widetilde{A}_1 > A_{n_2} = \widetilde{A}_2 \\
                                a_1 > 0: \quad A_{n_1} = \widetilde{A}_1 > \widetilde{A}_2
                            \end{array}.
                        \]
              \end{itemize}

              Аналогично, пока $n$ меняется от $n_k$ до $n_{k+1}$, то будем иметь либо $A_{n_k} < A_n < A_{n_{k+1}}$, либо $A_{n_k} > A_n > A_{n_{k+1}}$.

              Ряд $(\widetilde{A})$ -- сходится $\implies \exists \underset{k\rightarrow\infty}{\lim}\widetilde{A}_k = \underset{k\rightarrow\infty}{\lim}\widetilde{A}_{k+1} = \widetilde{A} \implies$ по теореме о 2-х миллиционерах:
              \[
                  \underset{k\rightarrow\infty}{\lim}A_n = \widetilde{A}.
              \]
    \end{enumerate}
\end{proof}

\subsection{Переместительное свойство сходящихся рядов}

\begin{theorem}[Переместительное свойство сходящихся рядов]
    Если ряд $(A)$ абсолютно сходится, то его сумма не зависит от перестановки членов ряда.
\end{theorem}

\begin{proof}[Доказательство теоремы]
    Пусть ряд $ (A) $ сходится абсолютно $\implies$ ряд $ (A^*) $ сходится. Пусть ряд
    \[
        (A') \ \sum_{n=1}^{\infty}a_n'
    \]
    получен из ряда $(A)$ путем перестановки его членов. Покажем, что ряд $(A')$ сходится и $A = A'$ (их суммы совпадают).

    \begin{enumerate}
        \item Пусть $(A)$ -- знакоположительный, то есть $\forall n \in \N \quad a_n > 0$. Рассмотрим частичные суммы ряда $(A')$:
              \[
                  A_k' = a_1' + a_2' + \ldots + a_k' = a_{n_1} + a_{n_2} + \ldots + a_{n_k}.
              \]

              Пусть $n' = \max\{n_1,n_2,\ldots,n_k\}$. Тогда:
              \[
                  A_k' \leqslant a_1 + a_2 + \ldots + a_{n_j} + \ldots + a_{n'} = A_{n'},
              \]
              где $A_{n'}$ -- $n'$-я частичная сумма ряда $(A)$. Так как $(A)$ сходится и знакоположительный $\implies A_{n'} \leqslant A$.

              Таким образом получаем, что $\forall k \ A_k' \leqslant A \implies$ последовательность $A_k' \nearrow$ и ограничена, тогда:
              \[
                  \exists\underset{k\rightarrow\infty}{\lim}A_k' = A' \leqslant A.
              \]

              С другой стороны, ряд $(A')$ получен перестановкой членов ряда $(A) \implies A' \geqslant A \implies A' \leqslant A \leqslant A' \implies A = A'$.

        \item Пусть ряд $(A)$ сходится абсолютно, то есть $(A^*)$ сходится. С рядом $(A)$ свяжем два ряда:
              \[
                  (P) \ \sum_{n=1}^{\infty}p_n, \quad (Q) \ \sum_{n=1}^{\infty}q_n,
              \]
              где $p_n$ -- положительные члены ряда $(A)$, $q_n$ -- отрицательные члены ряда $(A)$, взятые по модулю, причем все члены рядов $(P)$ и $(Q)$ взяты в том же порядке, как они стояли в ряде $(A)$.
    \end{enumerate}

    Если ряд $(A)$ сходится абсолютно, то сходится ряд $(A^*)$, $(A^*)$ -- положительный ряд $\implies (A^{*'})$ сходится (получен путем перестановки членов ряда $(A^*)$) $\implies$ по лемме сходятся ряды $(P')$ и $(Q')$ и $A' = P' - Q'$.

    \[
        \begin{array}{ccc}
                &          & (P) \\
                & \nearrow &     \\
            (A) &          &     \\
                & \searrow &     \\
                &          & (Q)
        \end{array} \quad \Longrightarrow \quad \begin{array}{ccccc}
            (A^*) & \rightarrow & \underbrace{(A^{*'})}_{\text{сх.}} &          &      \\
                  &             & \downarrow                         &          &      \\
                  &             & (A')                               &          &      \\
                  & \swarrow    &                                    & \searrow &      \\
            (P')  &             &                                    &          & (Q')
        \end{array}
    \]

    \begin{itemize}
        \item $(P')$ -- положительный ряд $\implies$ по пункту 1, $(P)$ -- сходится,
        \item $(Q')$ -- положительный ряд $\implies$ по пункту 1, $(Q)$ -- сходится
    \end{itemize}
    и $P' = P, \ Q' = Q \implies A' = P - Q = A$.
\end{proof}

\subsection{Теорема Римана о перестановке членов условно сходящегося ряда}

\begin{theorem}[Римана о перестановке членов условно сходящегося ряда]
    Если ряд $(A)$ условно сходится, то $\forall B \in \R$ (в том числе $B = \pm\infty$) $\exists$ перестановка ряда $(A)$ такая, что полученный ряд сходится и имеет сумму $B$. Более того, $\exists$ перестановка ряда $(A)$ такая, что частичные суммы полученного ряда не стремятся ни к конечному, ни к бесконечному пределу.
\end{theorem}

\begin{proof}[Доказательство теоремы]
    Пусть $B \in \R$. Возьмем номера:
    \[
        \begin{array}{l}
            n_1: \ p_1 + p_2 + \ldots + p_{n_1} \geqslant B, \\
            n_2: \ p_1 + p_2 + \ldots + p_{n_1} - q_1 - q_2 - \ldots - q_{n_2} \leqslant B.
        \end{array}
    \]

    Более того, элементы $p$ и $q$ будем брать столько, сколько это необходимо для выполнения этого условия.

    Возьмем:
    \[
        n_3: \ p_1 + p_2 + \ldots + p_{n_1} - q_1 - q_2 - \ldots - q_{n_2} + p_{n_1 + 1} + p_{n_1 + 2} + \ldots + p_{n_3} \geqslant B
    \] и так далее.

    Таким образом получим ряд
    \begin{multline*}
        (p_1 + \ldots + p_{n_1}) + (-q_1 - \ldots - q_{n_2}) + \\
        + (p_{n_1 + 1} + \ldots + p_{n_3}) + (-q_{n_2 + 1} - \ldots - q_{n_4}) + \ldots
    \end{multline*} -- этот ряд сходится к $B$.

    Действительно, так как ряд $(A)$ сходится, то $\underset{n\rightarrow\infty}{\lim} a_n = 0$.

    Так как количество членов $p_i$ и $q_i$ бралось лишь столько, сколько необходимо, то соответствующие частичные суммы отличаются от $B$ разве что на последнее слогаемое в этой частичной сумме, которое стремится к нулю $\implies \underset{n\rightarrow\infty}{\lim}A_n' = B$.
\end{proof}

\subsection{Теорема Коши о произведении рядов}

\begin{theorem}[Коши о произведении рядов]
    Если ряды $ (A),(B) $ абсолютно сходятся, $A$ и $B$ -- их суммы, то $\forall$ их произведение абсолютно сходится и равно $A \cdot B$.
\end{theorem}

\begin{proof}
    Рассмотрим $r$-тую частичную сумму ряда
    \[
        (A\cdot B)^* \quad \sum_{r=1}^{\infty}|a_{n_r}\cdot b_{k_r}|,
    \]
    \begin{multline*}
        S_r = |a_{n_1} \cdot b_{k_1}| + |a_{n_2} \cdot b_{k_2}| + \ldots + |a_{n_r} \cdot b_{k_r}| \leqslant \\
        \leqslant (|a_{n_1}| + |a_{n_2}| + \ldots + |a_{n_r}|) \cdot (|b_{k_1}| + |b_{k_2}| + \ldots + |b_{k_r}|) \leqslant \\
        \leqslant (|a_1| + |a_2| + \ldots + |a_m|) \cdot (|b_1| + |b_2| + \ldots + |b_m|),
    \end{multline*}
    где $m = \max\{n_1,n_2,\ldots,n_r,k_1,k_2,\ldots,k_r\}$.

    Так как ряды $(A)$ и $(B)$ сходятся абсолютно, то есть сходятся ряды $(A^*)$ и $(B^*)$, то $S_r \leqslant A^* \cdot B^* \implies$ последовательность $S_r \nearrow$ и ограничена $\implies \exists \underset{r\rightarrow\infty}{\lim} S_r \implies$ ряд $(A\cdot B)^*$ сходится $\implies$ ряд $ (A\cdot B) $ -- сходится, причем его сумма не зависит от порядка суммирования.

    Будем суммировать ряд $A\cdot B$ по квадратам:
    \[
        \underbrace{a_1b_1}_{c_1} + \underbrace{(a_1b_2 + a_2 b_2 + a_2 b_1)}_{c_2} + \underbrace{(a_1b_3 + a_2b_3 + a_3b_3 + a_3b_2 + b_3b_1)}_{c_3} + \ldots
    \]
    \[
        \begin{array}{l}
            S_1 = a_1b_1 = A_1\cdot B_1                                                                        \\
            S_2 = c_1 + c_2 = a_1b_1 + (a_1b_2 + a_2b_2 + a_2b_1) = (a_1 + a_2)\cdot(b_1 + b_2) = A_2\cdot B_2 \\
            S_3 = c_1 + c_2 + c_3 = (a_1 + a_2 + a_3)\cdot(b_1 + b_2 + b_3) = A_3\cdot b_3                     \\
            \vdots                                                                                             \\
            S_n = A_n \cdot B_n
        \end{array}
    \]
    \[
        \underset{n\rightarrow\infty}{\lim}S_n = \underset{n\rightarrow\infty}{\lim}(A_n \cdot B_n) = \underset{n\rightarrow\infty}{\lim}A_n \cdot \underset{n\rightarrow\infty}{\lim}B_n = A\cdot B
    \]
\end{proof}

\setcounter{subsection}{62}

\subsection{Теорема о связи сходимости простого и повторного рядов}

\begin{definition}[Повторный ряд]
    \emph{Повторным рядом} называются выражения
    \begin{equation}\label{eq:6.6.1}
        \sum_{n=1}^{\infty}\sum_{k=1}^{\infty}a_{nk},
    \end{equation}
    \begin{center}
        и
    \end{center}
    \begin{equation}\label{eq:6.6.2}
        \sum_{k=1}^{\infty}\sum_{n=1}^{\infty}a_{nk}.
    \end{equation}

    Говорят, что ряд \ref{eq:6.6.1} сходится, если сходятся все ряды $(A_n)$ по строкам $(\sum_{k=1}^{\infty}a_{n_k} = A_n)$ и сходится ряд $ \sum_{n=1}^{\infty}A_n $.
\end{definition}

\begin{definition}[Простой ряд]
    Пусть ряд
    \begin{equation}\label{eq:6.6.4}
        \sum_{r=1}^{\infty}U_r
    \end{equation}
    построен из элементов таблицы, взятых в произвольном порядке. Такой ряд будем называть \emph{простым}, связанным с данной таблицей.
\end{definition}

\begin{theorem}[О связи сходимости простого и повторного рядов]\leavevmode
    \begin{enumerate}
        \item Если ряд \ref{eq:6.6.4} абсолютно сходится, то ряд \ref{eq:6.6.1} сходится и его сумма равна $U$.

        \item Если после замены элементов таблицы $(\star)$ их модулями ряд \ref{eq:6.6.1}$ ^* $ ходится, то ряд \ref{eq:6.6.4} сходится абсолютно и суммы рядов \ref{eq:6.6.1} (без модулей) и \ref{eq:6.6.4} совпадают.
    \end{enumerate}
\end{theorem}

\begin{proof}\leavevmode
    \begin{enumerate}
        \item Пусть \ref{eq:6.6.1}$ ^* $ сходится. Покажем, что все ряды по строкам сходятся:
              \[
                  (A_n) \ \sum_{k=1}^{\infty}a_{nk} \quad (\forall n \in \N)
              \]
              и сходится ряд $ \sum_{n=1}^{\infty}A_n $.

              Рассмотрим
              \[
                  |a_{n1}| + |a_{n2}| + \ldots + |a_{nk}| \leqslant |u_1| + |u_2| + \ldots + |u_r|,
              \]
              где $r$ выбран таким образом, чтобы среди $|u_i|$ были все слагаемые $|a_{n1}, \ldots, a_{nk}|$.

              Таким образом,
              \[
                  \underbrace{|a_{n1}| + \ldots + |a_{nk}|}_{A_{nk}^*} \leqslant U^* \implies \exists \underset{k\rightarrow\infty}{\lim}A_{nk}^* = A_n^* \implies
              \]
              $\implies$ ряд $\sum_{k=1}^{\infty}a_{nk} \ \forall n \in \mathbb{N}$ сходится абсолютно $\implies$ он сходится.

              Далее, пусть $\epsilon > 0$ задано. Выберем номер $r_0: \ \forall r > r_0$
              \[
                  \sum_{i=1}^{\infty}|u_{r+i}| < \frac{\epsilon}{3}.
              \]

              Тогда
              \[
                  \bigg|\sum_{i=1}^{r} u_i - U\bigg| = \bigg| \sum_{i=1}^{\infty}u_{r + i}\bigg| \leqslant \sum_{i=1}^{\infty}|u_{r + i}| < \frac{\epsilon}{3}
              \]

              Так как ряды по строкам сходятся, то $\forall n$ выберем $m(n)$:
              \[
                  \bigg|\sum_{k=1}^{m(n)}a_{n_k} - A_n\bigg| < \frac{\epsilon}{3}.
              \]

              Наконец, выберем номер $N_0$ такой, что все числа $u_1, u_2,\ldots, u_{r_0}$ содержались бы в первых $N_0$ строках:
              \begin{multline*}
                  \bigg|\sum_{n=1}^{N_0}A_n - U\bigg| = \\
                  = \bigg|\sum_{n=1}^{N_0}A_n - \sum_{n=1}^{N_0}\sum_{k=1}^{m(n)}a_{n_k} + \sum_{n=1}^{N_0}\sum_{k=1}^{m(n)}a_{n_k} - \sum_{i=1}^{r_0}u_i + \sum_{i=1}^{r_0}u_i - U\bigg| \leqslant \\
                  \leqslant \sum_{n=1}^{N_0}\bigg|A_n - \sum_{k=1}^{m(n)}a_{n_k}\bigg| + \bigg|\sum_{n=1}^{N_0}\sum_{k=1}^{m(n)}a_{n_k} - \sum_{i=1}^{r_0}u_i\bigg| + \underbrace{\bigg|\sum_{i=1}^{r_0}u_i - U\bigg|}_{<\frac{\epsilon}{3}} < \\
                  < \frac{\epsilon}{3} + \sum_{i=r_0 + 1}^{\infty}(u_i) + \frac{\epsilon}{3} < \frac{\epsilon}{3} \cdot 3 = \epsilon.
              \end{multline*}

        \item Пусть ряд $\sum_{n=1}^{\infty}\sum_{k=1}^{\infty}|a_{n_k}| = A^*$ сходится.

              Тогда $\forall r \ \exists N,K$ такие, что числа $u_1,\ldots,u_r$ содержатся в $N$ первых строчках и $K$ первых столбцах таблицы:
              \[
                  \sum_{i=1}^{r}|u_i| \leqslant \sum_{n=1}^{N}\sum_{k=1}^{K}|a_{n_k}| \leqslant A^* \implies
              \]
              $\implies |u_r|\nearrow$ и ограничен $\implies$ ряд \ref{eq:6.6.4} сходится абсолютно $\implies$ по пункту 1. суммы рядов \ref{eq:6.6.4} и \ref{eq:6.6.1} равны.
    \end{enumerate}
\end{proof}

\subsection{Свойства двойного ряда}

\begin{definition}[Двойной ряд]
    \emph{Двойным рядом} называется выражение:
    \begin{equation}\label{eq:6.6.3}
        \sum_{n,k = 1}^{\infty} a_{nk}
    \end{equation}

    Говорят, что ряд \ref{eq:6.6.3} сходится, если:
    \[
        \exists A = \underset{N\rightarrow\infty}{\underset{K\rightarrow\infty}{\lim}}A_{NK} = \underset{N\rightarrow\infty}{\underset{K\rightarrow\infty}{\lim}}\sum_{n=1}^{N}\sum_{k=1}^{K}a_{nk}.
    \]

    То есть $\forall \epsilon > 0 \ \exists N_0$ и $K_0: \ \forall N > N_0$ и $\forall k > K_0$
    \[
        \bigg|\underbrace{\sum_{n=1}^{N}\sum_{k=1}^{K}a_{nk}}_{A_{NK}} - A\bigg| < \epsilon.
    \]
\end{definition}

\begin{theorem}[Свойства двойных рядов]\leavevmode
    \begin{enumerate}
        \item Если ряд \ref{eq:6.6.3} сходится, то
              \[
                  \underset{k\rightarrow\infty}{\underset{n\rightarrow\infty}{\lim}}a_{nk} = 0.
              \]

        \item (Критерий Коши)
              Ряд \ref{eq:6.6.3} сходится $\iff \forall \epsilon > 0 \ \exists N_0,K_0: \ \forall n > N_0, \ \forall k > K_0, \ \forall p > 0, \ \forall q > 0$
              \[
                  \bigg|\sum_{n=1}^{p}\sum_{k=1}^{q}a_{(N_0 + n)(K_0 + k)}\bigg| < \epsilon.
              \]

        \item Если ряд \ref{eq:6.6.3} сходится, то $\forall c \in \R$ ряд
              \[
                  \sum_{n,k=1}^{\infty}(c\cdot a_{nk})
              \]
              сходится, и его сумма равна $c\cdot A$ (где $A = \sum_{n,k=1}^{\infty}a_{nk}$).

        \item Если ряд \ref{eq:6.6.3} сходится и ряд
              \[
                  \sum_{n,k=1}^{\infty}b_{nk}
              \]
              сходится, то
              \[
                  \sum_{n,k=1}^{\infty}(a_{nk} + b_{nk}) = A + B,
              \]
              а к тому же -- сходится.

        \item Если $\forall n, \ \forall k \ a_{nk} \geqslant 0$, то ряд \ref{eq:6.6.3} сходится $ \iff $ его частичные суммы ограничены в совокупности.
    \end{enumerate}
\end{theorem}

\begin{proof}\leavevmode
    \begin{enumerate}
        \item Пусть ряд \ref{eq:6.6.3} сходится. Заметим, что
              \[
                  A_{nk} = \sum_{i,j=1}^{n,k},
              \]
              \[
                  a_{nk} = A_{nk} - A_{n(k-1)} - A_{(n-k)k} + A_{(n-1)(k-1)}
              \]
              $\implies a_{nk} \rightarrow 0 $.

        \item (Критерий Коши)
              На декартовом произведении $\N\times\N$ введем базу:
              \[
                  B_{nk} = \big\{(n,k): \ n > N_0, \ k > K_0\big\}.
              \]

              Тогда критерий Коши сходимости ряда -- это есть критерий Коши существования предела функции $A_{nk}$ по данной базе.

        \item Самостоятельно.

        \item Самостоятельно.

        \item \begin{itemize}
                  \item $ |\Rightarrow| $ Очевидно.

                  \item $ |\Leftarrow| $ Пусть множество $\{A_{nk}\}$ -- ограничено.

                        Пусть $A = \sup\{A_{nk}\}$. Покажем, что $A$ -- сумма ряда \ref{eq:6.6.3}.

                        Пусть $\epsilon > 0$ задано. Выберем $N_0$ и $K_0$:
                        \[
                            \begin{array}{c}
                                A - A_{N_0 K_0} < \epsilon \\
                                (\text{по опр. }\sup)
                            \end{array}
                        \]

                        Тогда $ \forall n > N_0$ и $ \forall k > K_0 \ A_{nk} \geqslant A_{N_0K_0} \implies 0 < A - A_{nk} \leqslant A - A_{N_0K_0} < \epsilon \implies |A - A_{nk}| < \epsilon $.
              \end{itemize}

              $ \implies  $ ряд \ref{eq:6.6.3} сходится.
    \end{enumerate}
\end{proof}

\subsection{Теорема о связи сходимости двойного и повторного рядов}

\begin{theorem}[О связи сходимости двойного ряда и повторного]
    Если
    \begin{itemize}
        \item ряд \ref{eq:6.6.3} сходится (двойной),
        \item все ряды по строкам сходятся,
    \end{itemize}
    тогда повторный ряд $ \sum_{n=1}^{\infty}\sum_{k=1}^{\infty}a_{nk} $ сходится и
    \[
        A = \sum_{n=1}^{\infty}\sum_{k=1}^{\infty}a_{nk} = \sum_{n,k=1}^{\infty}a_{nk}.
    \]
\end{theorem}

\begin{proof}
    Пусть $ \epsilon>0 $ задано. Выберем $ N_0,K_0: \ \forall n > N_0 $ и $ k>K_0 $
    \begin{equation}\label{eq:for_proof1}
        \left|\sum_{i=1}^{n}\sum_{j=1}^{k}a_{ij} - A\right| < \frac{\epsilon}{2}.
    \end{equation}
    \[
        \sum_{i,j=1}^{n,k}a_{ij} = A_{nk}\text{ двойного ряда}.
    \]

    В неравенстве \ref{eq:for_proof1} переходим к пределу при $ k \rightarrow\infty $.

    Тогда $ \forall n > N_0 $
    \[
        \left|\sum_{i=1}^{n}\sum_{j=1}^{\infty}a_{ij} - A\right| = \left|\sum_{i=1}^{n}A_n - A\right| < \frac{\epsilon}{2} < \epsilon
    \]
    $ \implies $ повторный ряд $ \sum_{n=1}^{\infty}\sum_{k=1}^{\infty}a_{nk} = A $.
\end{proof}

\subsection{Теорема о связи сходимости двойного и простого рядов}

\begin{theorem}[О связи сходимости двойного и простого рядов]
    Если ряд \ref{eq:6.6.3}$ ^* $ сходится, то сходится ряд \ref{eq:6.6.4}.

    И наоборот, если сходится ряд \ref{eq:6.6.4}$ ^* $, то сходится ряд \ref{eq:6.6.3}.

    И в обоих случаях суммы рядов равны:
    \[
        \sum_{n,k=1}^{\infty}a_{nk} = \sum_{r=1}^{\infty}u_r
    \]
\end{theorem}

\begin{proof}\leavevmode
    \begin{itemize}
        \item $ |\Rightarrow| $ Пусть двойной ряд сходится абсолютно, то есть сходится ряд $\sum_{n,k=1}^{\infty}|a_{nk}|$.

              Тогда для любого номера $S \ \exists N,K$ такие, что все числа $u_1,\ldots,u_S$ содержатся в первых $N$ строках и первых $K$ столбцах, тогда:
              \[
                |u_1| + |u_2| + \ldots + |u_S| \leqslant \sum_{n=1}^{N}\sum_{k=1}^{K}|a_{nk}| \leqslant A^* = \sum_{n,k=1}^{\infty}|a_{nk}| \implies
              \]
              $\implies$ последовательность $U_i^* \nearrow$ и ограничена $\implies$ ряд $\sum_{r=1}^{\infty}u_r$ сходится абсолютно $\implies$ сходится.

        \item $ |\Leftarrow| $ Пусть ряд $\sum_{r=1}^{\infty}|u_r|$ сходится $\implies \forall N,K \ \exists S:$ все числа $a_{11},a_{12},\ldots,a_{1K},a_{21},\ldots,a_{2K},\ldots,a_{N1},\ldots,a_{NK}$ содержатся среди чисел $u_1,\ldots,u_S$. Тогда
              \[
                A_{NK}^* = \sum_{n=1}^{N}\sum_{k=1}^{K}|a_{nk}| \leqslant\sum_{r=1}^{S}|u_r| \leqslant U^* = \sum_{r=1}^{\infty}|u_r| \implies
              \]
              $\implies$ ряд $\sum_{n,k=1}^{\infty}a_{nk}$ сходится.

              Покажем, что $\sum_{n,k=1}^{\infty}a_{nk} = \sum_{r=1}^{\infty}u_r$.

              Так как ряд $\sum_{r=1}^{\infty}u_r$ сходится абсолютно, то расположим элементы по квадратам:
              \[
                \begin{array}{l}
                    a_{11} = u_{r_1}                                       \\
                    a_{12} + a_{22} + a_{21} = u_{r_2} + u_{r_3} + u_{r_4} \\
                    \vdots                                                 \\
                    A_{nn} = a_{11} + \ldots + a_{nn} = U_n = u_{r_1} + \ldots + u_{r_n}
                \end{array},
              \]
              \[
                A = \underset{n\rightarrow\infty}{\lim}A_{nn} = \underset{n\rightarrow\infty}{\lim}U_n = U.
              \]
    \end{itemize}
\end{proof}