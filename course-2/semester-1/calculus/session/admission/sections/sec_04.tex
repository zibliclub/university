\section{Теория рядов}

\setcounter{subsection}{28}

\subsection{Числовой ряд}

\begin{definition}
    \emph{Рядом} называется выражение:
    \[
        a_1 + a_2 + \ldots + a_n + \ldots, \quad a_i \in \R.
    \]

    \begin{equation}\label{eq:6.1}
        \sum_{n=1}^{\infty}a_n
    \end{equation}

    Рассмотрим числа:
    \begin{align*}
         & A_1 = a_1,                      \\
         & A_2 = a_1 + a_2,                \\
         & \vdots                          \\
         & A_n = a_1 + a_2 + \ldots + a_n.
    \end{align*}

    Числа $ A_1,A_2,\ldots,A_n $ называются \emph{частичными суммами ряда} \ref{eq:6.1}.
\end{definition}

\begin{remark}
    Числа $ a_i $ называются \emph{членами ряда}, $ a_n $ -- \emph{$ n $-ым членом ряда}.
\end{remark}

\newpage

\subsection{Сходимость числового ряда}

\begin{definition}
    Ряд \ref{eq:6.1} \emph{сходится}, если
    \[
        \exists \underset{n\rightarrow\infty}{\lim}A_n = A.
    \]

    Тогда сумма бесконечного ряда \ref{eq:6.1} полагается равной
    \[
        A = \sum_{n=1}^{\infty}a_n.
    \]
\end{definition}

\subsection{Критерий Коши сходимости числовых рядов}

\begin{theorem}
    Ряд \ref{eq:6.1} сходится $\iff \forall \epsilon > 0 \ \exists N \in \mathbb{N}: \ \forall n > N, \ \forall p > 0$
    \[
        |a_{n+1} + \ldots + a_{n+p}| < \epsilon.
    \]
\end{theorem}

\subsection{Необходимое условие сходимости числового ряда}

\begin{theorem}
    Ряд \ref{eq:6.1} сходится $ \implies $
    \[
        \underset{n\rightarrow\infty}{\lim}a_n = 0.
    \]
\end{theorem}

\subsection{Теорема об остатке ряда}

\begin{definition}
    Пусть дан ряд \ref{eq:6.1}. Ряд вида
    \begin{equation}\label{eq:6.3}
        \sum_{n=m+1}^{\infty}a_n
    \end{equation}
    называется \emph{$ m $-ым остатным ряда} \ref{eq:6.1}.
\end{definition}

\begin{theorem}
    Следующие условия эквивалентны:
    \begin{enumerate}
        \item Ряд \ref{eq:6.1} сходится.
        \item Любой его остаток сходится.
        \item Некоторый его остаток \ref{eq:6.3} сходится.
    \end{enumerate}
\end{theorem}

\newpage

\subsection{Теорема о сумме рядов и умножении ряда на число}

\begin{theorem}
    Ряды $ (A),(B) $ сходятся $ \implies $
    \begin{enumerate}
        \item $ \forall \alpha \in \R \ \sum_{n=1}^{\infty}\alpha a_n $ сходится и его сумма равна $ \alpha \cdot A, \ A = \sum_{n=1}^{\infty}a_n $.

        \item Ряд $ (A+B) $ сходится и его сумма равна $ A^* + B^*, \ A^* = \sum_{n=1}^{\infty}a_n,$ $ B^* = \sum_{n=1}^{\infty}b_n $.
    \end{enumerate}
\end{theorem}

\subsection{Положительный числовой ряд}

\begin{definition}
    Ряд $ (A) $ называется \emph{положительным}, если $ \forall n \ a_n>0 $.
\end{definition}

\subsection{Основная теорема о сходимости положительных рядов}

\begin{theorem}\label{theorem:2.1}
    Положительный ряд $ (A) $ сходится $ \iff \exists M > 0: \ \forall n \ A_n < M $.
\end{theorem}

\subsection{Первый признак сравнения}

\begin{theorem}\label{theorem:6.2}
    Даны $ (A),(B), \ \forall n \ a_n, b_n > 0, \ \exists N \in \mathbb{N}: \ \forall n > N \ a_n \leqslant b_n \implies $
    \begin{enumerate}
        \item Из сходимости $(B) \implies$ сходимость $(A)$.
        \item Из расходимости $(A) \implies$ расходимость $(B)$.
    \end{enumerate}
\end{theorem}

\subsection{Второй признак сравнения}

\begin{theorem}
    Даны $ (A),(B), \ \forall n \ a_n, b_n > 0, \ \underset{n\rightarrow\infty}{\lim}\frac{a_n}{b_n}=k \in [0;\infty] \implies $
    \begin{enumerate}
        \item При $k=\infty$ из сходимости $(A) \implies$ сходимость $(B)$.
        \item При $k=0$ из сходимости $(B) \implies$ сходимость $(A)$.
        \item Иначе $(A)$ и $(B)$ ведут себя одинаково.
    \end{enumerate}
\end{theorem}

\subsection{Третий признак сравнения}

\begin{theorem}
    Даны $ (A),(B), \ \forall n \ a_n, b_n > 0, \ \exists N \in \N \cup \{0\}: \forall n > N \ \frac{a_{n+1}}{a_n}\leqslant\frac{b_{n+1}}{b_n} \implies$
    \begin{enumerate}
        \item Из сходимости $(B) \implies$ сходимость $(A)$.
        \item Из расходимости $(A) \implies$ расходимость $(B)$.
    \end{enumerate}
\end{theorem}

\subsection{Интегральный признак сходимости Коши-Маклорена}

\begin{theorem}
    Дан положительный ряд $ (A), \ f(x)$ удовлетворяет условиям:
    \begin{enumerate}
        \item $f(x): \ [1;+\infty) \rightarrow\R$.
        \item $f(x)$ -- непрерывна.
        \item $f(x)$ -- монотонна.
        \item $f(x) = a_n, \ \forall n \in \N$.
    \end{enumerate}

    Тогда $(A)$ и $\int_{1}^{\infty}f(x)dx$ ведут себя одинаково.
\end{theorem}

\subsection{Радикальный признак Коши}

\begin{theorem}
    Дан положительный ряд $(A), \ \underset{n\rightarrow\infty}{\overline{\lim}}\sqrt[n]{a_n} = q\implies $
    \begin{enumerate}
        \item $q < 1: (A)$ сходится.
        \item $q > 1: (A)$ расходится.
        \item $q = 1: (A)$ может как сходиться, так и расходиться.
    \end{enumerate}
\end{theorem}

\subsection{Признак Даламбера}

\begin{theorem}
    Дан положительный ряд $(A), \ \underset{n\rightarrow\infty}{\lim}\frac{a_{n+1}}{a_n} = d \implies $
    \begin{enumerate}
        \item $d < 1:(A)$ сходится.
        \item $d > 1:(A)$ расходится.
        \item $d = 1:(A)$ может как сходиться, так и расходиться.
    \end{enumerate}
\end{theorem}

\subsection{Признак Раббе}

\begin{theorem}
    Дан положительный ряд $(A), \ \underset{n\rightarrow\infty}{\lim}n \cdot \left(\frac{a_n}{a_{n+1}} - 1\right) = r \implies $
    \begin{enumerate}
        \item $r>1: (A)$ сходится.
        \item $r<1: (A)$ расходится.
        \item $r=1: (A)$ может как сходиться, так и расходиться.
    \end{enumerate}
\end{theorem}

\newpage

\subsection{Признак Кумера}

\begin{theorem}
    Дан положительный ряд $(A), \ c_1,c_2,\ldots,c_n,\ldots: \ \forall n > N \ c_n > 0$ и ряд $\sum_{n=1}^{\infty}c_n$ расходится. Если
    \[
        \underset{n\rightarrow\infty}{\lim}\left(c_n \cdot \frac{a_n}{a_{n+1}} - c_{n+1}\right) = k,
    \]
    то
    \begin{enumerate}
        \item $k > 0: (A)$ сходится.
        \item $k < 0: (A)$ расходится.
        \item $k = 0: (A)$ может как сходиться, так и расходиться.
    \end{enumerate}
\end{theorem}

\subsection{Признак Бертрана}

\begin{theorem}
    Дан положительный ряд $(A)$. Если
    \[
        \underset{n\rightarrow\infty}{\lim} \ln n \cdot \left[n \cdot \left(\frac{a_n}{a_{n+1}} - 1\right)\right] = B,
    \]
    то
    \begin{enumerate}
        \item $B > 1: (A)$ сходится.
        \item $B < 1: (A)$ расходится.
        \item $B = 1: (A)$ может как сходиться, так и расходиться.
    \end{enumerate}
\end{theorem}

\subsection{Признак Гаусса}

\begin{theorem}
    Ряд $(A), \ a_n > 0, \ \forall n \in \N, \ \lambda, \mu \in \R$. Если
    \[
        \frac{a_n}{a_{n+1}} = \left(\lambda + \frac{\mu}{n}\right) + O\left(\frac{1}{n^2}\right),
    \]
    то
    \begin{enumerate}
        \item $\lambda > 1: (A)$ сходится.
        \item $\lambda < 1: (A)$ расходится.
        \item $\lambda = 1$ и \begin{enumerate}
                  \item $\mu > 1 \implies (A)$ сходится.
                  \item $\mu \leqslant 1 \implies (A)$ расходится.
              \end{enumerate}
    \end{enumerate}
\end{theorem}

\newpage

\subsection{Знакопеременные ряды}

\begin{note}
    Дан ряд $(A)$. Если $\exists N: \ \forall n > N \ a_n$ не меняет знак, то исследование сходимости такого ряда сводится к исследованию сходимости положительных рядов. Будем считать, что «$+$» и «$-$» бесконечно много. Такие ряды будем называть \emph{знакопеременными}.
\end{note}

\subsection{Абсолютно сходящийся ряд}

\begin{definition}
    Ряд $(A)$ называется \emph{абсолютно сходящимся}, если сходится ряд
    \[
        (A^*) \ \sum_{n=1}^{\infty}|a_n|.
    \]
\end{definition}

\subsection{Условно сходящийся ряд}

\begin{definition}
    Если ряд $(A)$ сходится, а ряд $(A^*)$ расходится, то ряд $(A)$ называется \emph{условно сходящимся}.
\end{definition}

\subsection{Следствие абсолютной сходимости ряда}

\begin{statement}
    Если ряд $(A)$ абсолютно сходящийся, то он сходящийся.
\end{statement}

\subsection{Знакочередующиеся ряды}

\begin{definition}
    Ряд $(A)$ называется \emph{знакочередующимся}, если $\forall n \in \N $ \\ $ a_n \cdot a_{n+1} < 0$. Обозначим знакочередующийся ряд:
    \[
        (\overline{A}) \ \sum_{n=1}^{\infty}(-1)^{n-1}\cdot a_n, \quad a_n > 0 \ \forall n \in\N.
    \]
\end{definition}

\subsection{Признак Лейбница}

\begin{theorem}
    Пусть ряд $(\overline{A}), \forall n \ a_n > 0 $ удовлетворяет условиям:
    \begin{enumerate}
        \item $a_1 \geqslant a_2 \geqslant a_3 \geqslant \ldots \geqslant a_n \geqslant \ldots$.
        \item $\underset{n\rightarrow\infty}{\lim} a_n = 0$.
    \end{enumerate}

    Тогда ряд $(\overline{A})$ сходится и его сумма $S: \ 0 < S \leqslant a_1$.
\end{theorem}

\newpage

\subsection{Признак Абеля}

\begin{theorem}
    Если \begin{itemize}
        \item последовательность $\{a_n\}$ монотонна и ограничена,
        \item ряд $\sum_{n=1}^{\infty} b_n$ сходится,
    \end{itemize}
    то ряд $\sum_{n=1}^{\infty}a_n \cdot b_n$ сходится.
\end{theorem}

\subsection{Признак Дирихле}

\begin{theorem}
    Если \begin{itemize}
        \item последовательность $\{a_n\}$ монотонна и $\underset{n\rightarrow\infty}{\lim}a_n = 0$,
        \item частичные суммы ряда $(B)$ ограничены, то есть $\exists k > 0: $ \\ $ \forall n \ \left|\sum_{m=1}^{n} b_m\right|< k$,
    \end{itemize}
    то $\sum_{n=1}^{\infty}a_n \cdot b_n$ сходится.
\end{theorem}

\subsection{Сочетательное свойство сходящихся рядов}

\begin{theorem}\leavevmode
    \begin{enumerate}
        \item Если ряд $(A)$ сходится, то для любой возрастающей последовательности $n_k$ ряд $(\widetilde{A})$ сходится и их суммы совпадают ($A = \widetilde{A}$).
        \item Если ряд $(\widetilde{A})$ сходится и внутри каждой  скобки знак не меняется, то ряд $(A)$ сходится и их суммы совпадают, то есть $\widetilde{A} = A$.
    \end{enumerate}
\end{theorem}

\subsection{Переместительное свойство сходящихся рядов}

\begin{theorem}
    Если ряд $(A)$ абсолютно сходится, то его сумма не зависит от перестановки членов ряда.
\end{theorem}

\subsection{Теорема Римана о перестановке членов условно сходящегося ряда}

\begin{theorem}
    Если ряд $(A)$ условно сходится, то $\forall B \in \R$ (в том числе $B = \pm\infty$) $\exists$ перестановка ряда $(A)$ такая, что полученный ряд сходится и имеет сумму $B$. Более того, $\exists$ перестановка ряда $(A)$ такая, что частичные суммы полученного ряда не стремятся ни к конечному, ни к бесконечному пределу.
\end{theorem}

\newpage

\subsection{Произведение рядов}

\begin{definition}
    \emph{Произведением рядов} $(A)$ и $(B)$ назовем ряд, членами которого ялвяются элементы на строке таблицы $a_ib_j$, взятые в произвольном порядке.

    Если числа выбираются по диагоналям, то произведение называется \emph{формой Коши}:
    \[
        a_1 b_1 + (a_1 b_2 + a_2 b_1) + \ldots
    \]
\end{definition}

\subsection{Теорема Коши о произведении рядов}

\begin{theorem}
    Ряды $ (A),(B) $ абсолютно сходятся, $A, B$ -- их суммы $\implies \forall$ их произведение абсолютно сходится и равно $A \cdot B$.
\end{theorem}

\subsection{Повторный ряд}

\begin{definition}
    \emph{Повторным рядом} называются выражения
    \begin{equation}\label{eq:6.6.1}
        \sum_{n=1}^{\infty}\sum_{k=1}^{\infty}a_{nk},
    \end{equation}
    \begin{center}
        и
    \end{center}
    \begin{equation}\label{eq:6.6.2}
        \sum_{k=1}^{\infty}\sum_{n=1}^{\infty}a_{nk}.
    \end{equation}

    Ряд \ref{eq:6.6.1} сходится, если сходятся все ряды $(A_n)$ по строкам $(\sum_{k=1}^{\infty}a_{n_k} = A_n)$ и сходится ряд $ \sum_{n=1}^{\infty}A_n $.
\end{definition}

\subsection{Двойной ряд}

\begin{definition}
    \emph{Двойным рядом} называется выражение:
    \begin{equation}\label{eq:6.6.3}
        \sum_{n,k = 1}^{\infty} a_{nk}
    \end{equation}

    Ряд \ref{eq:6.6.3} сходится, если:
    \[
        \exists A = \underset{N\rightarrow\infty}{\underset{K\rightarrow\infty}{\lim}}A_{NK} = \underset{N\rightarrow\infty}{\underset{K\rightarrow\infty}{\lim}}\sum_{n=1}^{N}\sum_{k=1}^{K}a_{nk}.
    \]

    То есть $\forall \epsilon > 0 \ \exists N_0$ и $K_0: \ \forall N > N_0$ и $\forall k > K_0$
    \[
        \bigg|\underbrace{\sum_{n=1}^{N}\sum_{k=1}^{K}a_{nk}}_{A_{NK}} - A\bigg| < \epsilon.
    \]
\end{definition}

\subsection{Простой ряд}

\begin{definition}
    Пусть ряд
    \begin{equation}\label{eq:6.6.4}
        \sum_{r=1}^{\infty}U_r
    \end{equation}
    построен из элементов таблицы, взятых в произвольном порядке. Такой ряд будем называть \emph{простым}, связанным с данной таблицей.
\end{definition}

\subsection{Теорема о связи сходимости простого и повторного рядов}

\begin{theorem}\leavevmode
    \begin{enumerate}
        \item Ряд $ \sum_{r=1}^{\infty}U_r $ абсолютно сходится $\implies \sum_{n=1}^{\infty}\sum_{k=1}^{\infty}a_{nk} $ сходится и его сумма равна $U$.

        \item Если после замены элементов таблицы $(\star)$ их модулями ряд \\ $ \sum_{n=1}^{\infty}\sum_{k=1}^{\infty}|a_{nk}| $ cходится, то ряд $ \sum_{r=1}^{\infty}U_r $ сходится абсолютно и суммы рядов $ \sum_{n=1}^{\infty}\sum_{k=1}^{\infty}a_{nk} $ и $ \sum_{r=1}^{\infty}U_r $ совпадают.
    \end{enumerate}
\end{theorem}

\subsection{Свойства двойного ряда}

\begin{theorem}\leavevmode
    \begin{enumerate}
        \item Если ряд $ \sum_{n,k = 1}^{\infty} a_{nk} $ сходится, то
              \[
                  \underset{k\rightarrow\infty}{\underset{n\rightarrow\infty}{\lim}}a_{nk} = 0.
              \]

        \item (Критерий Коши)
              Ряд $ \sum_{n,k = 1}^{\infty} a_{nk} $ сходится $\iff \forall \epsilon > 0 \ \exists N_0,K_0: \ \forall n > N_0, \ \forall k > K_0, \ \forall p > 0, \ \forall q > 0$
              \[
                  \bigg|\sum_{n=1}^{p}\sum_{k=1}^{q}a_{(N_0 + n)(K_0 + k)}\bigg| < \epsilon.
              \]

        \item Если ряд $ \sum_{n,k = 1}^{\infty} a_{nk} $ сходится, то $\forall c \in \R$ ряд
              \[
                  \sum_{n,k=1}^{\infty}(c\cdot a_{nk})
              \]
              сходится, и его сумма равна $c\cdot A$ (где $A = \sum_{n,k=1}^{\infty}a_{nk}$).

        \item Если ряд $ \sum_{n,k = 1}^{\infty} a_{nk} $ сходится и ряд
              \[
                  \sum_{n,k=1}^{\infty}b_{nk}
              \]
              сходится, то
              \[
                  \sum_{n,k=1}^{\infty}(a_{nk} + b_{nk}) = A + B,
              \]
              а к тому же -- сходится.

        \item Если $\forall n, \ \forall k \ a_{nk} \geqslant 0$, то ряд $ \sum_{n,k = 1}^{\infty} a_{nk} $ сходится $ \iff $ его частичные суммы ограничены в совокупности.
    \end{enumerate}
\end{theorem}

\subsection{Теорема о связи сходимости двойного и повторного рядов}

\begin{theorem}
    Если
    \begin{itemize}
        \item ряд $ \sum_{n,k = 1}^{\infty} a_{nk} $ сходится (двойной),
        \item все ряды по строкам сходятся,
    \end{itemize}
    тогда повторный ряд $ \sum_{n=1}^{\infty}\sum_{k=1}^{\infty}a_{nk} $ сходится и
    \[
        A = \sum_{n=1}^{\infty}\sum_{k=1}^{\infty}a_{nk} = \sum_{n,k=1}^{\infty}a_{nk}.
    \]
\end{theorem}

\subsection{Теорема о связи сходимости двойного и простого рядов}

\begin{theorem}
    Ряд $ \sum_{n,k = 1}^{\infty} |a_{nk}| $ сходится $ \implies $ сходится $ \sum_{r=1}^{\infty}U_r $.

    И наоборот, если сходится $ \sum_{r=1}^{\infty}|U_r| \implies $ сходится $ \sum_{n,k = 1}^{\infty} a_{nk} $.

    В обоих случаях суммы рядов равны:
    \[
        \sum_{n,k=1}^{\infty}a_{nk} = \sum_{r=1}^{\infty}U_r
    \]
\end{theorem}
