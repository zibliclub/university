\section{Основные теоремы дифференциального исчисления функций многих переменных}

\setcounter{subsection}{4}

\subsection{Теорема о среднем (аналог теоремы Лагранжа)}

\begin{theorem}
    $ D \subset \R^n, \ [x;x+h]\subset D, \ f: D \rightarrow\R $ диф-ма на $ (x;x+h) $ и непрерывна на $ [x;x+h] \implies \exists \xi \in (x;x+h): $
    \[
        f(x+h)-f(x) = f'(\xi)\cdot h = \frac{\partial f}{\partial x_1}(\xi)\cdot h^1 + \frac{\partial f}{\partial x_2}(\xi)\cdot h^2 + \ldots + \frac{\partial f}{\partial x_n}(\xi)\cdot h^n.
    \]
\end{theorem}

\begin{note}
    $ \{1,2,\ldots,n\} $ над $ h $ -- индексы.
\end{note}

\subsection{Следствие теоремы о среднем}

\begin{corollary}
    $ D \subset \R^n, \ f:D \rightarrow \R $ -- диф-ма на $ D $ и $ \forall x \in D \ df(x) = 0 \implies f(x) = const $.
\end{corollary}

\subsection{Достаточное условие дифференцируемости функции}

\begin{theorem}
    $ D \subset \R^n, \ f:D \rightarrow\R $ имеет непрерывные частные производные в каждой окрестности $ x \in D \implies f $ диф-ма в $ x $.
\end{theorem}

\subsection{Производные высших порядков}

\begin{definition}
    $ D \subset \R^n, \ f:D \rightarrow\R $. Производная по $ x^j $ от производной по $ x^i $ называется \emph{второй производной функции $ f $ по} $ x^i,x^j $ и обозначается
    \[
        \frac{\partial^2f}{\partial x^i\partial x^j}(x)\text{ или }f_{x^i,x^j}''(x).
    \]
\end{definition}

\subsection{Теорема о смешанных производных}

\begin{theorem}
    $ D \subset \R^n, \ f: D \rightarrow\R $ имеет в $ D $ непрерывные смешанные производные (второго порядка) $ \implies $ производные не зависят от порядка диф-ния.
\end{theorem}

\subsection{Формула Тейлора}

\begin{theorem}
    $ D \subset \R^n, \ f:D\rightarrow\R, \ f\in C^{(k)}(D,\R), \ [x;x+h] \subset D \implies $
    \[
        f(x + h) = f(x) + \sum_{i=1}^{k-1}\frac{1}{i!}\left(\frac{\partial f}{\partial x^1}\cdot h^1 + \ldots + \frac{\partial f}{\partial x^n}\cdot h^n\right)^i \cdot f(x) + R^k.
    \]
\end{theorem}

\begin{note}
    $R^k$ -- остаточный член,
    \[
        R^k = \frac{1}{k!}\left(\frac{\partial f}{\partial x^1}\cdot h^1 + \ldots + \frac{\partial f}{\partial x^n}\cdot h^n\right)^k \cdot f(x + \xi \cdot h),
    \]
    \[
        x = (x^1,\ldots,x^n), \quad h = (h^1,\ldots,h^n).
    \]
\end{note}

\subsection{Локальный экстремум функции многих переменных}

\begin{definition}
    $ X $ -- МП, $ f:X \rightarrow\R, \ x_0 $ называется \emph{точкой локального максимума (минимума)}, если $ \exists U(x_0) \subset X: \ \forall x \in U(x_0) $
    \[
        f(x)\leqslant f(x_0) \quad \big(f(x) \geqslant f(x_0)\big)
    \]
\end{definition}

\subsection{Необходимое условие локального экстремума}

\begin{theorem}
    $ D \subset \R^n, \ f:D \rightarrow\R, \ x_0 \in D $ точка локального экстремума $ \implies $ в $ x_0 \ \forall i = \overline{1,n}$
    \[
        \frac{\partial f}{\partial x^i}(x_0) = 0.
    \]
\end{theorem}

\subsection{Критическая точка функции}

\begin{definition}
    $ D \subset \R^n, \ f:D \rightarrow\R^k $ диф-ма в $ x_0 \in D, \ x_0 $ называется \emph{критической точкой функции} $ f(x) $, если:
    \[
        rank \mathfrak{I} f(x_0) < \min(n,k).
    \]
\end{definition}

\begin{note}
    $ \mathfrak{I}f(x_0) $ -- матрица Якоби функции $ f(x_0) $.
\end{note}

\newpage

\subsection{Достаточное условие локального экстремума}

\begin{theorem}
    $ D \subset \R^n, \ f: D \rightarrow \R, \ f \in C^2(D,\R), \ f$ диф-ма в $x \in D $ -- критической точке для $ f $. Тогда, если:
    \begin{enumerate}
        \item $Q(h)$ -- знакоположительна $\implies x$ -- лок. $ \min $.
        \item $Q(h)$ -- знакоотрицательна $\implies x$ -- лок. $ \max $.
        \item $Q(h)$ может принимать различные значения ($>0, < 0$) $\implies $ в $ x$ нет экстремума.
    \end{enumerate}
\end{theorem}

\subsection{Неявная функция}

\begin{definition}
    $ D,\Omega \subset \R^k, \ F: D \times \Omega \rightarrow \mathbb{R}^k, \ f:D \rightarrow\Omega: $
    \[
        y = f(x) \iff F(x,y) = 0.
    \]

    Уравнение $ F(x,y) = 0 $ \emph{неявно задает} $ y $.
\end{definition}

\subsection{Теорема о неявной функции}

\begin{theorem}\label{theorem:1}
    $ U(x_0,y_0) \subset \R^2, \ F(x,y):U(x_0,y_0)\rightarrow\R $. Пусть $ F $ имеет следующие свойства:
    \begin{enumerate}
        \item $ F(x_0,y_0) = 0 $.
        \item $ F(x,y) \in C^P(U,\R), \ p \geqslant 1 $.
        \item $ \frac{\partial F}{\partial y}(x_0,y_0)\ne 0 $.
    \end{enumerate}

    Тогда $ \exists $ открезки $ I_x,I_y: \ f:I_x \rightarrow I_y $:
    \begin{enumerate}
        \item $ I_x \times I_y \subset U(x_0,y_0) $.
        \item $ \forall x \in I_x \ y = f(x) \iff F(x,y) = 0 $.
        \item $ f \in C^P(I_x,I_y) $.
        \item $ \forall x \in I_x \ f'(x) = -\frac{F_x'(x,y)}{F_y'(x,y)} $.
    \end{enumerate}
\end{theorem}

\newpage