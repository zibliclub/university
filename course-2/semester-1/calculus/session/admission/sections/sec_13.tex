\section{Определение и свойства кратного интеграла Римана}

\setcounter{subsection}{137}

\subsection{Разбиение множества}

\begin{definition}
    Пусть множество $ G \subset \R^n $ измеримо по Жордану.

    Совокупность измеримых по Жордану множеств $ G_i \subset \R^n, \ i = \overline{1,N} $, попарно пересекающихся $ G = \overset{n}{\underset{i=1}{\bigcup}}G_i $ называются \emph{разбиением} множества $ G $.
    \[
        \text{Обозначение:}\quad T = \{G_i\}
    \]
\end{definition}

\subsection{Мелкость разбиения}

\begin{definition}
    Число $ l(T) = \max d(G_i) $ называется \emph{мелкостью разбиения} $ T $.
\end{definition}

\subsection{Продолжение разбиения}

Не нашел.

\subsection{Интегральная сумма Римана}

\begin{definition}
    Пусть функция $ f(x) = f(x_1,\ldots,x_n): \ G \rightarrow \R $ определена на измеримом по Жордану множестве $ G \subset \R^n $, $T = \{G_{ij}\}$ -- разбиение множества $ G $.

    Возьмем $ \xi_i \in G_i, \ i = \overline{1,N} $.

    Выражение
    \[
        \sigma_T = \sigma_T(f,\xi,G) = \sum_{i=1}^{N}f(\xi_i)m(G_i)
    \]
    называется \emph{интегральной суммой Римана} от функции \\ $ f(x) = f(x_1,\ldots,x_n) $ на множестве, соответствующей разбиению $ T $ и выборке $ \xi = (\xi_1,\ldots,\xi_N) $.
\end{definition}

\subsection{Суммы Дарбу}

Не нашел.

\newpage

\subsection{Кратный интеграл Римана}

\begin{note}
    Число $ I $ будем называть \emph{кратным интегралом Римана} от функции $ f(x) $ по множеству $ G $, а функцию $ f(x) $ -- \emph{интегрируемой} на множестве $ G $.
    \[
        \begin{array}{ccccc}
                                      &          & \begin{array}{cc}
                                                       \text{Обозн.} \\
                                                       \text{инт. Римана}
                                                   \end{array} &          &                                                                                                    \\
                                      & \swarrow &                    & \searrow &                                                                                             \\
            \text{кратное}            &          &                    &          & \text{развернутое}                                                                          \\
            \underset{G}{\int} f(x)dx &          &                    &          & \underbrace{\underset{G}{\int\ldots\int}}_{n \text{ раз}}f(x_1,\ldots,x_n)dx_1,\ldots,d x_n
        \end{array}
    \]

    При $ n=2 $ кратный интеграл Римана называется \emph{двойным} и обозначается
    \[
        \iint\limits_G f(x,y)dxdy.
    \]

    При $ n=3 $ -- \emph{тройным} и обозначается
    \[
        \iiint\limits_G f(x,y,z)dxdydz.
    \]
\end{note}

\subsection{Критерий интегрируемости (без доказательства)}

\begin{theorem}
    Ограниченная формула $ f(x) $ интегрируема на измеримом по Нордану множестве $ G \subset \R^n \iff \forall \epsilon > 0 \ \exists\delta > 0: \ \forall T \ l(T)<\delta$
    \[
        \overline{S_T} - S_T < \epsilon
    \]
    \begin{center}
        (то есть $ \overline{S_T} - S_T \rightarrow 0 $ при $ l(T) \rightarrow 0 $)
    \end{center}
\end{theorem}

\subsection{Критерий интегрируемости более сильный (без доказательства)}

\begin{theorem}
    Ограниченная функция $ f(x) $ интегрируема на измеримом по Жордану множестве $ G \subset \R^n \iff \forall \epsilon > 0 \ \exists T $ множества $ G $:
    \[
        \overline{S_T} - S_T < \epsilon.
    \]
\end{theorem}

\newpage

\subsection{Классы интегрируемых функций (2 теоремы, без доказательств)}

\begin{theorem}
    Непрерывная на измеримом по Жордану компактном множестве $ G $, функция $ f(x) $ интегрируема на этом множестве.
\end{theorem}

\begin{theorem}
    Пусть функция $ f(x) $ ограничена на измеримом компакте $ G \subset \R^n $ и множество разрыва $ f(x) $ имеет Жорданову меру нуль.

    Тогда $ f(x) $ интегрируема на $ G $.
\end{theorem}

\subsection{Свойства кратного интеграла (1-8, без доказательств)}

\begin{itemize}
    \item Свойство $ \circled{1} $
          \begin{statement}
              Справедливо равенство
              \[
                  \underset{G}{\int}1 dx = m(G).
              \]
          \end{statement}

    \item Свойство $ \circled{2} $
          \begin{statement}
              Если $ f(x) > 0 $ и $ f(x) $ -- интегрируемая на измеримом по Жордану множестве $ G $ функция, то
              \[
                  \underset{G}{\int}f(x)dx \geqslant 0.
              \]
          \end{statement}

    \item Свойство $ \circled{3} $
          \begin{statement}
              Если $ f_1(x) $ и $ f_2(x) = f_2(x_1,\ldots,x_n) $ -- интегрируемые на измеримом по Жордану множестве $ G $ функции, $ \alpha,\beta \in \R $, то и функция $ \alpha \cdot f_1(x) + \beta \cdot f_2(x) $ интегрируема на $ G $ и
              \[
                  \underset{G}{\int}\big(\alpha\cdot f_1(x) + \beta \cdot f_2(x)\big)dx = \alpha \underset{G}{\int}f_1(x)dx + \beta \underset{G}{\int}f_2(x)dx.
              \]
          \end{statement}

    \item Свойство $ \circled{4} $
          \begin{statement}
              Если $ f_1(x) $ и $ f_2(x) $ -- интегралы на измеримом по Жордану множестве $ G $ и $ \forall x \in G \ f_1(x) \leqslant f_2(x) $, то
              \[
                  \underset{G}{\int} f_1(x)dx \leqslant \underset{G}{\int}f_2(x)dx.
              \]
          \end{statement}

    \item Свойство $ \circled{5} $
          \begin{statement}
              Если функция $ f(x) $ непрерывна на измеримом связном компакте $ G $, то $ \exists \xi \in G $:
              \[
                  \underset{G}{\int}f(x)dx = f(\xi)m(G).
              \]
          \end{statement}

    \item Свойство $ \circled{6} $
          \begin{statement}
              Если $ G_k, \ k = \overline{1,m} $ если разбиение множества $ G $, то функция $ f(x) $ интегрируема на $ G \iff f(x) $ интегрируема на $ G_k, \ k = \overline{1,m} $, при этом
              \[
                  \underset{G}{\int}f(x)dx = \sum_{k=1}^{m}\underset{G_k}{\int}f(x)dx.
              \]
          \end{statement}

    \item Свойство $ \circled{7} $
          \begin{statement}
              Произведение интегрируемых на измеримом множестве $ G $ функцией является интегрируемой на $ G $ функцией.
          \end{statement}

    \item Свойство $ \circled{8} $
          \begin{statement}
              Если $ f(x) $ интегрируема на множестве $ G $ функция, то функция $ \big| f(x) \big| $ также интегрируема
              \[
                  \left|\underset{G}{\int}f(x)dx\right| \leqslant \underset{G}{\int}\big|f(x)\big|dx.
              \]
          \end{statement}
\end{itemize}

\newpage

\subsection{Теорема о сведении двойного интеграла по прямоугольнику к повторному интегралу}

\begin{theorem}
    Пусть
    \begin{enumerate}
        \item Функция $ f(x,y) $ интегрируема на прямоугольнике
              \[
                  \Pi = \big\{(x,y): \ a \leqslant x \leqslant b, \ c \leqslant y \leqslant d\big\}.
              \]

        \item $ \int_{c}^{d}f(x,y)dy \ \exists \ \forall x \in [a;b] $.
    \end{enumerate}

    Тогда функция $ F(x) = \int_{c}^{d}f(x,y)dy $ интегрируема на отрезке $ [a;b] $ и справедлива формула:
    \[
        \boxed{\iint\limits_\Pi f(x,y)dxdy = \int_{a}^{b}dx \int_{c}^{d}f(x)dy}
    \]
\end{theorem}

\subsection{Теорема о сведении двойного интеграла по элементарной области к повторному интегралу}

\begin{theorem}
    Пусть $ \Omega $ -- элементарная относительно оси $ Oy $ область, функция $ f(x,y) $ интегрируема на $ \overline{\Omega} = \Omega \cup G\Omega $ и $ \forall x \in [a;b] \ \exists \ \int f(x,y)dx $.

    Тогда справедлива следующая формула:
    \begin{equation}\label{eq:for_proof9}
        \iint\limits_\Omega f(x,y)dxdy = \int_{a}^{b}dx \int_{\phi(x)}^{\psi(x)}f(x,y)dy.
    \end{equation}
\end{theorem}

\subsection{Теорема Фубини (без доказательства)}

КТО ЭТО

\newpage

\subsection{Формула замены переменной в кратном интеграле (без доказательства)}

\begin{theorem}
    Пусть отображение $ F: \Omega \rightarrow \R^n $ ($ \Omega \subset \R^n $ -- открытое множество) является взаимнооднозначным и удовлетворяет условиям 1.-3.

    Пусть $ G $ -- измеримый компат: $ G \subset \Omega $. Тогда, если функция $ f(x) = f(x_1,\ldots,x_n) $ непрерывна на множестве $ G'=F(G) $, то справедлива следующая формула замены переменных в кратном интеграле:
    \begin{multline*}
        \int\limits_{G'}f(x)dx = \underset{G'}{\int\ldots\int}f(x_1,\ldots,x_n)dx_1dx_2\ldots dx_n = \\
        = \int\limits_G f\big(\phi_1(u),\ldots,\phi_n(u)\big)\big|\mathfrak{I}(u)\big|du,
    \end{multline*}
    \[
        u = (u_1,\ldots,u_n), \quad du = du_1du_2\ldots du_n.
    \]
\end{theorem}