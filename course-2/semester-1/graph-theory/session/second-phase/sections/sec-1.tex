\section{Определение графа. Примеры графов. Степени вершин графа. Лемма о рукопожатиях.}

\begin{definition}[Неориентированный граф, вершины и ребра графа]
    \emph{Неориентированный граф} -- пара множеств $ G = (V,E) $, где
    \[
        \begin{array}{l}
            V \text{ -- непустое конечное множество,} \\
            E \text{ -- множество, состоящее из неупорядоченных пар элементов из } V.
        \end{array}
    \]

    Элементы множества $ V $ называются \emph{вершинами}, а элементы $ E $ -- \emph{ребрами} графа.
\end{definition}

\begin{note}
    Если $ u,v \in V, \ \{u,v\}\in E $, то будем записывать
    \[
        e = uv \ (=vu)
    \]
    и говорить, что вершины $ u $ и $ v $ \emph{смежны}, вершина $ u $ и ребро $ e $ -- \emph{инцидентны}.
\end{note}

\begin{definition}[Степень вершины]
    \emph{Степенью вершины} $ v $ называется число инцидентных ей ребер.
    \[
        \text{\textbf{Обозначение:}} \quad d(v)\ \bigl(deg(v)\bigr)
    \]
\end{definition}

\begin{example}
    $ deg(v) = 3 $
    \begin{figure}[H]
        \centering
        \incfig{fig_01}
        \label{fig:fig_01}
    \end{figure}
\end{example}

\begin{example}
    \emph{Пустой граф} -- граф без ребер: $ O_n $.
\end{example}

\begin{example}
    \emph{Полный граф} -- граф, любая пара которого смежна: $ K_n $.
\end{example}

\begin{note}
    \[
        | E | = C_n^2 = \frac{n(n-1)}{2} \text{ -- число ребер}.
    \]
\end{note}

\begin{example}
    \emph{Двудольный граф} -- граф, вершины которого разбиты на 2 непересекающиеся части (доли) так, что любое ребро ведет из одной доли в другую.

    Если любая вершина одной доли смежна с любой вершиной другой доли, то такой граф называется \emph{полным двудольным}.

    Полный двудольный граф с долями размера $ p $ и $ q $ обозначают: $ K_{p,q} $,
    \[
        | E | = p\cdot q.
    \]
    \begin{figure}[H]
        \centering
        \incfig{fig_02}
        \label{fig:fig_02}
    \end{figure}
\end{example}

\begin{example}
    \emph{Звезда} -- полный двудольный граф $ K_{1,q} $: одна доля состоит из одной вершины, а из нее веером расходятся лучи.
    \begin{figure}[H]
        \centering
        \incfig{fig_03}
        \label{fig:fig_03}
    \end{figure}
\end{example}

\begin{example}
    Графы многогранников
    \begin{figure}[H]
        \centering
        \incfig{fig_04}
        \label{fig:fig_04}
    \end{figure}
\end{example}

\begin{lemma}[О рукопожатиях]
    Пусть $ G=(V,E) $ -- произвольный граф. Сумма степеней всех вершин графа $ G $ -- четное число, равное удвоенному количеству его ребер:
    \begin{equation}\label{eq:1}
        \sum_{v \in V}deg_G(v) = 2 | E |
    \end{equation}
\end{lemma}

\begin{proof}
    Индукция по числу ребер графа $ G $.
    \begin{enumerate}
        \item Если $ | E | = 0 $, то формула \ref{eq:1} верно.
        \item Предположим, что формула \ref{eq:1} верна для любого графа, в котором число ребер $ \leqslant m $, где $ m \geqslant 0 $.
        \item Пусть $ | E | = m+1 $. Выберем произвольное ребро $ e = uv $ и удалим его из графа $ G $. Получим граф $ G' = (V,E') $, где $ | E' | = m $.

              По предположению индукции для графа $ G' $ формула \ref{eq:1} верна:
              \[
                  \sum_{v \in V}deg_{G'}(v) = 2 | E' | = 2m.
              \]

              Вернем ребро $ e = uv $:
              \[
                  \sum_{v \in V}deg_G(v) = \sum_{v \in V}deg_{G'}(v) + 2 = 2m+2 = 2(m+1) = 2 | E | .
              \]
    \end{enumerate}
\end{proof}

\section{Маршруты, цепи, циклы. Лемма о выделении простой цепи. Лемма об объединении простых цепей.}

\begin{definition}[Маршрут]
    \emph{Маршрутом}, соединяющим вершины $ u $ и $ v $ ($ (u,v) $-маршрут), называется чередующаяся последовательность вершин и ребер вида
    \[
        (u = v_1,e_1,v_2,\ldots,v_k,e_k,v_{k+1}=v)
    \]
    такая, что $ e_i = v_iv_{i+1}, \ i = \overline{1,k} $.
\end{definition}

\begin{definition}[Замкнутый маршрут]
    Маршрут называется \emph{замкнутым}, если первая вершина совпадает с последней, то есть
    \[
        v_1=v_{k+1}.
    \]
\end{definition}

\newpage

\begin{definition}[Цепь, простая цепь]
    Маршрут называется \emph{цепью}, если в нем все ребра различны и \emph{простой цепью}, если в нем все вершины различны (за исключением, быть может, первой и последней).
    \begin{figure}[H]
        \centering
        \incfig{fig_05}
        \label{fig:fig_05}
    \end{figure}
\end{definition}

\begin{definition}[Цикл, простой цикл]
    Замкнутая цепь называется \emph{циклом}, а замкнутая простая цепь -- \emph{простым циклом}.
    \begin{figure}[H]
        \centering
        \incfig{fig_06}
        \label{fig:fig_06}
    \end{figure}
\end{definition}

\begin{lemma}[О выделении простой цепи]
    Всякий незамкнутый \\ $ (u,v) $-маршрут содержит простую $ (u,v) $-цепь.
\end{lemma}

\begin{proof}\leavevmode
    \begin{enumerate}
        \item Если все вершины $ (u,v) $-маршрута различны, то $ (u,v) $ -- простая цепь.
        \item Пусть $ v_i $ -- первая из вершин, имеющая в нем повторение, а $ v_j $ -- последнее повторение.
              \begin{figure}[H]
                  \centering
                  \incfig{fig_07}
                  \label{fig:fig_07}
              \end{figure}

              $ (v_1,v_2,\ldots,v_{i-1},v_i,v_{j+1},\ldots) $ -- заменим на более короткий, исключив цикл. Если в более коротком маршруте еще есть повторяющиеся вершины, то поступаем также.

              В конце концов получим незамкнутый $ (u,v) $-маршрут, в котором все вершины различны, то есть простую цепь.
    \end{enumerate}
\end{proof}

\begin{lemma}[Об объединении простых цепей]
    Объединение двух несовпадающих простых $ (u,v) $-цепей содержит простой цикл.
\end{lemma}

\begin{proof}
    Предположим, что $ P = (u_1,\ldots,u_{k+1}), \ Q = (v_1,\ldots,v_{l+1}) $ -- две несовпадающие простые цепи:
    \[
        u = u_1 = v_1, \quad v = u_{k+1} = v_{l+1},
    \]
    \begin{figure}[H]
        \centering
        \incfig{fig_08}
        \label{fig:fig_08}
    \end{figure}

    Предположим, что $ u_{r+1} $ и $ v_{r+1} $ -- первые несовпадающие вершины этих цепей, а $ u_s = v_t $ -- первые совпадающие за $ v_{r+1} $ и $ u_{r+1} $. Тогда
    \[
        \begin{array}{l}
            (u_r,u_s) \text{ -- фрагмент } P \\
            (v_r,v_s) \text{ -- фрагмент } Q
        \end{array} \text{ -- образуют простой цикл.}
    \]
\end{proof}

\section{Эйлеровы циклы. Критерий существования эйлерова цикла (теорема Эйлера).}

\begin{definition}[Эйлеров цикл]
    Пусть $ G = (V,E) $ -- произвольный граф (мультиграф). Цикл в графе $ G $ называется \emph{эйлеровым}, если он содержит все ребра графа.
\end{definition}

\begin{definition}[Эйлеров граф]
    Граф называется \emph{эйлеровым}, если в нем есть эйлеров цикл.
\end{definition}

\begin{theorem}[Эйлер, 1736]
    В связном графе $ G = (V,E) $ существует эйлеров цикл $ \iff $ все вершины графа $ G $ четны (то есть имеют четную степень).
\end{theorem}

\begin{proof}\leavevmode
    \begin{description}
        \item[$ \boxed{\Rightarrow} $] (необходимость)

              Пусть граф $ G $ -- эйлеров. Эйлеров цикл, проходя через каждую вершину графа, входит в нее по одному ребру и выходт по другому. Значит каждая вершина должна быть инцидентна четному числу ребер.

        \item[$ \boxed{\Leftarrow} $] (достаточность)

              Пусть $ G $ -- связен, все его вершины имеют четную степень.

              Рассмотрим следующий алгоритм и докажем, что он обязательно построит эйлеров цикл.

              \begin{note}[Алгоритм построения эйлерова цикла]
                  Рассмотрим произвольную вершину $ v_0 $ и построим из нее маршрут $ C_0 $.

                  Пройденные вершины запоминаем, а ребра удаляем. Действуем так до тех пор, пока не получим граф $ G_1 $, в котором нет ребер инцидентных очередной вершине маршрута $ C_0 $.
                  \begin{figure}[H]
                      \centering
                      \incfig{fig_09}
                      \label{fig:fig_09}
                  \end{figure}

                  Если $ C_0 $ содержит все ребра графа $ G $, то он и есть эйлеров цикл и все доказано.

                  В противном случае, в силу связности графа $ G $ в цикле $ C_0 $ найдется вершина $ v_1 $, инцидентная некоторому ребру графа $ G_1 $. Начинаем стоить из нее ($ v_1 $) цикл $ C_1 $ в графе $ G_1 $.
                  \begin{figure}[H]
                      \centering
                      \incfig{fig_10}
                      \label{fig:fig_10}
                  \end{figure}

                  Если все циклы $ C_0 $ и $ C_1 $ содержат все ребра графа $ G_1 $, то алгоритм завершает работу.

                  В противном случае, в одном из циклов $ C_0,C_1 $ найдется вершина $ v_2 $, инцидентная какому-то ребру графа $ G_2 $. Строим из нее цикл $ C_2 $ в графе $ G_2 $ и так далее.

                  В конце концов, получим, что после построения цикла $ C_k $, оставшийся граф $ G_{k+1} $ пуст $ \implies $ в построенных циклах все ребра $ G $. Тогда контруируем в графе $ G $ эйлеров цикл из ребер построенных циклов.
              \end{note}
    \end{description}
\end{proof}

\section{Гамильтоновы циклы. Достаточные условия существования гамильтонова цикла (теоремы Оре и Дирака).}

\begin{definition}[Гамильтонов цикл, граф]
    Пусть $ G = (V,E) $ -- обыкновенный граф, $ | V | = n  $. Простой цикл в графе $ G $ называется \emph{гамильтоновым}, если он проходит по всем вершинам графа.

    Граф называется \emph{гамильтоновым}, если он содержит гамильтонов цикл.
\end{definition}

\begin{definition}[Гамильтонова цепь]
    Простая цепь в графе $ G $ называется \emph{гамильтоновой}, если она проходит по всем вершинам графа.
\end{definition}

\begin{theorem}[Оре, 1960]
    Пусть $ n \geqslant 3 $. Если в $ n $-вершинном графе $ G $ для любой пары несмежных вершин $ u,v $ выполнено условие
    \[
        deg(u) + deg(v) \geqslant n,
    \]
    то граф -- гамильтонов.
\end{theorem}

\begin{proof}
    От противного. Предположим, что граф $ G $ удовлетворяет условию теоремы, но $ G $ -- негамильтонов.

    Соединив любые две несмежные вершины графа ребром, мы вновь получим граф, удовлетворяющий условию теоремы. Поскольку полный граф гамильтонов, то существует мауксимальный негамильтонов граф $ G^* $, удовлетворяющий условию теоремы.

    Это значит, что соединив две несмежные вершины графа $ G^* $ ребром, мы получим гамильтонов цикл. Поэтому любые две вершины графа $ G^* $ соединены гамильтоновой цепью.

    Выберем в $ G^* $ пару несмежных вершин $ v_1,v_n $ и пусть \\ $ (v_1,v_2,\ldots,v_{n-1},v_n) $ -- гамильтонова цепь в $ G^* $.
    \begin{figure}[H]
        \centering
        \incfig{fig_11}
        \label{fig:fig_11}
    \end{figure}

    Если в графе $ G^* $ вершины $ v_1 $ и $ v_i $ -- смежные, то вершины $ v_{i-1} $ и $ v_n $ не могут быть смежными, иначе в $ G^* $ существовал бы гамильтонов цикл
    \[
        (v_1,v_i,v_n,v_{i-1},v_1),
    \]
    \begin{figure}[H]
        \centering
        \incfig{fig_12}
        \label{fig:fig_12}
    \end{figure}

    Отсюда следует, что
    \[
        deg(v_n) \leqslant n-1 - deg(v_1).
    \]

    Следовательно, $ deg(v_1) + deg(v_n) \leqslant n - 1 $ -- противоречие с условием.
\end{proof}

\begin{theorem}[Дирак, 1953]
    Пусть $ n \geqslant 3 $. Если в $ n $-вершинном графе $ G $ для любой вершины выполнено условие
    \[
        deg(v) \geqslant \frac{n}{2},
    \]
    то граф -- гамильтонов.
\end{theorem}

\begin{proof}
    Теорема Дирака следует из теоремы Оре.
\end{proof}

\section{Изоморфизм графов. Помеченные и непомеченные графы. Теорема о числе помеченных $n$-вершинных графов.}

\begin{definition}[Изоморфные графы]
    Графы $ G = (V_G,E_G),\\ H = (V_H,E_H) $ называются \emph{изоморфными}, если между множествами из вершин существует взаимнооднозначное соответствие
    \[
        \phi: V_G \rightarrow V_H,
    \]
    сохраняющее смежность, то есть $ \forall u,v \in V_G $
    \[
        uv \in E_G \iff \phi(u)\phi(v)\in E_H.
    \]
    \[
        \text{\textbf{Обозначение:}} \quad G \cong H
    \]
    \begin{figure}[H]
        \centering
        \incfig{fig_13}
        \label{fig:fig_13}
    \end{figure}
\end{definition}

\begin{definition}[Помеченный граф]
    Граф называется \emph{помеченным}, если его вершины отличаются одна от другой какими-то метками.
    \begin{figure}[H]
        \centering
        \incfig{fig_14}
        \caption*{3 разных помеченных графа}
        \label{fig:fig_14}
    \end{figure}
    \begin{figure}[H]
        \centering
        \incfig{fig_15}
        \caption*{2 одинаковых помеченных графа}
        \label{fig:fig_15}
    \end{figure}
\end{definition}

\begin{theorem}[О числе помеченных $ n $-вершинных графах]
    Число $ p_n $ различных помеченных $ n $-вершинных графов с фиксированным множеством вершин равно
    \[
        2^\frac{n(n-1)}{2}.
    \]
\end{theorem}

\begin{proof}
    В помеченном $ n $-вершинном графе $ G $ можно перенумеровать все пары вершин (таких пар всего $ C^2_n = \frac{n(n-1)}{2} $) и поставить графу $ G $ взаимнооднозначное соответствие его характеристический вектор длины $ k = \frac{n(n-1)}{2} $, $ i $-ая компонента которого равна
    \[
        e_i = \left\{\begin{array}{l}
            1, \text{ если пара вершин с номером }i \text{ смежна} \\
            0, \text{ в противном случае}
        \end{array}\right.
    \]

    \begin{example}
        $ e = (\underset{e_1}{1},\underset{e_2}{0},\underset{e_3}{1},\underset{e_4}{1},\underset{e_5}{1},\underset{e_6}{0}) $
        \begin{figure}[H]
            \centering
            \incfig{fig_16}
            \caption*{$ \underset{\circled{1}}{\{1,2\}},\underset{\circled{2}}{\{1,3\}},\underset{\circled{3}}{\{1,4\}},\underset{\circled{4}}{\{2,3\}},\underset{\circled{5}}{\{2,4\}},\underset{\circled{6}}{\{3,4\}} $}
            \label{fig:fig_16}
        \end{figure}
    \end{example}

    Тогда $ p_n $ равно числу булевых векторов длины $ k = \frac{n(n-1)}{2} $, то есть
    \[
        p_n = 2^k = 2^\frac{n(n-1)}{2}.
    \]
\end{proof}

\section{Проблема изоморфизма. Инварианты графа. Примеры.}

\begin{definition}[Инвариант графа]
    \emph{Инвариант графа} $ G = (V,E) $ -- это число, набор чисел, функция или свойство связанные с графом и принимающие одно и то же значение на любом графе, изоморфном $ G $, то есть
    \[
        G \cong H \implies i(G) = i(G).
    \]

    Инвариант $ i $ называется \emph{полным}, если
    \[
        i(G) = i(H) \implies G \cong H.
    \]
    \[
        \text{\textbf{Обозначение:}} \quad i(G)
    \]
\end{definition}

\begin{example}\leavevmode
    \begin{enumerate}
        \item $ n(G) $ -- число вершин.
        \item $ m(G) $ -- число ребер.
        \item $ \delta(G) $ -- $ \min $ степень.
        \item $ \Delta(G) $ -- $ \max $ степень.
        \item $ \phi(G) $ -- плотность графа $ G $ -- наибольшее число попарно смежных вершин.
        \item $ \epsilon(G) $ -- неплотность -- наибольшее число попарно несмежных вершин.
        \item $ ds(G) $ -- вектор степеней (или степенная последовательность) -- последовательность степеней всех вершин, выписанная в порядке неубывания.
        \item $ \chi(G) $ -- хроматическое число -- наименьшее число $ \chi $, для которого граф имеет правильную $ \chi $-раскраску множества вершин (правильная раскраска -- раскраска, при которой смежные вершины имеют разный цвет).
    \end{enumerate}
    \begin{figure}[H]
        \centering
        \incfig{fig_17}
        \label{fig:fig_17}
    \end{figure}
    \[
        \begin{array}{ll}
            n(Q_4) = 4      & \phi(Q_4) = 3       \\
            m(Q_4) = 5      & \epsilon(Q_4) = 2   \\
            \delta(Q_4) = 2 & ds(Q_4) = (2,2,3,3) \\
            \Delta(Q_4) = 3 & \chi(Q_4) = 3
        \end{array}
    \]
\end{example}

\section{Связные и несвязные графы. Лемма об удалении ребра. Оценки числа ребер связного графа.}

\begin{definition}[Соединимые вершины, связный граф]
    Две вершины $ u,v $ графа $ G $ называются \emph{соединимыми}, если в $ G \ \exists \ (u,v) $-маршрут.

    Граф называется \emph{связным}, если в нем любые две вершины соединимы.
\end{definition}

\begin{remark}
    Тривиальный граф считается связным.
\end{remark}

\begin{definition}[Циклическое, ациклическое ребро]
    Ребро $ e $ называется \emph{циклическим}, если оно принадлежит некоторому циклу, и \emph{ациклическим} -- в противном случае.
\end{definition}

\begin{lemma}[Об удалении ребра]
    Пусть $ G = (V,E) $ -- связный граф, $ e \in E $.
    \begin{enumerate}
        \item Если $ e $ -- циклическое ребро, то граф $ G - e $ -- связен.
        \item Если $ e $ -- ациклическое, то граф $ G - e $ имеет ровно две компоненты связности.
    \end{enumerate}
\end{lemma}

\begin{proof}\leavevmode
    \begin{enumerate}
        \item Пусть $ e = (u,v) $ -- циклическое, входит в цикл $ C $, который можно рассмотреть как объединение ребра $ e $ и $ (u,v) $-цепи $ P $.
              \begin{figure}[H]
                  \centering
                  \incfig{fig_18}
                  \label{fig:fig_18}
              \end{figure}

              Чтобы доказать, что $ G-e $ -- связен, нужно доказать, что любые его две вершины соединимы.

              Рассмотрим две произвольные вершины, назовем их $ s $ и $ t $. Так как по условию $ G $ -- связный, то $ \exists \ (s,t) $-маршрут.

              Если этот $ (s,t) $-маршрут проходит по ребру $ e $, то заменим в нем ребро $ e $ на $ (u,v) $-цепь $ P $, получили новый $ (s,t) $-маршрут, не проходящий по $ e \implies G-e $ -- связен.
              \begin{figure}[H]
                  \centering
                  \incfig{fig_19}
                  \label{fig:fig_19}
              \end{figure}

        \item Пусть $ e = uv $ ацикличен, очевидно, что $ G-e $ -- несвязный.
              \begin{figure}[H]
                  \centering
                  \incfig{fig_20}
                  \label{fig:fig_20}
              \end{figure}

              Чтобы доказать, что в $ G - e $ ровно 2 компоненты связности, нужно доказать, что любая вершина $ \omega $ содержится в одной компоненте $ c $ в $ u $ или $ v $.

              По условию $ G $ -- связен, значит в нем $ \exists $ простая $ (u,\omega) $-цепь и простая $ (v,\omega) $-цепь. Заметим, что ребро $ e $ может входить в одну, и только в одну, из этих цепей, иначе $ e $ было бы циклическим.
              \begin{figure}[H]
                  \centering
                  \incfig{fig_21}
                  \label{fig:fig_21}
              \end{figure}

              Предположим, что ребро $ e $ входит в $ (u,\omega) $-цепь. Тогда вершины $ v $ и $ \omega $ находятся в одной компоненте связности.
    \end{enumerate}
\end{proof}

\begin{theorem}[Оценки числа ребер связного графа]\label{theorem:1}
    Если $ G $ -- связный $ (n,m) $-граф, то
    \[
        n-1 \leqslant m \leqslant \frac{n(n-1)}{2}.
    \]
\end{theorem}

\begin{proof}
    Доказательство требует только нижняя оценка.

    Пусть $ G=(V,E) $ -- связный.

    Доказывать будем индукцией по числу $ | E | $ ребер. Если $ | E | = m =0 $, то $ G $ -- тривиальный граф, то есть $ | V | =n = 1 \implies m = n-1 = 0 $. Предположим, что для графа, где $ | E | < m $, неравенство верно. Пусть $ | E | = m \geqslant 1 $.
    \begin{enumerate}
        \item Если в $ G $ есть циклы, то рассмотрим какое-нибудь циклическое ребро $ e $ и удалим его из $ G $. Тогда по лемме об удалении ребра, $ G-e $ связен, а количество ребер $ m-1 $.

              По предположению индукции, $ m-1 \geqslant n-1 \implies m \geqslant n > n-1 $.
        \item Пусть в $ G $ нет циклов, рассмотрим произвольное ребро $ e $, оно ациклическое, удалим его, тогда в $ G - e $ ровно две компоненты связности.

              Обозначим их $ G_1 $ и $ G_2 $.
              \begin{figure}[H]
                  \centering
                  \incfig{fig_22}
                  \label{fig:fig_22}
              \end{figure}

              Пусть $ G_1 $ -- $ (n_1,m_1) $-граф, а $ G_2 $ -- $ (n_2,m_2) $-граф. Тогда
              \[
                  \begin{array}{l}
                      m_1 \geqslant n_1 - 1 \\
                      m_2 \geqslant n_2-1
                  \end{array}
              \]
              \begin{center}
                  (по предположению индукции, так как $ m_1<m, \ m_2<m $)
              \end{center}

              Следовательно,
              \[
                  m-1 = m_1 + m_2 \geqslant n_1-1+n_2-1=n_1+n_2-2 = n-2,
              \]
              то есть $ m-1 \geqslant n-2 \implies m \geqslant n-1 $.
    \end{enumerate}
\end{proof}

\newpage

\section{Плоские и планарные графы. Графы Куратовского. Формула Эйлера для плоских графов.}

\begin{definition}[Плоский, планарный граф]
    \emph{Плоский граф} -- это такой граф, вершины которого являются точками плоскости, а ребра -- непрерывными плоскими линиями без самопересечений, соединяющими вершины так, что никакие два ребра не имеют общих точек вне вершин.

    \emph{Планарный граф} -- это граф, изоморфный некоторому плоскому графу.
    \begin{figure}[H]
        \centering
        \incfig{fig_23}
        \label{fig:fig_23}
    \end{figure}
\end{definition}

\begin{remark}
    Несложно доказать, что графы $ K_{3,3} $ и $ K_5 $ -- непланарны.
    \begin{figure}[H]
        \centering
        \incfig{fig_24}
        \label{fig:fig_24}
    \end{figure}
\end{remark}

\begin{definition}[Гомеоморфные графы]
    Два графа называются \emph{гомеоморфными}, если их можно получить из одного и того же графа с помощью разбиения ребер, то есть замены некоторых ребер простыми цепями.
\end{definition}

\begin{theorem}[Понтрягин-Куратовский]
    Граф планарен $ \iff $ он не содержит подграфов, гомеоморфных $ K_{3,3} $ или $ K_5 $.
\end{theorem}

\begin{definition}[Грань]
    \emph{Гранью} плоского графа называется максимальное множество точек плоскости, каждая пара из которых может быть соединена непрерывной плоскоской линией, не пересекающей ребер графа.
\end{definition}

\newpage

\begin{theorem}[Формула Эйлера]
    Для всякого связного плоского графа верна формула
    \begin{equation}\label{eq:2}
        n-m+l=2,
    \end{equation}
    где $ n $ -- число вершин, $ m $ -- число ребер, $ l $ -- число граней графа.
\end{theorem}

\begin{proof}
    Рассмотрим две операции перехода от связного плоского графа $ G $ к его связному плоскому подграфу, не изменяющие величины $ n-m+l $.
    \begin{enumerate}
        \item Удаление ребра, принадлежащего сразу двум граням (одно из которых может быть внешней), при этом $ m $ и $ l $ уменьшаются на $ 1 $.
        \item Удаление висячей вершины вместе с инцидентным ребром. При этом $ n $ и $ m $ уменьшаются на $ 1 $.
    \end{enumerate}

    Очевидно, что любой связный плоский граф, выполняя эти две операции, можно превратить в тривиальный граф, не меняя величины $ n-m+l $, а для тривиального графа:
    \[
        n-m+l=2.
    \]

    Значит формула \ref{eq:2} верна для любого связного плоского графа.
\end{proof}