\section{Определение графа. Примеры графов. Степени вершин графа. Лемма о рукопожатиях.}

\begin{definition}[Неориентированный граф, вершины и ребра графа]
    \emph{Неориентированный граф} -- пара множеств $ G = (V,E) $, где
    \[
        \begin{array}{l}
            V \text{ -- непустое конечное множество,} \\
            E \text{ -- множество, состоящее из неупорядоченных пар элементов из } V.
        \end{array}
    \]

    Элементы множества $ V $ называются \emph{вершинами}, а элементы $ E $ -- \emph{ребрами} графа.
\end{definition}

\begin{note}
    Если $ u,v \in V, \ \{u,v\}\in E $, то будем записывать
    \[
        e = uv \ (=vu)
    \]
    и говорить, что вершины $ u $ и $ v $ \emph{смежны}, вершина $ u $ и ребро $ e $ -- \emph{инцидентны}.
\end{note}

\begin{definition}[Степень вершины]
    \emph{Степенью вершины} $ v $ называется число инцидентных ей ребер.
    \[
        \text{\textbf{Обозначение:}} \quad d(v)\ \bigl(deg(v)\bigr)
    \]
\end{definition}

\begin{example}
    $ deg(v) = 3 $
    \begin{figure}[H]
        \centering
        \incfig{fig_01}
        \label{fig:fig_01}
    \end{figure}
\end{example}

\begin{example}
    \emph{Пустой граф} -- граф без ребер: $ O_n $.
\end{example}

\begin{example}
    \emph{Полный граф} -- граф, любая пара которого смежна: $ K_n $.
\end{example}

\begin{note}
    \[
        | E | = C_n^2 = \frac{n(n-1)}{2} \text{ -- число ребер}.
    \]
\end{note}

\begin{example}
    \emph{Двудольный граф} -- граф, вершины которого разбиты на 2 непересекающиеся части (доли) так, что любое ребро ведет из одной доли в другую.

    Если любая вершина одной доли смежна с любой вершиной другой доли, то такой граф называется \emph{полным двудольным}.

    Полный двудольный граф с долями размера $ p $ и $ q $ обозначают: $ K_{p,q} $,
    \[
        | E | = p\cdot q.
    \]
    \begin{figure}[H]
        \centering
        \incfig{fig_02}
        \label{fig:fig_02}
    \end{figure}
\end{example}

\begin{example}
    \emph{Звезда} -- полный двудольный граф $ K_{1,q} $: одна доля состоит из одной вершины, а из нее веером расходятся лучи.
    \begin{figure}[H]
        \centering
        \incfig{fig_03}
        \label{fig:fig_03}
    \end{figure}
\end{example}

\begin{example}
    Графы многогранников
    \begin{figure}[H]
        \centering
        \incfig{fig_04}
        \label{fig:fig_04}
    \end{figure}
\end{example}

\begin{lemma}[О рукопожатиях]
    Пусть $ G=(V,E) $ -- произвольный граф. Сумма степеней всех вершин графа $ G $ -- четное число, равное удвоенному количеству его ребер:
    \begin{equation}\label{eq:1}
        \sum_{v \in V}deg_G(v) = 2 | E |
    \end{equation}
\end{lemma}

\section{Маршруты, цепи, циклы. Лемма о выделении простой цепи. Лемма об объединении простых цепей.}

\begin{definition}[Маршрут]
    \emph{Маршрутом}, соединяющим вершины $ u $ и $ v $ ($ (u,v) $-маршрут), называется чередующаяся последовательность вершин и ребер вида
    \[
        (u = v_1,e_1,v_2,\ldots,v_k,e_k,v_{k+1}=v)
    \]
    такая, что $ e_i = v_iv_{i+1}, \ i = \overline{1,k} $.
\end{definition}

\begin{definition}[Замкнутый маршрут]
    Маршрут называется \emph{замкнутым}, если первая вершина совпадает с последней, то есть
    \[
        v_1=v_{k+1}.
    \]
\end{definition}

\begin{definition}[Цепь, простая цепь]
    Маршрут называется \emph{цепью}, если в нем все ребра различны и \emph{простой цепью}, если в нем все вершины различны (за исключением, быть может, первой и последней).
    \begin{figure}[H]
        \centering
        \incfig{fig_05}
        \label{fig:fig_05}
    \end{figure}
\end{definition}

\begin{definition}[Цикл, простой цикл]
    Замкнутая цепь называется \emph{циклом}, а замкнутая простая цепь -- \emph{простым циклом}.
    \begin{figure}[H]
        \centering
        \incfig{fig_06}
        \label{fig:fig_06}
    \end{figure}
\end{definition}

\begin{lemma}[О выделении простой цепи]
    Всякий незамкнутый \\ $ (u,v) $-маршрут содержит простую $ (u,v) $-цепь.
\end{lemma}

\begin{lemma}[Об объединении простых цепей]
    Объединение двух несовпадающих простых $ (u,v) $-цепей содержит простой цикл.
\end{lemma}

\section{Эйлеровы циклы. Критерий существования эйлерова цикла (теорема Эйлера).}

\begin{definition}[Эйлеров цикл]
    Пусть $ G = (V,E) $ -- произвольный граф (мультиграф). Цикл в графе $ G $ называется \emph{эйлеровым}, если он содержит все ребра графа.
\end{definition}

\begin{definition}[Эйлеров граф]
    Граф называется \emph{эйлеровым}, если в нем есть эйлеров цикл.
\end{definition}

\begin{theorem}[Эйлер, 1736]
    В связном графе $ G = (V,E) $ существует эйлеров цикл $ \iff $ все вершины графа $ G $ четны (то есть имеют четную степень).
\end{theorem}

\section{Гамильтоновы циклы. Достаточные условия существования гамильтонова цикла (теоремы Оре и Дирака).}

\begin{definition}[Гамильтонов цикл, граф]
    Пусть $ G = (V,E) $ -- обыкновенный граф, $ | V | = n  $. Простой цикл в графе $ G $ называется \emph{гамильтоновым}, если он проходит по всем вершинам графа.

    Граф называется \emph{гамильтоновым}, если он содержит гамильтонов цикл.
\end{definition}

\begin{definition}[Гамильтонова цепь]
    Простая цепь в графе $ G $ называется \emph{гамильтоновой}, если она проходит по всем вершинам графа.
\end{definition}

\begin{theorem}[Оре, 1960]
    Пусть $ n \geqslant 3 $. Если в $ n $-вершинном графе $ G $ для любой пары несмежных вершин $ u,v $ выполнено условие
    \[
        deg(u) + deg(v) \geqslant n,
    \]
    то граф -- гамильтонов.
\end{theorem}

\begin{theorem}[Дирак, 1953]
    Пусть $ n \geqslant 3 $. Если в $ n $-вершинном графе $ G $ для любой вершины выполнено условие
    \[
        deg(v) \geqslant \frac{n}{2},
    \]
    то граф -- гамильтонов.
\end{theorem}

\section{Изоморфизм графов. Помеченные и непомеченные графы. Теорема о числе помеченных $n$-вершинных графов.}

\begin{definition}[Изоморфные графы]
    Графы $ G = (V_G,E_G),\\ H = (V_H,E_H) $ называются \emph{изоморфными}, если между множествами из вершин существует взаимнооднозначное соответствие
    \[
        \phi: V_G \rightarrow V_H,
    \]
    сохраняющее смежность, то есть $ \forall u,v \in V_G $
    \[
        uv \in E_G \iff \phi(u)\phi(v)\in E_H.
    \]
    \[
        \text{\textbf{Обозначение:}} \quad G \cong H
    \]
    \begin{figure}[H]
        \centering
        \incfig{fig_13}
        \label{fig:fig_13}
    \end{figure}
\end{definition}

\begin{definition}[Помеченный граф]
    Граф называется \emph{помеченным}, если его вершины отличаются одна от другой какими-то метками.
    \begin{figure}[H]
        \centering
        \incfig{fig_14}
        \caption*{3 разных помеченных графа}
        \label{fig:fig_14}
    \end{figure}
    \begin{figure}[H]
        \centering
        \incfig{fig_15}
        \caption*{2 одинаковых помеченных графа}
        \label{fig:fig_15}
    \end{figure}
\end{definition}
    
\begin{theorem}[О числе помеченных $ n $-вершинных графах]
    Число $ p_n $ различных помеченных $ n $-вершинных графов с фиксированным множеством вершин равно
    \[
        2^\frac{n(n-1)}{2}.
    \]
\end{theorem}

\section{Проблема изоморфизма. Инварианты графа. Примеры.}

\begin{definition}[Инвариант графа]
    \emph{Инвариант графа} $ G = (V,E) $ -- это число, набор чисел, функция или свойство связанные с графом и принимающие одно и то же значение на любом графе, изоморфном $ G $, то есть
    \[
        G \cong H \implies i(G) = i(G).
    \]

    Инвариант $ i $ называется \emph{полным}, если
    \[
        i(G) = i(H) \implies G \cong H.
    \]
    \[
        \text{\textbf{Обозначение:}} \quad i(G)
    \]
\end{definition}

\begin{example}\leavevmode
    \begin{enumerate}
        \item $ n(G) $ -- число вершин.
        \item $ m(G) $ -- число ребер.
        \item $ \delta(G) $ -- $ \min $ степень.
        \item $ \Delta(G) $ -- $ \max $ степень.
        \item $ \phi(G) $ -- плотность графа $ G $ -- наибольшее число попарно смежных вершин.
        \item $ \epsilon(G) $ -- неплотность -- наибольшее число попарно несмежных вершин.
        \item $ ds(G) $ -- вектор степеней (или степенная последовательность) -- последовательность степеней всех вершин, выписанная в порядке неубывания.
        \item $ \chi(G) $ -- хроматическое число -- наименьшее число $ \chi $, для которого граф имеет правильную $ \chi $-раскраску множества вершин (правильная раскраска -- раскраска, при которой смежные вершины имеют разный цвет).
    \end{enumerate}
    \begin{figure}[H]
        \centering
        \incfig{fig_17}
        \label{fig:fig_17}
    \end{figure}
    \[
        \begin{array}{ll}
            n(Q_4) = 4      & \phi(Q_4) = 3       \\
            m(Q_4) = 5      & \epsilon(Q_4) = 2   \\
            \delta(Q_4) = 2 & ds(Q_4) = (2,2,3,3) \\
            \Delta(Q_4) = 3 & \chi(Q_4) = 3
        \end{array}
    \]
\end{example}

\section{Связные и несвязные графы. Лемма об удалении ребра. Оценки числа ребер связного графа.}

\begin{definition}[Соединимые вершины, связный граф]
    Две вершины $ u,v $ графа $ G $ называются \emph{соединимыми}, если в $ G \ \exists \ (u,v) $-маршрут.

    Граф называется \emph{связным}, если в нем любые две вершины соединимы.
\end{definition}

\begin{remark}
    Тривиальный граф считается связным.
\end{remark}

\begin{definition}[Циклическое, ациклическое ребро]
    Ребро $ e $ называется \emph{циклическим}, если оно принадлежит некоторому циклу, и \emph{ациклическим} -- в противном случае.
\end{definition}

\begin{lemma}[Об удалении ребра]
    Пусть $ G = (V,E) $ -- связный граф, $ e \in E $.
    \begin{enumerate}
        \item Если $ e $ -- циклическое ребро, то граф $ G - e $ -- связен.
        \item Если $ e $ -- ациклическое, то граф $ G - e $ имеет ровно две компоненты связности.
    \end{enumerate}
\end{lemma}

\begin{theorem}[Оценки числа ребер связного графа]\label{theorem:1}
    Если $ G $ -- связный $ (n,m) $-граф, то
    \[
        n-1 \leqslant m \leqslant \frac{n(n-1)}{2}.
    \]
\end{theorem}

\section{Плоские и планарные графы. Графы Куратовского. Формула Эйлера для плоских графов.}

\begin{definition}[Плоский, планарный граф]
    \emph{Плоский граф} -- это такой граф, вершины которого являются точками плоскости, а ребра -- непрерывными плоскими линиями без самопересечений, соединяющими вершины так, что никакие два ребра не имеют общих точек вне вершин.

    \emph{Планарный граф} -- это граф, изоморфный некоторому плоскому графу.
    \begin{figure}[H]
        \centering
        \incfig{fig_23}
        \label{fig:fig_23}
    \end{figure}
\end{definition}

\begin{remark}
    Несложно доказать, что графы $ K_{3,3} $ и $ K_5 $ -- непланарны.
    \begin{figure}[H]
        \centering
        \incfig{fig_24}
        \label{fig:fig_24}
    \end{figure}
\end{remark}

\begin{definition}[Гомеоморфные графы]
    Два графа называются \emph{гомеоморфными}, если их можно получить из одного и того же графа с помощью разбиения ребер, то есть замены некоторых ребер простыми цепями.
\end{definition}

\begin{theorem}[Понтрягин-Куратовский]
    Граф планарен $ \iff $ он не содержит подграфов, гомеоморфных $ K_{3,3} $ или $ K_5 $.
\end{theorem}

\begin{definition}[Грань]
    \emph{Гранью} плоского графа называется максимальное множество точек плоскости, каждая пара из которых может быть соединена непрерывной плоскоской линией, не пересекающей ребер графа.
\end{definition}

\begin{theorem}[Формула Эйлера]
    Для всякого связного плоского графа верна формула
    \begin{equation}\label{eq:2}
        n-m+l=2,
    \end{equation}
    где $ n $ -- число вершин, $ m $ -- число ребер, $ l $ -- число граней графа.
\end{theorem}
