\section{Формула Пуассона}

\begin{note}
	Взято не из лекций.
\end{note}

\begin{theorem}[Формула Пуассона]
	\[
		p_n(k) \approx \frac{\lambda ^k}{k!}e^{-\lambda } , \qquad \lambda = np < 10.
	\]
\end{theorem}

\section{Определение вероятности в геометрической схеме}

\begin{note}
	Взято не из лекций.
\end{note}

\begin{definition}[Вероятность в геометрической схеме]
	Пусть $\mu $ -- мера в $\R^n$ (чаще всего мера Жордана). $\Omega $ -- некоторое измеримое множество.

	Исход: точка в $\Omega $ (все исходы равновозможны), $\mathcal{F}$ -- $\sigma $-алгебра измеримых подмножеств $\Omega $. $A \in \mathcal{F}$, то
	\[
		P(A) = \frac{m(A)}{m(\Omega )}
	\]
	называется \emph{вероятностью в геометрической схеме}.
\end{definition}

\section{Случайная величина}

\begin{definition}[Случайная величина]
	Пусть $(\Omega , \mathcal{F}, P)$ -- вероятностное пространство. Функция $\xi : \Omega  \rightarrow \R$ такая, что для любого $B \in $ борелевская $\sigma $-алгебры и $\R$
	\[
		\xi^{-1} (B) \in \mathcal{F},
	\]
	$\xi $ -- измеримая функция. Тогда $\xi $ -- \emph{случайная величина}.
\end{definition}

\section{Распределение случайной величины}

\begin{definition}[Распределение случайной величины]
	$(\Omega, \mathcal{F},P )$ -- В.П., $\xi $ -- случайная величина, тогда распределением $\xi $ называется $P_{\xi } $ на $\R$.
\end{definition}

\newpage

\section{Функция распределения случайной величины}

\begin{definition}[Функция распределения случайной величины]
	$(\Omega, \mathcal{F},P )$ -- В.П., $\xi $ -- случайная величина,
	\[
		P_{\xi } \big((-\infty ; x)\big) = P \big(\xi^{-1} (-\infty ; x)\big) = P \big(\xi (\omega ) < x\big) = F_{\xi } (x),
	\]

	$F_{\xi } (x)$ -- функция распределения $\xi $.
\end{definition}

\section{Абсолютно непрерывная мера}

\begin{note}
	Взято не из лекций.
\end{note}

\begin{definition}[Абсолютно непрерывная мера]
	Мера $\mu $ называется \emph{абсолютно непрерывной} относительно другой меры $v$, если $\mu $ -- абсолютно непрерывная функция относительно $v$.
\end{definition}

\section{Сингулярные меры}

\begin{note}
	Взято не из лекций.
\end{note}

\begin{definition}[Сингулярные меры]
	Две положительные меры $\mu $ и $v$ определенные на измеримом пространстве $(\Omega , \Sigma )$ называются \emph{сингулярными}, если $\exists A,B \in \Sigma $ (непересекающиеся и измеримые), объединение которых $\Omega $ такого, что $\mu =0$ во всех измеримых подмножествах $B$, в то время как $v$ равно нулю во всех измеримых подмножествах $A$.
	\begin{notation}
		$\mu \perp v$.
	\end{notation}
\end{definition}

\section{Теорема Лебега о разложении}

\begin{note}
	Взято не из лекций.
\end{note}

\begin{theorem}
	Любую меру Лебега можно представить в виде трех мер -- дискретной, абсолютно непрерывной и сингулярной.
\end{theorem}

\newpage

\section{Дискретная случайная величина}

\begin{note}
	Взято не из лекций.
\end{note}

\begin{definition}
	Случайная величина $\xi $, имеющая дискретное распределение, называется \emph{дискретной}.

	Распределение случайной величины $\xi $ \emph{дискретно}, если число значений $\xi $ не более, чем счетно.
\end{definition}

\section{Абсолютно непрерывная случайная величина}

\begin{note}
	Взято не из лекций.
\end{note}

\begin{definition}
	Случайная величина $\xi $, имеющая абсолютно непрерывное распределение, называется \emph{абсолютно непрерывной}.

	Распределение случайной величины $\xi $ называется \emph{абсолютно непрерывным}, если $\exists $ неотрицательная функция $P_{\xi } (x)$ такая, что:
	\[
		\forall x \quad F_{\xi } (x) = P(\xi < x) = \int_{-\infty }^{x} P_{\xi } (t)\d t.
	\]
\end{definition}
