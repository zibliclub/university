\section{Формула числа сочетаний с повторениями}

\begin{note}
	Взято не из лекций.
\end{note}

\begin{definition}[Число сочетаний с повторениями]
	\[
		\overline{C_{n}^{k}} = C_{n+k-1}^{k}.
	\]
\end{definition}

\section{Условная вероятность $P(A | B)$}

\begin{definition}[Условная вероятность]
	Пусть $P(B)> 0$ условий. Все исходы -- это $B$, исходы $AB$ и тогда:
	\[
		P(A | B) = \frac{P(AB)}{P(B)}.
	\]
\end{definition}

\section{Определение независимых событий}

\begin{note}
	Взято не из лекций.
\end{note}

\begin{definition}[Независимое событие]
	$(\Omega , \mathcal{F}, P)$ -- В.П., $A,B \in \mathcal{F}$ -- случайные события. Говорят, что $A$ и $B$ \emph{независимы}, если $P(A | B) = P(A)$ ($A$ не зависит от $B$).
\end{definition}

\section{Формула полной вероятности}

\begin{theorem}[Формула полной вероятности]
	Пусть $A$ -- случайные события, $H_1,\ldots , H_n$ -- разбиения, тогда:
	\[
		P(A) = \sum_{j=1}^{n} P(A | H_j) \cdot P(H_j).
	\]
\end{theorem}

\newpage

\section{Формула Байеса}

\begin{theorem}[Формула Байесса]
	$H_1,\ldots ,H_n$ -- разбиение, $A \in \mathcal{F}$,
	\[
		P(H_i | A) = \frac{P(A | H_i)P(H_i)}{\sum_{j} P(A | H_j)P(H_j)},
	\]
	апостериорные вероятности, $P(H_i )$ -- априорные вероятности.
\end{theorem}

\section{Формула Бернулли}

\begin{note}
	Взято не из лекций.
\end{note}

\begin{theorem}[Формула Бернулли]
	Вероятность того, что в $n$ независимых испытаниях, в каждом из которых вероятность успеха равна $0 < p < 1$, а вероятность неудачи равна $q = 1 -p$, событие наступит ровно $k$ раз, безразлично в какой последовательности равна:
	\[
		P(B_k) = C_{n}^{k} \cdot p^k \cdot q^{n-k}.
	\]
\end{theorem}

\section{Наиболее вероятное число успехов в схеме Бернулли}

\begin{note}
	Взято не из лекций.
\end{note}

\begin{note}
	Наиболее вероятное число успехов в схеме Бернулли:
	\[
		np - q \leqslant k \leqslant np + p.
	\]

	Наиболее вероятное число успехов $(np + p)$. Число успехов с наибольшей вероятностью:
	\[
		[np + p] \quad \text{или} \quad [np + p] + 1.
	\]
\end{note}

\newpage

\section{Закон больших чисел для схемы Бернулли}

\begin{note}
	Взято не из лекций.
\end{note}

\begin{theorem}
	Пусть производится последовательность независимых испытаний, в результате каждого из которых может наступить или не наступить событие $A$, причем вероятность наступления этого события одна и та же при каждом испытании и равна $p$.

	Если событие $A$ фактически произошло $m$ раз в $n$ испытаниях, то отношение $\frac{m}{n} $ называют, как мы знаем, частотой появляения события $A$. Частота есть случайная величина, причем вероятность того, что частота принимает значение $\frac{m}{n} $, выражается по формуле Бернулли:
	\[
		P_n(m) = C_{n}^{m} p^m q^{n-m} .
	\]

	Закон больших чисел в форме Бернулли состоит в следующем:
	\[
		\underset{n \rightarrow \infty }{\lim} P \left[\abs{\frac{m}{n} - p} < \epsilon \right] = 1.
	\]
\end{theorem}

\section{Локальная формула Муавра-Лапласа}

\begin{note}
	Взято не из лекций.
\end{note}

\begin{theorem}
	\[
		P_n(k) = \frac{1}{\sqrt{npq} } \phi (x),
	\]
	\[
		\phi (x) = \frac{1}{\sqrt{2\pi} } \cdot e^{-\frac{x^2}{2} } , \qquad x = \frac{k - np}{\sqrt{npq} } .
	\]
\end{theorem}

\section{Интегральная формула Муавра-Лапласа}

\begin{theorem}[Интегральная теорема Муавра-Лапласа]
	В схеме Бернулли $\sigma = \sqrt{npq} \rightarrow \infty $, то для $a,b \in \R \ m$ -- число успехов в схеме Бернулли, то:
	\[
		P \left(a \leqslant \frac{m - np}{\sqrt{npq} } \leqslant b\right) - \frac{1}{\sqrt{2\pi} } \cdot \int_{n}^{b} e^{-\frac{x^2}{2} } \d x \xrightarrow[n \rightarrow \infty]{} 0.
	\]
\end{theorem}

\begin{note}
	\[
		p_n (k_1 \leqslant k \leqslant k_2) = \Phi (x_2) - \Phi (x_1), \quad \frac{k_i - np}{\sqrt{npq} } ,
	\]
	\[
		\Phi (x) = \frac{1}{\sqrt{2\pi} } \int_{-\infty }^{x} e^{-\frac{t^2}{2} } \d t.
	\]
\end{note}
