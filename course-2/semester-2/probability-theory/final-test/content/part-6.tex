\section{Некоррелированные случайные величины}

\begin{definition}[Некоррелированные случайные величины]
	Две случайные величины $X,Y$ называются \emph{некоррелированными}, если их ковариация равна нулю.
\end{definition}

\section{Дисперсия}

\begin{definition}[Дисперсия]
	\emph{Дисперсия} -- это мера разброса значений случайной величины относительно ее математического ожидания,
	\[
		D(\xi ) = \mu \Big(\big(\xi - \mu (x \xi )\big)^2\Big).
	\]
\end{definition}

\begin{remark}
	\[
		D(\xi ) = \mu \Big(\big(\xi - \mu (\xi )\big)^2\Big) = \mu (\xi^2) - \big(\mu (\xi )\big)^2.
	\]
\end{remark}

\section{Стандартное отклонение}

\begin{definition}[Стандартное отклонение]
	\emph{Стандартное отклонение ($\sigma $)} -- это мера разброса значений случайной величины $X$ относительно ее математического ожидания, аналогичная дисперсии, но выраженная в тех же единицах, что и сама случайная величина,
	\[
		\sigma = \sqrt{\Var(X)} .
	\]
\end{definition}

\section{Формула для дисперсии дискретной случайной величины}

\[
	\Var(X) = \mu (X^2) - \big(\mu (X)\big)^2.
\]

\newpage

\section{Формула для дисперсии абсолютно непрерывной случайной величины}

Если возможные значения случайной величины $X$ лежат на $[a ; b]$, то
\[
	D(x) = \int_{a}^{b} \big(x - M(x)\big)^2 f(x) \d x.
\]

Если же возможные значения случайной величины $X$ заполняют всю числовую ось, то
\[
	D(X) = \int_{-\infty }^{+\infty } \big(x - M(x)\big)^2 f(x) \d x.
\]

\[
	M(X) = \int_{-\infty }^{+\infty } x f(x) \d x,
\]
где $f(x)$ -- функция распределения случайной величины.

\section{Начальный момент}

\begin{definition}[Начальный момент]
	Если дана случайная величина $X$, определенная на некотором В.П., то $k$-м \emph{начальным моментом} случайной величины $X$, где $k \in \N$, называется величина
	\[
		\nu _k = M(X^k).
	\]
\end{definition}

\section{Центральный момент}

\begin{definition}[Центральный момент]
	$k$-м \emph{центральным моментом} случайной величины $X$ называется величина
	\[
		\mu _k = M \big(\abs{X - M(X)} ^k\big).
	\]
\end{definition}

\section{Неравенство Коши-Буняковского}

Пусть $X,Y$ -- случайные величины на одном и том же вероятностном пространстве. Тогда
\[
	\big(M(XY)\big)^2 \leqslant M(X^2) \cdot M(Y^2) \quad (M \text{ -- мат. ожидание}).
\]

\section{Неравенство Гёльдера}

Пусть $\alpha ,\beta \geqslant 0, \ \frac{1}{\alpha } + \frac{1}{\beta } =1$,
\[
	\mu \big(\abs{\xi _1 \cdot \xi _2} \big)\leqslant \big(\mu \abs{\xi_{1}^{\alpha } } \big)^{\frac{1}{\alpha } } \cdot \Big(\mu \abs{\xi_{2}^{\beta } } \Big)^{\frac{1}{\beta } } .
\]

\section{Неравенство Иенсена}

Если есть математическое ожидание $\mu \xi $ и $g(x)$ -- выпуклая вниз измеримая функция, то
\[
	g(\mu \xi ) \leqslant  \mu \big(g(\xi )\big).
\]
