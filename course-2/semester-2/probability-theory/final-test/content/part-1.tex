\section{Алгебра событий}

\begin{definition}[Алгебра]
	$\mathcal{F}$ -- семейство подмножеств $\Omega $, $\mathcal{F}$ -- \emph{алгебра}, если:
	\begin{enumerate}
		\item $\Omega \in \mathcal{F} \ (\varnothing \in \mathcal{F})$.
		\item $A \in \mathcal{F} \implies \overline{A} \in \mathcal{F}$.
		\item $A,B \in \mathcal{F} \implies AB \in \mathcal{F}, \ A + B \in \mathcal{F}$.
	\end{enumerate}
\end{definition}

\section{Сигма-алгебра событий}

\begin{definition}[$\sigma $-алгебра]
	$\mathcal{F}$ -- семейство подмножеств $\Omega $, $\mathcal{F}$ -- \emph{$\sigma $-алгебра}, если:
	\begin{enumerate}
		\item $\Omega \in \mathcal{F} \ (\varnothing \in \mathcal{F})$.
		\item $A \in \mathcal{F} \implies \overline{A} \in \mathcal{F}$.
		\item $A,B \in \mathcal{F} \implies AB \in \mathcal{F}, \ A + B \in \mathcal{F}$.
		\item $\forall \{A_{\alpha} \} \subset \mathcal{F} \ \underset{\alpha }{\bigcap} A_{\alpha} \in \mathcal{F} $ и $\underset{\alpha }{\bigcup} A_{\alpha } \in \mathcal{F}$.
	\end{enumerate}
\end{definition}

\section{Борелевская сигма-алгебра}

\begin{note}
	Взято не из лекций.
\end{note}

\begin{definition}[Борелевская $\sigma $-алгебра]
	$\R$ -- топологическое пространство, \emph{борелевская $\sigma $-алгебра} -- это алгебра, порожденная всеми интервалами.
\end{definition}

\newpage

\section{Определение меры}

\begin{note}
	Взято не из лекций.
\end{note}

\begin{definition}[Мера]
	Пусть $X$ -- некоторое множество и $\Sigma $ -- $\alpha $-алгебра над $X$.

	Функция $\mu $ из $\Sigma $ в расширенную действительную числовую прямую называется \emph{мерой}, если она удовлетворяет следующим свойствам:
	\begin{itemize}
		\item неотрицательность: для всех $E$ в $\Sigma $ имеем
		      \[
			      \mu (E) \geqslant 0 ;
		      \]
		\item пустой набор:
		      \[
			      \mu (\varnothing ) = 0 ;
		      \]
		\item счетная аддитивность (или $\sigma $-аддитивность): для всех счетных наборов попарно непересекающихся множеств в $\Sigma $:
		      \[
			      \{E_k\}_{k=1}^{\infty},
		      \]
		      \[
			      \mu \left(\bigcup_{k=1}^{\infty}E_k\right) = \sum_{k=1}^{\infty} \mu (E_k).
		      \]
	\end{itemize}
\end{definition}

\section{Аксиомы вероятности}

\begin{definition}
	Вероятностное пространство $(\Omega , \mathcal{F}, P)$.

	$\Omega $ -- множество элементарных исходов, $\mathcal{F}$ -- $\sigma $-алгебра подмножеств $\Omega $, $P$ -- мера на $\mathcal{F}$, $P: \mathcal{F} \rightarrow \R$,
	\begin{description}
		\item[\circled{$A_1$}] $\forall A \in \mathcal{F} \ P(A) \geqslant 0$.
		\item[\circled{$A_2$}] $P(\Omega ) = 1$ (условие нормировки).
		\item[\circled{$A_3$}] $\forall A,B \in \mathcal{F} \ AB = \varnothing \implies P(A+B) = P(A) + P(B)$.
		\item[\circled{$A_4$}] $\{A_n\} \subset \mathcal{F} \ A_{n+1} \leqslant A_n \ \underset{n}{\bigcap} A_n = \varnothing $,
		      \[
			      \underset{n \rightarrow \infty}{\lim}P(A_n) = 0 \ \text{(непрерывность меры)}.
		      \]
	\end{description}
\end{definition}

\newpage

\section{Определение вероятности в классической схеме}

\begin{definition}[Вероятность в классической схеме]
	$\Omega $ -- конечное множество равновозможных исходов. $\mathcal{F}$ -- все подмножества $\Omega $ (их $2^{\abs{\Omega } } $),
	\[
		P(A) = \frac{\abs{A} }{\abs{\Omega } } .
	\]
\end{definition}

\section{Формула числа перестановок}

\begin{definition}[формула числа перестановок]
	\emph{Число перестановок} $n$ различных шаров (перестановки отличаются порядком шаров) -- $P(n)$,
	\[
		P(n) = n!, \quad n \cdot (n-1)\cdot (n-2)\cdot \ldots \cdot 1 = n!, \ 0! = 1.
	\]

	Пусть есть $n_1,\ldots,n_m$ шаров $m$ видов, $n =n_1 + \ldots + n_m$.

	\emph{Число перестановок} этих $n$ шаров равно $P(n_1,\ldots ,n_m)$,
	\[
		P(n_1,\ldots ,n_m) = \frac{(n_1 + \ldots  + n_m)!}{n_1!n_2!\ldots n_n!}.
	\]
\end{definition}

\section{Формула числа размещений}

\begin{definition}[Формула числа размещений]
	\emph{Размещения} $n$ элементов по $k$ местам. Выкладываем в ряд $k$ шариков из имеющихся $n$:
	\[
		A_{n}^{k} = n \cdot (n-1)\cdot \ldots \cdot (n-k+1) = \frac{n!}{(n-k)!} .
	\]
\end{definition}

\section{Формула числа сочетаний}

\begin{definition}[Формула числа сочетаний]
	\emph{Сочетания} $k$ элементов из $n$. Число $k$-элементарных подмножеств из $n$-элементов множества -- $C_{n}^{k} $,
	\[
		C_{n}^{k} = \frac{A_{n}^{k}}{k!} = \frac{n!}{(n-k)!k!}.
	\]
\end{definition}

\section{Формула числа размещений с повторениями}

\begin{remark}
	Если мы разрешим шарики повторять, то получим размещения с повторениями:
	\[
		\overline{A_{n}^{k}} = n \cdot n \cdot \ldots \cdot n = n^k.
	\]
\end{remark}
