\section{Плотность распределения}

\begin{note}
	\emph{С этого момента начинаю копировать с файла второй группы, так как устал лишний раз заглядывать в конспекты в ожидании найти хоть что-то (вероятно это имеется в конспектах, просто я не вижу).}
\end{note}

\begin{definition}[Плотность распределения]
	Будем говорить, что функция $P(x) = P_{\xi } (x)$ называется \emph{плотностью распределения} случайной величины $\xi $, если:
	\[
		\forall x_1 < x_2 \quad P(x_1 < \xi < x_2) = \int_{x_1}^{x_2} P_{\xi } (x)\d x.
	\]
\end{definition}

\newpage

\section{Биномиальное распределение, его среднее и дисперсия}

\begin{definition}[Биномиальное распределение]
	\emph{Биномиальное распределение} $B(n,p), \ \Bin(n,p)$, $\xi $ -- случайная величина $\big(\xi \Subset B(n,p) \big)$, если:
	\[
		P(\xi = k) = C_{n}^{k} p^k (1- p)^{n-k}, \quad k = \overline{0,n} ,
	\]
	$\xi $ -- число успехов в схеме Бернулли.

	\begin{description}
		\item[Среднее:] $np$.
		\item[Дисперсия:] $D \big(B(n,p)\big) = np(1-p)$.
	\end{description}
\end{definition}

\section{Геометрическое распределение, его среднее и дисперсия}

\begin{definition}[Геометрическое распределение]
	\[
		P(\xi = k) = p(1-p)^{k-1} , \quad k = \overline{1,n} .
	\]

	$\varnothing \_ \varnothing$

	\begin{description}
		\item[Среднее:] $\frac{1-p}{p} $.
		\item[Дисперсия:] $\frac{1-p}{p^2} $.
	\end{description}
\end{definition}

\section{Гипергеометрическое распределение}

\begin{definition}[Гипергеометрическое распределение]
	\[
		P(\xi = m) = \frac{C_{M}^{m}C_{N-M}^{n-m}  }{C_{N}^{n} } , \quad m=\overline{0,\min(M,n)}.
	\]
	$N,M,n \in \N: \ M \leqslant N, \ n \leqslant N$.
\end{definition}

\newpage

\section{Распределение Пуассона, его среднее и дисперсия}

\begin{definition}[Распределение Пуассона]
	Распределение Пуассона:
	\[
		P(\lambda ), \Pi(\lambda ),
	\]

	\[
		\xi \Subset P(\lambda ) \iff P(\xi = k) = \frac{\lambda^k}{k!} e^{-\lambda } , \quad k = 0,1,2,\ldots
	\]

	\begin{description}
		\item[Среднее:] $\lambda $.
		\item[Дисперсия:] $\lambda $.
	\end{description}
\end{definition}

\section{Равномерное распределение на отрезке $[a ; b]$, его среднее и дисперсия}

\begin{definition}[Равномерное распределение]
	Равномерное распределение $U_{[a ; b]} $.

	Если $\xi $ -- случайная величина равномерна на $[a ; b]$, то:
	\[
		F_{\xi } = P(\xi < x) = \left\{\begin{array}{ll}
			0,                & x \leqslant a     \\
			\frac{x-a}{b-a} , & a < x \leqslant b \\
			1,                & x > b
		\end{array}\right.
	\]

	\begin{description}
		\item[Среднее:] $\frac{a+b}{2} $.
		\item[Дисперсия:] $\frac{(b-a)^2}{12} $.
	\end{description}
\end{definition}

\newpage

\section{Нормальное распределение, его среднее и дисперсия}

\begin{definition}[Нормальное распределение]
	Нормальное распределение $\mathcal{N}(a,\sigma ^2)$.

	$\xi $ -- нормальное распределение $\iff \xi \Subset \mathcal{N}(a,\sigma ^2)\iff P(\xi \in B) = \frac{1}{\sigma \sqrt{2\pi} } \int_{B} e^{-\frac{(x-a)^2}{2 \sigma ^2} } \d x$, $\mathcal{N}(0,1)$ -- стандартное нормальное распределение.
	\[
		P(\xi \in B) = \frac{1}{\sqrt{2 \pi} } \int_{B} e^{- \frac{x^2}{2} } \d x,
	\]
	\[
		F_{\xi } = \frac{1}{\sqrt{2\pi} } , \quad \int_{-\infty }^{x} e^{-\frac{t^2}{2} } \d t = \Phi (x).
	\]

	\begin{description}
		\item[Среднее:] $a $.
		\item[Дисперсия:] $\sigma ^2 $.
	\end{description}
\end{definition}

\section{Распределение Коши, его среднее и дисперсия}

\begin{definition}[Распределение Коши]
	Распределение Коши $K(a,\sigma ), \ a \in \R, \ \sigma > 0$.

	\[
		\sigma \Subset K(a,\sigma ) \iff P(B) = \frac{1}{\pi \sigma } \int_{B} \frac{1}{1 + \left(\frac{x-a}{\sigma } \right)^2} \d x.
	\]
	\[
		K = K(0,1), \ F_{\xi } = K(x), \quad \eta \Subset K(a,\sigma ), \ F_{\eta} (x) = K \left(\frac{x-a}{\sigma } \right).
	\]

	\begin{description}
		\item[Среднее:] $a $.
		\item[Дисперсия:] Не существует.
	\end{description}
\end{definition}

\newpage

\section{Экспоненциальное распределение, его среднее и дисперсия}

\begin{definition}[Экспоненциальное распределение]
	$\Exp(a), \ E(a)$.

	\[
		\xi \underset{a \geqslant 0}{\Subset} \Exp(a), \ \text{если} \ P(\xi \in B) = a \int_{B \cap [0 ; +\infty )}e^{-at} \d t,
	\]

	\[
		\left\{\begin{array}{ll}
			0,                                                                  & x \leqslant 0 \\
			a \int_{0}^{x} e^{-at} \d t = -e^{-at} \Big|_{0}^{x} = 1 - e^{ax} , & x > 0
		\end{array}\right.
	\]

	\begin{description}
		\item[Среднее:] $a^{-1}  $.
		\item[Дисперсия:] $a^{-2} $.
	\end{description}
\end{definition}

\section{Случайный вектор}

\begin{definition}[Случайный вектор]
	$(\Omega , \mathcal{F}, P)$ -- вероятностное пространство, $\xi_1, \ldots , \xi_n$ -- случайные величины. Функция
	\[
		\xi (w) = \big(\xi _1(w), \ldots , \xi _1(w)\big): \ \Omega \rightarrow \R^n
	\]
	называется \emph{случайным вектором ($n$-мерным случайным вектором)}.
\end{definition}
