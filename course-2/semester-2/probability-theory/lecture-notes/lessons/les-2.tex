\lesson{2}{от 21 фев 2024 8:45}{Продолжение}


\section{Классическая схема и комбинаторика}

\begin{definition}[Классическая схема]
  $\Omega$ -- конечное множество равновозможных исходов, $\mathcal{F}$ -- все подмножества $\Omega$ (их $2^{\abs{\Omega}}$),
  \[
    P(A) = \frac{\abs{A}}{\abs{\Omega}}.
  \]

  Это \emph{классическая схема}.
\end{definition}

\begin{eg}
  $2 \diff G$ нас интересует сумма,
  \[
    \left.\begin{array}{c}
      \Omega_{1} \text{ -- исход сумма от 2 до 12} \\
      \text{(2 и 7 неравновозможные)}
    \end{array}\right\} \text{не классическая схема}
  \]
  \[
    \left.\begin{array}{c}
      \Omega_{2} \text{ -- множество очков на кубиках} \\
      \text{($\{1,1\}$ и $\{1,2\}$ неравновозможные)}
    \end{array}\right\} \text{не классическая схема}
  \]
  \[
    \left.\begin{array}{c}
      \Omega_{3} \text{ -- упорядоченная пара очков на кубиках} \\
      \text{(все 36 исходов равновозможныe)}
    \end{array}\right\} \text{классическая схема}
  \]
\end{eg}

\begin{definition}[Число перестановок различных шаров]
  \emph{Число перестановок $n$ различных шаров} (перестановки отличаются порядком шаров) -- $P(n)$,
  \[
    P(n) = n!.
  \]
\end{definition}

\begin{definition}[Число перестановок шаров разных видов]
  Пусть есть $n_1,\ldots,n_m$ шаров $m$ видов,
  \[
    n = n_1 + \ldots + n_m.
  \]
  \emph{Число перестановок этих $n$ шаров} равно $P(n_1,\ldots,n_m)$,
  \[
    P(n_1,\ldots,n_m) = \frac{(n_1+\ldots+n_m)!}{n_1!n_2!\ldots n_m!}.
  \]
\end{definition}

\begin{definition}[Размещения элементов по местам]
  \emph{Размещения $n$ элементов по $k$ местам}.

  Выкладываем в ряд $k$ шариков из имеющихся $n$:
  \[
    A^{k}_{n} = n \cdot (n-1) \cdot \ldots \cdot (n-k+1) = \frac{n!}{(n-k)!}.
  \]
\end{definition}

\begin{remark}
  Если мы разрешим шарики повторять, то получим размещения с повторениями:
  \[
    \overline{A^{k}_{n}} = n \cdot n \cdot \ldots \cdot n = n^{k}.
  \]
\end{remark}

\begin{definition}[Сочетания $k$ элементов из $n$]
  Число $k$-элементных подмножеств из $n$-элементов множества -- $C^{k}_{n}$:
  \[
    C^{k}_{n} = \frac{A^{k}_{n}}{k!} = \frac{n!}{(n-k)!k!}.
  \]
\end{definition}

\begin{eg}
  $\sum_{k=0}^{n}C^{k}_{n} = 2^n$ ($n$ -- число всех подмножеств),
  \begin{figure}[H]
    \centering
    \incfig[0.5]{fig-3}
    \label{fig:fig-3}
  \end{figure}

  Каждая полоска взаимно однозначно задает подмножество $\implies $ есть биекция между подмножествами и полосками $\implies $ число подмножеств равно числу полосок и равно
  \[
    \overline{A_{2}^{n}} = 2^n.
  \]
\end{eg}

\begin{eg}
  Хотим разложить $n$ одинаковых монет по $k$ кошелькам (различным):
  \begin{enumerate}
    \item Нет пустых кошельков.
    \[
      \underbrace{\circ \circ \circ \ldots \circ}_{n \text{ монет}}
    \]
         
    Будем ставить перегородки, монеты между перегородками попадают в один кошелек.
    \[
      \begin{array}{l|l}
        \text{до }1^{\text{й}} \text{ перегородки} & 1^{\text{й}} \text{ кошелек} \\
        \text{от }1^{\text{й}} \text{ до } 2^{\text{й}} \text{ перегородки} & 2^{\text{й}} \text{ кошелек} \\
        \vdots \\
        \text{от }k-1^{\text{й}} \text{ до } k^{\text{й}} \text{ перегородки} & k^{\text{й}} \text{ кошелек} \\
      \end{array}
    \]
    Есть биекция между разложением монет по кошелькам и расстановкой перегородок.

    Считаем способы расстановки перегородок.

    Перегородки ставятся по одной между монетами.

    \[
      C_{n-1}^{k-1} \text{ -- число способов поставить }k-1 \text{ перегородку и }n-1 \text{ мест}.
    \]
    \item Могут быть пустые кошельки.
    \[
      \underset{n+k-1 \text{ -- клетка}}{\boxed{\square \square \square \ldots \square \square}}                
    \]

    Закрасим $k-1$ клетку, отличающую перегородку.

    Между $k$-ой и $(k+1)$-ой крашенными клетками лежат монеты.

    Тогда есть биекция между крашенными полосками и монетами в кошельках.
    \[
      \text{Таких полосок } \begin{array}{ccc}
        C_{n+k-1}^{k-1} & = & \overline{C_{k}^{n}} \\
        \verteq & & \\
        C_{n+k-1}^{n} 
      \end{array}
    \]
  \end{enumerate}
\end{eg}

\begin{center}
  \begin{tabular}{lcc} 
    \toprule
    & Без повторений & С повторениями \\ 
    \midrule
    Важен порядок & $A_{n}^{k} = \frac{n!}{(n-k)!}$ & $\overline{A_{n}^{k}} = n^k$ \\ 
    Не важен порядок & $C_{n}^{k} = \frac{n!}{(n-k)!k!}$ & $\overline{C_{n}^{k}} = C_{n+k-1}^{k}$ \\ 
    \bottomrule
  \end{tabular}
\end{center}

\begin{eg}[Гипергеометрическое распределение]
  $N$ различных шариков (есть номера). Из них $M$ белых шариков.

  Вынимаем $n$ шариков,
  \[
    P(\text{ровно }k \text{ белых среди }n) =\ ?
  \]

  Исход -- неупорядоченный набор из $n$ шаров (всего $N$); шары не повторяются,
  \[
    A = \text{``ровно $k$ белых''},\qquad \abs{A} = C_{M}^{k}\cdot C_{N-M}^{n-k},
  \]
  \[
    P(A) = \frac{C_{M}^{k}\cdot C_{N-M}^{n-k}}{C_{N}^{n}} = \frac{M! (N-M)! n! (N-n)!}{k! (M-k)! (n-k)! (N-M-n+k)! N!}.
  \]
\end{eg}

\section{Геометрическая схема}

Пусть $\mu$ -- \emph{мера в $\R^n$} (чаще всего Жордана).

$\Omega$ -- некоторое измеримое множество. \emph{Исход} -- точка $\Omega$, все исходы равновозможны.

$\mathcal{F}$ -- $\sigma$-алгебра измеримых подмножеств $\Omega$,
\[
  A \in \mathcal{F} \implies P(A) = \frac{\mu(A)}{\mu(\Omega)}.
\]

\begin{eg}
  Палку 1м ломаем на 2 части, затем большую часть ломаем еще на две:
  \[
    P(\text{из них составляем треугольник})
  \]
  \begin{figure}[H]
    \centering
    \incfig[0.5]{fig-4}
    \label{fig:fig-4}
  \end{figure}

  \[
    \begin{array}{ll}
      x\text{ -- место }1^{\text{го}}\text{ разлома}, \ x \leqslant \frac{1}{2} \\
      y\text{ -- место }2^{\text{го}}\text{ разлома}
    \end{array}
  \]
  
  Исход -- пара чисел $(x;y)$ -- точка с координатами $(x;y)$.

  $\Omega$ -- все исходы; $0 \leqslant x \leqslant \frac{1}{2}, \ x \leqslant y \leqslant 1$.

  \begin{figure}[H]
    \centering
    \incfig{fig-5}
    \label{fig:fig-5}
  \end{figure}

  Длины сторон: $x,\ y-x,\ 1-y$,
  \[
    \left\{\begin{array}{ll}
      x + y - x &> 1-y \\
      x+1 - y &> y-x \\
      y-x+1-y&>x
    \end{array}\right., \qquad \begin{array}{l}
      y > \frac{1}{2} \\
      y < x + \frac{1}{2} \\
      x < \frac{1}{2}
    \end{array},
  \]
  \[
    P = \frac{\mu(A)}{\mu(\Omega)} = \frac{\frac{1}{8}}{\frac{3}{8}} = \frac{1}{3}.
  \]
\end{eg}

\begin{eg}
  $AB$ -- хорда, $P(AB > \sqrt{3})$,
  \begin{figure}[H]
    \centering
    \incfig[0.5]{fig-6}
    \label{fig:fig-6}
  \end{figure}

  \begin{enumerate}
    \item Исход -- точка $B$ -- точка на окружности.

    $\Omega$ -- вся окружность, $\abs{\Omega} = 360\degree \implies P(AB > \sqrt{3}) = \frac{1}{3}$.
    \begin{figure}[H]
      \centering
      \incfig[0.5]{fig-7}
      \label{fig:fig-6}
    \end{figure}

    \item $AB$ -- всегда вертикальна.

    Исход -- точка $M$ на диаметре. $\Omega$ -- диаметр, $\abs{\Omega} = 2 \implies P(AB > \sqrt{3}) = \frac{1}{2}$.
    \begin{figure}[H]
      \centering
      \incfig[0.5]{fig-8}
      \label{fig:fig-8}
    \end{figure}

    \item Исход -- точка $M$ -- середина хорды.

    $\Omega$ -- круг, $\abs{\Omega} = \pi \implies P(AB > \sqrt{3}) = \frac{\pi(\frac{1}{2})^2}{\pi} = \frac{1}{4}$.
    \begin{figure}[ht]
      \centering
      \incfig[0.5]{fig-9}
      \label{fig:fig-9}
    \end{figure}
  \end{enumerate}
\end{eg}
