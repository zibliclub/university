\lesson{1}{от 14 фев 2024 8:45}{Введение}


\setcounter{section}{-1}
\section{Введение}

\emph{Массовое явление} -- явление, для которого можно неоднократно повторить исходные условия.

\emph{Случайное событие} -- результат эксперимента.
\begin{figure}[H]
  \centering
  \incfig[0.5]{fig-1}
  \label{fig:fig-1}
\end{figure}

$\Omega$ -- множество всех элементарных случайных событий (элементарных исходов), $w \in \Omega$ -- \emph{элементарный исход}.

\begin{definition}[Благоприятный элементарный исход]
  Пусть $A$ -- исходное событие, $w \in \Omega$ -- \emph{благоприятный} для $A$, если $w$ влечет $A$.

  Тогда $A$ -- это \emph{подмножество $\Omega$ с подмножеством всех благоприятных для $A$ исходов}.
\end{definition}

\begin{note}
  $A,B$ -- случайные события ($A,B \subset \Omega$).
  \[
    \begin{array}{ccccc}
      \text{Не } A & = & \overline{A} & = & \Omega \setminus A \\
      A \text{ и } B & = & A \cdot B & = & A \cap B, \\
      A \text{ или } B & = & A + B & = & A \cup B.
    \end{array} 
  \]
  $\Omega$ -- достоверное, $\varnothing = \overline{\Omega}$ -- невозможное событие.
\end{note}

\begin{definition}[Алгебра]
  $\mathcal{F}$ -- семейство подмножеств $\Omega, \ \mathcal{F}$ -- \emph{алгебра}, если 
  \begin{enumerate}
    \item $\Omega \in \mathcal{F} \ (\phi \in \mathcal{F})$.
    \item $A \in \mathcal{F}\implies \overline{A} \in \mathcal{F}$.
    \item $A,B \in \mathcal{F}\implies AB \in \mathcal{F}, \ A+B\in \mathcal{F}$.

    Если, кроме этого, верное еще и
    \item $\forall \{A_{\alpha}\}\subset \mathcal{F}$
    \[
      \bigcap_{\alpha}A_{\alpha}\in \mathcal{F} \text{ и } \bigcup_{\alpha}A_{\alpha}\in \mathcal{F}.
    \]
  \end{enumerate}
  то $\mathcal{F}$ -- $\sigma$-алгебра.
\end{definition}

\begin{remark}
  Случайные события должны образовывать $\sigma$-алгебру.
\end{remark}

\begin{remark}
  Очевидно, что 
  \[
    \overline{\sum_{\alpha}A_{\alpha}}=\prod_{\alpha}\overline{A_{\alpha}} \text{ и } \overline{\prod_{\alpha}A_{\alpha}} = \sum_{\alpha}\overline{A_{\alpha}}.
  \]
\end{remark}

\begin{remark}
  \[
    \begin{array}{c}
      A \implies B \text{ то же самое, что и } A \leqslant B \\
      A \iff B \text{ то же самое, что и } A = B
    \end{array}
  \]
\end{remark}

\begin{definition}[Вероятностное пространство]
  \emph{Вероятностное пространство} ($\Omega,\mathcal{F},P$), где $\Omega$ -- множество элементарных исходов, $\mathcal{F}$ -- $\sigma$-алгебра подмножеств $\Omega$, $P$ -- мера на $\mathcal{F}$, $P: \mathcal{F} \rightarrow \R $.
  \begin{description}
    \item[\circled{$A_1$}] $\forall A \in \mathcal{F} \ P(A) \geqslant 0$.
    \item[\circled{$A_2$}] $P(\Omega) = 1$ (условие нормировки). 
    \item[\circled{$A_3$}] $\forall A,B \in \mathcal{F} \ AB = \varnothing \implies P(A+B) = P(A) + P(B)$.
    \item[\circled{$A_4$}] $\{A_{n}\}\subset \mathcal{F} \ A_{n+1} \leqslant A_n, \ \bigcap_n A_n = \varnothing$
    \[
      \lim_{n \rightarrow \infty}P(A_n) = 0 \text{ (непрерывность меры)}. 
    \]
  \end{description}
\end{definition}

\begin{theorem}[Свойства вероятностей]
  $(\Omega,\mathcal{F},P)$ -- вероятностное пространство.
  \begin{enumerate}
    \item $A \in \mathcal{F}\implies P(\overline{A}) = 1 - P(A)$.
    \begin{proof}
      $B = \overline{A}, \ AB = \varnothing, \ A+B = \Omega,$
      \[
        1 = P(\Omega) = P(A+\overline{A}) = P(A) + P(\overline{A}).
      \]
    \end{proof}

    Следовательно, $P(\varnothing) = 0$.

    \item $A \subseteq B \implies P(A) \leqslant P(B)$.
    \begin{proof}
      $C = B \setminus A = B \cap \overline{A} \in \mathcal{F}, \ B = A + C, \ AC = \varnothing$,
      \[
        P(B) = P(A) + \underset{\geqslant 0}{P(C)} \geqslant P(A).
      \]
    \end{proof}

    Следовательно, $\forall A \in \mathcal{F} \ 0 \geqslant P(A) \geqslant 1 \ (\phi \subseteq A \subseteq \Omega)$.

    \item $A_1,\ldots,A_n \in \mathcal{F}, \ A_iA_j = \varnothing \ (i\ne j)$.

    Тогда $P(A_1 + \ldots + A_n) = \sum_{k=1}^{n}P(A_k)$.
      \begin{proof}
        Индукция по $n$.
      \end{proof}

    \item $A_1,\ldots,A_n \in \mathcal{F}\implies P(A_1 + \ldots + A_n) \leqslant \sum_{k=1}^{\infty}P(A_k)$.
    \begin{proof}
      $B_k = A_k \setminus \left(\sum_{i=1}^{k-1}A_i\right), \ \sum_{k=1}^{n}A_k = \sum_{k=1}^{n}B_k$.
      \[
        P \left(\sum A_k\right) = P(\sum B_k) = \sum P(B_k) \leqslant \sum P(A_k) \ (B_k \leqslant A_k).
      \]
    \end{proof}

    \item $P(A \cup B) = P(A) + P(B) - P(AB)$.
    \begin{proof}
      $C = A \setminus B, \ P(A) = P(C + AB) = P(C) + P(AB)$,
      \[
        P(C) = P(A) - P(AB),
      \]
      \[
        P(A \cup B) = P(B + C) = P(B) + P(C) = P(B) + P(A) - P(AB).
      \]
    \end{proof}
  \end{enumerate}
\end{theorem}

\begin{description}
  \item[\circled{$A_3^*$} ($G$-аддитивность)]
  \[
    \{A_n\}\subset \mathcal{F}, \ A_iA_j = \varnothing \implies P \left(\sum_{n=1}^{\infty}A_n\right) = \sum_{n=1}^{\infty }P(A_n).
  \]
\end{description}

\begin{theorem}
  \circled{$A_1$},\circled{$A_2$},\circled{$A_3$} и \circled{$A_4$} $\iff$ \circled{$A_1$},\circled{$A_2$},\circled{$A_3^*$}.
\end{theorem}

\begin{proof}\leavevmode
  \begin{description}
    \item[$\boxed{\Rightarrow}$] Покажем, что $\circled{$A_3$}$ и $\circled{$A_4$} \implies \circled{$A_3^*$}$: $\{A_n\}\subset \mathcal{F}, \ A_iA_j = \varnothing$,
    \[
      B_n = \sum_{k=n+1}^{\infty }A_k, \qquad A = \sum_{k=1}^{\infty }A_k,
    \]
    \[
      \begin{array}{c}
        A = A_1 + \ldots + A_n + B_n, \\
        P(A) = P(A_1) + \ldots + P(A_n) + P(B_n).
      \end{array}
    \]
    $B_{n+1} \leqslant B_n, \ \bigcap_{n}B_n = \varnothing \implies B_n \rightarrow 0, \ P(B_n) \xrightarrow[n \rightarrow \infty ]{} 0$,
    \[
      \begin{array}{ccc}
        P(A) & = & \sum_{k=1}^{n}P(A_k) + P(B_n) \rightarrow \sum_{n=1}^{\infty }P(A_n) + 0 \\
        \downarrow & & \\
        P(A) & & 
      \end{array}.
    \]

    \item[$\boxed{\Leftarrow}$] Пусть есть \circled{$A_3^*$}. Построим последовательность $A_1,\ldots,A_n,\varnothing,\ldots,\varnothing,\ldots$:
    \[
      \begin{array}{ccc}
        P \left(\sum_{n=1}^{\infty }A_n\right) & = & P \left(\sum_{k=1}^{n}A_k\right) \\
        \verteq & & \\
        \sum_{n=1}^{\infty }P(A_n) & = & \sum_{k=1}^{n}P(A_k)
      \end{array} \implies \circled{$A_3$}.
    \]
    \begin{figure}[H]
      \centering
      \incfig[0.5]{fig-2}
      \label{fig:fig-2}
    \end{figure}

    $\{A_n\} \subset \mathcal{F}, \ A_n \geqslant A_{n+1}$ и $\bigcap A_n = \varnothing $.

    $B_{n+1} = A_n \setminus A_{n+1}, \ B_iB_j = \varnothing, \ \bigcup B_n = \bigcup A_n, \qquad B_1 = A_1$,
    \[
      \begin{array}{ccc}
        P \left(\sum_{n=1}^{\infty }A_n\right) & = & P \left(\sum_{n=1}^{\infty }B_n\right) = \sum_{n=1}^{\infty }P(B_n) \text{ -- сходится} \\
        \verteq & & \\
        P(A_1) & &
      \end{array},
    \]
    \[
      \begin{array}{r}
        A_n = \sum_{k=n+1}^{\infty }B_k \implies  P(A_n) = \sum_{k=n+1}^{\infty }P(B_k) \rightarrow 0 \\
        \left|\sum_{k=1}^{\infty }B_k - \sum_{k=1}^{n}P(B_k)\right| \rightarrow 0
      \end{array} \implies \circled{$A_4$}. 
    \]
  \end{description} 
\end{proof}

\begin{eg}
  $\Omega = \{B,H\}, \ \mathcal{F} = \{\varnothing ,\Omega,\{B\},\{H\}\}$,
  \[
    P(\Omega) = 1, \qquad P(\varnothing ) = 0,
  \]
  \[
    \begin{array}{ccc}
      \circled{М} & \qquad & \circled{Ж} \\
      P(B) = 0 & \qquad & P(B) = \frac{1}{2} \\
      P(H) = 1 & \qquad & P(H) = \frac{1}{2}
    \end{array}
  \]
\end{eg}
