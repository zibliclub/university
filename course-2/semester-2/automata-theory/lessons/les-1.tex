\lesson{1}{от 12 фев 2024 8:45}{Регулярные языки и регулярные выражения}


\begin{definition}[Алфавит, слово]
    $ \Sigma $ -- \emph{алфавит} (как правило, конечный),
    \[
        \Sigma = \{a_1,\ldots,a_n\}.
    \]

    \emph{Слово над алфавитом} $ \Sigma $ -- конечный упорядоченный набор символов из $ \Sigma $.

    $ \Sigma^* $ -- все слова над $ \Sigma $.
\end{definition}

\begin{example}
    $ \Sigma = \{\text{а,б,}\ldots \text{,я}\} $, слова: яблоко, абвежр.

    $ \epsilon $ -- пустое слово.
\end{example}

\begin{definition}[Язык]
    \emph{Язык} $ L $ -- подмножество слов над $ \Sigma $,
    \[
        L \subseteq \Sigma^*.
    \]
\end{definition}

\begin{example}
    $ \Sigma_2 = \{0,1\}, \quad \Sigma_2^* = \{\underset{\epsilon}{\emptyset},1,\emptyset\emptyset,\emptyset1,1\emptyset,\ldots\} $
\end{example}

\begin{remark}[Как «конечно» описать бесконечный язык?]\leavevmode
    \begin{enumerate}
        \item Конечный набор правил построения языков.
        \item Алгоритм-распознаватель:
              \begin{figure}[H]
                  \centering
                  \incfig{fig_01}
                  \label{fig:fig_01}
              \end{figure}
    \end{enumerate}
\end{remark}

\begin{remark}[Конструкции]\leavevmode
    \begin{enumerate}
        \item $ L_1,L_2 $ -- языки,
              \[
                  L_1 \cup L_2 \text{ -- объединение,}
              \]
              \[
                  L_1 \cap L_2 \text{ -- пересечение,}
              \]
              \[
                  \overline{L_1} = \Sigma^* \setminus L_1 \text{ -- дополнение.}
              \]
        \item Конструкция: $ w_1,w_2 \longrightarrow w_1w_2 $,
              \[
                  L_1,L_2 \longrightarrow L_1L_2 = \{w_1w_2,\quad w_1\in L_1, \ w_2 \in L_2\}.
              \]

              \begin{example}
                  \[
                      \begin{array}{l}
                          L_1 = \{a\} \\
                          L_2 = \{b\}
                      \end{array} \quad L_1L_2 = \{ab\}, \ \epsilon b = b.
                  \]
                  \[
                      \begin{array}{rc}
                          L_1L_2 = & \{ab\}   \\
                                   & \vertneq \\
                          L_2L_1 = & \{ba\}
                      \end{array}
                  \]
              \end{example}

              \begin{example}
                  \[
                      \begin{array}{l}
                          L_1 = \{a,b\} \\
                          L_2 = \{a,b\}
                      \end{array}, \quad \underset{\text{степень}}{L_1L_1} = L_1L_2 = L_1^2 = \{aa,ab,ba,bb\},
                  \]
                  \[
                      L_1^2,L_1^3 = L_1^2L_1 = \left\{\begin{array}{l}
                          aaa,aba,baa,bba, \\
                          aab,abb,bab,bbb
                      \end{array}\right\}
                  \]
              \end{example}

        \item Итерация (Ж. Клани)
              \[
                  L \longrightarrow L^* = \bigcup_{n=\emptyset}^{\infty}L^n, \quad \boxed{L^1 \subset L},
              \]
              \[
                  L^0 = \{\epsilon\}.
              \]

              \begin{example}
                  $ \Sigma = \{a,b\} $,
                  \[
                      L = \{a,b\}, \quad L^* = \left\{\begin{array}{l}
                          \epsilon,a,b,aa,ab,ba,bb, \\
                          aaa,\ldots,bbb,\ldots
                      \end{array}\right\}.
                  \]
              \end{example}
    \end{enumerate}
\end{remark}

\begin{definition}[Регулярная языка, регулярные языки]\leavevmode
    \begin{enumerate}
        \item $ \emptyset, \{\epsilon\}, \{a_i\}, \ a_i \in \Sigma $ -- \emph{регулярная языка}.
        \item $ L_1,L_2 $ -- \emph{регулярные языки},
              \[
                  L_1 \cup L_2, \ L_1L_2, \ L_1^* \text{ -- тоже регулярные языки}.
              \]
    \end{enumerate}
\end{definition}

\begin{example}
    $ L = \left(\{a\}\equalto{\big(\{a\}\cup\{b\}\big)^*}{\text{все слова из }a,b}\{b\}\right)^* $ -- все слова, начинающиеся на $ a $ и заканчивающиеся на $ b $, например:
    \[
        \underbrace{abbab}\underbrace{ab}\underbrace{aab}.
    \]
\end{example}

\begin{definition}[Регулярное выражение]\leavevmode
    \begin{enumerate}
        \item $ \emptyset,\{\epsilon\},\{a_i\}, \ a_i \in \Sigma $ -- \emph{регулярные выражения}.
        \item Если $ R_1,R_2 $ -- регулярные выражение, то
              \[
                  R_1 + R_2, \ R_1R_2, \ (R_1)^* \text{ -- тоже регулярные выражения}.
              \]
    \end{enumerate}
\end{definition}

\begin{example}
    Любой конечный язык -- регулярный.
\end{example}

\begin{example}
    $ \Sigma = \{a,b\}, \ L_1 = \{\text{все слова из }a,b \text{ четной длины (включая $ \epsilon $)}\} $:
    \[
        \left\{\begin{array}{l}
            \epsilon,aa,ab,ba,bb, \\
            aaaa,\ldots,bbbb,\ldots
        \end{array}\right\} = \bigl((a+b)(a+b)\bigr)^*.
    \]

    $ \Sigma = \{a,b\}, \ L_2 = \{w: \ \text{в }w \text{ четное число }a \text{ и }b\} $
    \[
        L_2 \subset L_1 \quad \left(\begin{array}{l}
                ab \in L_1 \\
                ab \notin L_2
            \end{array}\right)
    \]
    \[
        \bigl((aa + bb)^*(ab + ba)(aa + bb)^*(ab + ba)(aa + bb)^*\bigr)^*
    \]

    $ L_3 = \{w: \ \text{в }w\text{ число }a\text{ четно}\} $
    \[
        b\ldots b a b \ldots b a b \ldots b = \bigl(b^*ab^*ab^*\bigr)^*
    \]
\end{example}

\begin{example}[Нерегулярный язык]
    $ L = \{\text{симметричные слова из }a,b\} $
    \[
        abba, \ aba, \ aa.
    \]
\end{example}