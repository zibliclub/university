\usepackage[utf8]{inputenc}
\usepackage[T2A]{fontenc}
\usepackage[russian]{babel}

\usepackage{amsmath}
\usepackage{amssymb}
\usepackage{amsthm}
\usepackage{float}
\usepackage{tikz}
\usepackage{booktabs} % tabular \toprule, ...
\usepackage{gensymb} % \degree
\usepackage{mathtools} % for \coloneqq
\usepackage{multicol}


% for title
\author{
  Основано на лекциях \lecturer \\
  \small Конспект написан Заблоцким Данилом
}
\date{\term\ \year}
\title{\course}

\makeatletter

\let\@real@maketitle\maketitle
\renewcommand{\maketitle}{
  {\let\newpage\relax\@real@maketitle}
  \begin{center}
    \begin{minipage}[c]{0.9\textwidth}
      \centering\footnotesize Эти записи не одобряются лекторами, и я вношу в них изменения (часто существенно) после лекций. Они далеко не точно отражают то, что на самом деле читалось, и, в частности, все ошибки почти наверняка мои.
    \end{minipage}
  \end{center}
}

\let\@real@tableofcontents\tableofcontents
\renewcommand{\tableofcontents}{\@real@tableofcontents\newpage}

\makeatother


% hyperref
\usepackage{url}

\usepackage{hyperref}
\hypersetup{
  colorlinks,
  linkcolor={black},
  citecolor={black},
  urlcolor={blue!80!black}
}


% horizontal rule
\newcommand\hr{
  \noindent\rule[0.5ex]{\linewidth}{0.5pt}
}


% theorems
\usepackage{thmtools}
\usepackage[framemethod=TikZ]{mdframed}
\mdfsetup{skipabove=1em,skipbelow=0em, innertopmargin=5pt, innerbottommargin=6pt}

\theoremstyle{definition}

\declaretheoremstyle[
  headfont=\bfseries\sffamily,
  bodyfont=\normalfont,
  mdframed={ nobreak }
]{thmbox}

\declaretheoremstyle[
  headfont=\bfseries\sffamily,
  bodyfont=\normalfont
]{thmunbox}

\declaretheoremstyle[
  headfont=\bfseries\sffamily,
  bodyfont=\normalfont,
  numbered=no,
  mdframed={ rightline=false, topline=false, bottomline=false, },
  qed=\qedsymbol
]{thmproofline}

\declaretheorem[style=thmbox, name=Определение]{definition}
\declaretheorem[style=thmbox, name=Следствие]{corollary}
\declaretheorem[style=thmbox, name=Предложение]{prop}
\declaretheorem[style=thmbox, name=Теорема]{theorem}
\declaretheorem[style=thmbox, name=Лемма]{lemma}
\declaretheorem[style=thmbox, name=Критерий]{crit}

\declaretheorem[numbered=no, style=thmproofline, name=Доказательство]{replacementproof}
\declaretheorem[style=thmunbox, numbered=no, name=Упражнение]{ex}
\declaretheorem[style=thmunbox, numbered=no, name=Пример]{eg}
\declaretheorem[style=thmunbox, numbered=no, name=Замечание]{remark}
\declaretheorem[style=thmunbox, numbered=no, name=Примечание]{note}

\renewenvironment{proof}[1][\proofname]{\begin{replacementproof}}{\end{replacementproof}}

\AtEndEnvironment{eg}{\null\hfill$\diamond$}

\newtheorem*{notation}{Обозначение}
\newtheorem*{previouslyseen}{Как было замечено ранее}
\newtheorem*{problem}{Проблема}
\newtheorem*{observe}{Наблюдение}
\newtheorem*{property}{Свойство}
\newtheorem*{intuition}{Предположение}

\usepackage{etoolbox}
\AtEndEnvironment{vb}{\null\hfill$\diamond$}
\AtEndEnvironment{intermezzo}{\null\hfill$\diamond$}


% lesson
\usepackage{xifthen}

\def\testdateparts#1{\dateparts#1\relax}
\def\dateparts#1 #2 #3 #4 #5\relax{
  \marginpar{\small\textsf{\mbox{#1 #2 #3 #5}}}
}

\def\@lesson{}
\newcommand{\lesson}[3]{
  \ifthenelse{\isempty{#3}}{
    \def\@lesson{Лекция #1}
  }{
    \def\@lesson{Лекция #1: #3}
  }
  \subsection*{\@lesson}
  \testdateparts{#2}
}


% fancy headers
\usepackage{fancyhdr}
\pagestyle{fancy}

\fancyhead[RO,LE]{\course}
\fancyhead[RE,LO]{\@lesson}
\fancyfoot[LE,RO]{\thepage}
\fancyfoot[C]{\leftmark}
\renewcommand{\headrulewidth}{0.4pt}


% incfig
\usepackage{import}
\usepackage{pdfpages}
\usepackage{transparent}
\usepackage{xcolor}

\newcommand{\incfig}[2][1]{%
  \def\svgwidth{#1\columnwidth}
  \import{./figures/}{#2.pdf_tex}
}

\pdfsuppresswarningpagegroup=1


% custom commands
\let\epsilon\varepsilon

\newcommand\N{\ensuremath{\mathbb{N}}}
\newcommand\R{\ensuremath{\mathbb{R}}}
\newcommand\Z{\ensuremath{\mathbb{Z}}}
\newcommand\Q{\ensuremath{\mathbb{Q}}}
\renewcommand\C{\ensuremath{\mathbb{C}}}
\newcommand{\diff}{\ensuremath{\operatorname{d}\!}}
\renewcommand{\d}{\operatorname{d}}
\renewcommand{\Re}{\operatorname{Re}}
\renewcommand{\Im}{\operatorname{Im}}
\newcommand{\Arg}{\operatorname{Arg}}
\newcommand{\dist}{\operatorname{dist}}
\newcommand{\cl}{\operatorname{cl}}
\newcommand{\Cl}{\operatorname{Cl}}

\newcommand{\abs}[1]{\left\lvert #1\right\rvert}
\newcommand{\verteq}[0]{\rotatebox{90}{$=$}}
\newcommand{\vertneq}[0]{\rotatebox{90}{$\ne$}}
\newcommand{\equalto}[2]{\underset{\scriptstyle\overset{\mkern4mu\verteq}{#2}}{#1}}
\newcommand{\nequalto}[2]{\underset{\scriptstyle\overset{\mkern4mu\vertneq}{#2}}{#1}}
\newcommand{\RomanNumeralCaps}[1]{\MakeUppercase{\romannumeral #1}}

\newcommand*\circled[1]{
  \tikz[baseline=(char.base)]{
    \node[shape=circle,draw,inner sep=1pt] (char) {#1};
  }
}


% for \xrightrightarrows
\makeatletter

\newcommand*{\relrelbarsep}{.450ex}
\newcommand*{\relrelbar}{
  \mathrel{
    \mathpalette\@relrelbar\relrelbarsep
  }
}
\newcommand*{\@relrelbar}[2]{
  \raise#2\hbox to 0pt{$\m@th#1\relbar$\hss}
  \lower#2\hbox{$\m@th#1\relbar$}
}

\providecommand*{\rightrightarrowsfill@}{
  \arrowfill@\relrelbar\relrelbar\rightrightarrows
}
\providecommand*{\leftleftarrowsfill@}{
  \arrowfill@\leftleftarrows\relrelbar\relrelbar
}
\providecommand*{\xrightrightarrows}[2][]{
  \ext@arrow 0359\rightrightarrowsfill@{#1}{#2}
}
\providecommand*{\xleftleftarrows}[2][]{
  \ext@arrow 3095\leftleftarrowsfill@{#1}{#2}
}

\makeatother
