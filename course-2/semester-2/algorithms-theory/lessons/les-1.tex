\lesson{1}{от 13 фев 2024 12:45}{Начало}


\begin{note}[История]
    ?

    $ \begin{array}{ll}
            \boxed{1936\text{г.}} & \text{алгоритм Евклида (>2 тыс.)}             \\
                                  & \text{алгоритм сложения, умножения (>1 тыс.)} \\
                                  & \text{метод Гаусса}                           \\
                                  & \vdots
        \end{array} $

    $ \begin{array}{ll}
            \boxed{1900\text{г.}} & \text{Гильберт}
        \end{array} $

    \begin{center}
        \underline{аксиомы} $ \longrightarrow $ теоремы
    \end{center}

    $ \begin{array}{ll}
            \boxed{1931\text{г.}} & \text{Гёдель}
        \end{array} $

    $ \begin{array}{ll}
            \boxed{1936\text{г.}} & \text{формализация понятия алгоритма: модели вычислений} \\
                                  & \text{Чёрч -- }\lambda\text{-исчисление}                 \\
                                  & \text{Тьюринг -- машина Тьюринга}                        \\
                                  & \text{Пост -- машина Поста}                              \\
                                  & \text{Марков -- алгорифмы Маркова}                       \\
                                  & \text{Клини -- рекурсивные функции}
        \end{array} $

    \[
        \begin{array}{ccccc}
                                                                       &             & \boxed{\begin{array}{c}
                                                                                                      \text{поиск}         \\
                                                                                                      \text{неразрешенных} \\
                                                                                                      \text{проблем}
                                                                                                  \end{array}}                                                \\
                                                                       & \nearrow    &                           &                      &                         \\
            \underset{\text{понятие алгоритма}}{\boxed{1936\text{г.}}} & \rightarrow & \boxed{\begin{array}{c}
                                                                                                      \text{абстрактная} \\
                                                                                                      \text{теория}      \\
                                                                                                      \text{алгоритмов}
                                                                                                  \end{array}  }   &                      &                           \\
                                                                       & \searrow    &                           &                      & \boxed{\begin{array}{c}
                                                                                                                                                         \text{сложность} \\
                                                                                                                                                         \text{вычисления}
                                                                                                                                                     \end{array}} \\
                                                                       &             & \boxed{\text{компьютеры}} & \begin{matrix}
                                                                                                                       \nearrow \\ \searrow
                                                                                                                   \end{matrix} &                            \\
                                                                       &             &                           &                      & \boxed{\begin{array}{c}
                                                                                                                                                         \text{сложность} \\
                                                                                                                                                         \text{вычисления}
                                                                                                                                                     \end{array}} \\
        \end{array}
    \]
\end{note}

\begin{note}
    \[
        A = \{a_1,\ldots,a_m\} \text{ -- рабочий алфавит}
    \]
\end{note}

\begin{definition}[Машина Тьюринга]
    \emph{Машина Тьюринга} (МТ) над алфавитом $ A $ состоит из:
    \begin{enumerate}
        \item Бесконечной в обе стороны ленты, разбитой на ячейки. Ячейка может быть пустой (записан $ \square $), или содержать символ из $ A $.
              \begin{figure}[H]
                  \centering
                  \incfig{fig_01}
                  \label{fig:fig_01}
              \end{figure}
        \item Каретка, которая двигается над лентой, читает и пишет символы в ячейки.
        \item Внутренние состояния:
              \[
                  q_0,q_1,q_2,\ldots,q_n \quad \Bigg| \ \begin{array}{l}
                      q_0 \text{ -- конечное} \\
                      q_1 \text{ -- начальное}
                  \end{array}
              \]
        \item Программа -- набор правил вида:
              \[
                  (q_i,a)\longrightarrow (q_j,b,S),
              \]
              где $ \begin{array}{ll}
                      q_i \text{ -- любое состояние }\ne q_0       \\
                      a,b \text{ -- символы из }A \cup \{\square\} \\
                      q_j \text{ -- набор состояний }              \\
                      S \text{ -- сдвиг }R \text{ и }L
                  \end{array} $,
              по одному правилу: $ \forall $ комбинации $ (q_i,a) $
              \[
                  q_i \ne q_0, \quad a \in A \cup \{\square\}.
              \]
    \end{enumerate}
\end{definition}

\begin{definition}[Работа МТ]
    \emph{Работа МТ} $ M $ на слове $ w \in A^* $:
    \begin{enumerate}
        \item (на рисунке)
              \begin{figure}[H]
                  \centering
                  \incfig{fig_02}
                  \label{fig:fig_02}
              \end{figure}
        \item Согласно программе $ M $ работает.
        \item $ M $ останавливается, если она нападает в $ q_0 $
              \[
                  \bigl(M(w)\downarrow\bigr)
              \]
              и результат работы $ M(w) $ -- это слово, которое остается записанным. Иначе $ M $ не останавливается на $ w $
              \[
                  \bigl(M(w)\uparrow\bigr).
              \]
    \end{enumerate}
\end{definition}

\begin{definition}[Вычисление фукнкции МТ]
    МТ $ M $ \emph{вычисляет функцию} $ f_M: A^* \longrightarrow A^* $, если $ \forall w \in A^* $ если $ f_M(w) $ определена, то $ M(w)\downarrow $ и $ M(w) = f_M(w) $, а если $ f_M(w) $ не определена, то $ M(W)\uparrow $.
\end{definition}

\begin{note}[Тезис Тьюринга]
    Если $ f: A^* \longrightarrow A^* $ вычислима интуитивно, то $ \exists $ МТ $ M $, которая ее вычисляет.
\end{note}

\begin{example}
    \[
        f(w) = \left\{\begin{array}{ll}
            1, & \text{если }| w |\emph{ -- четная длина }w \\
            0, & \text{иначе}
        \end{array}\right.
    \]
    \begin{figure}[H]
        \centering
        \incfig{fig_03}
        \label{fig:fig_03}
    \end{figure}
    \[
        \begin{array}{l}
            q_1 \implies \text{четная} \\
            q_2 \implies \text{нечетная}
        \end{array}
    \]
    \[
        \begin{array}{ll}
            (q_1,0) \longrightarrow (q_2,\square,R) & (q_2,0) \longrightarrow (q_1,\square,R) \\
            (q_1,1) \longrightarrow (q_2,\square,R) & (q_2,1) \longrightarrow (q_1,\square,R) \\
            (q_1,\square) \longrightarrow (q_0,1,L) & (q_2,\square) \longrightarrow (q_0,0,R) \\
        \end{array}
    \]
\end{example}
