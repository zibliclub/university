\usepackage[utf8]{inputenc}
\usepackage[T2A]{fontenc}
\usepackage[russian]{babel}

\usepackage{hyperref}
\hypersetup{
	colorlinks,
	linkcolor={black},
	citecolor={black},
	urlcolor={blue!80!black}
}

\usepackage{amsmath, amsfonts, mathtools, amsthm, amssymb}
\usepackage{tikz}

% theorems
\usepackage{thmtools}
\usepackage[framemethod=TikZ]{mdframed}
\mdfsetup{skipabove=1em,skipbelow=0em, innertopmargin=5pt, innerbottommargin=6pt}

\declaretheoremstyle[
	headfont=\bfseries\sffamily,
	bodyfont=\normalfont,
	mdframed={ nobreak }
]{thmbox}

\declaretheoremstyle[
	headfont=\bfseries\sffamily,
	bodyfont=\normalfont
]{thmunbox}

\declaretheoremstyle[
	headfont=\bfseries\sffamily,
	bodyfont=\normalfont,
	numbered=no,
	mdframed={ rightline=false, topline=false, bottomline=false, },
	qed=\qedsymbol
]{thmproofline}

\declaretheorem[numbered=no, style=thmbox, name=Определение]{definition}
\declaretheorem[numbered=no, style=thmbox, name=Следствие]{corollary}
\declaretheorem[numbered=no, style=thmbox, name=Предложение]{prop}
\declaretheorem[numbered=no, style=thmbox, name=Теорема]{theorem}
\declaretheorem[numbered=no, style=thmbox, name=Лемма]{lemma}
\declaretheorem[numbered=no, style=thmbox, name=Критерий]{crit}

\declaretheorem[numbered=no, style=thmproofline, name=Доказательство]{replacementproof}
\declaretheorem[style=thmunbox, numbered=no, name=Упражнение]{ex}
\declaretheorem[style=thmunbox, numbered=no, name=Пример]{eg}
\declaretheorem[style=thmunbox, numbered=no, name=Замечание]{remark}
\declaretheorem[style=thmunbox, numbered=no, name=Примечание]{note}
\declaretheorem[style=thmunbox, numbered=no, name=Задача]{task}

\renewenvironment{proof}[1][\proofname]{\begin{replacementproof}}{\end{replacementproof}}

\AtEndEnvironment{eg}{\null\hfill$\diamond$}

\newtheorem*{notation}{Обозначение}
\newtheorem*{previouslyseen}{Как было замечено ранее}
\newtheorem*{problem}{Проблема}
\newtheorem*{observe}{Наблюдение}
\newtheorem*{property}{Свойство}
\newtheorem*{intuition}{Предположение}

\usepackage{etoolbox}
\AtEndEnvironment{vb}{\null\hfill$\diamond$}
\AtEndEnvironment{intermezzo}{\null\hfill$\diamond$}


% custom commands
\let\epsilon\varepsilon

\newcommand\N{\ensuremath{\mathbb{N}}}
\newcommand\R{\ensuremath{\mathbb{R}}}
\newcommand\Z{\ensuremath{\mathbb{Z}}}
\newcommand\Q{\ensuremath{\mathbb{Q}}}
\renewcommand\C{\ensuremath{\mathbb{C}}}
\newcommand{\diff}{\ensuremath{\operatorname{d}\!}}
\renewcommand{\d}{\operatorname{d}}
\renewcommand{\Re}{\operatorname{Re}}
\renewcommand{\Im}{\operatorname{Im}}
\newcommand{\Arg}{\operatorname{Arg}}
\newcommand{\dist}{\operatorname{dist}}
\newcommand{\cl}{\operatorname{cl}}
\newcommand{\Cl}{\operatorname{Cl}}
\newcommand{\Ln}{\operatorname{Ln}}
\newcommand{\Arcsin}{\operatorname{Arcsin}}
\newcommand{\Arccos}{\operatorname{Arccos}}
\newcommand{\const}{\operatorname{const}}
\newcommand{\dom}{\operatorname{dom}}

\newcommand{\abs}[1]{\left\lvert #1\right\rvert}
\newcommand{\verteq}[0]{\rotatebox{90}{$=$}}
\newcommand{\vertneq}[0]{\rotatebox{90}{$\ne$}}
\newcommand{\equalto}[2]{\underset{\scriptstyle\overset{\mkern4mu\verteq}{#2}}{#1}}
\newcommand{\nequalto}[2]{\underset{\scriptstyle\overset{\mkern4mu\vertneq}{#2}}{#1}}
\newcommand{\RomanNumeralCaps}[1]{\MakeUppercase{\romannumeral #1}}

\newcommand*\circled[1]{
	\tikz[baseline=(char.base)]{
		\node[shape=circle,draw,inner sep=1pt] (char) {#1};
	}
}
