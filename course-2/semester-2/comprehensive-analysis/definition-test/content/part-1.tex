\section{Модуль и аргумент коплексного числа}

\begin{definition}
  \emph{Полярные координаты} комплексного числа:
  \[
    z = x + iy,
  \]
  \emph{полярный радиус}:
  \[
    r = \sqrt{x^2 + y^2}
  \]
  и \emph{полярный угол} $\phi$, то есть угол между положительным направлением оси $OX$ и вектора $z$, соответственно называется его  \emph{молулем и аргументом}.

  Модуль определяется однозначно, а аргумент -- с точностью до слагаемого $2\pi k, \ k \in \Z$.
\end{definition}

\section{Алгербраическая, показательная и тригонометрическая формы записи комплексного числа}

Алгебраическая форма записи:
\[
	z = (x,y) = x + iy, \quad x,y \in \R,
\]
Показательная форма записи:
\[
	z = \abs{z} \cdot e^{i\arg z}.
\]
Тригонометрическая форма записи:
\[
	z = \abs{z} \cdot (\cos \arg z + i \sin \arg z),
\]

\newpage

\section{Сопряжённое к комплексному числу}

\[
	\overline{z} = x - iy.
\]

\section{Сложение, умножение и деление комплексных чисел}

\[
  z_1 + z_2 = (x_1, y_1) + (x_2, y_2) = (x_1 + x_2, y_1 + y_2),
\]

\begin{multline*}
  z_1 \cdot z_2 = (x_1, y_1) \cdot (x_2, y_2) = \\
  = (x_1x_2 - y_1y_2, x_1 y_2 + x_2y_1) = \\
  = x_1x_2 - y_1y_2 + i(x_1y_2 + x_2y_1),
\end{multline*}

\[
  \frac{z_1}{z_2} = \frac{z_1 \cdot \overline{z_2}}{z_2 \cdot \overline{z_2}} = \frac{z_1 \cdot \overline{z_2}}{\abs{z_2}^{2}},
\]

\[
  z \cdot \overline{z} = \abs{z}^{2}.
\]

\section{Формула Эйлера}

\[
	e^{i \phi} = \cos \phi + i \sin \phi, \quad \forall \phi \in \R.
\]

\section{Формула Муавра}

\[
	z^n = r^n(\cos n \phi + i\sin n \phi).
\]

\begin{definition}
  \emph{Корнем $n$-ой степени} комплексного числа $z$ называется комплексное число, $n$-ая степень которого равна $z$,
  \[
    z^n = z_0,
  \]
  \[
    \sqrt[n]{z_0} = \sqrt[n]{\abs{z_0}} \cdot e^{i \frac{\arg z_0 + 2\pi k}{n}}, \quad 0 \leqslant k \leqslant n-1.
  \]
\end{definition}

\section{Расстояние между двумя конечными точками на комплексной плоскости}

\[
  \dist(z_1,z_2) \coloneq \abs{z_1 - z_2}, \text{ где }z_1,z_2 \in \C.
\]

\section{Окрестность конечной точки на комплексной плоскости}

\begin{definition}
	Множество называется \emph{окрестностью} точки, если оно содержит некоторый шарик с центром в этой точке.
	\begin{notation}
		\[
			O_z, \quad z \in \overline{\C}.
		\]
	\end{notation}
\end{definition}

\section{Окрестность бесконечно удалённой точки}

\begin{definition}
	Множество $V \subset \overline{\C}$ является \emph{окрестностью бесконечно удаленной точки}, если $\exists \epsilon > 0:$
	\[
		\big\{z \in \overline{\C}: \abs{z} > \epsilon\big\} \subset V.
	\]
\end{definition}

\section{Предельная точка множества}

\begin{definition}
	Точка называется \emph{предельной} точкой множества, если в любой ее окрестности есть точки множества, отличные от данной.
\end{definition}

\begin{remark}
	Точка является предельной точкой множества на расширенной комплексной плоскости ($\overline{\C}$) $\iff \forall $ ее окрестность содержит бесконечное число точек данного множества.
\end{remark}
