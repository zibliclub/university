\section{Внутренняя точка множества}

\begin{definition}
	$\mathfrak{D}\subset \C$. $z \in \mathfrak{D}$ называется \emph{внутренней точкой} множества $\mathfrak{D}$, если $\mathfrak{D} \in O_z$.
\end{definition}

\section{Граничная точка множества}

\begin{definition}
	Точка называется \emph{граничной} точкой множества, если в любой ее окрестности есть как точки множества, так и точки его дополнения.
	\begin{notation}
		\[
			\partial \mathfrak{D}.
		\]
	\end{notation}
\end{definition}

\newpage

\section{Предел последовательности комплексных чисел}

\begin{definition}
	Комплексное число $z_0$ называется \emph{пределом последовательности комплексных чисел} $\{z_n\}_{n \in \N}$, если $\forall \epsilon > 0 \ \exists n_0 \in \N: \ \forall n \geqslant n_0$
	\[
		\abs{z_n - z_0} < \epsilon \quad \big(\d(z_n,z_0)\xrightarrow[n \rightarrow \infty]{} 0\big),
	\]
	\[
		\underset{n \rightarrow \infty}{\lim}z_n = z_0.
	\]
\end{definition}

\section{Предел функции}

\begin{definition}
	$\mathfrak{D}\subset \dom f, \ z_0 \in \overline{\C}$ -- предельная точка $\mathfrak{D}$. Тогда $\omega_0 \in \overline{\C}$ называется \emph{пределом отображения} $f, \ \omega_0 \coloneq \underset{\mathfrak{D}\ni z \rightarrow z_0}{\lim}f(z)$, если $\forall V \in O_{\omega_0} \ \exists U \in O_{z_0}$
	\[
		f(\mathring{U} \cap \mathfrak{D}) \subset V.
	\]

	$\dom f$ -- область определения функции.
\end{definition}

\section{Непрерывность функции в точке}

\begin{definition}
	Функция $f$ называется \emph{непрерывной в точке} $z_0 \in \overline{\C}$, если
	\begin{enumerate}
		\item $z_0 \in \dom f$.
		\item $\forall \epsilon > 0 \ \exists \delta > 0: \ \forall z \in \mathfrak{D}$
		      \[
			      0 < \abs{z - z_0} < \delta \implies \abs{f(z) - \omega_0} < \epsilon.
		      \]
	\end{enumerate}
\end{definition}

\section{Производная функции в точке}

\begin{definition}
	Если $f: \C \rightarrow \C$ определена в некоторой окрестности точки $z_0 \in \C$ и $\exists \underset{z \rightarrow z_0}{\lim}\frac{f(z) - f(z_0)}{z - z_0} \eqqcolon f'(z_0)$ называется \emph{производной функции в точке} $z_0$.
\end{definition}

\newpage

\section{Равномерная сходимость последовательности функций на множестве}

\begin{definition}
	Пусть ($n\in \N$), $f_n : \C \rightarrow \C$, $D \coloneq \underset{n \in \N}{\bigcap}\dom f_n$.

	$A \subset \mathfrak{D}, \ f: A \rightarrow \C$. Говорят, что последовательность $f_n \rightrightarrows f$ на $A$, если $\forall \epsilon> 0 \ \exists n_0 \in \N: \ \forall z \in A \ \forall n \geqslant n_0$
	\[
		\big|f_n(z) - f(z)\big| < \epsilon
	\]
	\begin{multline*}
		\big(\forall \epsilon > 0 \ \exists n_0 \in \N : \ \forall n \geqslant n_0 \\
		\underset{z \in A}{\sup}\big|f_n(z)-f(z)\big| < \epsilon, \ \abs{z-z_0} < \delta \implies \big|f(z) - f(z_0)\big|< \epsilon\big).
	\end{multline*}
\end{definition}

\section{Признак Вейерштрасса равномерной сходимости функционального ряда}

\begin{note}
	Предположим, что $\sum_{n=1}^{\infty}f_n$ таков, что $\forall n \in \N \ \forall z \in A \ \big|f_n(z)\big|$, причем $\sum_{n=1}^{\infty}c_n$ сходится. Тогда ряд $\sum_{n=1}^{\infty}f_n$ равномерно абсолютно сходится на $A$.
\end{note}

\section{Теорема Вейерштрасса (о равномерно сходящейся последовательности непрерывных функций)}

\begin{theorem}
	Если $\{f_n\}_{n \in \N} \subset C(A), \ f_n \rightrightarrows f$, то $f \in C(A)$.
\end{theorem}

\section{Путь, эквивалентные пути, жорданов путь, кривая, кривая Жордана, гладкая кривая, кусочногладкая кривая (это разные вопросы)}

\begin{definition}[Путь]
	\emph{Путем} $\gamma:[a ; b]\rightarrow \C$ называется непрерывное отображение $[a ; b]$ в $\C$.
\end{definition}

\begin{definition}[Эквивалентные пути]
	$\gamma_1:[a_1 ; b_2] \rightarrow \C, \ \gamma_2:[a_2 ; b_2] \rightarrow \C$.

	$\gamma_1 \sim \gamma_2$, если $\exists $ возрастающая непрерывная функция
	\[
		\phi:[a_1 ; b_1] \xrightarrow[]{\text{на}} [a_2 ; b_2]: \ \gamma_1 (t) = \gamma_2 \big(\phi(t)\big), \quad \forall t \in [a_1 ; b_1].
	\]
\end{definition}

\begin{definition}[Жорданов путь]
	Путь называется \emph{жордановым}, если он является взаимно однозначной функцией.
\end{definition}

\begin{definition}[Гладкая кривая]
	Кривая называется \emph{гладкой}, если в каждой ее точке $\exists $ касательная, непрерывно изменяющаяся вдоль кривой (если в каждой ее точке $\exists $ непрерывная производная).
\end{definition}

\begin{definition}[Кусочногладкая кривая]
	Кривая, состоящая из конечного числа гладких дуг, называется \emph{кусочно-гладкой}.
\end{definition}
