\section{Внутренняя точка множества}

 {\huge НАЙТИ}

\section{Граничная точка множества}

\begin{definition}
	Точка называется \emph{граничной} точкой множества, если в любой ее окрестности есть как точки множества, так и точки его дополнения.
	\begin{notation}
		\[
			\partial \mathfrak{D}.
		\]
	\end{notation}
\end{definition}

\section{Предел последовательности комплексных чисел}

 {\huge НАЙТИ}

\section{Предел функции}

 {\huge НАЙТИ}

\section{Непрерывность функции в точке}

 {\huge НАЙТИ}

\section{Производная функции в точке}

 {\huge НАЙТИ}

\section{Равномерная сходимость последовательности функций на множестве}

 {\huge НАЙТИ}

\section{Признак Вейерштрасса равномерной сходимости функционального ряда}

 {\huge НАЙТИ}

\section{Теорема Вейерштрасса (о равномерно сходящейся последовательности непрерывных функций)}

 {\huge НАЙТИ}

\section{Путь, эквивалентные пути, жорданов путь, кривая, кривая Жордана, гладкая кривая, кусочногладкая кривая (это разные вопросы)}

\begin{definition}[Путь]
	\emph{Путем} $\gamma:[a ; b]\rightarrow \C$ называется непрерывное отображение $[a ; b]$ в $\C$.
\end{definition}

\begin{definition}[Эквивалентные пути]
	$\gamma_1:[a_1 ; b_2] \rightarrow \C, \ \gamma_2:[a_2 ; b_2] \rightarrow \C$.

	$\gamma_1 \sim \gamma_2$, если $\exists $ возрастающая непрерывная функция
	\[
		\phi:[a_1 ; b_1] \xrightarrow[]{\text{на}} [a_2 ; b_2]: \ \gamma_1 (t) = \gamma_2 \big(\phi(t)\big), \quad \forall t \in [a_1 ; b_1].
	\]
\end{definition}

\begin{definition}[Жорданов путь]
  Путь называется \emph{жордановым}, если он является взаимно однозначной функцией.
\end{definition}

\begin{definition}[Гладкая кривая]
  {\huge НАЙТИ}
\end{definition}

\begin{definition}[Кусочногладкая кривая]
  {\huge НАЙТИ}
\end{definition}
