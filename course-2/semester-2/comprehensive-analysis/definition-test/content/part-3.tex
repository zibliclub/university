\section{Множество связное}

\begin{definition}
	Множество $A \subset \overline{\C}$ называется \emph{связным}, если не существует $U,V \in O_p \overline{\C}: \ U \cap A \ne \varnothing, \ V \cap A \ne \varnothing, \ U \cap V = \varnothing$.

	\begin{notation}
		$O_p \overline{\C}$ -- совокупность всех открытых множеств.
	\end{notation}
\end{definition}

\begin{definition}
	Множество называется \emph{линейно связным}, если $\forall $ две его точки можно соединить путем, значения которого лежат в этом множестве.
\end{definition}

\section{Область, односвязная область}

\begin{definition}
	\emph{Областью в $\C$} ($\overline{\C}$) называется $\forall $ непустое открытое связное множество.

	Область называется \emph{односвязной}, если $\forall $ замкнутая кривая гомотопна некоторой точки этой прямой (кривая гомотопна точке, если она стягивается в эту точку).
\end{definition}

\section{Производная функции в точке}

\begin{definition}
	Если $f: \C \rightarrow \C$ определена в некоторой окрестности точки $z_0 \in \C$ и $\exists \underset{z \rightarrow z_0}{\lim}\frac{f(z) - f(z_0)}{z - z_0} \eqqcolon f'(z_0)$ называется \emph{производной функции в точке} $z_0$.
\end{definition}

\newpage

\section{Моногенная в точке функция}

\begin{definition}
	Если $f: \C \rightarrow \C$ определена в некоторой окрестности точки $z_0 \in \C$, то она называется \emph{моногенной в точке} $z_0$, если $\exists $ конечный $\underset{z \rightarrow z_0}{\lim}\frac{f(z) - f(z_0)}{z-z_0}\eqqcolon f'(z_0)$.

	Другими словами, функция называется моногенной в точке, если она имеет в этой точке конечную производную.
\end{definition}

\section{Голоморфная в точке функция}

\begin{definition}
	Функция называется \emph{голоморфной в точке}, если она моногенна в некоторой ее окрестности, то есть дифференцируема в каждой точке ее окрестности.
\end{definition}

\section{Голоморфная в области функция}

\begin{definition}
	Функция называется \emph{голоморфной} в области, если она моногенна в каждой точке этой области.
\end{definition}

\section{Условия Коши-Римана}

\begin{note}
	Если функция
	\[
		f(z) = f(x+iy) = u(x,y) + iv(x,y)
	\]
	дифференцируема в точке $z$, то ее действительная и мнимая части обладают частными производными первого порядка, которые удовлетворяют условиям Коши-Римана:
	\[
		\left\{\begin{array}{l}
			\frac{\partial u}{\partial x}(x_0,y_0) = \frac{\partial v}{\partial y}(x_0,y_0), \\
			\frac{\partial u}{\partial y}(x_0,y_0) = - \frac{\partial v}{\partial x}(x_0,y_0)
		\end{array}\right.
	\]
\end{note}

\section{Степенной ряд}

\[
	\sum_{n=0}^{\infty}a_n (z-z_0)^n, \quad \text{где }\{a_n\}_{n\in\N}\subset \C, \ z,z_0 \in \C.
\]

\newpage

\section{1-я теорема Абеля}

\begin{theorem}
	Если ряд $\sum_{n=0}^{\infty}a_n(z-z_0)^n$ сходится в точке $\nequalto{z_1}{z_0} \in \C$, то он абсолютно сходится при $\abs{z-z_0} < \abs{z_1 - z_0}$.

	А если ряд $\sum_{n=0}^{\infty}a_n(z-z_0)^n$ расходится в точке $\nequalto{z_1}{z_0}\in \C$, то он расходится и при $\abs{z-z_0} > \abs{z_1-z_0}$.
\end{theorem}

\section{Радиус сходимости степенного ряда}

\begin{definition}
	Элемент $R \in [0 ; +\infty]$ называется \emph{радиусом сходимости} ряда $\sum_{n=0}^{\infty}a_n(z-z_0)^n$, если при $\abs{z-z_0} < R$ исходный ряд абсолютно сходится, а при $\abs{z-z_0} > R$ исходный ряд расходится.
\end{definition}
