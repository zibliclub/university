\section{Множество связное}

{\huge НАЙТИ}

\section{Область, односвязная область}

{\huge НАЙТИ}

\section{Производная функции в точке}

{\huge НАЙТИ}

\section{Моногенная в точке функция}

{\huge НАЙТИ}

\section{Голоморфная в точке функция}

{\huge НАЙТИ}

\section{Голоморфная в области функция}

\begin{definition}
  Функция называется \emph{голоморфной} в области, если она моногенна в каждой точке этой области.
\end{definition}

\section{Условия Коши-Римана}

{\huge НАЙТИ}

\section{Степенной ряд}

\[
  \sum_{n=0}^{\infty}a_n (z-z_0)^n, \quad \text{где }\{a_n\}_{n\in\N}\subset \C, \ z,z_0 \in \C.
\]

\section{1-я теорема Абеля}

\begin{theorem}
	Если ряд $\sum_{n=0}^{\infty}a_n(z-z_0)^n$ сходится в точке $\nequalto{z_1}{z_0} \in \C$, то он абсолютно сходится при $\abs{z-z_0} < \abs{z_1 - z_0}$.

	А если ряд $\sum_{n=0}^{\infty}a_n(z-z_0)^n$ расходится в точке $\nequalto{z_1}{z_0}\in \C$, то он расходится и при $\abs{z-z_0} > \abs{z_1-z_0}$.
\end{theorem}

\section{Радиус сходимости степенного ряда}

\begin{definition}
	Элемент $R \in [0 ; +\infty]$ называется \emph{радиусом сходимости} ряда $\sum_{n=0}^{\infty}a_n(z-z_0)^n$, если при $\abs{z-z_0} < R$ исходный ряд абсолютно сходится, а при $\abs{z-z_0} > R$ исходный ряд расходится.
\end{definition}
