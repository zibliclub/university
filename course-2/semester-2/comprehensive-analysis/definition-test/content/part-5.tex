\section{Лемма Гурса (Гауса?)}

\begin{lemma}
	Если функция $f$ непрерывна в области $\mathfrak{D}$, то для любой спрямляемой кривой Жордана $\gamma \subset \mathfrak{D}$, для любого $\epsilon > 0 \ \exists $ вписанная в $\gamma$ ломанная $P$ такая, что
	\[
		\left| \int_{\gamma}f(z)dz - \int_{P}f(z)dz \right| < \epsilon.
	\]
\end{lemma}

\section{Интегральная теорема Коши}

\begin{theorem}
	Пусть $\mathfrak{D}$ -- односвязная область в $\C$, функция $f$ голоморфна в $\mathfrak{D}$. Тогда для любой замкнутой спрямляемой кривой Жордана $\gamma$
	\[
		\int_{\gamma}f(z)dz = 0.
	\]
\end{theorem}

\section{Интеграл Коши от степенной функции по замкнутому контуру}

 {\huge НАЙТИ}

\section{Интегральная формула Коши}

\begin{theorem}
  Если функция $f$ голоморфна в односвязной области $D$, ограничена замкнутой спрямляемой кривой Жордана $\gamma$, непрерывна вплоть до границы, то
  \[
    \frac{1}{2\pi i}\int_{\gamma}\frac{f(z)}{z - z_0}dz = \left\{\begin{array}{ll}
        f(z_0), & \text{если } z_0 \in \mathfrak{D} \\
        0, & \text{если } z_0 \notin \cl \mathfrak{D}
    \end{array}\right..
  \]
\end{theorem}

\section{Интеграл типа Коши}

\begin{definition}
  Пусть односвязная область $\mathfrak{D}$ ограничена замкнутой спрямляемой кривой Жордана $\gamma$, а функция $f$ непрерывна на $\gamma$. Положим
  \[
    F(z) = \frac{1}{2\pi i}\int_{\gamma}\frac{f(\zeta)}{\zeta - z}d \zeta, \quad z \in \mathfrak{D}.
  \]

  Функция $F$ называется \emph{интегралом типа Коши}.
\end{definition}

\section{Теорема Лиувилля}

\begin{theorem}
  Если функция $f$ голоморфна в $\C$ и ограничена, то $f = \const$.
\end{theorem}

\section{Теоремы Мореры и Вейерштрасса}

\begin{theorem}[Морера]
  Для того, чтобы непрерывная в односвязной области функция была голоморфна в этой области, необходимо и достаточно, чтобы интеграл от этой функции по $\forall $ замкнутому контуру (то есть по $\forall $ замкнутой спрямляемой кривой Жордана), лежащему в области, был равен $0$.
\end{theorem}

\begin{theorem}[Вейерштрасса]
  Равномерный предел последовательности голоморфных функция является голоморфной функцией, то есть если $\{f_n\}_{n \in \N}\subset \mathcal{H}(\mathfrak{D})$ и $f_n \rightrightarrows f$ внутри $\mathfrak{D}$, то $f \in \mathcal{H}(\mathfrak{D})$.
\end{theorem}

\section{Ряд Тейлора голоморфной в круге функции}

 {\huge НАЙТИ}

\section{Ряд Лорана голоморфной в кольце функции}

\begin{theorem}
  Если функция $f$ голоморфна в кольце $r \subset \abs{z-z_0}\subset R$, то в этом кольце она разлагается в ряд Лорана:
  \[
    f(z) = \sum_{n=-\infty}^{\infty}c_n(z-z_0)^n
  \]
  с коэфициентами $c_n$, определяемыми формулами:
  \[
    c_n = \frac{1}{2\pi i}\int_{\abs{\xi - z_0 = \rho}}\frac{f(\xi)}{(\xi-z_0)^{n+1}}d \xi, \quad \forall \rho \in (r,R).
  \]
\end{theorem}

\section{Правильная и главная части ряда Лорана в конечной точке и в бесконечно удалённой точке}

 {\huge НАЙТИ}

\section{Определение вычета в конечной точке и в бесконечно удалённой}

 {\huge НАЙТИ}

\section{Вормулы для вычисления вычета в полюсе $k$-го порядка в конечной точке и в бесконечно удалённой}

 {\huge НАЙТИ}

\section{Гармоническая функция}

\begin{definition}
  Определенная в односвязной области $\mathfrak{D}\subset \R^2$ функция $u(x,y)$ называется \emph{гармонической функцией}, если $u \in C^2(\mathfrak{D})$ и
  \[
    \triangle u \coloneq \frac{\partial^2 u}{\partial x^2} + \frac{\partial^2 u}{\partial y^2} \equiv 0,
  \]
  $\triangle$ -- оператор Лапласа.
\end{definition}

\section{Определения целой и мероморфной функций}

\begin{definition}[Целая функция]
  Голоморфная в $\C$ функция называется \emph{целой функцией}.
\end{definition}

\begin{definition}[Мероморфная функция]
  Функция, голоморфная в области $\mathfrak{D}$ всюду, за исключением полюсов, называется \emph{мероморфной} в этой области функцией.
\end{definition}

\section{Теорема Римана}

 {\huge НАЙТИ}
