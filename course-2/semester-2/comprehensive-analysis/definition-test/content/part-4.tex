\section{Формула Коши-Адамара}

\begin{theorem}
	Для степенного ряда $\sum_{n=0}^{\infty}a_n(z-z_0)^n$ положим $l\coloneq \underset{n \rightarrow \infty}{\overline{\lim}}\sqrt[n]{\abs{a_n}}$. Тогда:
	\begin{enumerate}
		\item Если $l=0$, то исходный ряд сходится $\forall z \in \C$.
		\item Если $l = \infty$, то исходный ряд сходится только в точке $z_0$.
		\item Если $l \in (0 ; +\infty)$, то при $\abs{z-z_0} < \frac{1}{l}$, а при $\abs{z-z_0} > \frac{1}{l}$ исходный ряд расходится.
	\end{enumerate}
\end{theorem}

\section{Формула Даламбера}

\begin{remark}
	Если $\exists \underset{n \rightarrow \infty}{\lim}\frac{\abs{a_n}}{\abs{a_{n+1}}}$, то этот предел равен $R$ (радиусу сходимости).
\end{remark}

\section{Конформное в точке отображение}

\begin{definition}
	$f: \mathfrak{D}\rightarrow \C$ называется \emph{конформным отображением}, если оно является гомеоморфизмом и оно конфорно в каждой точке области $\mathfrak{D}$.
\end{definition}

\newpage

\section{Регулярное в точке отображение}

\begin{definition}
	Функция называется \emph{регулярной в точке}, если она имеет в этой точке конечную производную от $0$.
\end{definition}

\section{Связь между голоморфностью и конформностью}

\begin{note}
	Каждое конфорное в области отображение голоморфно и регулярно в этой области.

	Любое однолистное голоморфное и регулярное в области отображение конформно в этой области.
\end{note}

\section{Определение функций $e^z$, $\sin z$, $\cos z$, $\ln z$, $\Ln z$, $\Arg z$, $\Arcsin z$, $\Arccos z$, выражение тригонометрических функций через экспоненту}

\[
	\begin{array}{l}
		e^z = \omega = \abs{\omega}e^{i\arg \omega} = e^{\ln \abs{\omega} + i\arg \omega} = e^{\ln \abs{\omega} + i\arg \omega + 2\pi k i}, \ k \in \Z, \\
		z = \ln \abs{\omega} + i\arg \omega + 2\pi k i = \ln \abs{\omega} + i \Arg \omega,                                                              \\
		\ln z = \ln \abs{z} + i\arg z,                                                                                                                  \\
		\Ln z = \ln z + 2\pi k i = \ln \abs{z} + i\Arg z                                                                                                \\
		\Arg z \coloneq \arg z + 2\pi k, \ k \in \Z,                                                                                                    \\
		\sin z = \frac{e^{iz} - e^{-iz}}{2i} = \sum_{n=0}^{\infty}(-1)^{n}\frac{z^{2n}}{(2n)!}                                                          \\
		\cos z = \frac{e^{iz} + e^{-iz}}{2} = \sum_{n=0}^{\infty}(-1)^{n}\frac{z^{2n+1}}{(2n+1)!},                                                      \\
		\Arcsin z = -i \Ln i (z+\sqrt{z^2 - 1}),                                                                                                        \\
		\Arccos z = -i \Ln(z + \sqrt{z^2 -1}),
	\end{array}
\]

\[
	\begin{array}{l}
		e^{-iz} = \cos z - i\sin z, \\
		e^{iz} = \cos z + i\sin z,  \\
		\sqrt[n]{z} = \sqrt[n]{\abs{z}} \cdot e^{i \frac{\arg z}{2}+?}
	\end{array}
\]

\section{Дробно-линейная функция}

\begin{definition}
	\emph{Дробно-линейным отображением} называется функция вида
	\[
		f(z) = \frac{az + b}{cz + d}.
	\]
\end{definition}

\newpage

\section{Общий вид дробно-линейного автоморфизма верхней полуплоскости}

\begin{note}
	Каждый дробно-линейный автоморфизм верхней полуплоскости представим в виде $f(z) = \frac{az + b}{cz + d}$, где $a,b,c,d \in \R$ и $ad - bc > 0$.

	$\forall $ отображение такого вида является отображением верхней полуплоскости на себя (то есть ее автоморфизмом).
\end{note}

\section{Общий вид дробно-линейного автоморфизма единичного круга}

\begin{note}
	Каждый дробно-линейный автоморфизм единичного круга на себя можно представить в виде $f(z) = e^{i \theta}\frac{z - a}{1 - \overline{a}z}$, где $\theta \in \R, \ \abs{a} < 1$.

	$\forall $ отображение такого вида является автоморфизмом единичного круга.
\end{note}

\section{Общий вид дробно-линейного изоморфизма верхней полуплоскости на единичный круг}

\begin{note}
	Каждый дробно-линейный изоморфизм верхней полуплоскости на единичный круг можно представить в виде
	\[
		f(z) = e^{i\theta}\frac{z-a}{z - \overline{a}}, \text{ где } \theta \in \R, \ \Im a > 0.
	\]

	$\forall $ отображение такого вида является изоморфизмом верхней полуплоскости на единичный круг.
\end{note}
