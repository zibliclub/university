\section{Формула Коши-Адамара}

\begin{theorem}
	Для степенного ряда $\sum_{n=0}^{\infty}a_n(z-z_0)^n$ положим $l\coloneq \underset{n \rightarrow \infty}{\overline{\lim}}\sqrt[n]{\abs{a_n}}$. Тогда:
	\begin{enumerate}
		\item Если $l=0$, то исходный ряд сходится $\forall z \in \C$.
		\item Если $l = \infty$, то исходный ряд сходится только в точке $z_0$.
		\item Если $l \in (0 ; +\infty)$, то при $\abs{z-z_0} < \frac{1}{l}$, а при $\abs{z-z_0} > \frac{1}{l}$ исходный ряд расходится.
	\end{enumerate}
\end{theorem}

\section{Формула Даламбера}

\begin{remark}
	Если $\exists \underset{n \rightarrow \infty}{\lim}\frac{\abs{a_n}}{\abs{a_{n+1}}}$, то этот предел равен $R$ (радиусу сходимости).
\end{remark}

\section{Конформное в точке отображение}

 {\huge НАЙТИ}

\section{Регулярное в точке отображение}

\begin{definition}
	Функция называется \emph{регулярной в точке}, если она имеет в этой точке конечную производную от $0$.
\end{definition}

\section{Связь между голоморфностью и конформностью}

 {\huge НАЙТИ}

\section{Определение функций $e^z$, $\sin z$, $\cos z$, $\ln z$, $\Ln z$, $\Arg z$, $\Arcsin z$, $\Arccos z$, выражение тригонометрических функций через экспоненту}

\[
	\begin{array}{l}
		e^z = e^{x + iy} = e^x \cdot e^{iy},         \\
		\sin z = ???????????????????????             \\
		\cos z = ???????????????????????             \\
		\ln z = ???????????????????????              \\
		\Ln z = ???????????????????????              \\
		\Arg z \coloneq \arg z + 2\pi k, \ k \in \Z, \\
		\Arcsin z = ???????????????????????          \\
		\Arccos z = ???????????????????????
	\end{array}
\]

\[
	{\huge \text{НАЙТИ}}
\]

\section{Дробно-линейная функция}

\begin{definition}
  \emph{Дробно-линейным отображением} называется функция вида
  \[
    f(z) = \frac{az + b}{cz + d}.
  \]
\end{definition}

\section{Общий вид дробно-линейного автоморфизма верхней полуплоскости}

 {\huge НАЙТИ}

\section{Общий вид дробно-линейного автоморфизма единичного круга}

 {\huge НАЙТИ}

\section{Общий вид дробно-линейного изоморфизма верхней полуплоскости на единичный круг}

 {\huge НАЙТИ}
