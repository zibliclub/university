\lesson{5}{от 21 мар 2024 12:45}{Продолжение}


\begin{theorem}
  $ f=u_iv $ у и в вещественные мнимые части комлпексной функции ф, если у и в непрерывно дифференцируемы в функции и в этой области в точке консерватизм угло
\end{theorem}


\begin{theorem}
  $ f = u+iv $ если функции у и в непрерывно дифференцируемы в области и в этой области функция ф обладает свойством постоянства искажения масштаба, то фунцкия ф голоморфна или антиголоморфна.
\end{theorem}

Функция антиголоморфна, если голоморфны ее сопряженные.

\begin{definition}
  Говорят, что фунцкия $ f $ голоморфна (моногенна) в бесконечно удаленной точке, если функция $ g(z) := f \left(\frac{1}{z}\right) $ голоморфна (моногенна) в нуле.
  \[
      f(z) = \frac{1}{z}, \quad g(z) = f \left(\frac{1}{z}\right) = z.
  \]
\end{definition}

\begin{note}
    \[
        df(z_0): \ \Comp \ni h \rightarrow f^{(z_0)}\cdot h \in \Comp.
    \]
    \[
        d f(z_0) = f^{(z_0)}dz,
    \]
    \[
        f(z)=z, \quad \Comp\ni h \rightarrow h \in \Comp,
    \]
    \[
        \forall z \in \Comp f^{(z)} \ f^{(z)} = 1,
    \]
    \[
        d f(z_0) (h) = f^{(z_0)}\cdot h,
    \]
\end{note}

\begin{note}[Правила дифференцирования]\leavevmode
  \begin{enumerate}
      \item $ \forall \alpha,\beta \in \Comp \quad (\alpha f + \beta g)^{'} = \alpha \cdot f^{'} + \beta \cdot g^{'} $;
  \end{enumerate} 
\end{note}

\subsection{Дробно линейные отображения}

