\lesson{2}{от 22 фев 2024 12:45}{Продолжение}


\[
  \forall z \in \C \ d(z ; \infty )\coloneqq +\infty, \quad \begin{array}{ll}
    d: & \C^2 \rightarrow \R \\
    d: & \C^2 \rightarrow \overline{\R} \\
    \rho: & \overline{\C}^2 \rightarrow \R, \quad \rho(z ; \infty ) \in \R
  \end{array}
\]

\begin{property}[Свойства окрестностей]
  $\forall z \in \overline{\C}$:
  \begin{enumerate}
    \item $\forall V \in O_z \quad z \in V$.
    \item $\forall U,V \in O_z \quad U \cap V \in O_z$.
    \item $\forall U \in O_z, \ \forall V \supset U \quad V \in O_z$.
    \item $\forall V \in O_z, \ \exists U \in O_z : \ U \subset V \ \& \ \forall w \in U \quad U \in O_w$.
  \end{enumerate}
\end{property}

\begin{definition}[Открытое множество]
  Множество называется \emph{открытым}, если оно является окрестностью каждой своей точки.
\end{definition}

\begin{definition}[Окрестность множества]
  \emph{Окрестностью множества} называется множество, являющееся окрестностью каждой точки исходного множества ($V$ -- окрестность множества $A$, если $\forall z \in A \ V \in O_z$).
\end{definition}

\begin{definition}
  $D \subset \overline{\C}, \ z \in \C$,
  \begin{multicols}{2}
    \[
      \dist(z,D)\coloneqq \underset{w \in D}{\inf}\d(z,w),
    \]

    \columnbreak

    \begin{figure}[H]
      \centering
      \incfig{fig-4}
      \label{fig:fig-4}
    \end{figure}
  \end{multicols}
\end{definition}

\begin{definition}
  $D_1,D_2 \subset \overline{\C}$,
  \begin{multicols}{2}
    \[
      \dist(D_1,D_2) \coloneqq \underset{z \in D_1, \ w \in D_2}{\inf}\d(z,w),
    \]

    \columnbreak

    \begin{figure}[H]
      \centering
      \incfig{fig-5}
      \label{fig:fig-5}
    \end{figure}
  \end{multicols}
\end{definition}

\begin{definition}[Внутренность]
  Множество всех внутренних точек называется \emph{внутренностью}.
  \begin{notation}
    \[
      \operatorname{int}D.
    \]
  \end{notation}
\end{definition}

\begin{definition}[Предельная точка множества]
  Точка называется \emph{предельной точкой множества}, если в любой ее окрестности есть точки множества, отличные от данной.
\end{definition}

\begin{remark}
  Точка является предельной точкой множества на расширенной комплексной плоскости $(\overline{\C}) \iff \forall $ ее окрестность содержит бесконечное число точек данного множества.
\end{remark}

\begin{definition}[Окрестность бесконечно удаленной точки]
  Множество $V \subset \overline{\C}$ является \emph{окрестностью бесконечно удаленной точки}, если $\exists \epsilon > 0 : \big\{z \in \overline{\C}: \ \abs{z}>\epsilon\big\} \subset V$.
  \begin{figure}[H]
    \centering
    \incfig[0.7]{fig-6}
    \label{fig:fig-6}
  \end{figure}
\end{definition}

\begin{definition}[Точка прикосновения множества]
  Точка $z \in \overline{\C}$ расширенной комплексной плоскости называется \emph{точкой прикосновения множества} $D \subset \overline{\C}$, если пересечение $\forall V \in O_z \quad V \cap D \ne \varnothing$.
  \begin{notation}
    \[
      \cl D \text{ -- \emph{замыкание} (closure)}
    \]
  \end{notation}
\end{definition}

\begin{definition}[Замкнутое множество]
  Множество называется \emph{замкнутым}, если его дополнение открыто.
  \begin{notation}
    \[
      \partial D
    \]
  \end{notation}
\end{definition}

\begin{definition}[Граничная точка]
  Точка называется \emph{граничной точкой множества}, если в любой ее окрестности есть как точки множества, так и точки его дополнения.
  \begin{notation}
    Множество всех замкнутых подмножеств в $\overline{\C}$:
    \[
      \Cl \overline{\C} \ \text{(closed)}
    \]
  \end{notation}
\end{definition}

\begin{definition}[Компактное множество]
  Множество в $\overline{\C}$ называется \emph{компактным}, если $\forall $ его открытое покрытие имеет конечное подпокрытие.
  \begin{notation}
    \[
      v \text{ -- покрытие множества }D \text{, если }D \underset{V \in v}{\subset }UV,
    \]
  \end{notation}
  \begin{notation}
    \[
      \mathcal{P}(\overline{\C}) \text{ -- совокупность всех подмножеств }\overline{\C}.
    \]
  \end{notation}
\end{definition}

\begin{crit}[Компактности]
  Подмножество $\C$ компактно $\iff $ оно замкнуто и ограничено.
\end{crit}

\begin{note}
  Множество ограничено, если оно содержится в некотором шаре.
\end{note}

\begin{remark}
  $\overline{\C}$ -- компактно.
\end{remark}

\begin{definition}
  Последовательность $\{z_n\}_{n \in \N}\subset \C$ сходится к $z \in \C$, если $\forall \epsilon > 0 \ \exists n_0 \in \N: \ \forall n \geqslant n_0$
  \[
    \abs{z_n - z} < \epsilon.
  \]
  \[
    \d(z_n,z) \xrightarrow[n \rightarrow \infty ]{} 0, \qquad z_n \rightarrow \infty \text{, если } \lim_{n \rightarrow \infty }\abs{z_n} = \pm \infty.
  \]
  \[
    z = \lim_{n \rightarrow \infty } z_n, \qquad z_n \xrightarrow[n \rightarrow \infty ]{} z.
  \]
\end{definition}

\begin{remark}
  \[
    z_n \rightarrow z \text{ в } \C \iff \left\{\begin{array}{l}
      \Re z_n \rightarrow \Re z \\
      \Im z_n \rightarrow \Im z
    \end{array}\right. \text{ в } \R,
  \]

  \[
    \abs{z_n - z} = \sqrt{(\Re z_n - \Re z)^2 + (\Im z_n - \Im z)^2} \geqslant \abs{\Re z_n - \Re z},
  \]
  \[
    \Re(z_1 \pm z_2) = \Re z_1 \pm \Re z_2.
  \]
\end{remark}

\begin{crit}[Коши]
  Последовательность $\{z_n\}_{n \in \N}\subset \C$ сходится $\iff \forall \epsilon > 0 \ \exists n_0 \in \N: \ \forall n,m \geqslant n_0$
  \[
    \abs{z_n - z_m} < \epsilon.
  \]
\end{crit}

\begin{crit}[Коши в $\overline{\C}$]
  Последовательность $\{z_n\}_{n \in \N}\subset \overline{\C}$ сходится $\iff \forall \epsilon > 0 \ \exists n_0 \in \N: \ \forall n,m \geqslant n_0$
  \[
    \rho(z_n,z_m) < \epsilon,
  \]
  \[
    z_n \xrightarrow[n \rightarrow \infty ]{} z \iff \rho(z_n,z)\xrightarrow[n \rightarrow \infty ]{} 0.
  \]
\end{crit}

\begin{crit}[Компактности (расширенный)]
  Подмножество $D \subset \overline{\C}$ компактно $\iff \forall $ его последовательность имеет сходящуюся подпоследовательность: $D \subset \overline{\C} \ \forall \{z_n\}_{n \in \N} \subset D \ \exists \{z_{n_k}\}_{k \in \N} \subset \{z_n\}_{n \in \N}:$ 
  \[
    z_{n_k} \rightarrow z \in D.
  \]

  Пусть $\{z_n\}_{n \in \N} \subset \C:$
  \[
    \sum_{n=1}^{\infty }z_n = \lim_{n \rightarrow \infty }S_n.
  \]
\end{crit}

\begin{definition}[Числовой ряд]
  \emph{Числовым рядом} называется формальная сумма членов.
\end{definition}

\begin{definition}[Абсолютно сходящийся числовой ряд]
  Числовой ряд называется \emph{абсолютно сходящимся}, если сходится ряд
  \[
    \sum_{n=1}^{\infty }\abs{z_n}.
  \]
\end{definition}

\begin{crit}[Коши (сходимости ряда)]
  $\sum_{n=1}^{\infty }z_n$ сходится $\iff \forall \epsilon > 0 \ \exists m \in \N: \ \forall n \geqslant m \ \forall k \in \N$
  \[
    \underbrace{\abs{z_{n+1}+z_{n+2}+\ldots+z_{n+k}}}_{\abs{S_{n+k}-S_n}} < \epsilon.
  \]
\end{crit}

\begin{corollary}
  Если ряд сходится, то его общий член стремится к $0$.
\end{corollary}

\begin{corollary}
  Каждый абсолютно сходящийся числовой ряд сходится.
\end{corollary}

\subsection{Пути, кривые и области}

\begin{definition}[Путь]
  Путем $\gamma:[a ; b] \rightarrow \C$ называется непрерывное отображение $[a ; b]$ в $\C$.
\end{definition}

\begin{eg}
  $\gamma(t) = e^{it},$
  \begin{multicols}{2}
    \begin{figure}[H]
      \centering
      \incfig{fig-7}
      \label{fig:fig-7}
    \end{figure}

    \columnbreak

    \[
      0 \leqslant t \leqslant 2\pi.
    \]
  \end{multicols}
\end{eg}

\begin{definition}
  $\gamma_1 : [a_1 ; b_1] \rightarrow \C, \ \gamma_2 : [a_2 ; b_2] \rightarrow \C$. $\gamma_1 \sim \gamma_2$, если $\exists $ возрастающая непрерывная функция $\phi: [a_1 ; b_1] \xrightarrow[]{\text{на}} [a_2 ; b_2]:$
  \[
    \gamma_1(t) = \gamma_2\big(\phi(t)\big), \quad \forall t \in [a_1 ; b_1].
  \]
\end{definition}

\begin{eg}
  \begin{multicols}{2}
    \[
      \begin{array}{ll}
        \gamma_1(t) = t & 0 \leqslant t \leqslant 1 \\
        \gamma_2(t) = \sin t & 0 \leqslant t \leqslant \frac{\pi}{2} \\
        \gamma_3(t) = \sin t & 0 \leqslant t \leqslant \pi \\
        \gamma_4(t) = \cos t & 0 \leqslant t \leqslant \frac{\pi}{2}
      \end{array}
    \]
    $\phi(t) = \arcsin t$,
    \[
      \gamma_1(t) = \gamma_2\left(\phi(t)\right).
    \]
    \begin{figure}[H]
      \centering
      \incfig{fig-8}
      \label{fig:fig-8}
    \end{figure}
  \end{multicols}
\end{eg}
