\documentclass{report}

\usepackage[utf8]{inputenc}
\usepackage[T2A]{fontenc}
\usepackage[russian]{babel}

\usepackage{hyperref}
\hypersetup{
    colorlinks,
    linkcolor={black},
    citecolor={black},
    urlcolor={blue!80!black}
}

\usepackage{amsmath, amsfonts, mathtools, amsthm, amssymb}
\usepackage{tikz}

% theorems
\usepackage{thmtools}
\usepackage[framemethod=TikZ]{mdframed}
\mdfsetup{skipabove=1em,skipbelow=0em, innertopmargin=5pt, innerbottommargin=6pt}

\declaretheoremstyle[headfont=\bfseries\sffamily, bodyfont=\normalfont, mdframed={ nobreak } ]{boxstyle}

\declaretheorem[numberwithin=section, style=boxstyle, name=Определение]{definition}
\declaretheorem[numberwithin=section, style=boxstyle, name=Теорема]{theorem}

\newcommand{\RomanNumeralCaps}[1]
    {\MakeUppercase{\romannumeral #1}}


\def\course   {Комплексный Анализ}
\def\lecturer {Мельникова Е.В.}
\author{
	Основано на лекциях \lecturer \\
	\small Конспект написан Заблоцким Данилом и Кручининым Максимом
}
\def\term     {Весенний семестр}
\def\year     {2024}

\begin{document}
\maketitle
\newpage
\tableofcontents

\chapter*{Введение}

\lesson{1}{от 9 фев 2024 8:45}{Начало}

\emph{Методы оптимизации} -- раздел прикладной математики, предметом изучения которого является теория и методы оптимизации.

\begin{definition}[Оптимизационная задача]
	\emph{Оптимизационная задача} -- задача выбора из множества возможных вариантов, наилучших в некотором смысле,
	\[
		\left\{\begin{array}{l}
			f(x) \rightarrow \min \ (\max) \\
			x \in D
		\end{array}\right.,
	\]
	где
	\[
		\begin{array}{rl}
			D       & \text{-- множество допустимых решений},           \\
			x \in D & \text{-- допустимое решение},                     \\
			f(x)    & \text{-- целевая функция (критерий оптимизации)}.
		\end{array}
	\]
\end{definition}

\section*{Задачи математического программирования (МП) и их классификация}

$\begin{array}{ll}
		\text{1939 г.}  & \text{Л.В. Конторович}                                        \\
		\text{1947 г.}  & \text{Д. Данциг}                                              \\
		\text{с 50 гг.} & \text{бурное развитие}                                        \\
		\text{1975 г.}  & \text{Нобелевская премия по экономике Конторовичу и Купмаксу}
	\end{array}$

\begin{definition}[Задача математического программирования]
	\[
		\begin{array}{ll}
			(1) & f(x) \rightarrow \max (\min)                                                     \\
			(2) & g_i(x) \# 0, \ i = \overline{1,m}, \ \# \in \{\leqslant ,\geqslant ,=\}          \\
			(3) & x_j \in \R, \ j = \overline{1,n} \ \big(x \in \R^n, \ x = (x_1,\ldots ,x_n)\big)
		\end{array}
	\]

	Множество точек $x$, удовлетворяющих условиям $(2)$--$(3)$, называется \emph{множеством $D$ доп. решений}.
\end{definition}

\begin{definition}[Оптимальное решение]
	$x^* \in D$ называется \emph{оптимальным решением} задачи $(1)$--$(3)$, если $\forall x \in D \ f(x^*)\geqslant f(x)$ для задачи на $\max$ и $\forall x \in D \ f(x^*)\leqslant f(x)$ для задачи на $\min$.

	$x^*$ является \emph{глобальным экстремумом}.
\end{definition}

\begin{definition}[Разрешимая задача]
	Задача $(1)$--$(3)$, которая обладает оптимальным решением, называется \emph{разрешимой}, и \emph{неразрешимой} в противном случае.

	$D = \R^n$ -- задача \emph{безусловной оптимизации}, в противном случае -- \emph{задача условной оптимизации}.
\end{definition}

\begin{note}[Классификация]\leavevmode
	\begin{enumerate}
		\item Если $f,g_i$ являются линейными, то задача является задачей \emph{линейного программирования (ЛП)}.
		\item Если хотя бы одна из функций $f$ и $g_i$ нелинейная, то задача \emph{нелинейного програмирования}.
		\item $f,g_i$ -- выпуклые, то \emph{выпукголого программирования}.
	\end{enumerate}
\end{note}

\chapter{Линейное программирование}

\section{Постановка задачи, теорема эквивалентности}

\begin{definition}[Общая задача ЛП (ЗЛП)]
	\[
		\begin{array}{ll}
			f(x) = & c_0 + \sum_{j=1}^{n} c_jx_j \rightarrow \max \ (\min)                                         \\
			       & \sum_{j=1}^{n} a_{ij} x_j \# b_i, \ i = \overline{1,m} , \ \# \in \{\leqslant ,\geqslant ,=\} \\
			       & x_j \geqslant 0, \ j \in \mathcal{J} \leqslant \{1,\ldots ,n\}
		\end{array}
	\]

	\[
		\underbrace{A = \left(\begin{array}{cccc}
				a_{11} & a_{12} & \cdots & a_{1n} \\
				\vdots & \vdots & \ddots & \vdots \\
				a_{m1} & a_{m2} & \cdots & a_{mn}
			\end{array}\right)^C \quad b = \left(\begin{array}{c}
				b_1 \\ \vdots \\ b_m
			\end{array}\right)}_{\text{дано}} \quad x = \left(\begin{array}{c}
				x_1 \\ x_2 \\ \vdots \\ x_n
			\end{array}\right)
	\]
\end{definition}

\begin{remark}
	Б.о.о. далее полагаем $c_0 = 0$, так как добавление константы не влияет на процесс нахожденя оптимального решения.
\end{remark}

\begin{note}[Матричная задача]
	\[
		\begin{array}{ll}
			f(x) = & (c,x) \rightarrow \max \ (\min)                                \\
			       & Ax \# b                                                        \\
			       & x_j \geqslant 0, \ j \in \mathcal{J} \subseteq \{1,\ldots ,n\}
		\end{array}
	\]
\end{note}

\begin{note}[Каноническая ЗЛП (КЗЛП)]
	\[
		\begin{array}{ll}
			f(x) = & (c,x) \rightarrow \max                                             \\
			       & Ax = b                                                             \\
			       & x \geqslant \overline{0} \ \big(\overline{0}  = (0,\ldots ,0)\big)
		\end{array}
	\]
\end{note}

\begin{note}[Симметричная ЗЛП]
	\[
		\begin{array}{ll}
			f(x) = & (c,x) \rightarrow \max   \\
			       & Ax < b                   \\
			       & x \geqslant \overline{0}
		\end{array} \quad \text{или} \quad \begin{array}{ll}
			f(x) = & (c,x) \rightarrow \min   \\
			       & Ax \geqslant b           \\
			       & x \geqslant \overline{0}
		\end{array}
	\]
\end{note}

\subsection*{Примеры моделей ЛП}

\begin{eg}
	Задача о составлении оптимального плана пространства

	\[
		\begin{array}{cll}
			m & \text{ресурсов}        & i = \overline{1,m} \\
			n & \text{видов продукции} & j = \overline{1,n}
		\end{array}
	\]

	Известно: $\begin{array}{l}
			b_i \text{ -- запас }i \text{-го ресурса, }i = \overline{1,m}                  \\
			a_{ij} \text{ -- кол-во рес. }i \text{, требуемое для пр-ва 1ед. прод. вида }j \\
			c_j \text{ -- прибыль от продажи 1ед. }j \text{го продукта}
		\end{array}$

	Необходимо составить план производства, максимизирующий суммарную прибыль.

	Переменные: $x_j$ ед. продукции вида $j$ производства ($j = \overline{1,n} $),
	\[
		\begin{array}{l}
			\sum_{j=1}^{n} c_jx_j \rightarrow \max                        \\
			\sum_{j=1}^{n} a_{ij} x_j \leqslant b_i, \ i = \overline{1,m} \\
			x_j \geqslant 0, \ j = \overline{1,n}
		\end{array}
	\]
\end{eg}

\begin{eg}
	О максимальном потоке

	\[
		\begin{array}{l}
			G = (V,E) \text{ ориент. взвешенный}                              \\
			c: E \rightarrow \R \text{ -- веса дуг -- пропускная способность} \\
			s \text{ -- источник}                                             \\
			t \text{ -- сток}
		\end{array}
	\]

	Пусть $x_{ij} $ -- поток по дуге $(i,j) \in E$,
	\[
		\left\{\begin{array}{ll}
			f = & \sum_{j:(s,j) \in E} x_{sj} \rightarrow \max                                          \\
			    & \sum_{j:(j,i)\in E} x_{ji} = \sum_{k:(i,k)\in E} x_{ik} , \ i \in V \setminus \{s,t\} \\
			    & 0 \leqslant x_{ij} \leqslant c_{ij} , \ (i,j) \in E
		\end{array}\right.
	\]
\end{eg}

\newpage

\subsection*{Теорема эквивалентности задач ЛП}

\begin{definition}[Эквивалентные задачи]
	Две задачи МП
	\[
		\overset{(\RomanNumeralCaps{1})}{\left\{\begin{array}{l}
				f(x) \rightarrow opt \\
				x \in D
			\end{array}\right.} \quad \overset{(\overline{\RomanNumeralCaps{1}})}{\left\{\begin{array}{l}
				\overline{f} (\overline{x} ) \rightarrow \overline{opt} \\
				\overline{x} \rightarrow \overline{D}
			\end{array}\right.} \qquad \begin{array}{l}
			D \xrightarrow[]{\phi } \overline{D} \\
			\overline{D} \xrightarrow[]{\overline{\phi } } D
		\end{array}
	\]
	называются \emph{эквивалентными}, если любому допустимому решению каждой из них по некоторому правилу соответсвует допустимое решение другой задачи, причем оптимальному решению соответсвует оптимальное.
\end{definition}

\begin{theorem}[Первая теорема эквивалентности]
	Для любой задачи ЛП $\exists $ эквивалентная ей каноническая ЗЛП.
\end{theorem}

\begin{note}[Идея доказательства]
	$n=2, \ m = 3$

	\[
		\begin{array}{l}
			f = c_1x_1 + c_2x_2 \rightarrow \min   \\
			a_{11} x_1 + a_{12} x_2 = b_1          \\
			a_{21} x_1 + a_{22} x_2 \leqslant b_2  \\
			a_{31} x_1 + a_{32}  x_2 \geqslant b_3 \\
			x_1 \geqslant 0                        \\
			x_2 \in \R
		\end{array} \qquad \begin{array}{l}
			\overline{f} = -c_1x_1 - c_2x_2 \rightarrow \max \\
			a_{11} x_1 + a_{12} x_2 = b_1                    \\
			a_{21} x_1 + a_{22} x_2 + x_3 = b_2              \\
			a_{31} x_1 + a_{32} x_2 - x_4 = b_3              \\
			x_1,x_3,x_4 \geqslant 0                          \\
			x_2 = x_{2}' - x_{2}'', \ x_{2}' \geqslant 0, \ x_{2}'' \geqslant 0
		\end{array}
	\]

	\[
		\text{КЗЛП} \qquad \begin{array}{l}
			\overline{f}  = -c_1x_1 - c_2x_2 ' + c_2 x_2 '' \rightarrow \max \\
			a_{11} x_1 + a_{12} x_2 ' - a_{12} x_2 '' = b_1                  \\
			a_{21} x_1 + a_{22} x_2 ' - a_{22} x_2 '' + x_3 = b_2            \\
			a_{31} x_1 + a_{32} x_2 ' - a_{32} x_2 '' - x_4 = b_3            \\
			x_1,x_2 ', x_2 '', x_3,x_4 \geqslant 0
		\end{array}
	\]

	Неоднозначность -- разность, $\forall x \in D \ f(x) = -f(\overline{x} ), \ \overline{x}  \in \overline{D} $
	\[
		\overline{x} = \phi (x).
	\]

	Очевидно, что оптимальность также сохраняется при таких преобразованиях.
\end{note}

\begin{theorem}[Вторая теорема эквивалентности]
	Для любой задачи ЛП $\exists $ эквивалентная ей симметричная задача ЛП.
\end{theorem}

\begin{note}[Идея доказательства]
	\[
		\alpha = \beta \iff \left\{\begin{array}{l}
			\alpha \leqslant \beta \\
			\alpha \geqslant \beta
		\end{array}\right. \qquad \begin{array}{l}
			(c,x)\rightarrow \max \\
			Ax \leqslant b        \\
			x \geqslant 0
		\end{array} \Bigg| \begin{array}{l}
			(c,x) \rightarrow \min \\
			Ax \geqslant b         \\
			x \geqslant 0
		\end{array}
	\]
\end{note}

\begin{remark}
	Смысл теоремы 1 в том, чтобы свести решение ЗЛП к КЗЛП.
\end{remark}

\begin{note}[Геометрическая интерпретация]
	$n=2$,
	\[
		\begin{array}{l}
			f = c_1x_1 + c_2x_2 \rightarrow \max \\
			a_{i1} x_1 + a_{i2} x_2 \leqslant b_i, \ i = \overline{1,m}
		\end{array}
	\]

	\begin{figure}[H]
		\centering
		\incfig[0.7]{fig-1}
		\label{fig:fig-1}
	\end{figure}

	Линии уровня целевой функции
	\[
		\begin{array}{l}
			c_1x_1 + c_2x_2 = const \\
			\perp \nabla f = (c_1,c_2)
		\end{array}
	\]
	\[
		\begin{array}{ccl}
			                     &          & \exists ! x^* \text{ -- опт. решение} \\
			                     & \nearrow &                                       \\
			\text{ЗЛП разрешима} &          &                                       \\
			                     & \searrow &                                       \\
			                     &          & \text{беск. много опт. реш.}
		\end{array}
	\]
	\[
		\begin{array}{ccl}
			                       &          & f \rightarrow +\infty \text{ на мн-ве }D \text{ (ф-я неогр. сверху на }D \text{)} \\
			                       & \nearrow &                                                                                   \\
			\text{ЗЛП неразрешима} &          &                                                                                   \\
			                       & \searrow &                                                                                   \\
			                       &          & D = \varnothing \text{ нет доп. реш.}
		\end{array}
	\]
\end{note}

\section{Базисные решения КЗЛП}

\[
	\text{КЗЛП} \quad \begin{array}{l}
		(1) \ f =  (c,x) \rightarrow \max \\
		\left.\begin{array}{ll}
			      (2) & Ax = b                   \\
			      (3) & x \geqslant \overline{0}
		      \end{array}\right\} D
	\end{array}
\]

\[
	A_{m \times n} = (A^1,A^2,\ldots ,A^n) \quad A^j = \left(\begin{array}{c}
			a_{1j} \\ a_{2j} \\ \vdots \\ a_{mj}
		\end{array}\right) \text{ -- }j \text{-ый столбец матрицы }A
\]

\begin{definition}[Базисное решение системы $(2)$]
	Пусть $\overline{x} $ -- решение системы $(2)$. Вектор $\overline{x}$ называется \emph{базисным решением системы $(2)$}, если система векторных столбцов матрицы $A$, соответствующая ненулевым компонентам вектора $\overline{x} $, линейно независима.
\end{definition}

\begin{remark}
	В случае однородной системы ($b=0$), решение $x=0$ является базисным.
\end{remark}

\begin{definition}[Базисное решение КЗЛП]
	Неотрицательное базисное решение системы $(2)$ называется \emph{базисным (опорным) решением КЗЛП}.
\end{definition}

\begin{eg}
	\[
		\begin{array}{l}
			3x_1 - 4x_2 + x_3 \rightarrow \max \\
			\left\{\begin{array}{l}
				       2x_1 + 2x_2 + 3x_3 - x_4 + x_5 = 1 \\
				       2x_1 + 4x_2 + x_4 + 2x_5 = 2       \\
				       x_j \geqslant 0, \ j = \overline{1,5}
			       \end{array}\right.
		\end{array}
	\]
	\[
		A = \left(\begin{array}{ccccc}
				2 & 2 & \textbf{3} & \textbf{-1} & 1 \\
				2 & 4 & \textbf{0} & \textbf{1}  & 2
			\end{array}\right)
	\]

	$x^1 = (0,0,1,2,0)$ -- базисное решение системы, так как $\left|\begin{array}{cc}
			3 & -1 \\ 0 & 1
		\end{array}\right| \ne 0$ соответствует базису $\{A^3,A^4\}$.

	\[
		\begin{array}{ll}
			x^1   & \text{и БР КЗЛП}                                               \\
			x^2 = & (1,0,\textbf{$-\frac{1}{3}$ },0,0) \text{ БР СЛАУ, но не КЗЛП} \\
			x^3 = & (0,0,0,0,1) \text{ БР КЗЛП}
		\end{array}
	\]
\end{eg}

\begin{definition}[Вырожденное решение]
	$x$ -- базисное решение КЗЛП называется \emph{вырожденным}, если число ненулевых компонент вектора $x$ меньше ранга матрицы $A$.

\end{definition}

\begin{note}
	$x^3$ -- вырожденное, недост.: соответствует разным наборам баз. столбцов матрицы.

	$x^3$ соответствует $\{A_1,A_5\}, \{A_3,A_5\}, \{A_4,A_5\}$.
\end{note}

\lesson{2}{от 22 фев 2024 12:45}{Продолжение}


\[
  \forall z \in \C \ d(z ; \infty )\coloneqq +\infty, \quad \begin{array}{ll}
    d: & \C^2 \rightarrow \R \\
    d: & \C^2 \rightarrow \overline{\R} \\
    \rho: & \overline{\C}^2 \rightarrow \R, \quad \rho(z ; \infty ) \in \R
  \end{array}
\]

\begin{property}[Свойства окрестностей]
  $\forall z \in \overline{\C}$:
  \begin{enumerate}
    \item $\forall V \in O_z \quad z \in V$.
    \item $\forall U,V \in O_z \quad U \cap V \in O_z$.
    \item $\forall U \in O_z, \ \forall V \supset U \quad V \in O_z$.
    \item $\forall V \in O_z, \ \exists U \in O_z : \ U \subset V \ \& \ \forall w \in U \quad U \in O_w$.
  \end{enumerate}
\end{property}

\begin{definition}[Открытое множество]
  Множество называется \emph{открытым}, если оно является окрестностью каждой своей точки.
\end{definition}

\begin{definition}[Окрестность множества]
  \emph{Окрестностью множества} называется множество, являющееся окрестностью каждой точки исходного множества ($V$ -- окрестность множества $A$, если $\forall z \in A \ V \in O_z$).
\end{definition}

\begin{definition}
  $D \subset \overline{\C}, \ z \in \C$,
  \begin{multicols}{2}
    \[
      \dist(z,D)\coloneqq \underset{w \in D}{\inf}\d(z,w),
    \]

    \columnbreak

    \begin{figure}[H]
      \centering
      \incfig{fig-4}
      \label{fig:fig-4}
    \end{figure}
  \end{multicols}
\end{definition}

\begin{definition}
  $D_1,D_2 \subset \overline{\C}$,
  \begin{multicols}{2}
    \[
      \dist(D_1,D_2) \coloneqq \underset{z \in D_1, \ w \in D_2}{\inf}\d(z,w),
    \]

    \columnbreak

    \begin{figure}[H]
      \centering
      \incfig{fig-5}
      \label{fig:fig-5}
    \end{figure}
  \end{multicols}
\end{definition}

\begin{definition}[Внутренность]
  Множество всех внутренних точек называется \emph{внутренностью}.
  \begin{notation}
    \[
      \operatorname{int}D.
    \]
  \end{notation}
\end{definition}

\begin{definition}[Предельная точка множества]
  Точка называется \emph{предельной точкой множества}, если в любой ее окрестности есть точки множества, отличные от данной.
\end{definition}

\begin{remark}
  Точка является предельной точкой множества на расширенной комплексной плоскости $(\overline{\C}) \iff \forall $ ее окрестность содержит бесконечное число точек данного множества.
\end{remark}

\begin{definition}[Окрестность бесконечно удаленной точки]
  Множество $V \subset \overline{\C}$ является \emph{окрестностью бесконечно удаленной точки}, если $\exists \epsilon > 0 : \big\{z \in \overline{\C}: \ \abs{z}>\epsilon\big\} \subset V$.
  \begin{figure}[H]
    \centering
    \incfig[0.7]{fig-6}
    \label{fig:fig-6}
  \end{figure}
\end{definition}

\begin{definition}[Точка прикосновения множества]
  Точка $z \in \overline{\C}$ расширенной комплексной плоскости называется \emph{точкой прикосновения множества} $D \subset \overline{\C}$, если пересечение $\forall V \in O_z \quad V \cap D \ne \varnothing$.
  \begin{notation}
    \[
      \cl D \text{ -- \emph{замыкание} (closure)}
    \]
  \end{notation}
\end{definition}

\begin{definition}[Замкнутое множество]
  Множество называется \emph{замкнутым}, если его дополнение открыто.
  \begin{notation}
    \[
      \partial D
    \]
  \end{notation}
\end{definition}

\begin{definition}[Граничная точка]
  Точка называется \emph{граничной точкой множества}, если в любой ее окрестности есть как точки множества, так и точки его дополнения.
  \begin{notation}
    Множество всех замкнутых подмножеств в $\overline{\C}$:
    \[
      \Cl \overline{\C} \ \text{(closed)}
    \]
  \end{notation}
\end{definition}

\begin{definition}[Компактное множество]
  Множество в $\overline{\C}$ называется \emph{компактным}, если $\forall $ его открытое покрытие имеет конечное подпокрытие.
  \begin{notation}
    \[
      v \text{ -- покрытие множества }D \text{, если }D \underset{V \in v}{\subset }UV,
    \]
  \end{notation}
  \begin{notation}
    \[
      \mathcal{P}(\overline{\C}) \text{ -- совокупность всех подмножеств }\overline{\C}.
    \]
  \end{notation}
\end{definition}

\begin{crit}[Компактности]
  Подмножество $\C$ компактно $\iff $ оно замкнуто и ограничено.
\end{crit}

\begin{note}
  Множество ограничено, если оно содержится в некотором шаре.
\end{note}

\begin{remark}
  $\overline{\C}$ -- компактно.
\end{remark}

\begin{definition}
  Последовательность $\{z_n\}_{n \in \N}\subset \C$ сходится к $z \in \C$, если $\forall \epsilon > 0 \ \exists n_0 \in \N: \ \forall n \geqslant n_0$
  \[
    \abs{z_n - z} < \epsilon.
  \]
  \[
    \d(z_n,z) \xrightarrow[n \rightarrow \infty ]{} 0, \qquad z_n \rightarrow \infty \text{, если } \lim_{n \rightarrow \infty }\abs{z_n} = \pm \infty.
  \]
  \[
    z = \lim_{n \rightarrow \infty } z_n, \qquad z_n \xrightarrow[n \rightarrow \infty ]{} z.
  \]
\end{definition}

\begin{remark}
  \[
    z_n \rightarrow z \text{ в } \C \iff \left\{\begin{array}{l}
      \Re z_n \rightarrow \Re z \\
      \Im z_n \rightarrow \Im z
    \end{array}\right. \text{ в } \R,
  \]

  \[
    \abs{z_n - z} = \sqrt{(\Re z_n - \Re z)^2 + (\Im z_n - \Im z)^2} \geqslant \abs{\Re z_n - \Re z},
  \]
  \[
    \Re(z_1 \pm z_2) = \Re z_1 \pm \Re z_2.
  \]
\end{remark}

\begin{crit}[Коши]
  Последовательность $\{z_n\}_{n \in \N}\subset \C$ сходится $\iff \forall \epsilon > 0 \ \exists n_0 \in \N: \ \forall n,m \geqslant n_0$
  \[
    \abs{z_n - z_m} < \epsilon.
  \]
\end{crit}

\begin{crit}[Коши в $\overline{\C}$]
  Последовательность $\{z_n\}_{n \in \N}\subset \overline{\C}$ сходится $\iff \forall \epsilon > 0 \ \exists n_0 \in \N: \ \forall n,m \geqslant n_0$
  \[
    \rho(z_n,z_m) < \epsilon,
  \]
  \[
    z_n \xrightarrow[n \rightarrow \infty ]{} z \iff \rho(z_n,z)\xrightarrow[n \rightarrow \infty ]{} 0.
  \]
\end{crit}

\begin{crit}[Компактности (расширенный)]
  Подмножество $D \subset \overline{\C}$ компактно $\iff \forall $ его последовательность имеет сходящуюся подпоследовательность: $D \subset \overline{\C} \ \forall \{z_n\}_{n \in \N} \subset D \ \exists \{z_{n_k}\}_{k \in \N} \subset \{z_n\}_{n \in \N}:$ 
  \[
    z_{n_k} \rightarrow z \in D.
  \]

  Пусть $\{z_n\}_{n \in \N} \subset \C:$
  \[
    \sum_{n=1}^{\infty }z_n = \lim_{n \rightarrow \infty }S_n.
  \]
\end{crit}

\begin{definition}[Числовой ряд]
  \emph{Числовым рядом} называется формальная сумма членов.
\end{definition}

\begin{definition}[Абсолютно сходящийся числовой ряд]
  Числовой ряд называется \emph{абсолютно сходящимся}, если сходится ряд
  \[
    \sum_{n=1}^{\infty }\abs{z_n}.
  \]
\end{definition}

\begin{crit}[Коши (сходимости ряда)]
  $\sum_{n=1}^{\infty }z_n$ сходится $\iff \forall \epsilon > 0 \ \exists m \in \N: \ \forall n \geqslant m \ \forall k \in \N$
  \[
    \underbrace{\abs{z_{n+1}+z_{n+2}+\ldots+z_{n+k}}}_{\abs{S_{n+k}-S_n}} < \epsilon.
  \]
\end{crit}

\begin{corollary}
  Если ряд сходится, то его общий член стремится к $0$.
\end{corollary}

\begin{corollary}
  Каждый абсолютно сходящийся числовой ряд сходится.
\end{corollary}

\subsection{Пути, кривые и области}

\begin{definition}[Путь]
  Путем $\gamma:[a ; b] \rightarrow \C$ называется непрерывное отображение $[a ; b]$ в $\C$.
\end{definition}

\begin{eg}
  $\gamma(t) = e^{it},$
  \begin{multicols}{2}
    \begin{figure}[H]
      \centering
      \incfig{fig-7}
      \label{fig:fig-7}
    \end{figure}

    \columnbreak

    \[
      0 \leqslant t \leqslant 2\pi.
    \]
  \end{multicols}
\end{eg}

\begin{definition}
  $\gamma_1 : [a_1 ; b_1] \rightarrow \C, \ \gamma_2 : [a_2 ; b_2] \rightarrow \C$. $\gamma_1 \sim \gamma_2$, если $\exists $ возрастающая непрерывная функция $\phi: [a_1 ; b_1] \xrightarrow[]{\text{на}} [a_2 ; b_2]:$
  \[
    \gamma_1(t) = \gamma_2\big(\phi(t)\big), \quad \forall t \in [a_1 ; b_1].
  \]
\end{definition}

\begin{eg}
  \begin{multicols}{2}
    \[
      \begin{array}{ll}
        \gamma_1(t) = t & 0 \leqslant t \leqslant 1 \\
        \gamma_2(t) = \sin t & 0 \leqslant t \leqslant \frac{\pi}{2} \\
        \gamma_3(t) = \sin t & 0 \leqslant t \leqslant \pi \\
        \gamma_4(t) = \cos t & 0 \leqslant t \leqslant \frac{\pi}{2}
      \end{array}
    \]
    $\phi(t) = \arcsin t$,
    \[
      \gamma_1(t) = \gamma_2\left(\phi(t)\right).
    \]
    \begin{figure}[H]
      \centering
      \incfig{fig-8}
      \label{fig:fig-8}
    \end{figure}
  \end{multicols}
\end{eg}

\lesson{3}{от 21 фев 2024 8:46}{Продолжение}


\section{Независимость событий}

Пусть $(\Omega,\mathcal{F},P)$ -- вероятностное пространство, $B \in \mathcal{F}, \ P(B) > 0$.

\begin{definition}
  Пусть $P(B)>0$ условий. Все исходы -- это $B$, исходы $AB \implies$
  \[
    P(A|B) = \frac{P(AB)}{P(B)}.
  \]
  \begin{figure}[H]
    \centering
    \incfig[0.4]{fig-10}
    \label{fig:fig-10}
  \end{figure}
\end{definition}

\begin{theorem}[Умножение вероятностей]
  \[
    P(AB) = P(A|B)\cdot P(B).
  \]
\end{theorem}

\begin{proof}
  Очевидно.
\end{proof}

Получили новое вероятностное пространство $(\Omega,\mathcal{F},P_B))$,
\[
  P_B(A) = \frac{P(AB)}{P(B)}\geqslant 0, \qquad P_B(\Omega) = \frac{P(\Omega B)}{P(B)} = 1,
\]
\[
  A_n \searrow \varnothing \implies A_nB \searrow \varnothing,
\]

\[
  P_B(A_n) = \frac{P(A_nB)}{P(B)} \rightarrow 0.
\]

\begin{eg}
  $N$ шаров, $M$ белых.

  Вытаскиваем два исхода по очереди:
  \[
    P(\text{оба белых})
  \]
  \[
    \begin{array}{lcl}
      A \text{ -- }& \text{``1'' белый }P(A)&=\frac{M}{N}, \\
      B \text{ -- }& \text{``2'' белый }P(B|A)&=\frac{M-1}{N-1}
    \end{array}
  \]
  \[
    P(AB) = P(B|A)\cdot P(A)=\frac{M(M-1)}{N(N-1)}.
  \]
\end{eg}

\begin{eg}
  Два шара одновременно.

  Исход -- пара шаров, неупорядоченных, без повторений.
  \[
    (\Omega) = C_{N}^{2}, \ \abs{A} = C_{M}^{2},
  \]
  \[
    P(\text{оба белых}) = \frac{C_{M}^{2}}{C_{N}^{2}} = \frac{M!2!(N-2)!}{2!(M-2)!N!} = \frac{M(M-1)}{N(N-1)}.
  \]
\end{eg}

\begin{theorem}
  Пусть $A_1,A_2,\ldots,A_n \in \mathcal{F}$, тогда
  \begin{align*}
    P(A_1 \ldots A_n) &=P(A_1 \ldots A_{n-1}|A_n)\cdot P(A_n) \\
    &=P(A_1|A_2 \ldots A_n)\cdot P(A_2|A_3 \ldots A_n)\cdot \ldots \cdot P(A_{n-1}|A_n)\cdot P(A_n).
  \end{align*}
\end{theorem}

\begin{proof}\leavevmode
  \begin{description}
    \item[\boxed{\text{База}}] $P(A_1A_2) = P(A_1 | A_2)P(A_2)$.
    \item[\boxed{\text{Переход}}] Пусть верно для $A_2 \ldots A_{n+1}$, добавим $A_1$:
    \[
      B = A_2 \ldots A_{n+1},
    \]
    \begin{align*}
      P(A_1B) &= P(A_1A_2 \ldots A_{n+1}) \\
      &= P(A_1|A_2 \ldots A_{n+1})\cdot \big(P(A_2|A_3 \ldots A_{n+1})\cdot P(A_{n+1})\big).
    \end{align*}
  \end{description}
\end{proof}

\begin{definition}[Разбиение]
  \emph{Разбиение} -- множество событий $H_1,\ldots,H_n$ таких, что
  \begin{enumerate}
    \item $H_iH_j = \varnothing, \ \forall i \ne j$.
    \item $\sum_{j=1}^{n}H_j = \Omega$.
  \end{enumerate}
\end{definition}

\begin{theorem}[Формула полной вероятности]
  Пусть $A$ -- случайные события, $H_1,\ldots,H_n$ -- разбиения, тогда
  \[
    P(A) = \sum_{j=1}^{n}P(A|H_j)\cdot P(H_j).
  \]
\end{theorem}

\begin{proof}
  $\sum H_j = \Omega, \quad A \cdot \sum H_j = A$,
  \[
    A \cdot H_j \cap AH_j = \varnothing,
  \]
  \begin{align*}
    P(A) &= P\left(A \cdot \sum H_j\right) \\
    &= P\left(\sum_{j}A H_j\right) \\
    &= \sum_{j}P(AH_j) \\
    &= \sum_{j}P(A|H_j)P(H_j)
  \end{align*}
\end{proof}

\begin{eg}
  $N$ билетов, $M$ хороших.
  \[
    \begin{array}{l}
      A_1 \text{ -- зашли 1ым и вытащили хороший билет} \\
      A_2 \text{ -- зашли 2ым и вытащили хороший билет}
    \end{array}
  \]
  $P(A_1) = \frac{M}{N}, \quad A_1$ и $\overline{A_1}$ -- разбиение,
  \begin{align*}
    P(A_2) &= P(A_2|A_1)\cdot P(A_1) + P(A_2|\overline{A_1})P(\overline{A_1}) \\
    &= \frac{M-1}{N-1} \cdot \frac{M}{N} + \frac{M}{N-1}\cdot \frac{N-M}{N} \\
    &= \frac{M}{N(N-1)}(M-1+N-M) \\
    &= \frac{M}{N}
  \end{align*}
\end{eg}

\begin{theorem}[Формула Байесса]
  $H_1,\ldots,H_n$ -- разбиение, $A \in \mathcal{F}$,
  \[
    P(H_i|A) = \frac{P(A|H_i)P(H_i)}{\sum_{j}P(A|H_j)P(H_j)},
  \]
  апостериорные вероятности,
  \[
    P(H_i) \text{ -- априорные вероятности}.
  \]
\end{theorem}

\begin{proof}
  \[
    P(H_i|A) = \frac{P(AH_j)}{P(A)} = \frac{P(A|H_i)\cdot P(H_i)}{\sum_{j}P(A|H_j)\cdot P(H_j)}.
  \]
\end{proof}

\begin{eg}
  $N$ билетов, $M$ -- хороших, $A_1,A_2$,
  \begin{align*}
    P(A_1) &= \frac{M}{N}, \\
    P(A_1|A_2) &= \frac{P(A_1A_2)}{P(A_2)} = \frac{P(A_2|A_1)P(A_1)}{P(A_2)} = \frac{\frac{M-1}{N-1}\cdot \frac{N}{M}}{\frac{M}{N}} = \frac{M-1}{N-1}, \\
    P(A_2|A_1) &= \frac{M-1}{N-1}.
  \end{align*}
\end{eg}

\begin{definition}[Независимые события]
  $(\Omega,\mathcal{F},P)$ -- в.п., $A,B \in \mathcal{F}$.

  Говорят, что $A$ и $B$ \emph{независимы}, если $P(A|B) = P(A)$ ($A$ не зависит от $B$).
\end{definition}

\begin{remark}
  Если $A$ не зависит от $B$, то $B$ не зависит от $A$:
  \[
    P(B|A) = \frac{P(AB)}{P(A)} = \frac{P(A|B)\cdot P(B)}{P(A)} = \frac{P(A)\cdot P(B)}{P(A)} = P(B).
  \]
\end{remark}

\begin{remark}
  Если $A$ и $B$ независимы, то
  \[
    P(AB) = P(A|B)\cdot P(B) = P(A)\cdot P(B).
  \]
\end{remark}

\begin{definition}[Независимые в совокупности]
  $A_1,\ldots,A_n$ \emph{независимы в совокупности}, если
  \[
    P(A_1A_2 \ldots A_n) = P(A_1)P(A_2)\ldots P(A_n).
  \]
\end{definition}

\begin{eg}
  $A_1,A_2$ -- орел в $1^{\text{м}}$ и $2^{\text{м}}$ бросках,
  \begin{align*}
    P(A_1) &= \frac{1}{2} = P(A_2) \\
    P(A_1A_2) &= \frac{1}{4} = P(A_1)P(A_2)
  \end{align*}
\end{eg}

\begin{eg}
  52 карты, вытаскиваем одну, $A$ -- туз, $B$ -- бубновая,
  \begin{align*}
    P(A) &= \frac{4}{52} = \frac{1}{13} \\
    P(B) &= \frac{13}{52} = \frac{1}{4} \\
    P(AB) &= \frac{1}{52} = P(A)\cdot P(B)
  \end{align*}
\end{eg}

\begin{eg}
  $2,3,5,30$, $A_2,A_3,A_5$, $A_k: \text{число} \vdots k$.

  Выбираем одно число:
  \begin{align*}
    P(A_2) &= \frac{1}{2} = P(A_3) = P(A_5) \\
    P(A_2A_3) &= \frac{1}{4} = P(A_2)\cdot P(A_3) \implies A_2 \text{ и } A_3 \text{ независимы}
  \end{align*}
  \begin{align*}
    A_2 \text{ и } A_5 & \text{ -- независимы}, \\
    A_3 \text{ и } A_5 & \text{ -- независимы},
  \end{align*}
  \[
    P\equalto{(A_2A_3A_5)}{\frac{1}{4}} \ne \underset{\frac{1}{2}\cdot \frac{1}{2} \cdot \frac{1}{2}}{P(A_2)P(A_3)P(A_5))}.
  \]
\end{eg}

\begin{remark}
  ``зависимые'' = ``не являются независимыми''.
\end{remark}

\begin{theorem}
  Если $A_1,\ldots,A_n$ независимы в совокупности, $P(A_1 \ldots A_n) > 0, \ i_1 \ldots i_k j_1 \ldots j_m$ -- различные индексы от $1$ до $n$, то
  \[
    P(A_{i_1}\ldots A_{i_k}|A_{j_1}\ldots A_{j_m}) = P(A_{i_1}\ldots A_{i_k}).
  \]
\end{theorem}

\begin{proof}
  $P(A_{i_1}\ldots A_{i_k}) = P(A_{i_1})\ldots P(A_{i_k})$,
  \[
    P \left(\underbrace{A_{i_1}\ldots A_{i_k}}_{A}\underbrace{A_{j_1}\ldots A_{j_m}}_{B}\right) = P(A_{i_1})\ldots P(A_{i_k})P(A_{j_1})\ldots P(A_{j_m}),
  \]
  \[
    P(AB) = P(A)P(B),
  \]
  $\implies P(A|B) = \frac{P(AB)}{P(B)} = P(A)$.
\end{proof}

\begin{definition}[Алгебра, порожденная $\gamma$]
  $\gamma$ -- некоторое конечное семейство множеств $A_1,\ldots,A_n$.

  \emph{Алгебра ($\sigma$-алгебра), порожденная $\gamma$} -- это минимальная по включению алгебра ($\sigma$-алгебра), содержащая все элементы $\gamma$.

  Пусть $A_1 \ldots A_n$ -- разбиение $\Omega$ -- $\alpha$,
  \begin{align*}
    \mathcal{A}(\alpha) &\text{ -- } \sigma \text{-алгебра, порожденная } \alpha, \\
    \mathcal{A}(\alpha) &\text{ -- конечно, содержит все множества вида }A_{i_1} + \ldots + A_{i_k}.
  \end{align*}
\end{definition}

\begin{theorem}
  Любая конечная $\sigma$-алгебра порождена некоторым разбиением.
\end{theorem}

\begin{proof}
  $\mathcal{B}$ -- конечная $\sigma$-алгебра, $w \in \Omega, \quad \mathcal{B}_w = \{B \in \mathcal{B}: \ w \in B\}$,
  \[
    B_w = \bigcap_{B \in \mathcal{B}_w}B \ni w \quad \text{и} \quad w \in B \implies B \supset B_w. 
  \]

  $w_1,w_2 \in \Omega$, покажем, что $B_{w_1} = B_{w_2}$ или $B_{w_1}\cap B_{w_2} = \varnothing$.

  Пусть $B_{w_1}\cap B_{w_2} \ne \varnothing $, $w \in B_{w_1}\cap B_{w_2}$,
  \[
    w \in B_{w_1} \implies B_{w_1} \supset B_w \implies \forall B \in \mathcal{B}_{w_1} \quad w \in B,
  \]
  \[
    \begin{array}{cc}
      \bigcap_{B \in \mathcal{B}_{w_1}} &\subset \bigcap_{w\in B}B = B_w \\
      \verteq & \\
      B_{w_1} &
    \end{array} \implies B_{w_1} = B_w = B_{w_2}.
  \]
\end{proof}

\begin{definition}[Независимые $\sigma$-алгебры]
  $\mathcal{A}_1 \ldots \mathcal{A}_n$ -- $\sigma$-алгебры, они \emph{независимы}, если $A_1 \ldots A_n$ независимы для всех $A_i \in \mathcal{A}_i$.
\end{definition}

\begin{theorem}
  Конечные $\mathcal{A}_1,\ldots,\mathcal{A}_n$ $\sigma$-алгебры независимы $\iff$ независимы порождающие их разбиения.
\end{theorem}

\begin{lemma}
  $A$ и $B$ независимы, тогда $A$ и $\overline{B}$ тоже независимы.
\end{lemma}

\begin{proof}
  \begin{align*}
    P(A\overline{B}) &= P(A) - P(AB) \\
    &= P(A)-P(A)P(B) \\
    &= P(A)\left(1-P(B)\right) \\
    &= P(A)P(\overline{B}).
  \end{align*}
\end{proof}

\newpage

\lesson{4}{от 28 мар 2024 10:30}{Продолжение}


\section{Фазовый портрет нелинейной системы}

\begin{definition}[Предельный цикл системы]
	\emph{Предельным циклом системы} называется замкнутая фазовая кривая, у которой существует окружность, целиком заполненная траекториями, точки на траектории движутся к этой замкнутой привой при $t \rightarrow +\infty $ или $t \rightarrow - \infty $.
\end{definition}

\begin{note}
	Устойчивый предельный цикл содержит неустойчивый фокус. Неустойчивый предельный цикл содержит устойчивый фокус.
\end{note}

\subsection{Исследование устойчивости с помощью функций Ляпунова}

Рассмотрим нелинейную систему:
\[
	(1) \ \dot{\overline{x} } = \overline{F} (t,\overline{x} ), \ F = \left(\begin{array}{c}
			f ' \\ \vdots \\ f_n
		\end{array}\right), \ \overline{x} = \left(\begin{array}{c}
			x_1 \\ \vdots \\ x_n
		\end{array}\right)f_i,
\]
\[
	\frac{\partial f_i}{\partial x_k} \in C(D), \ t \geqslant t_0, \ v(t,\overline{x} )\in C(R^{n} ).
\]

\begin{definition}
	Производная $v(t,\overline{x} )$ в силу системы $(1)$ определяется по формуле:
	\[
		\frac{dv}{dt} \Big|_{(1)} = \frac{\partial v}{\partial k} + \frac{\partial v}{\partial x_1} f_1 + \frac{\partial v}{\partial x_2} f_2 + \ldots + \frac{\partial v}{\partial x_k} f_n = \frac{\partial v}{\partial k} + \sum_{i=1}^{n} \frac{\partial v}{\partial x_i} f_i.
	\]
\end{definition}

Рассмотрим автономную систему:
\[
	(2) \ \dot{\overline{x} } = F(\overline{x} ).
\]

Будем предполагать, что $F(\overline{0} ) = \overline{0} $, $\overline{0} $ -- особая точка $(2)$.

\begin{definition}[Функция Ляпунова]
	Функция $V(x) = V(x_1,x_2,\ldots ,x_n)$, определенная на шаре $\|\overline{x} \| < R$, называется \emph{функцией Ляпунова}, если:
	\begin{enumerate}
		\item $V(\overline{x} ) \in C^1 \ (\|\overline{x} \| < R)$.
		\item $V(\overline{x} ) \geqslant 0$ в $\|\overline{x} \| < R ; \ V(\overline{x} ) = 0 \iff \overline{x} = \overline{0} $.
		\item $\frac{dV}{dt} \Big|_{(1)} = \frac{\partial V}{\partial x_1} f_1 + \frac{\partial V}{\partial x_2} f_2 + \ldots + \frac{\partial V}{\partial x_n} f_n = (\grad V, \ F) \leqslant 0$ в $0 < \|\overline{x} \| < R$.
	\end{enumerate}
\end{definition}

\begin{theorem}[Ляпунова об устойчивости]
	Если $\exists $ функция Ляпунова для системы $(2)$, то нулевое решение устойчиво по Ляпунову.
\end{theorem}

\begin{theorem}[Ляпунова об асимптотической устойчивости]
	Если $\exists $ функция $V(\overline{x} ) = V(x_1,\ldots ,x_n)$, опр. на шаре $\|\overline{x} \| < R$, со свойствами:
	\begin{enumerate}
		\item $V(\overline{x} ) \in C^1 \ (\|\overline{x} \| < R)$.
		\item $V(\overline{x} )\geqslant 0$ в $\|\overline{x} \| < R, \ V(\overline{x} ) = 0 \iff \overline{x}  = \overline{0}  $.
		\item $\frac{dV}{dt} \Big|_{(1)} = (\grad V, F)\leqslant -w(x) < 0$ в $0 < \|\overline{x} \| < R, \ w(x) \in C(\|\overline{x} \| < R)$.
	\end{enumerate}

	Тогда нулевое решение $(2)$ асимптотически устойчиво.
\end{theorem}

\begin{theorem}[Ляпунова о неустойчивости]
	Если $\exists $ функция \\ $V(\overline{x} ) = V(x_1,x_2,\ldots ,x_n)$ опр. на шаре $\|\overline{x} \| < R$, со свойствами:
	\begin{enumerate}
		\item $V(\overline{x} ) \in C^1 \ (\|\overline{x} \| < R)$.
		\item $V(\overline{x} ) \geqslant 0$ в $\|\overline{x} \| < R ; \ V(\overline{x} ) = 0 \iff \overline{x} = \overline{0} $.
		\item $\frac{dV}{dt} \Big|_{(2)} = (\grad V, F) \geqslant w(x)> 0$ в $0 < \| \overline{x} \| < R,\ t \geqslant t_0,\ w(x) \in C \ (\|\overline{x} \| < R)$.
	\end{enumerate}

	Тогда нулевое решение $(2)$ неустойчиво.
\end{theorem}

\begin{theorem}[Четаева о неустойчивости]
	Если $\exists $ область $D$, причем $\overline{0} \in \partial D$ и $\exists $ функция $V(\overline{x} ) = V(x_1,\ldots ,x_n)$ опр. в $\|\overline{x} \| < R$, удовлетворяет условиям:
	\begin{enumerate}
		\item $V(\overline{x} )\in C^1 \ (\|\overline{x} \| < R)$.
		\item $V(\overline{x} )\geqslant 0$ в $D, \ V(\overline{x} ) = 0 \iff \overline{x} \in \partial D$.
	\end{enumerate}
\end{theorem}

\newpage

\lesson{5}{от 14 мар 2024 12:45}{Продолжение}


\section{Теория интеграла Коши}

\subsection{Определения и основные свойства интеграла Коши}

\begin{definition}[Разбиение кривой Жордана]
	Пусть $\gamma $ -- кривая Жордана, $\gamma \in \C$ с концами $\alpha , \beta  \in \C$.

	\begin{figure}[H]
		\centering
		\incfig[0.7]{fig-14}
		\label{fig:fig-14}
	\end{figure}

	Разбиением кривой Жордана назовем $\sigma \coloneq \{z_0,z_1,\ldots ,z_n,\xi_0,\ldots \xi_{n-1} \}$, где $n \in \N, \ z_0 = \alpha , \ z_n = \beta , \ z_{k+1} \notin \overbrace{z_0z_k} \ \forall k \in \overline{0,n-1} , \ \zeta _k \in \overbrace{z_k,z_{k+1} } $
	\[
		\triangle z_k \coloneq z_{k+1} -z_k,
	\]
	\[
		\d(\sigma ) \coloneq \underset{0 \leqslant k < n-1}{\max} \abs{\triangle z_k} \text{ -- диаметр разбиения } \sigma.
	\]
\end{definition}

\begin{definition}
	Если $f: \gamma \rightarrow \C, \ \sigma $ -- интегральная сумма, то
	\[
		S_\sigma (f) \coloneq \sum_{k=1}^{n-1} f(S_k)\underbrace{(z_{k+1} -z_n)}_{\triangle z_k}.
	\]
\end{definition}

\begin{definition}
	$\prod (\gamma )$ -- множество всех разбиений кривой $\gamma $,
	\[
		\Phi : \prod(\gamma ) \rightarrow \C.
	\]

	Будем говорить, что $\exists \underset{d(\sigma )\rightarrow 0}{\lim} \Phi (v) =w \in \C$, если $\forall \epsilon > 0 \ \exists \delta > 0: \ \forall \sigma \in \prod(\gamma ) \ \d(\sigma )< \delta \implies \big|\Phi (\sigma ) - w\big| < \epsilon $.
\end{definition}

\begin{definition}[Интеграл Коши]
	Если $f: \gamma  \rightarrow \C$ и $\exists \underset{\d(\sigma )\rightarrow 0}{\lim} S_{\sigma } (f)\in \C$, то
	\[
		\int_{\gamma } f(z)\d z \coloneq \underset{\d(\sigma )\rightarrow 0}{\lim} S_{\sigma } (f)
	\]
	называется \emph{интегралом Коши} от функции $f$ по кривой $\gamma $.
\end{definition}

\begin{theorem}
	Если $f$ непрерывна на спрямляемой кривой Жордана $\gamma $, то $\int_{\gamma } f(z)\d z$ существует (то есть является элементом $\C$).
\end{theorem}

\begin{proof}
	$f(z) = f(x + iy) = u(x,y) + iv(x,y)$,
	\begin{multline*}
		\int_{\gamma } f(z)\d z = \int_{\gamma } \big(u(x,y)+iv(x,y)\big)\d(x+iy) = \\ = \int_{\gamma } u\d x - v\d y + \int_{\gamma } v\d x + u\d y \in \C.
	\end{multline*}
\end{proof}

\subsection{Интегральная теорема Коши}

\begin{lemma}[Грусса]
	Если функция $f$ непрерывна в области $D$, то для любой спрямляемой кривой Жордана $\gamma \subset D$, для любого $\epsilon > 0$ существует вписанная в $\gamma $ ломанная $P$ такая, что
	\[
		\left|\int_{\gamma } f(z)\d z - \int_{P} f(z)\d z\right| < \epsilon.
	\]
\end{lemma}

\begin{theorem}[Интегральная теорема Коши]
	Пусть $D$ -- односвязная область в $\C$, функция $f$ голоморфна в $D$. Тогда для любой замкнутой спрямляемой кривой Жордана $\gamma $
	\[
		\int_{\gamma } f(z)\d z = 0.
	\]
\end{theorem}

\begin{proof}
	Пусть $\gamma $ -- $\triangle$ в $D$.
	\begin{figure}[H]
		\centering
		\incfig[0.7]{fig-15}
		\label{fig:fig-15}
	\end{figure}

	Докажем, что интеграл по этому треугольнику равен нулю. Допустим противное:
	\[
		\left|\int_{\gamma } f(z)\d z\right| \eqcolon M \ne 0.
	\]

	$\gamma _1, \gamma _2, \gamma _3, \gamma _4, \quad \int_{\gamma } f(z)\d z = \sum_{k=1}^{4} \int_{\gamma _n} f(z)\d z$,
	\[
		\left|\int_{\gamma } f(z)\d z\right| \leqslant \sum_{k=1}^{n} \left|\int_{\gamma k} f(z)\d z\right|,
	\]
	\[
		\overline{\triangle_0} \coloneq \gamma , \quad \overline{\triangle_1} \coloneq \gamma _i : \ \left|\int_{\gamma _i} f(z)\d z\right| \geqslant  \frac{M}{4} ,
	\]
	\[
		\exists \overline{\triangle_2} : \ \left|\int_{\overline{\triangle_2} }f(z)dz \right| \geqslant  \frac{M}{4^2} .
	\]

	Продолжая этот процесс, мы получм последовательность $\{\overline{\triangle_k} \}:$
	\[
		\left|\int_{\overline{\triangle_k} } f(z)\d z\right| \geqslant \frac{M}{4^k} ,
	\]
	\[
		D(\overline{\triangle_{k+1} } ) \subset D(\overline{\triangle_k} ).
	\]

	То есть можем считать эту последовательность $\{\overline{\triangle_k} \}$ как последовательность вложенных множеств $\implies \exists z_0 \in \underset{k \in \N}{\bigcap} D(\overline{\triangle_k} ) \ne \varnothing $.

	????????

	В силу произвольности $\epsilon $ получаем, что $M = 0$,
	\[
		\left|\int_{\gamma } f(z)\d z - \int_{P} f(z)\d z\right| < \epsilon .
	\]
\end{proof}

\begin{theorem}[Обобщенная интегральная теорема Коши]
	Если функция $f$ голоморфна в односвязной области $D$, ограниченной замкнутой спрямляемой кривой Жордана $\gamma $ и $f$ непрерывна вплоть до границы, то есть $\forall z_0 \in \gamma $
	\[
		\underset{D \ni z \rightarrow z_0}{\lim} f(z) = f(z_0) \implies \int_{\gamma } f(z)\d z = 0.
	\]
\end{theorem}

\begin{corollary}
	Если область $D$ ограничена конечным числом замкнутых спрямляемых кривых Жордана. Если $f$ голоморфна в этой области ???
\end{corollary}

\begin{corollary}
	Утверждение обобщенной теоремы остается в силе, если условие голоморфности функции $f$ в области нарушается в конечном количестве точек $z_1,\ldots z_n \in D$, в которых функция ведет себя так:
	\[
		\underset{\exists \rightarrow z_k}{\lim} (z-z_k)f(z) = 0 \quad (0 \leqslant k \leqslant n).
	\]
\end{corollary}

\subsection{Интегральная формула Коши, интеграл типа Коши}

\begin{theorem}[Интегральная формула Коши]
	Если функция $f$ голоморфна в односвязной области $D$, ограничена замкнутой спрямляемой кривой Жордана $\gamma $, непрерывна вплоть до границы, то
	\[
		\frac{1}{2\pi i} \int_{\gamma } \frac{f(z)}{z-z_0} \d z = \left\{ \begin{array}{l}
			f(z_0), \text{ если } z_0 \in D \\
			0, \text{ если } z_0 \notin \cl D
		\end{array}\right.
	\]
\end{theorem}

\begin{definition}[Интеграл типа Коши]
	Пусть односвязная область $D$ ограничена замкнутой спрямляемой кривой Жордана $\gamma $, а функция $f$ непрерывна на $\gamma $. Положим
	\[
		F(z) = \frac{1}{2\pi i} \int_{\gamma } \frac{f(\xi )}{\xi - z} \d \xi , \ z \in D.
	\]

	Эта функция $F$ называется \emph{интегралом типа Коши}.
\end{definition}

\begin{theorem}[Лиувилль]
	Если функция $f$ голоморфна в $\C$ и ограничена, то $f \equiv const$.
\end{theorem}

\newpage

\begin{proof}
	$R > 0, \ z \in \C$
	\[
		f'(z) = \frac{1}{2\pi i} \int_{|\xi - z| = R} \frac{f(\xi )}{(\xi  - z)^2} \d \xi .
	\]

	Пусть $M > 0: \ \underset{z \in \C}{\sup} \big|f(z)\big|\leqslant M \implies$
	\[
		\big|f '(z)\big| \leqslant \frac{1}{2\pi} \int_{|\xi - z| = R} \frac{\abs{f(\xi )} }{\abs{\xi -z}^2 } \abs{\d \xi } \leqslant \frac{1}{2\pi} \cdot \frac{M}{R^2} \cdot 2\pi R = \frac{M}{R} \xrightarrow[R \rightarrow +\infty ]{}0 \implies
	\]
	$\implies f '(z) = 0 \ (\forall z \in \C)$.

	\[
		u_{x}^{'} = u_{y}^{'} = v_{x}^{'} = v_{y}^{'} = 0 \implies  u = const, \ v = const \implies f = const.
	\]
\end{proof}

\subsection{Неопределенный интеграл теорем Мореры и Вейерштрасса}

\begin{theorem}
	Непрерывная в односвязной области $D$ функция $f$ голоморфна в этой области $\iff \forall z_0,z \in D \ \int_{z_0}^{z} f(\xi )\d \xi $ не зависит от пути интегрирования, соединяющего области $D$ точек $z_0,z$.
\end{theorem}

\begin{definition}[Первообразная голоморфной в области]
	\emph{Первообразной голоморфной в области} $D$ функции $f$ называется голоморфная в $D$ функция $F: \ \forall  z \in D \ F '(z) = f(z)$.
\end{definition}

\begin{remark}
	Любые две первообразные голоморфной функции отличаются только на константу.
\end{remark}

\begin{definition}[Неопределенный интеграл]
	Совокупность всех первообразных голоморфной функции называется ее \emph{неопределенным интегралом}.
	\begin{notation}
		$\int f(z)\d z = F(z) + c $.
	\end{notation}
\end{definition}

\begin{remark}
	Если функция $f$ голоморфна в области $D$ и $F$ -- ее первообразная, то $\forall z_0,z \in D$
	\[
		\int_{z_0}^{z} f(\xi )\d \xi = F(z) - F(z_0).
	\]
\end{remark}

\begin{theorem}[Морера]
	Для того, чтобы непрерывная в односвязной области функция была голоморфна в этой области, необходимо и достаточно, чтобы интеграл от этой функции по любому замкнутому контуру, лежащему в области, был равен $0$.
\end{theorem}

\begin{remark}
	В сторону достаточности условия теоремы Мореры можно ослабить. Если функция непрерывна в односвязной области и $\int_{\triangle} f(z)\d z = 0$, то функция голоморфна $\forall \triangle \in D$.
\end{remark}

\begin{definition}
	Пусть $\{f_n\}_{n \in \N} \subset C(D)$. Говорят, что эта последовательность сходится равномерно к $f$ внутри $D$, если $\forall K \in D \Subset D \ f_n \rightrightarrows f$ на $K$, то есть $\forall \epsilon > 0 \exists n \in \N: \ \forall n \geqslant n_0$
	\[
		\underset{I \in K}{\sup} \big|f_n(z)- f(z)\big| < \epsilon .
	\]
\end{definition}

\begin{theorem}[Вейерштрасса]
	Равномерный предел последовательности голоморфных функций является голоморфной функцией, то есть если $\{f_n\}_{n \in \N} \subset \mathcal{H}(D)$ и $f_n \rightrightarrows f$ внутри $D$, то $f \in \mathcal{H}(D)$.
\end{theorem}

\begin{definition}[Корень многочлена]
	\emph{Корнем многочлена} $P(z)\coloneq a_n z^n + \ldots + a_1z + a_0,$ где $a_0,a_1,\ldots ,a_n \in \C$, называется число $z_0 \in \C: \ P(z_0) = 0$.
\end{definition}

\begin{theorem}[Безу]
	Если $z_0$ -- корень многочлена $P$, то $\exists $ многочлен $Q: \ P(z) = (z-z_0)\cdot Q(z_0)$.
\end{theorem}

\begin{theorem}[Основная теорема алгебры]
	Каждый многочлен с комплексными коэффициентами в $\deg \geqslant 1$ имеет к.б. один комплексный корень.
\end{theorem}

\begin{corollary}
	Каждый многочлен $n$-ой степени имеет $n$ корней.
\end{corollary}

\newpage

\lesson{6}{от 21 мар 2024 12:45}{Продолжение}


\section{Ряды Тейлора и Лорана. Элементы теории вычетов}

\subsection{Разложение голоморфной функции в ряд Тейлора}

\begin{theorem}
	Пусть $f \in \mathcal{H}(D)$. Тогда $\forall z_0 \in D \ \exists r > 0$: при $\abs{z-z_0} < r$
	\[
		f(x) = \sum_{n=0}^{\infty } \frac{f^{(n)} (z_0)}{n!} (z - z_n)^n.
	\]
\end{theorem}

\begin{corollary}
	$\mathcal{H}(D) = \mathcal{A}(D)$.
\end{corollary}

\begin{theorem}
	Пусть $f$ голоморфна в $B_r(z_0) \ \forall z \in B_r(z_0) \ f(z) = \sum_{n=0}^{\infty } C_n (z-z_0)^n$.

	Тогда $\forall n \in \overline{\N}$
	\[
		C_n = \frac{1}{2\pi i} \int_{\abs{\xi -z_0} = \rho} \frac{f(z)}{(\xi - z_0)^{n+1} } \d \xi \quad \forall \rho \in (0,r).
	\]

	То есть любой степенной ряд является рядом Тейлора для своей суммы.
\end{theorem}

\begin{proof}
	Радиус сходимости $\geqslant r, \ \rho \in (0 ; r)$.

	$\abs{z-z_0} = \rho \implies $ ряд сходится, рассмотрим:
	\begin{multline*}
		f(z) = \frac{1}{2\pi i} \int_{\abs{\xi -z_0}= \rho } \frac{f(\xi )}{(\xi -z_0)^{k+1} } \d \xi = \\
		= \frac{1}{2\pi i} \int_{\abs{\xi -z_0} = \rho } \frac{\sum_{n=0}^{\infty } C_n(\xi -z_0)^n}{(\xi -z_0)^{k+1} } \d \xi = \\
		= -\frac{k!}{2\pi i} \cdot C_k \cdot 2\pi i = C_n \cdot k!,
	\end{multline*}
	\[
		C_k = \frac{f^{(k)} (z_0)}{k!} .
	\]
\end{proof}

\begin{theorem}[Неравенство Коши]
	Пусть $f$ голоморфна в $D$ и $B_r[z_0] \subset D, \ f(z) = \sum_{n=0}^{\infty } C_n(z-z_0)^n$.

	Пусть $M \coloneq \underset{\abs{z-z_0} \leqslant r}{\sup} \big|f(z)\big|$. Тогда $\forall n \in \overline{\N} \ \abs{C_n}  \leqslant \frac{M}{r^n} $.
\end{theorem}

\begin{definition}[Предельная точка]
	Точка называется \emph{предельной точкой множества}, если в любой ее окрестности есть точки множества, отличные от данной.
\end{definition}

\begin{corollary}
	Любые две аналитические в области функции, совпадающие на множестве, имеющем в этом множестве предельную точку, тождественно равны.
\end{corollary}

\subsection{Ряды Лорана}

\begin{definition}[Ряд Лорана]
	\emph{Рядом Лорана} называется степенной ряд вида $\sum_{n=-\infty }^{\infty } C_n(z-z_0)^n$. Ряд Лорана раскладывается на сумму двух рядов:
	\[
		\sum_{n=-\infty }^{\infty } C_n (z-z_0)^n \coloneq \sum_{n=0}^{\infty } C_n (z-z_0)^n + \sum_{n=1}^{\infty } C_{-n} (z-z_0)^{-n} .
	\]

	Ряд Лорана сходится $\iff $ сходятся обе его составляющие.

	Область сходимости ряда Лорана: $0 \leqslant r < \abs{z-z_0} < R \leqslant +\infty $.
\end{definition}

\begin{theorem}[О ряде Лорана]
	Если функция $f$ голоморфна в кольце $r < \abs{z-z_0} < R$, то в этом кольце она разлагается в ряд Лорана.

	$f(z) = \sum_{n=-\infty }^{\infty } C_n(z-z_0)^n$ с коэфициентами $C_n$, определяемыми формулами:
	\[
		C_n = \frac{1}{2\pi i} \int_{\abs{\xi -z_0} = \rho} \frac{f(\xi )}{(\xi -z_0)^{n+1} } \d \xi \quad \forall \rho \in (r,R).
	\]
\end{theorem}

\subsection{Классификация изолированных особых точек}

\begin{definition}[Правильная точка]
	Точка $z_0 \in \dom f$ называется \emph{правильной} точкой функции $f$, если $f$ определена в некоторой области и непрерывна в самой функции.
\end{definition}

\begin{definition}[Особая точка]
	\emph{Особой} точкой функции называется предельная точка ее области определения, этой области не принадлежащая.
\end{definition}

\begin{definition}[Изолированная особая точка]
	Особая точка называется \emph{изолированной} особой точкой, если в некоторой ее окрестности других особых точек нет.
\end{definition}

\begin{remark}
	Особая точка функции называется изолированной, если в проколотой окрестности этой точки функция голоморфна.
\end{remark}

\begin{eg}
	\[
		f(z) = \frac{1}{\sin \frac{1}{z} },
	\]

	$z_0 = 0$ -- особая точка,

	$\sin \frac{1}{z} = 0 \implies  \frac{1}{z} = \pi k, \ k \in Z$,

	$z_k = \frac{1}{\pi k} , \ k \in \Z$ -- особые точки,

	\[
		\frac{1}{\pi(k+1)} < \frac{1}{\pi k} < \frac{1}{\pi (k-1)}.
	\]
\end{eg}

\begin{theorem}[О путях и полюсах]
	Изолированная особая точка $z_0$ функции $f$ является полюсом порядка $m$ функции $f \iff $ она является путем $m$-го порядка функции $\rho (z)= \frac{1}{f(x)} $.
\end{theorem}

\begin{proof}
	Самостоятельно.
\end{proof}

\begin{theorem}[Сохоцкий]
	Изолированная особая точка функции является существенно особой точкой $\iff $ в любой ее окрестности функция принимает значения сколь угодно близкие к любому числу $a \in \overline{\C} $.
\end{theorem}

\begin{definition}[$A$-точка]
	Пусть $A \in \C$, точка $z$ называется \emph{$A$-точкой} функции $f$, если $f(z) = A$.
\end{definition}

\begin{theorem}[Большая теорема Пикара]
	В окрестности существенно особой точки $z_0$ голоморфной функции $f \ \forall A \in \C$, за исключением быть может одного, существует последовательность $A$-точек функции $f$, сходящаяся к точке $z_0$.
\end{theorem}

\subsection{Вычеты}

\begin{definition}[Вычет функции относительно точки]
	Если $z_0$ -- изолированная особая точка функции $f$, то \emph{вычетом} $f$ относительно $z_0$ называется интеграл $\frac{1}{2\pi i} \int_{\gamma } f(z)\d z$, где $\gamma $ -- произволный контур, ограничивающий область $D$: $f$ непрерывна в $\cl D \setminus \{z_0\}$ и голоморфна в $D \setminus \{z_0\}$, то есть в качестве $\gamma $ можно брать любую окрестность сколь угодно малого радиуса с центром в точке $z_0$.
	\begin{notation}
		$\Res f\big|_{z=z_0}\coloneq \frac{1}{2\pi i} \int_{\gamma } f(z)\d z$.
	\end{notation}
\end{definition}

\begin{theorem}[Основная теорема теории вычетов]
	Пусть $\gamma $ -- замкнутый контур, ограничивающий односвязную область $D$, функция $f$ непрерывна на $\cl D = D \cup \gamma $ и голоморфна внутри $D$, за исключением конечного числа точек. Тогда:
	\[
		\int_{\gamma } f(z)\d z = 2\pi i \sum_{k=1}^{m} \underset{z_k}{\Res} f.
	\]
\end{theorem}

\begin{proof}
	$m = 3$,
	\begin{figure}[H]
		\centering
		\incfig[0.7]{fig-16}
		\label{fig:fig-16}
	\end{figure}
	\[
		\Gamma = \gamma \cup \gamma_{1}^{-} \cup \gamma_{2}^{-} \cup \gamma_{3}^{-},
	\]
	\[
		\int_{\Gamma } f(z)\d z = 0 = \int_{\gamma } f(z)\d z - \int_{\gamma _1} f(z)\d z - \int_{\gamma _2} f(z)\d z - \int_{\gamma _3} f(z)\d z.
	\]
\end{proof}

\begin{theorem}[О сумме вычетов]
	Если функция голоморфна в $\overline{\C} $, за исключением конечного числа изолированных о.т., то
	\[
		\sum_{k=0}^{m} \underset{z_k}{\Res} f = 0.
	\]
\end{theorem}

\subsection{Вычисление интегралов}

\begin{definition}
	Главным значнием по Коши интеграла $\int_{-\infty }^{+\infty } f(x)dx$ называется
	\[
		\underset{R \rightarrow \infty }{\lim} \int_{-R}^{R} f(x)\d x \eqcolon Vp \int_{-\infty }^{\infty } f(x)\d x.
	\]
\end{definition}

\begin{remark}
	Если несобственный интеграл $\int_{-\infty }^{\infty } f(x)\d x$ сходится, то его значение совпадает с его главным значением по Коши, Обратно неверно.
\end{remark}

\begin{lemma}
	Пусть
	\begin{enumerate}
		\item Для некоторого $R_0 > 0$ функция $f$ непрерывна при $\abs{z} > R_0$ и $\Im z \geqslant 0$.
		\item $\underset{R \rightarrow \infty }{\lim} \underset{z \in \gamma _R}{\sup} \big|zf(z)\big| = 0$.
	\end{enumerate}

	Тогда $\underset{R \rightarrow \infty }{\lim} \int_{\gamma_R } f(z)\d z = 0$.
\end{lemma}

\begin{lemma}[Жордана]
	Пусть $\alpha >0$,
	\begin{enumerate}
		\item Для некоторого $R_0 > 0$ функция $f$ непрерывна при $\abs{z} > R_0$ и $\Im z \geqslant 0$.
		\item $\underset{R \rightarrow \infty }{\lim} \underset{z \in \gamma _A}{\sup} \big|f(z)\big| = 0$.
	\end{enumerate}

	Тогда $\underset{R \rightarrow \infty }{\lim} \int_{\gamma R} e^{i \alpha z} f(z)\d z = 0$.
\end{lemma}

\newpage

\subsection{Гармонические функции}

\begin{definition}[Гармоническая функция]
	Определенная в односвязной области $D \subset \R^2$ функция $u(x,y)$ называется \emph{гармонической функцией}, если $u \in C^2(D)$ и
	\[
		\triangle u \coloneq \frac{\partial^2 u}{\partial x^2} + \frac{\partial^2 u}{\partial y^2} \equiv 0,
	\]
	где $\triangle$ -- оператор Лапласа.
\end{definition}

\begin{theorem}
	Если функция $f$ голоморфна в односвязной области $D \subset \C$, то ее вещественная и мнимая части являются гармоническими функциями в этой области.
\end{theorem}

\begin{proof}
	\[
		f(z) = f(x + iy) = u(x,y) + iv(x,y),
	\]
	\[
		\frac{\partial u}{\partial x} = \frac{\partial v}{\partial y} , \ \frac{partial u}{\partial y} = - \frac{\partial v}{\partial x},
	\]
	\[
		\frac{partial^2 u}{\partial y^2}  = \frac{\partial}{\partial y} \left(\frac{\partial u}{\partial y} \right) = \frac{\partial}{\partial y} \left(-\frac{\partial v}{\partial x} \right) = - \frac{\partial^2 v}{\partial y \partial x} .
	\]

	Получаем, что смеш. производные непрерывны, зачит они равны $\implies \frac{\partial^2u}{\partial x^2} + \frac{\partial^2u}{\partial y^2} \equiv 0 \implies $ вещественная и мнимая части являются гармоническими.
\end{proof}

\subsection{Целые и мероморфные функции}

\begin{definition}[Целая функция]
	Голоморфная в $\C$ функция называется \emph{целой функцией}. Целая функция называется \emph{трансцендентной}, если бесконечность является ее существенно о.т.
\end{definition}

\begin{definition}[Мероморфная функция]
	Функция, голоморфная в области $D$ всюду, за исключением полюсов, называется \emph{мероморфной} в этой области функцией.
\end{definition}

\begin{theorem}[О мероморфной функции]
	Если $\infty $ является устранимой о.т. мероморфной функции, то данная функция является частным двух многочленов, то есть является рациональной функцией.
\end{theorem}

\begin{proof}
	$\infty $ -- изолированная о.т. (в силу условия), $z_1,\ldots,z_n$ -- конечное число оптимальных точек.

	\[
		f(z) = h(z) + \sum_{k=1}^{m} f_k \left(\frac{1}{z - z_0} \right),
	\]
	\[
		\underset{z \rightarrow \infty }{\lim} f(z) = C_0, \quad \underset{z \rightarrow \infty }{\lim} h(z) = C_0
	\]
	$\implies h = const$ (по теореме Лиувиля) $\implies f(z) = C_0 + \sum_{k=1}^{m} f_k \left(\frac{1}{z-z_k} \right) = \frac{P(z)}{Q(z)} $.
\end{proof}

\lesson{7}{от 28 мар 2024 12:45}{Продолжение}


\section{Основные принципы комплексного анализа}

\subsection{Принцип аргумента и Теорема Руше}

\begin{definition}
	Пусть $f$ голоморфна в некоторой проколотой окружности точки $z_0$, а $z_0$ не хуже, чем полюс, тогда:
	\[
		f(z) = \sum_{n}^{\infty } C_n (z-z_0)^n,
	\]
	\[
		M_f(z_0) \coloneq \inf \{n \in \Z: \ C_n \ne 0\}.
	\]
\end{definition}

\begin{lemma}
  Пусть $z_0$ -- обычная точка или полюс функции $f$. Тогда
  \[
    \underset{z_0}{\Res} \frac{f'}{f} = M_f(z_0).
  \]
\end{lemma}

\begin{note}
  $\frac{f'}{f} = \big(\ln f(z)\big)' $ -- логарифмическая производная функции $f$.
\end{note}

\begin{remark}
  Предположим, что есть многозначная функция $\phi $ и кривая $\gamma $. Если мы можем выделить ветвь функции $\phi $, которая будет непрерывна в окружности $\gamma : [a,b] \rightarrow \C, \ \gamma (a), \gamma (b)$, то вариацией этой функции вдоль кривой $\gamma $
\end{remark}

\lesson{8}{от 4 июн 2024 12:45}{Продолжение}


\begin{theorem}[Принцип взаимно однозначного соответствия]
	Пусть $f$ -- голоморфна в области $D, \ \gamma $ -- простой контур в $D: \ D_{\gamma } \subset D$. Если функция $f$ взаимно однозначна на $\gamma $, то $f$ однолистна в $D_{\gamma } $ и, следовательно, осуществяет конформное отображение области $D_{\gamma } $.
\end{theorem}

\subsection{Принцип компактности}

\begin{definition}[Относительно компактное подмножество]
	Подмножество МП называется \emph{относительно компактным}, если его замыкание компактно.
\end{definition}

\begin{definition}[Секвенциально компактное подмножество]
	Подмножество МП называется \emph{севенциально компактным}, если каждая его последовательность имеет подпоследовательность, сходящуюся к элементу этого подмножества.
\end{definition}

\begin{note}
	В МП севенциальная компактность равносильна компактности.
\end{note}

\begin{lemma}
	Подмножество МП является относительно компактным $\iff \forall $ его последовательность имеет сходящуюся подпоследовательность.
\end{lemma}

\begin{remark}
	Подмножество комплексной плоскости компактно $\iff $ оно замкнуто и ограничено.
\end{remark}

\begin{definition}[Равномерно ограниченное множество функций]
	Множество $\mathcal{F}$ функция из $\C$ в $\C$ называется \emph{равномерно ограниченным} на $A \subset \C$, если
	\[
		\underset{f \in \mathcal{F}}{\sup} \underset{z \in A}{\sup} \big|f(z)\big| < +\infty .
	\]
\end{definition}

\begin{definition}[Равностепенно непрерывное множество функций]
	Множество $\mathcal{F}$ функций из $\C$ в $\C$ называется \emph{равностепенно непрерывным} на $A \subset \C$, если $\forall z \in A \ \forall \epsilon > 0 \ \exists \delta > 0 : \ \forall f \in \mathcal{F} \ \forall z ' \in A$ из $\abs{z ' - z} \subset \delta $ следует, что $\big|f(z ') - f(z)\big| < \epsilon $.
\end{definition}

\begin{remark}
	Пусть $K$ -- компакт в $\C$,
	\[
		C(K) \ d (f_1,f_2) \coloneq \underset{z \in K}{\sup} \big|f_1(z) - f_2(z)\big|.
	\]
\end{remark}

\begin{theorem}[Арцела-Асколи]
	Пусть $K$ -- компакт в $\C$. Множество $\mathcal{F}\subset C(k)$ относительно компактно $\iff $ оно равномерно ограничено на $K$ и равностепенно непрерывно на $K$.
\end{theorem}

\begin{definition}[Относительно компактное множество]
	$\mathcal{F} \subset C(D)$ называется \emph{относительно компактным} в $D$, если для $\forall \{f_n\}_{n \in N} \subset \mathcal{F} \ \exists $ ее подпоследовательность $\{f_{n_k} \}_{k \in N} : \ \forall K \Subset D \ f_{n_k} \rightrightarrows f$ на $K$.
\end{definition}

\begin{lemma}
	Пусть $D$ -- область в $\C, \ \mathcal{F}\subset C(D)$ и $\mathcal{F}$ относительно компактно в $C(K) \ \forall K \Subset D$. Тогда $\mathcal{F}$ относительно компактно в $D$.
\end{lemma}

\begin{proof}
	$K_n \eqcolon \{z \in D : \dist (z,\partial D)\geqslant \frac{1}{n} \& \abs{z} \leqslant n\}, \ n \in \N $.

	$K_n \subset K_{n+1} $ и $\underset{n \in \N}{\bigcup} K_n = D$ (стандартная последовательность).

	$\exists \{f_{n_k} \}_{n\in\N} $ -- подпоследовательность $\{f_n\}$.

	\[
		\{f_{n}^{n} \}_{n\in\N} \quad f_{n}^{n} \rightrightarrows \text{ на } K_m \ \forall m \in \N,
	\]

	$\exists m \in \N : \ \frac{1}{m} \subset \dist \implies  K \subset K_m$.
\end{proof}

\begin{definition}[Равномерно ограниченное отображение]
	$f \subset C(D)$ называется \emph{равномерно ограниченным} в $D$, если это множество равномерно ограничено на каждом компакте.
\end{definition}

\begin{lemma}
	Пусть $D$ -- область в $\C$, $K$ -- компакт в $D$ и $V \in O_p \C : \ K \subset V \subset \cl V \subset D :$ если $\mathcal{F} \subset \mathcal{H}(D): \ \mathcal{F}_1 \eqcolon \{f ' : f \in \mathcal{F}\}$ равномерно ограничен на $V$, то $\mathcal{F}$ равностепенно непрерывен на $K$.
\end{lemma}

\begin{proof}
	$z \in K \ \exists \delta _1 > 0: \ B_{\delta _1} [z]\subset V$,
	\[
		z ' \in V : \ \abs{z ' - z } < \delta _1,
	\]
	\[
		\big|f(z ') - f(z)\big| = \left|\int_{z}^{z '}f '(\xi )\d \xi \right| \leqslant \underset{\xi  \in V}{\sup} \big|f '(\xi )\big| \cdot \abs{z ' - z} < M \cdot \delta _1 \leqslant \epsilon .
	\]

	$\delta \eqcolon \min \{\delta _1 ; \frac{\epsilon }{M + 1} \}$.
\end{proof}

\begin{theorem}[Принцип компактности, теорема Ментеля]
	Если $\mathcal{F}\subset A(D)$ равномерно ограничен в $D$, то $\mathcal{F}$ относительно компактно в $A(D)$.
\end{theorem}

\begin{proof}
	$K \subset D \ \exists \gamma : \ K \subset D_{\gamma } \ \& \ \cl D_{\gamma } \subset D$,
	\[
		\delta = \dist(K,\gamma ) > 0.
	\]

	$f '(z) = \frac{1}{2\pi} \int_{\gamma } \frac{f(\xi )}{(\xi - z)^2} \d \xi \ \forall z \in K $.

	$\forall z \in K \ \big|f '(z)\big| \leqslant \frac{M \cdot l (\gamma )}{2\pi \delta ^2} $, где $M = \underset{z \in \gamma }{\sup} \big|f(z)\big| < +\infty $, $l (\gamma )$ -- длина $\gamma \implies $ по пред. лемме, множество равномерно ограниченно $\implies $ по теореме Ацела-Аскяли все доказано.
\end{proof}

\subsection{Принцип непрерывности}

\begin{theorem}[Принцип непрерывности]
	Пусть $f$ непрерывно в $D$ и голоморфна в $D \setminus \gamma  $, где $\gamma $ -- ломанная в $D$, состоящая из конечного числа дуг окружностей. Тогда $f$ голоморфна в $D$.
\end{theorem}

\begin{proof}
	$\gamma = I \coloneq [z_1,z_2]$, возьмем $\triangle \subset D, \ \partial \triangle$,
	\[
		\cl \triangle \cap I = \varnothing , \quad \int_{\partial \triangle}f(z)\d z = 0,
	\]
	\[
		\cl \triangle \cap I \ne \varnothing .
	\]
	$\int_{\partial \triangle} = \int_{\gamma _1}  + \int_{\gamma _2} =0 $ (по обобщенной теореме Коши) каждый из интегралов равен нулю.
\end{proof}

\begin{remark}
	Утверждение теоремы остается в силе, если $\gamma $ будет спрямляемой кривой Жордана.
\end{remark}

\subsection{Принцип симметрии}

\begin{theorem}[Принцип симметрии]
	Пусть $D$ -- область в $\C$ и часть $\gamma $ ее границы является дугой окрестности.

	Если функция $f$ голоморфна в $D$ и непрерывна вплоть до $\gamma $, то функция $\widetilde{f}$, определенная в области $D^{*} $, симметричной области $D$ относительно $\gamma $, равенство $\widetilde{f}(z^{*} ) = (f^{(z)} )^{*} $, будет голоморфно в области $D \cup \gamma \cup P^{*} $.
\end{theorem}

\begin{proof}
	$z^{*} = \overline{z}  $.

	\[
		\widetilde{f}(\overline{z} ) = \overline{f(z)} = \overline{\sum_{n=0}^{\infty } C_n(z-z_0)} = \sum_{n=0}^{\infty } C_n (\overline{z}  - \overline{z_0} )^n
	\]
	раскладывается в ряд Тейлора, $z_0 \in D, \ z_0^{*} = \overline{z_0} \implies $ голоморфна в $D$, на отрезке $\gamma $ непрерывна $\implies $ по теореме $f$ голоморфно в объединении.
\end{proof}

\section{Конформные отображения}

\begin{theorem}[Лемма Шварца]
	$\mathbb{D} \coloneq \{z \in \C: \ \abs{z} \subset 1\}$.

	Пусть $f$ голоморфна в $\mathbb{D}$ и $f(0) = 0, \ \abs{f(z)} \leqslant 1$ в $\mathbb{D}$. Тогда $\big|f '(0)\big| \leqslant 1$ и $\big|f(z)\big| \leqslant \abs{z} \ \forall z \in \mathbb{D}$.
\end{theorem}

\begin{remark}
	Если в одном из неравенств имеет место равенство, то $\exists $ число $\lambda \in \C: \ \abs{\lambda } =1 : \forall z \in \mathbb{D}$
	\[
		f(z) = \lambda z.
	\]
\end{remark}

\begin{remark}
	Если $\phi $ -- конформное отображение $\mathbb{D}$ на себя и $\phi (0) \coloneq 0$, то $\forall  z \in \mathbb{D} \ \phi (z) = \lambda z$, где $\lambda = const, \ \abs{\lambda }  = 1$.
\end{remark}

\begin{theorem}[Римана]
	Если $D$ -- односвязная область, отличная от $\C$, то $\forall z_0 \in D \ \exists $ единственное конформное отображение $\phi : D \rightarrow \mathbb{D}: \ \phi (z_0)= 0$ и $\phi ' (z_0) > 0$.
\end{theorem}


\addcontentsline{toc}{section}{Список используемой литературы}
\begin{thebibliography}{}
	\bibitem{litlink1}  Шабат                         -- «Введение в комплексный анализ, 1976» (том 1)
	\bibitem{litlink2}  Привалов                      -- «Введение в ТФКП, 1967»
	\bibitem{litlink3}  Бицадзе                       -- «Основы теории аналитических функций комплексного переменного, 1984»
	\bibitem{litlink4}  Волковыский, Лунц, Араманович -- «Сборник задач по ТФКП», 1975»
	\bibitem{litlink5}  Гилев В.М.                    -- «Основы комплексного анализа. Ч.1», 2000»
	\bibitem{litlink6}  Исапенко К.А.                 -- «Комплексный анализ в примерах и упражнениях (Ч.1, 2017, Ч.2, 2018)»
	\bibitem{litlink7}  Мещеряков Е.А., Чемеркин А.А. -- «Комплексный анализ. Практикум»
	\bibitem{litlink8}  Боярчук А.К.                  -- «Справочное пособие по высшей математике» (том 4)
\end{thebibliography}
\end{document}
