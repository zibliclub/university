\documentclass{report}

\usepackage[utf8]{inputenc}
\usepackage[T2A]{fontenc}
\usepackage[russian]{babel}

\usepackage{amsmath}
\usepackage{amssymb}
\usepackage{amsthm}
\usepackage{float}
\usepackage{tikz}
\usepackage{booktabs} % tabular \toprule, ...
\usepackage{gensymb} % \degree


% for title
\author{
  Основано на лекциях \lecturer \\
  \small Конспект написан Заблоцким Данилом
}
\date{\term\ \year}
\title{\course}

\makeatletter

\let\@real@maketitle\maketitle
\renewcommand{\maketitle}{
  {\let\newpage\relax\@real@maketitle}
  \begin{center}
    \begin{minipage}[c]{0.9\textwidth}
      \centering\footnotesize Эти записи не одобряются лекторами, и я вношу в них изменения (часто существенно) после лекций. Они далеко не точно отражают то, что на самом деле читалось, и, в частности, все ошибки почти наверняка мои.
    \end{minipage}
  \end{center}
}

\let\@real@tableofcontents\tableofcontents
\renewcommand{\tableofcontents}{\@real@tableofcontents\newpage}

\makeatother


% hyperref
\usepackage{url}

\usepackage{hyperref}
\hypersetup{
  colorlinks,
  linkcolor={black},
  citecolor={black},
  urlcolor={blue!80!black}
}


% horizontal rule
\newcommand\hr{
  \noindent\rule[0.5ex]{\linewidth}{0.5pt}
}


% theorems
\usepackage{thmtools}
\usepackage[framemethod=TikZ]{mdframed}
\mdfsetup{skipabove=1em,skipbelow=0em, innertopmargin=5pt, innerbottommargin=6pt}

\theoremstyle{definition}

\declaretheoremstyle[
  headfont=\bfseries\sffamily,
  bodyfont=\normalfont,
  mdframed={ nobreak }
]{thmbox}

\declaretheoremstyle[
  headfont=\bfseries\sffamily,
  bodyfont=\normalfont
]{thmunbox}

\declaretheoremstyle[
  headfont=\bfseries\sffamily,
  bodyfont=\normalfont,
  numbered=no,
  mdframed={ rightline=false, topline=false, bottomline=false, },
  qed=\qedsymbol
]{thmproofline}

\declaretheorem[style=thmbox, name=Определение]{definition}
\declaretheorem[style=thmbox, name=Следствие]{corollary}
\declaretheorem[style=thmbox, name=Предложение]{prop}
\declaretheorem[style=thmbox, name=Теорема]{theorem}
\declaretheorem[style=thmbox, name=Лемма]{lemma}

\declaretheorem[numbered=no, style=thmproofline, name=Доказательство]{replacementproof}
\declaretheorem[style=thmunbox, numbered=no, name=Упражнение]{ex}
\declaretheorem[style=thmunbox, numbered=no, name=Пример]{eg}
\declaretheorem[style=thmunbox, numbered=no, name=Замечание]{remark}
\declaretheorem[style=thmunbox, numbered=no, name=Примечание]{note}

\renewenvironment{proof}[1][\proofname]{\begin{replacementproof}}{\end{replacementproof}}

\AtEndEnvironment{eg}{\null\hfill$\diamond$}

\newtheorem*{notation}{Обозначение}
\newtheorem*{previouslyseen}{Как было замечено ранее}
\newtheorem*{problem}{Проблема}
\newtheorem*{observe}{Наблюдение}
\newtheorem*{property}{Свойство}
\newtheorem*{intuition}{Предположение}

\usepackage{etoolbox}
\AtEndEnvironment{vb}{\null\hfill$\diamond$}
\AtEndEnvironment{intermezzo}{\null\hfill$\diamond$}


% lesson
\usepackage{xifthen}

\def\testdateparts#1{\dateparts#1\relax}
\def\dateparts#1 #2 #3 #4 #5\relax{
  \marginpar{\small\textsf{\mbox{#1 #2 #3 #5}}}
}

\def\@lesson{}
\newcommand{\lesson}[3]{
  \ifthenelse{\isempty{#3}}{
    \def\@lesson{Лекция #1}
  }{
    \def\@lesson{Лекция #1: #3}
  }
  \subsection*{\@lesson}
  \testdateparts{#2}
}


% fancy headers
\usepackage{fancyhdr}
\pagestyle{fancy}

\fancyhead[RO,LE]{\course}
\fancyhead[RE,LO]{\@lesson}
\fancyfoot[LE,RO]{\thepage}
\fancyfoot[C]{\leftmark}
\renewcommand{\headrulewidth}{0.4pt}


% incfig
\usepackage{import}
\usepackage{pdfpages}
\usepackage{transparent}
\usepackage{xcolor}

\newcommand{\incfig}[2][1]{%
  \def\svgwidth{#1\columnwidth}
  \import{./figures/}{#2.pdf_tex}
}

\pdfsuppresswarningpagegroup=1


% custom commands
\let\epsilon\varepsilon

\newcommand\N{\ensuremath{\mathbb{N}}}
\newcommand\R{\ensuremath{\mathbb{R}}}
\newcommand\Z{\ensuremath{\mathbb{Z}}}
\newcommand\Q{\ensuremath{\mathbb{Q}}}
\renewcommand\C{\ensuremath{\mathbb{C}}}
\newcommand{\diff}{\ensuremath{\operatorname{d}\!}}

\newcommand{\abs}[1]{\left\lvert #1\right\rvert}
\newcommand{\verteq}[0]{\rotatebox{90}{$=$}}
\newcommand{\vertneq}[0]{\rotatebox{90}{$\ne$}}
\newcommand{\equalto}[2]{\underset{\scriptstyle\overset{\mkern4mu\verteq}{#2}}{#1}}
\newcommand{\nequalto}[2]{\underset{\scriptstyle\overset{\mkern4mu\vertneq}{#2}}{#1}}
\newcommand{\RomanNumeralCaps}[1]{\MakeUppercase{\romannumeral #1}}

\newcommand*\circled[1]{
  \tikz[baseline=(char.base)]{
    \node[shape=circle,draw,inner sep=1pt] (char) {#1};
  }
}


% for \xrightrightarrows
\makeatletter

\newcommand*{\relrelbarsep}{.450ex}
\newcommand*{\relrelbar}{
  \mathrel{
    \mathpalette\@relrelbar\relrelbarsep
  }
}
\newcommand*{\@relrelbar}[2]{
  \raise#2\hbox to 0pt{$\m@th#1\relbar$\hss}
  \lower#2\hbox{$\m@th#1\relbar$}
}

\providecommand*{\rightrightarrowsfill@}{
  \arrowfill@\relrelbar\relrelbar\rightrightarrows
}
\providecommand*{\leftleftarrowsfill@}{
  \arrowfill@\leftleftarrows\relrelbar\relrelbar
}
\providecommand*{\xrightrightarrows}[2][]{
  \ext@arrow 0359\rightrightarrowsfill@{#1}{#2}
}
\providecommand*{\xleftleftarrows}[2][]{
  \ext@arrow 3095\leftleftarrowsfill@{#1}{#2}
}

\makeatother


\def\course   {Комплексный Анализ}
\def\lecturer {Мельникова Е.В.}
\author{
	Основано на лекциях \lecturer \\
	\small Конспект написан Заблоцким Данилом и Кручининым Максимом
}
\def\term     {Весенний семестр}
\def\year     {2024}

\begin{document}
\maketitle
\newpage
\tableofcontents

\lesson{1}{от 9 фев 2024 8:45}{Начало}

\begin{definition}[Методы оптимизации]
    \emph{Методы оптимизации} -- раздел прикладной математики, предметом изучения которого является теория и методы оптимизационных задач.
\end{definition}

\begin{definition}[Оптимизационная задача]
    \emph{Оптимизационная задача} -- задача выбора из множества возможных вариантов наилучших в некотором смысле.
\end{definition}

\begin{note}
    \[
        \left\{\begin{array}{l}
            f(x) \rightarrow \min(\max) \\
            x \in D
        \end{array}\right.,
    \]
    где
    \[
        \begin{array}{rl}
            D       & \text{-- множество допустимых решений, }         \\
            x \in D & \text{-- допустимое решение, }                   \\
            f(x)    & \text{-- целевая функция (критерий оптимизации)}
        \end{array}
    \]
\end{note}

\section*{Задачи математического программирования (МП) и их классификация}

\begin{note}
    Немного истории:
    \begin{center}
        \boxed{\begin{array}{l}
                 1939\text{г. Л.В. Конторович} \\
                 1947\text{г. Д. Данциг}       
               \end{array}}
    \end{center}

    С 50-х годов -- бурное развитие
    \begin{center}
        \boxed{$ 1975\text{г. Нобелевская премия по экономике Конторовичу и Купмаксу} $}
    \end{center}
\end{note}

\newpage

\begin{note}[Задача математического программирования]\leavevmode
    \begin{enumerate}
        \item \label{task:1} $ f(x) \rightarrow \max(\min) $.
        \item \label{task:2} $ g(x) \# 0, \ i = \overline{1,m}, \quad \# \in \{\leqslant,\geqslant,=\} $.
        \item \label{task:3} $ \underset{(x\in R^n)}{x_j \in R}, \ j = \overline{1,n} $.
    \end{enumerate}
    \[
        x = (x_1,\ldots,x_n)
    \]
\end{note}

\begin{definition}[Оптимальное решение, глобальный экстремум]
    $ x^* \in D $ называется \emph{оптимальным решением} задачи \ref{task:1}--\ref{task:3}, если $ \forall x \in D $
    \[
        f(x^*) \geqslant f(x)
    \]
    для задачи на $ \max $ и $ \forall x \in D $
    \[
        f(x^*) \leqslant f(x)
    \]
    для задачи на $ \min $.

    $ x^* $ является \emph{глобальным экстремумом}.
\end{definition}

\begin{definition}[Разделимая, неразделимая задача]
    Задача \ref{task:1}--\ref{task:3}, которая обладает оптимальным решением, называется \emph{разделимой}, и \emph{неразделимой} в противном случае.

    $ D = R^n $ -- задача \emph{безусловной оптимизации}, в противном случае -- задача \emph{условной оптимизации}.
\end{definition}

\begin{note}[Классификация]\leavevmode
    \begin{enumerate}
        \item Если $ f,g_i $ являются линейными, то задача является задачей \emph{линейного программирования} (ЛП).
        \item Если хотя бы одна из функций $ f,g_i $ нелинейная, то задача \emph{нелинейного программирования}.
    \end{enumerate}
    $ f,g_i $ -- выпуклые, то \emph{выпуклого программирования}.
\end{note}

\chapter{Линейное программирование}

\section{Постановка задачи, теорема эквивалентности}

\begin{definition}[Общая задача ЛП (ЗЛП)]
    \[
        f(x) = c_0 + \sum_{j=1}^{n}c_jx_j \longrightarrow \max(\min),
    \]
    \[
        \sum_{j=1}^{n}a_{ij}x_j \# b_i, \quad i = \overline{1,n}, \ \# \in \{\leqslant,\geqslant,=\}
    \]
    \[
        x_j \geqslant 0, \quad j \in \mathfrak{I} \subseteq \{1,\ldots,n\}
    \]
    \[
        A_{m\times n} = \left(\begin{matrix}
                a_{11} & a_{12} & \cdots & a_{1n} \\
                \vdots & \vdots & \ddots & \vdots \\
                a_{m1} & a_{m2} & \cdots & a_{mn}
            \end{matrix}\right), \ b = \left(\begin{matrix}
                b_1 \\ \vdots \\ b_m
            \end{matrix}\right), \ x = \left(\begin{matrix}
                x_1 \\ \vdots \\ x_n
            \end{matrix}\right) \text{ -- }\begin{array}{ll}
            \text{переменные} \\ \text{задачи}
        \end{array}
    \]
\end{definition}

\begin{note}[Матричная задача]
    \[
        f(x) = (c,x) \longrightarrow \max(\min)
    \]
    \[
        Ax \# b
    \]
    \[
        x_i \geqslant 0, \quad j \in \mathfrak{I} \subseteq \{1,\ldots,n\}
    \]
\end{note}

\begin{note}[Каноническая ЗЛП (КЗЛП)]
    \[
        f(x) = (c,x) \longrightarrow \max
    \]
    \[
        Ax = b
    \]
    \[
        x \geqslant \vec{0}, \quad \vec{0} = (0,\ldots,0)
    \]
\end{note}

\begin{note}[Симметричная ЗЛП]
    \[
        \begin{array}{ccc}
            f(x) = (c,x) \longrightarrow \max                 &            & f(x) = (c,x)\longrightarrow \min \\
            Ax \leqslant b                                    & \text{или} & Ax \geqslant b                   \\
            x \geqslant \vec{0}, \quad \vec{0} = (0,\ldots,0) &            & x \geqslant \vec{0}
        \end{array}
    \]
\end{note}

\begin{remark}
    Без ограничения общности далее положим $ c_0 = 0 $, так как добавление константы не влияет на процесс нахождения оптимального решения.
\end{remark}

\subsection{Примеры моделей ЛП}

\begin{example}
    Задача о составлении оптимального плана производства.
    \[
        \begin{array}{ll}
            m \text{ ресурсов},        & i = \overline{1,m} \\
            n \text{ видов продукции}, & j = \overline{1,n}
        \end{array}
    \]

    Известно:
    \[
        \begin{array}{l}
            b_i \text{ -- запас }i\text{-го ресурса, }i = \overline{1,m}                      \\
            a_{ij} \text{ -- }\begin{array}{l}
                                  \text{количество ресурса }i\text{, требуемое для производства } \\
                                  1\text{ единицы продукции вида }j
                              \end{array} \\
            c_j \text{ -- прибыль от продажи }1\text{ единицы }j\text{-го продукта}           \\
        \end{array}
    \]

    Необходимо составить план производства, максимализирующий суммарную прибыль.

    Переменные: $ x_j $ единицы продукции вида $ j $ производства, $ j = \overline{1,n} $,
    \[
        \sum_{j=1}^{n}c_jx_j \longrightarrow \max,
    \]
    \[
        \sum_{j=1}^{n}a_{ij}x_j \leqslant b_i, \ i = \overline{1,m},
    \]
    \[
        x_j \geqslant 0, \quad j = \overline{1,n}.
    \]
\end{example}

\begin{example}
    О максимальном потоке в сети.
    \[
        \begin{array}{ll}
            G = (V,E)          & \text{-- ориентированный взвешенный граф}    \\
            c: E \rightarrow R & \text{-- веса дуг -- пропускная способность}
        \end{array}
    \]
    \[
        \begin{array}{ll}
            s & \text{-- источник} \\
            t & \text{-- сток}
        \end{array}
    \]

    Пусть $ x_{ij} $ -- поток по дуге $ (i,j)\in E $
    \[
        f = \sum_{j: (s,j)\in E} x_{sj} \longrightarrow \max,
    \]
    \[
        \sum_{j:(j,i)\in E}x_{ji} = \sum_{k:(i,k)\in E}x_ik, \quad i \in V \setminus\{s,t\},
    \]
    \[
        0 \leqslant x_{ij} \leqslant c_{ij}, \quad (i,j) \in E.
    \]
\end{example}

\lesson{2}{от 22 фев 2024 12:45}{Продолжение}


\[
	\forall z \in \C \ d(z ; \infty )\coloneqq +\infty, \quad \begin{array}{ll}
		d:    & \C^2 \rightarrow \R                                            \\
		d:    & \C^2 \rightarrow \overline{\R}                                 \\
		\rho: & \overline{\C}^2 \rightarrow \R, \quad \rho(z ; \infty ) \in \R
	\end{array}
\]

\begin{property}[Свойства окрестностей]
	$\forall z \in \overline{\C}$:
	\begin{enumerate}
		\item $\forall V \in O_z \quad z \in V$.
		\item $\forall U,V \in O_z \quad U \cap V \in O_z$.
		\item $\forall U \in O_z, \ \forall V \supset U \quad V \in O_z$.
		\item $\forall V \in O_z, \ \exists U \in O_z : \ U \subset V \ \& \ \forall w \in U \quad U \in O_w$.
	\end{enumerate}
\end{property}

\begin{definition}[Открытое множество]
	Множество называется \emph{открытым}, если оно является окрестностью каждой своей точки.
\end{definition}

\begin{definition}[Окрестность множества]
	\emph{Окрестностью множества} называется множество, являющееся окрестностью каждой точки исходного множества ($V$ -- окрестность множества $A$, если $\forall z \in A \ V \in O_z$).
\end{definition}

\begin{definition}
	$D \subset \overline{\C}, \ z \in \C$,
	\begin{multicols}{2}
		\[
			\dist(z,D)\coloneqq \underset{w \in D}{\inf}\d(z,w),
		\]

		\columnbreak

		\begin{figure}[H]
			\centering
			\incfig{fig-4}
			\label{fig:fig-4}
		\end{figure}
	\end{multicols}
\end{definition}

\begin{definition}
	$D_1,D_2 \subset \overline{\C}$,
	\begin{multicols}{2}
		\[
			\dist(D_1,D_2) \coloneqq \underset{z \in D_1, \ w \in D_2}{\inf}\d(z,w),
		\]

		\columnbreak

		\begin{figure}[H]
			\centering
			\incfig{fig-5}
			\label{fig:fig-5}
		\end{figure}
	\end{multicols}
\end{definition}

\begin{definition}[Внутренность]
	Множество всех внутренних точек называется \emph{внутренностью}.
	\begin{notation}
		\[
			\operatorname{int}D.
		\]
	\end{notation}
\end{definition}

\begin{definition}[Предельная точка множества]
	Точка называется \emph{предельной точкой множества}, если в любой ее окрестности есть точки множества, отличные от данной.
\end{definition}

\begin{remark}
	Точка является предельной точкой множества на расширенной комплексной плоскости $(\overline{\C}) \iff \forall $ ее окрестность содержит бесконечное число точек данного множества.
\end{remark}

\begin{definition}[Окрестность бесконечно удаленной точки]
	Множество $V \subset \overline{\C}$ является \emph{окрестностью бесконечно удаленной точки}, если $\exists \epsilon > 0 : \big\{z \in \overline{\C}: \ \abs{z}>\epsilon\big\} \subset V$.
	\begin{figure}[H]
		\centering
		\incfig[0.7]{fig-6}
		\label{fig:fig-6}
	\end{figure}
\end{definition}

\begin{definition}[Точка прикосновения множества]
	Точка $z \in \overline{\C}$ расширенной комплексной плоскости называется \emph{точкой прикосновения множества} $D \subset \overline{\C}$, если пересечение $\forall V \in O_z \quad V \cap D \ne \varnothing$.
	\begin{notation}
		\[
			\cl D \text{ -- \emph{замыкание} (closure)}
		\]
	\end{notation}
\end{definition}

\begin{definition}[Замкнутое множество]
	Множество называется \emph{замкнутым}, если его дополнение открыто.
	\begin{notation}
		\[
			\partial D
		\]
	\end{notation}
\end{definition}

\begin{definition}[Граничная точка]
	Точка называется \emph{граничной точкой множества}, если в любой ее окрестности есть как точки множества, так и точки его дополнения.
	\begin{notation}
		Множество всех замкнутых подмножеств в $\overline{\C}$:
		\[
			\Cl \overline{\C} \ \text{(closed)}
		\]
	\end{notation}
\end{definition}

\begin{definition}[Компактное множество]
	Множество в $\overline{\C}$ называется \emph{компактным}, если $\forall $ его открытое покрытие имеет конечное подпокрытие.
	\begin{notation}
		\[
			v \text{ -- покрытие множества }D \text{, если }D \underset{V \in v}{\subset }UV,
		\]
	\end{notation}
	\begin{notation}
		\[
			\mathcal{P}(\overline{\C}) \text{ -- совокупность всех подмножеств }\overline{\C}.
		\]
	\end{notation}
\end{definition}

\begin{crit}[Компактности]
	Подмножество $\C$ компактно $\iff $ оно замкнуто и ограничено.
\end{crit}

\begin{note}
	Множество ограничено, если оно содержится в некотором шаре.
\end{note}

\begin{remark}
	$\overline{\C}$ -- компактно.
\end{remark}

\begin{definition}
	Последовательность $\{z_n\}_{n \in \N}\subset \C$ сходится к $z \in \C$, если $\forall \epsilon > 0 \ \exists n_0 \in \N: \ \forall n \geqslant n_0$
	\[
		\abs{z_n - z} < \epsilon.
	\]
	\[
		\d(z_n,z) \xrightarrow[n \rightarrow \infty ]{} 0, \qquad z_n \rightarrow \infty \text{, если } \lim_{n \rightarrow \infty }\abs{z_n} = \pm \infty.
	\]
	\[
		z = \lim_{n \rightarrow \infty } z_n, \qquad z_n \xrightarrow[n \rightarrow \infty ]{} z.
	\]
\end{definition}

\begin{remark}
	\[
		z_n \rightarrow z \text{ в } \C \iff \left\{\begin{array}{l}
			\Re z_n \rightarrow \Re z \\
			\Im z_n \rightarrow \Im z
		\end{array}\right. \text{ в } \R,
	\]

	\[
		\abs{z_n - z} = \sqrt{(\Re z_n - \Re z)^2 + (\Im z_n - \Im z)^2} \geqslant \abs{\Re z_n - \Re z},
	\]
	\[
		\Re(z_1 \pm z_2) = \Re z_1 \pm \Re z_2.
	\]
\end{remark}

\begin{crit}[Коши]
	Последовательность $\{z_n\}_{n \in \N}\subset \C$ сходится $\iff \forall \epsilon > 0 \ \exists n_0 \in \N: \ \forall n,m \geqslant n_0$
	\[
		\abs{z_n - z_m} < \epsilon.
	\]
\end{crit}

\begin{crit}[Коши в $\overline{\C}$]
	Последовательность $\{z_n\}_{n \in \N}\subset \overline{\C}$ сходится $\iff \forall \epsilon > 0 \ \exists n_0 \in \N: \ \forall n,m \geqslant n_0$
	\[
		\rho(z_n,z_m) < \epsilon,
	\]
	\[
		z_n \xrightarrow[n \rightarrow \infty ]{} z \iff \rho(z_n,z)\xrightarrow[n \rightarrow \infty ]{} 0.
	\]
\end{crit}

\begin{crit}[Компактности (расширенный)]
	Подмножество $D \subset \overline{\C}$ компактно $\iff \forall $ его последовательность имеет сходящуюся подпоследовательность: $D \subset \overline{\C} \ \forall \{z_n\}_{n \in \N} \subset D \ \exists \{z_{n_k}\}_{k \in \N} \subset \{z_n\}_{n \in \N}:$
	\[
		z_{n_k} \rightarrow z \in D.
	\]

	Пусть $\{z_n\}_{n \in \N} \subset \C:$
	\[
		\sum_{n=1}^{\infty }z_n = \lim_{n \rightarrow \infty }S_n.
	\]
\end{crit}

\begin{definition}[Числовой ряд]
	\emph{Числовым рядом} называется формальная сумма членов.
\end{definition}

\begin{definition}[Абсолютно сходящийся числовой ряд]
	Числовой ряд называется \emph{абсолютно сходящимся}, если сходится ряд
	\[
		\sum_{n=1}^{\infty }\abs{z_n}.
	\]
\end{definition}

\begin{crit}[Коши (сходимости ряда)]
	$\sum_{n=1}^{\infty }z_n$ сходится $\iff \forall \epsilon > 0 \ \exists m \in \N: \ \forall n \geqslant m \ \forall k \in \N$
	\[
		\underbrace{\abs{z_{n+1}+z_{n+2}+\ldots+z_{n+k}}}_{\abs{S_{n+k}-S_n}} < \epsilon.
	\]
\end{crit}

\begin{corollary}
	Если ряд сходится, то его общий член стремится к $0$.
\end{corollary}

\begin{corollary}
	Каждый абсолютно сходящийся числовой ряд сходится.
\end{corollary}

\subsection{Пути, кривые и области}

\begin{definition}[Путь]
	Путем $\gamma:[a ; b] \rightarrow \C$ называется непрерывное отображение $[a ; b]$ в $\C$.
\end{definition}

\begin{eg}
	$\gamma(t) = e^{it},$
	\begin{multicols}{2}
		\begin{figure}[H]
			\centering
			\incfig{fig-7}
			\label{fig:fig-7}
		\end{figure}

		\columnbreak

		\[
			0 \leqslant t \leqslant 2\pi.
		\]
	\end{multicols}
\end{eg}

\begin{definition}
	$\gamma_1 : [a_1 ; b_1] \rightarrow \C, \ \gamma_2 : [a_2 ; b_2] \rightarrow \C$. $\gamma_1 \sim \gamma_2$, если $\exists $ возрастающая непрерывная функция $\phi: [a_1 ; b_1] \xrightarrow[]{\text{на}} [a_2 ; b_2]:$
	\[
		\gamma_1(t) = \gamma_2\big(\phi(t)\big), \quad \forall t \in [a_1 ; b_1].
	\]
\end{definition}

\begin{eg}
	\begin{multicols}{2}
		\[
			\begin{array}{ll}
				\gamma_1(t) = t      & 0 \leqslant t \leqslant 1             \\
				\gamma_2(t) = \sin t & 0 \leqslant t \leqslant \frac{\pi}{2} \\
				\gamma_3(t) = \sin t & 0 \leqslant t \leqslant \pi           \\
				\gamma_4(t) = \cos t & 0 \leqslant t \leqslant \frac{\pi}{2}
			\end{array}
		\]
		$\phi(t) = \arcsin t$,
		\[
			\gamma_1(t) = \gamma_2\left(\phi(t)\right).
		\]
		\begin{figure}[H]
			\centering
			\incfig{fig-8}
			\label{fig:fig-8}
		\end{figure}
	\end{multicols}
\end{eg}

\lesson{3}{от 29 фев 2024 12:45}{Продолжение}


\begin{definition}[Связное множество]
    $ A \subset \overline{\Comp} $ называется \emph{связным}, если $ \nexists U,V \in O_P \overline{\Comp}: \ U \cap A \ne \O, \ U \cap V = \O $.
    \[
        O_P \overline{\Comp} \text{ -- совокупность всех открытых множеств}
    \]
\end{definition}

\begin{example}
    $ A = \big\{(0,y): \ -1 \leqslant y \leqslant 1\big\} \cup \left\{\left(x,\sin\frac{1}{x}\right): \ 0 < x \leqslant 1\right\} $ -- связное.
\end{example}

\begin{definition}[Линейно связное множество]
    Множество называется \emph{линейно связным}, если любые его точки можно соединить путем, значения которого лежат в этом множестве.
\end{definition}

\begin{remark}
    В пространстве $ \R^n $, и в частности $ \overline{\Comp} $, любое открытое множество связно $ \iff $ оно линейно связно.
\end{remark}

\begin{definition}[Область]
    \emph{Областью} в $ \overline{\Comp} $ называется любое непустое открытое связное множество.
\end{definition}

\begin{definition}[Замкнутая область]
    \emph{Замкнутой областью} будем называть замыкание области.
\end{definition}

\section{Функции комплексного переменного}

\subsection{Структура функции комплексного переменного}

\begin{note}
    $ f : \Comp \longrightarrow \Comp $
    \[
        \begin{array}{l}
            dom f \text{ -- область определения функции} \\
            im f \text{ -- область значения функции}
        \end{array}
    \]
\end{note}

\begin{definition}[Предел отображения]
    $ D \subset dom f, \ z_0 \in \overline{\Comp} $ -- предельная точка $ D $. Тогда $ w_0 \in \overline{\Comp} $ называется \emph{пределом отображения} $ f $,
    \[
        w_0 \coloneqq \underset{D \circ z \rightarrow z_0}{\lim}f(z) \text{, если }\forall V \in O_{w_0} \ \exists U \in O_{z_0}: \ f(\mathring{U}\cap D)\subset V,
    \]
    \[
        U \in O_{z_0}, \quad \mathring{U} = U\setminus\{z_0\}.
    \]
\end{definition}

\begin{note}
    В случае, когда $ z_0,w_0 \in\Comp $ следует, что $ \forall \epsilon > 0 \ \exists \delta > 0 : \ \forall z \in D $
    \[
        0 < | z - z_0 | < \delta \implies \big| f(z) - w_0 \big| < \epsilon.
    \]
\end{note}

\begin{definition}[Непрерывная функция в точке]
    Функция $ f $ называется \emph{непрерывной в точке} $ z_0 \in \Comp $, если:
    \begin{enumerate}
        \item $ z_0 \in dom f $.
        \item $ \forall \epsilon > 0 \ \exists \delta > 0: \ \forall z \in D $
              \[
                  0 < | z - z_0 | < \delta \implies | f(z) - w_0 | < \epsilon.
              \]
    \end{enumerate}
\end{definition}

\newpage

\begin{definition}[Непрерывная функция на множестве]
    Функция $ f : \Comp \longrightarrow \Comp $ непрерывна на $ D \subset \Comp $, если
    \begin{enumerate}
        \item $ D \subset dom f $.
        \item $ \forall z_0 \in D \ \forall \epsilon > 0 \ \exists \delta > 0 \ \forall z \in D $
              \[
                  | z - z_0 | < \delta \implies \big|f(z) - f(z_0)\big| < \epsilon.
              \]
    \end{enumerate}
\end{definition}

\begin{note}[Функция Дирихле]
    $ D(x) = \left\{\begin{array}{ll}
            1, & x \in \Q              \\
            0, & x \in \R \setminus \Q
        \end{array}\right. $, непрерывна на $ \Q $, непрерывна на $ \R \setminus \Q $.
\end{note}

\begin{remark}
    Если множество является открытым или совпадает с областью определения функции, то непрерывность функции на этом множестве равносильно ее непрерывности в каждой точке.
    \[
        f_n: \Comp \rightarrow \Comp (n \in \N), \quad D \coloneqq \underset{n \in \N}{\bigcap} dom f_n.
    \]
\end{remark}

\begin{definition}
    $ A \subset D, \ f : A \rightarrow \Comp, \ f_n \rightrightarrows f $ на $ A $, если $ \forall \epsilon > 0 \ \exists n_0 \in \N : \ \forall z \in A \ \forall n \geqslant n_0 $
    \[
        \big|f_n(z) - f(z)\big| < \epsilon.
    \]

    ($ \forall \epsilon > 0 \ \exists n_0 \in \N:  \forall n \geqslant n_0 \quad \underset{z \in A}{\sup} \big| f_n(z) - f(z) \big| < \epsilon, \ | z - z_0 | < \delta \implies \big| f(z) - f(z_0) \big| < \epsilon $).
\end{definition}

\begin{theorem}[Вейерштрасса]
    Если $ \{f_n\}_{n\in\N} \subset C(A), \ f_n \rightrightarrows f $, то $ f \in C(A) $.
\end{theorem}

\begin{definition}[Функциональный ряд]
    \emph{Функциональным рядом} называется формальная сумма членов последовательности функции.
    \[
        \text{Обозначение: } \sum_{n=1}^{\infty}f_n.
    \]
\end{definition}

\begin{definition}[Числовой ряд]
    $ \forall z \in D \ \sum_{n=1}^{\infty}f_n(z) $ называется \emph{числовым рядом} $ \{f_n(z)\}_{n \in \N} $.
    \[
        S_n \coloneqq \sum_{k=1}^{n}f_k \text{ -- частичная сумма}.
    \]
\end{definition}

\begin{theorem}[Признак Вейерштрасса]
    $ \sum_{n=1}^{\infty}f_n $ таков, что $ \forall n \in \N \ \forall z \in A \ | f_n | \leqslant c_n $, причем $ \sum_{n=1}^{\infty} c_n $ сходится. Тогда ряд $ \sum_{n=1}^{\infty} f_n $ равномерно абсолютно сходится на $ A $.
\end{theorem}

\begin{theorem}[Критерий Коши (равномерная сходимость)]
    $ \{f_n\}_{n\in\N} $ равномерно сходится на $ A \iff \forall \epsilon > 0 \ \exists n_0 \in \N: \forall n,m \geqslant n_0 $
    \[
        \underset{z \in A}{\sup}\big|f_n(z) - f_n(z_0)\big| < \epsilon.
    \]
\end{theorem}

\begin{definition}[Линейная функция]
    Функция $ f : \Comp \longrightarrow \Comp $ называется \emph{линейной}, если $ \forall \alpha, \beta \in \Comp \ \forall z_1,z_2 \in \Comp $
    \[
        f(\alpha z_1 + \beta z_2) = \alpha f(z_1) + \beta f(z_2).
    \]
\end{definition}

\begin{remark}
    Функция $ f: \Comp \longrightarrow \Comp $ является линейной $ \iff \exists a \in \Comp : \forall z \in \Comp $
    \[
        f(z) = az
    \]
\end{remark}

\subsection{Степенные ряды}

\begin{note}
    $ \sum_{n=0}^{\infty}a_n(z-z_0)^n $, где $ \{a_n\}_{n\in\N} \subset \Comp, \ z,z_0 \in \Comp $.
\end{note}

\begin{theorem}[1-я теорема Абеля]
    Если ряд $ \sum_{n=0}^{\infty}a_n(z-z_0)^n $ сходится в точке $ \nequalto{z_1}{z_0} \in \Comp $, то он абсолютно сходится при $ | z - z_0 | < | z_1 - z_0 | $. А если ряд $ \sum_{n=0}^{\infty}a_n(z - z_0)^n $ расходится в точке $ \nequalto{z_1}{z_0} \in \Comp $, то он расходится и при $ | z - z_0 | > | z_1 - z_0 | $.
\end{theorem}

\begin{proof}\leavevmode
    \begin{enumerate}
        \item $ \sum_{n=0}^{\infty}a_n(z_1 - z_0)^n $ сходится $ \implies \big|a_n(z_1 - z_0)^n\big| \xrightarrow[n \rightarrow\infty]{} 0 $.
              \[
                  c \coloneqq \underset{n\in\N}{\sup} \big|a_n(z_1 - z_0)^n\big| < +\infty, \quad | z - z_0 | < | z_1 - z_0 |.
              \]

              Рассмотрим
              \[
                  \sum_{n=0}^{\infty}\big|a_n(z-z_0)^n\big| = \sum_{n=0}^{\infty}\big|a_n(z_1 - z_0)^n\big|\cdot \left|\frac{z-z_0}{z_1-z_0}\right|^n \leqslant c \cdot \sum_{n=0}^{\infty}\left|\frac{z-z_0}{z_1-z_0}\right|^n < + \infty.
              \]
        \item \textbf{добавить}
    \end{enumerate}
\end{proof}

\begin{definition}[Радиус сходимости]
    Элемент $ R \in [0;+\infty] $ называется \emph{радиусом сходимости} ряда $ \sum_{n=0}^{\infty}a_n(z-z_0)^n $, если при $ | z - z_0 | < R $ исходный ряд абсолютно сходится, а при $ | z-z_0 | > R $ исходный ряд расходится.
\end{definition}

\begin{theorem}[Коши-Адамара]
    Для степенного ряда $ \sum_{n=0}^{\infty}a_n(z-z_0)^n $ положим $ l \coloneqq \underset{n \rightarrow\infty}{\overline{\lim}} \sqrt[n]{| a_n |} $. Тогда:
    \begin{enumerate}
        \item Если $ l=0 $, то исходный ряд сходится $ \forall z \in \Comp $.
        \item Если $ l=\infty $, то исходный ряд сходится только в точке $ z_0 $.
        \item Если $ l\in(0;+\infty) $, то при $ | z-z_0 | < \frac{1}{l} $, а при $ | z-z_0 | > \frac{1}{l} $ исходный ряд расходится.
    \end{enumerate}
\end{theorem}

\begin{proof}\leavevmode
    \begin{enumerate}
        \item $ \underset{n \rightarrow\infty}{\overline{\lim}} \sqrt[n]{| a_n |} = \underset{n \rightarrow\infty}{\lim} \sqrt[n]{| a_n |} = 0 $,
              \[
                  z \in \Comp, \ \sum_{n=0}^{\infty}\big| a_n(z-z_0)^n \big|.
              \]
              $ \underset{n \rightarrow \infty}{\lim} \sqrt[n]{\big| a_n(z-z_0)^n \big|} = \underset{n \rightarrow \infty}{\lim} \sqrt[n]{\big| a_n \big|} \cdot | z - z_0 | = 0 \implies $ ряд сходится.
        \item $ \underset{n \rightarrow\infty}{\overline{\lim}} \sqrt[n]{| a_n |} = \infty $,
              \[
                  \exists\{a_{n_k}\}_{k\in\N} \subset \{a_n\}_{n\in\N}, \quad \sqrt[n_k]{| a_{n_k} |} \rightarrow + \infty.
              \]
              $ \sqrt[n_k]{| a_{n_k} |} \cdot | z - z_0 | \rightarrow +\infty \implies | a_{n_k} | $.
        \item $ | z - z_0 | < \frac{1}{l} \implies l | z - z_0 | < 1 $.
    \end{enumerate}
\end{proof}

\lesson{4}{от 7 мар 2024 12:45}{Продолжение}

\newpage

\lesson{5}{от 14 мар 2024 12:45}{Продолжение}


\section{Теория интеграла Коши}

\subsection{Определения и основные свойства интеграла Коши}

\begin{definition}[Разбиение кривой Жордана]
	Пусть $\gamma $ -- кривая Жордана, $\gamma \in \C$ с концами $\alpha , \beta  \in \C$.

	\begin{figure}[H]
		\centering
		\incfig[0.7]{fig-14}
		\label{fig:fig-14}
	\end{figure}

	Разбиением кривой Жордана назовем $\sigma \coloneq \{z_0,z_1,\ldots ,z_n,\xi_0,\ldots \xi_{n-1} \}$, где $n \in \N, \ z_0 = \alpha , \ z_n = \beta , \ z_{k+1} \notin \overbrace{z_0z_k} \ \forall k \in \overline{0,n-1} , \ \zeta _k \in \overbrace{z_k,z_{k+1} } $
	\[
		\triangle z_k \coloneq z_{k+1} -z_k,
	\]
	\[
		\d(\sigma ) \coloneq \underset{0 \leqslant k < n-1}{\max} \abs{\triangle z_k} \text{ -- диаметр разбиения } \sigma.
	\]
\end{definition}

\begin{definition}
	Если $f: \gamma \rightarrow \C, \ \sigma $ -- интегральная сумма, то
	\[
		S_\sigma (f) \coloneq \sum_{k=1}^{n-1} f(S_k)\underbrace{(z_{k+1} -z_n)}_{\triangle z_k}.
	\]
\end{definition}

\begin{definition}
	$\prod (\gamma )$ -- множество всех разбиений кривой $\gamma $,
	\[
		\Phi : \prod(\gamma ) \rightarrow \C.
	\]

	Будем говорить, что $\exists \underset{d(\sigma )\rightarrow 0}{\lim} \Phi (v) =w \in \C$, если $\forall \epsilon > 0 \ \exists \delta > 0: \ \forall \sigma \in \prod(\gamma ) \ \d(\sigma )< \delta \implies \big|\Phi (\sigma ) - w\big| < \epsilon $.
\end{definition}

\begin{definition}[Интеграл Коши]
	Если $f: \gamma  \rightarrow \C$ и $\exists \underset{\d(\sigma )\rightarrow 0}{\lim} S_{\sigma } (f)\in \C$, то
	\[
		\int_{\gamma } f(z)\d z \coloneq \underset{\d(\sigma )\rightarrow 0}{\lim} S_{\sigma } (f)
	\]
	называется \emph{интегралом Коши} от функции $f$ по кривой $\gamma $.
\end{definition}

\begin{theorem}
	Если $f$ непрерывна на спрямляемой кривой Жордана $\gamma $, то $\int_{\gamma } f(z)\d z$ существует (то есть является элементом $\C$).
\end{theorem}

\begin{proof}
	$f(z) = f(x + iy) = u(x,y) + iv(x,y)$,
	\begin{multline*}
		\int_{\gamma } f(z)\d z = \int_{\gamma } \big(u(x,y)+iv(x,y)\big)\d(x+iy) = \\ = \int_{\gamma } u\d x - v\d y + \int_{\gamma } v\d x + u\d y \in \C.
	\end{multline*}
\end{proof}

\subsection{Интегральная теорема Коши}

\begin{lemma}[Гауса]
	Если функция $f$ непрерывна в области $D$, то для любой спрямляемой кривой Жордана $\gamma \subset D$, для любого $\epsilon > 0$ существует вписанная в $\gamma $ ломанная $P$ такая, что
	\[
		\left|\int_{\gamma } f(z)\d z - \int_{P} f(z)\d z\right| < \epsilon.
	\]
\end{lemma}

\begin{theorem}[Интегральная теорема Коши]
	Пусть $D$ -- односвязная область в $\C$, функция $f$ голоморфна в $D$. Тогда для любой замкнутой спрямляемой кривой Жордана $\gamma $
	\[
		\int_{\gamma } f(z)\d z = 0.
	\]
\end{theorem}

\begin{proof}
	Пусть $\gamma $ -- $\triangle$ в $D$.
	\begin{figure}[H]
		\centering
		\incfig[0.7]{fig-15}
		\label{fig:fig-15}
	\end{figure}

	Докажем, что интеграл по этому треугольнику равен нулю. Допустим противное:
	\[
		\left|\int_{\gamma } f(z)\d z\right| \eqcolon M \ne 0.
	\]

	$\gamma _1, \gamma _2, \gamma _3, \gamma _4, \quad \int_{\gamma } f(z)\d z = \sum_{k=1}^{4} \int_{\gamma _n} f(z)\d z$,
	\[
		\left|\int_{\gamma } f(z)\d z\right| \leqslant \sum_{k=1}^{n} \left|\int_{\gamma k} f(z)\d z\right|,
	\]
	\[
		\overline{\triangle_0} \coloneq \gamma , \quad \overline{\triangle_1} \coloneq \gamma _i : \ \left|\int_{\gamma _i} f(z)\d z\right| \geqslant  \frac{M}{4} ,
	\]
	\[
		\exists \overline{\triangle_2} : \ \left|\int_{\overline{\triangle_2} }f(z)dz \right| \geqslant  \frac{M}{4^2} .
	\]

	Продолжая этот процесс, мы получм последовательность $\{\overline{\triangle_k} \}:$
	\[
		\left|\int_{\overline{\triangle_k} } f(z)\d z\right| \geqslant \frac{M}{4^k} ,
	\]
	\[
		D(\overline{\triangle_{k+1} } ) \subset D(\overline{\triangle_k} ).
	\]

	То есть можем считать эту последовательность $\{\overline{\triangle_k} \}$ как последовательность вложенных множеств $\implies \exists z_0 \in \underset{k \in \N}{\bigcap} D(\overline{\triangle_k} ) \ne \varnothing $.

	????????

	В силу произвольности $\epsilon $ получаем, что $M = 0$,
	\[
		\left|\int_{\gamma } f(z)\d z - \int_{P} f(z)\d z\right| < \epsilon .
	\]
\end{proof}

\begin{theorem}[Обобщенная интегральная теорема Коши]
	Если функция $f$ голоморфна в односвязной области $D$, ограниченной замкнутой спрямляемой кривой Жордана $\gamma $ и $f$ непрерывна вплоть до границы, то есть $\forall z_0 \in \gamma $
	\[
		\underset{D \ni z \rightarrow z_0}{\lim} f(z) = f(z_0) \implies \int_{\gamma } f(z)\d z = 0.
	\]
\end{theorem}

\begin{corollary}
	Если область $D$ ограничена конечным числом замкнутых спрямляемых кривых Жордана. Если $f$ голоморфна в этой области ???
\end{corollary}

\begin{corollary}
	Утверждение обобщенной теоремы остается в силе, если условие голоморфности функции $f$ в области нарушается в конечном количестве точек $z_1,\ldots z_n \in D$, в которых функция ведет себя так:
	\[
		\underset{\exists \rightarrow z_k}{\lim} (z-z_k)f(z) = 0 \quad (0 \leqslant k \leqslant n).
	\]
\end{corollary}

\subsection{Интегральная формула Коши, интеграл типа Коши}

\begin{theorem}[Интегральная формула Коши]
	Если функция $f$ голоморфна в односвязной области $D$, ограничена замкнутой спрямляемой кривой Жордана $\gamma $, непрерывна вплоть до границы, то
	\[
		\frac{1}{2\pi i} \int_{\gamma } \frac{f(z)}{z-z_0} \d z = \left\{ \begin{array}{l}
			f(z_0), \text{ если } z_0 \in D \\
			0, \text{ если } z_0 \notin \cl D
		\end{array}\right.
	\]
\end{theorem}

\begin{definition}[Интеграл типа Коши]
	Пусть односвязная область $D$ ограничена замкнутой спрямляемой кривой Жордана $\gamma $, а функция $f$ непрерывна на $\gamma $. Положим
	\[
		F(z) = \frac{1}{2\pi i} \int_{\gamma } \frac{f(\xi )}{\xi - z} \d \xi , \ z \in D.
	\]

	Эта функция $F$ называется \emph{интегралом типа Коши}.
\end{definition}

\begin{theorem}[Лиувилль]
	Если функция $f$ голоморфна в $\C$ и ограничена, то $f \equiv const$.
\end{theorem}

\newpage

\begin{proof}
	$R > 0, \ z \in \C$
	\[
		f'(z) = \frac{1}{2\pi i} \int_{|\xi - z| = R} \frac{f(\xi )}{(\xi  - z)^2} \d \xi .
	\]

	Пусть $M > 0: \ \underset{z \in \C}{\sup} \big|f(z)\big|\leqslant M \implies$
	\[
		\big|f '(z)\big| \leqslant \frac{1}{2\pi} \int_{|\xi - z| = R} \frac{\abs{f(\xi )} }{\abs{\xi -z}^2 } \abs{\d \xi } \leqslant \frac{1}{2\pi} \cdot \frac{M}{R^2} \cdot 2\pi R = \frac{M}{R} \xrightarrow[R \rightarrow +\infty ]{}0 \implies
	\]
	$\implies f '(z) = 0 \ (\forall z \in \C)$.

	\[
		u_{x}^{'} = u_{y}^{'} = v_{x}^{'} = v_{y}^{'} = 0 \implies  u = const, \ v = const \implies f = const.
	\]
\end{proof}

\subsection{Неопределенный интеграл теорем Мореры и Вейерштрасса}

\begin{theorem}
	Непрерывная в односвязной области $D$ функция $f$ голоморфна в этой области $\iff \forall z_0,z \in D \ \int_{z_0}^{z} f(\xi )\d \xi $ не зависит от пути интегрирования, соединяющего области $D$ точек $z_0,z$.
\end{theorem}

\begin{definition}[Первообразная голоморфной в области]
	\emph{Первообразной голоморфной в области} $D$ функции $f$ называется голоморфная в $D$ функция $F: \ \forall  z \in D \ F '(z) = f(z)$.
\end{definition}

\begin{remark}
	Любые две первообразные голоморфной функции отличаются только на константу.
\end{remark}

\begin{definition}[Неопределенный интеграл]
	Совокупность всех первообразных голоморфной функции называется ее \emph{неопределенным интегралом}.
	\begin{notation}
		$\int f(z)\d z = F(z) + c $.
	\end{notation}
\end{definition}

\begin{remark}
	Если функция $f$ голоморфна в области $D$ и $F$ -- ее первообразная, то $\forall z_0,z \in D$
	\[
		\int_{z_0}^{z} f(\xi )\d \xi = F(z) - F(z_0).
	\]
\end{remark}

\begin{theorem}[Морера]
	Для того, чтобы непрерывная в односвязной области функция была голоморфна в этой области, необходимо и достаточно, чтобы интеграл от этой функции по любому замкнутому контуру, лежащему в области, был равен $0$.
\end{theorem}

\begin{remark}
	В сторону достаточности условия теоремы Мореры можно ослабить. Если функция непрерывна в односвязной области и $\int_{\triangle} f(z)\d z = 0$, то функция голоморфна $\forall \triangle \in D$.
\end{remark}

\begin{definition}
	Пусть $\{f_n\}_{n \in \N} \subset C(D)$. Говорят, что эта последовательность сходится равномерно к $f$ внутри $D$, если $\forall K \in D \Subset D \ f_n \rightrightarrows f$ на $K$, то есть $\forall \epsilon > 0 \exists n \in \N: \ \forall n \geqslant n_0$
	\[
		\underset{I \in K}{\sup} \big|f_n(z)- f(z)\big| < \epsilon .
	\]
\end{definition}

\begin{theorem}[Вейерштрасса]
	Равномерный предел последовательности голоморфных функций является голоморфной функцией, то есть если $\{f_n\}_{n \in \N} \subset \mathcal{H}(D)$ и $f_n \rightrightarrows f$ внутри $D$, то $f \in \mathcal{H}(D)$.
\end{theorem}

\begin{definition}[Корень многочлена]
	\emph{Корнем многочлена} $P(z)\coloneq a_n z^n + \ldots + a_1z + a_0,$ где $a_0,a_1,\ldots ,a_n \in \C$, называется число $z_0 \in \C: \ P(z_0) = 0$.
\end{definition}

\begin{theorem}[Безу]
	Если $z_0$ -- корень многочлена $P$, то $\exists $ многочлен $Q: \ P(z) = (z-z_0)\cdot Q(z_0)$.
\end{theorem}

\begin{theorem}[Основная теорема алгебры]
	Каждый многочлен с комплексными коэффициентами в $\deg \geqslant 1$ имеет к.б. один комплексный корень.
\end{theorem}

\begin{corollary}
	Каждый многочлен $n$-ой степени имеет $n$ корней.
\end{corollary}

\newpage

\lesson{6}{от 21 мар 2024 12:45}{Продолжение}


\section{Ряды Тейлора и Лорана. Элементы теории вычетов}

\subsection{Разложение голоморфной функции в ряд Тейлора}

\begin{theorem}
	Пусть $f \in \mathcal{H}(D)$. Тогда $\forall z_0 \in D \ \exists r > 0$: при $\abs{z-z_0} < r$
	\[
		f(x) = \sum_{n=0}^{\infty } \frac{f^{(n)} (z_0)}{n!} (z - z_n)^n.
	\]
\end{theorem}

\begin{corollary}
	$\mathcal{H}(D) = \mathcal{A}(D)$.
\end{corollary}

\begin{theorem}
	Пусть $f$ голоморфна в $B_r(z_0) \ \forall z \in B_r(z_0) \ f(z) = \sum_{n=0}^{\infty } C_n (z-z_0)^n$.

	Тогда $\forall n \in \overline{\N}$
	\[
		C_n = \frac{1}{2\pi i} \int_{\abs{\xi -z_0} = \rho} \frac{f(z)}{(\xi - z_0)^{n+1} } \d \xi \quad \forall \rho \in (0,r).
	\]

	То есть любой степенной ряд является рядом Тейлора для своей суммы.
\end{theorem}

\begin{proof}
	Радиус сходимости $\geqslant r, \ \rho \in (0 ; r)$.

	$\abs{z-z_0} = \rho \implies $ ряд сходится, рассмотрим:
	\begin{multline*}
		f(z) = \frac{1}{2\pi i} \int_{\abs{\xi -z_0}= \rho } \frac{f(\xi )}{(\xi -z_0)^{k+1} } \d \xi = \\
		= \frac{1}{2\pi i} \int_{\abs{\xi -z_0} = \rho } \frac{\sum_{n=0}^{\infty } C_n(\xi -z_0)^n}{(\xi -z_0)^{k+1} } \d \xi = \\
		= -\frac{k!}{2\pi i} \cdot C_k \cdot 2\pi i = C_n \cdot k!,
	\end{multline*}
	\[
		C_k = \frac{f^{(k)} (z_0)}{k!} .
	\]
\end{proof}

\begin{theorem}[Неравенство Коши]
	Пусть $f$ голоморфна в $D$ и $B_r[z_0] \subset D, \ f(z) = \sum_{n=0}^{\infty } C_n(z-z_0)^n$.

	Пусть $M \coloneq \underset{\abs{z-z_0} \leqslant r}{\sup} \big|f(z)\big|$. Тогда $\forall n \in \overline{\N} \ \abs{C_n}  \leqslant \frac{M}{r^n} $.
\end{theorem}

\begin{definition}[Предельная точка]
	Точка называется \emph{предельной точкой множества}, если в любой ее окрестности есть точки множества, отличные от данной.
\end{definition}

\begin{corollary}
	Любые две аналитические в области функции, совпадающие на множестве, имеющем в этом множестве предельную точку, тождественно равны.
\end{corollary}

\subsection{Ряды Лорана}

\begin{definition}[Ряд Лорана]
	\emph{Рядом Лорана} называется степенной ряд вида $\sum_{n=-\infty }^{\infty } C_n(z-z_0)^n$. Ряд Лорана раскладывается на сумму двух рядов:
	\[
		\sum_{n=-\infty }^{\infty } C_n (z-z_0)^n \coloneq \sum_{n=0}^{\infty } C_n (z-z_0)^n + \sum_{n=1}^{\infty } C_{-n} (z-z_0)^{-n} .
	\]

	Ряд Лорана сходится $\iff $ сходятся обе его составляющие.

	Область сходимости ряда Лорана: $0 \leqslant r < \abs{z-z_0} < R \leqslant +\infty $.
\end{definition}

\begin{theorem}[О ряде Лорана]
	Если функция $f$ голоморфна в кольце $r < \abs{z-z_0} < R$, то в этом кольце она разлагается в ряд Лорана.

	$f(z) = \sum_{n=-\infty }^{\infty } C_n(z-z_0)^n$ с коэфициентами $C_n$, определяемыми формулами:
	\[
		C_n = \frac{1}{2\pi i} \int_{\abs{\xi -z_0} = \rho} \frac{f(\xi )}{(\xi -z_0)^{n+1} } \d \xi \quad \forall \rho \in (r,R).
	\]
\end{theorem}

\subsection{Классификация изолированных особых точек}

\begin{definition}[Правильная точка]
	Точка $z_0 \in \dom f$ называется \emph{правильной} точкой функции $f$, если $f$ определена в некоторой области и непрерывна в самой функции.
\end{definition}

\begin{definition}[Особая точка]
	\emph{Особой} точкой функции называется предельная точка ее области определения, этой области не принадлежащая.
\end{definition}

\begin{definition}[Изолированная особая точка]
	Особая точка называется \emph{изолированной} особой точкой, если в некоторой ее окрестности других особых точек нет.
\end{definition}

\begin{remark}
	Особая точка функции называется изолированной, если в проколотой окрестности этой точки функция голоморфна.
\end{remark}

\begin{eg}
	\[
		f(z) = \frac{1}{\sin \frac{1}{z} },
	\]

	$z_0 = 0$ -- особая точка,

	$\sin \frac{1}{z} = 0 \implies  \frac{1}{z} = \pi k, \ k \in Z$,

	$z_k = \frac{1}{\pi k} , \ k \in \Z$ -- особые точки,

	\[
		\frac{1}{\pi(k+1)} < \frac{1}{\pi k} < \frac{1}{\pi (k-1)}.
	\]
\end{eg}

\begin{theorem}[О путях и полюсах]
	Изолированная особая точка $z_0$ функции $f$ является полюсом порядка $m$ функции $f \iff $ она является путем $m$-го порядка функции $\rho (z)= \frac{1}{f(x)} $.
\end{theorem}

\begin{proof}
	Самостоятельно.
\end{proof}

\begin{theorem}[Сохоцкий]
	Изолированная особая точка функции является существенно особой точкой $\iff $ в любой ее окрестности функция принимает значения сколь угодно близкие к любому числу $a \in \overline{\C} $.
\end{theorem}

\begin{definition}[$A$-точка]
	Пусть $A \in \C$, точка $z$ называется \emph{$A$-точкой} функции $f$, если $f(z) = A$.
\end{definition}

\begin{theorem}[Большая теорема Пикара]
	В окрестности существенно особой точки $z_0$ голоморфной функции $f \ \forall A \in \C$, за исключением быть может одного, существует последовательность $A$-точек функции $f$, сходящаяся к точке $z_0$.
\end{theorem}

\subsection{Вычеты}

\begin{definition}[Вычет функции относительно точки]
	Если $z_0$ -- изолированная особая точка функции $f$, то \emph{вычетом} $f$ относительно $z_0$ называется интеграл $\frac{1}{2\pi i} \int_{\gamma } f(z)\d z$, где $\gamma $ -- произволный контур, ограничивающий область $D$: $f$ непрерывна в $\cl D \setminus \{z_0\}$ и голоморфна в $D \setminus \{z_0\}$, то есть в качестве $\gamma $ можно брать любую окрестность сколь угодно малого радиуса с центром в точке $z_0$.
	\begin{notation}
		$\Res f\big|_{z=z_0}\coloneq \frac{1}{2\pi i} \int_{\gamma } f(z)\d z$.
	\end{notation}
\end{definition}

\begin{theorem}[Основная теорема теории вычетов]
	Пусть $\gamma $ -- замкнутый контур, ограничивающий односвязную область $D$, функция $f$ непрерывна на $\cl D = D \cup \gamma $ и голоморфна внутри $D$, за исключением конечного числа точек. Тогда:
	\[
		\int_{\gamma } f(z)\d z = 2\pi i \sum_{k=1}^{m} \underset{z_k}{\Res} f.
	\]
\end{theorem}

\begin{proof}
	$m = 3$,
	\begin{figure}[H]
		\centering
		\incfig[0.7]{fig-16}
		\label{fig:fig-16}
	\end{figure}
	\[
		\Gamma = \gamma \cup \gamma_{1}^{-} \cup \gamma_{2}^{-} \cup \gamma_{3}^{-},
	\]
	\[
		\int_{\Gamma } f(z)\d z = 0 = \int_{\gamma } f(z)\d z - \int_{\gamma _1} f(z)\d z - \int_{\gamma _2} f(z)\d z - \int_{\gamma _3} f(z)\d z.
	\]
\end{proof}

\begin{theorem}[О сумме вычетов]
	Если функция голоморфна в $\overline{\C} $, за исключением конечного числа изолированных о.т., то
	\[
		\sum_{k=0}^{m} \underset{z_k}{\Res} f = 0.
	\]
\end{theorem}

\subsection{Вычисление интегралов}

\begin{definition}
	Главным значнием по Коши интеграла $\int_{-\infty }^{+\infty } f(x)dx$ называется
	\[
		\underset{R \rightarrow \infty }{\lim} \int_{-R}^{R} f(x)\d x \eqcolon Vp \int_{-\infty }^{\infty } f(x)\d x.
	\]
\end{definition}

\begin{remark}
	Если несобственный интеграл $\int_{-\infty }^{\infty } f(x)\d x$ сходится, то его значение совпадает с его главным значением по Коши, Обратно неверно.
\end{remark}

\begin{lemma}
	Пусть
	\begin{enumerate}
		\item Для некоторого $R_0 > 0$ функция $f$ непрерывна при $\abs{z} > R_0$ и $\Im z \geqslant 0$.
		\item $\underset{R \rightarrow \infty }{\lim} \underset{z \in \gamma _R}{\sup} \big|zf(z)\big| = 0$.
	\end{enumerate}

	Тогда $\underset{R \rightarrow \infty }{\lim} \int_{\gamma_R } f(z)\d z = 0$.
\end{lemma}

\begin{lemma}[Жордана]
	Пусть $\alpha >0$,
	\begin{enumerate}
		\item Для некоторого $R_0 > 0$ функция $f$ непрерывна при $\abs{z} > R_0$ и $\Im z \geqslant 0$.
		\item $\underset{R \rightarrow \infty }{\lim} \underset{z \in \gamma _A}{\sup} \big|f(z)\big| = 0$.
	\end{enumerate}

	Тогда $\underset{R \rightarrow \infty }{\lim} \int_{\gamma R} e^{i \alpha z} f(z)\d z = 0$.
\end{lemma}

\newpage

\subsection{Гармонические функции}

\begin{definition}[Гармоническая функция]
	Определенная в односвязной области $D \subset \R^2$ функция $u(x,y)$ называется \emph{гармонической функцией}, если $u \in C^2(D)$ и
	\[
		\triangle u \coloneq \frac{\partial^2 u}{\partial x^2} + \frac{\partial^2 u}{\partial y^2} \equiv 0,
	\]
	где $\triangle$ -- оператор Лапласа.
\end{definition}

\begin{theorem}
	Если функция $f$ голоморфна в односвязной области $D \subset \C$, то ее вещественная и мнимая части являются гармоническими функциями в этой области.
\end{theorem}

\begin{proof}
	\[
		f(z) = f(x + iy) = u(x,y) + iv(x,y),
	\]
	\[
		\frac{\partial u}{\partial x} = \frac{\partial v}{\partial y} , \ \frac{partial u}{\partial y} = - \frac{\partial v}{\partial x},
	\]
	\[
		\frac{partial^2 u}{\partial y^2}  = \frac{\partial}{\partial y} \left(\frac{\partial u}{\partial y} \right) = \frac{\partial}{\partial y} \left(-\frac{\partial v}{\partial x} \right) = - \frac{\partial^2 v}{\partial y \partial x} .
	\]

	Получаем, что смеш. производные непрерывны, зачит они равны $\implies \frac{\partial^2u}{\partial x^2} + \frac{\partial^2u}{\partial y^2} \equiv 0 \implies $ вещественная и мнимая части являются гармоническими.
\end{proof}

\subsection{Целые и мероморфные функции}

\begin{definition}[Целая функция]
	Голоморфная в $\C$ функция называется \emph{целой функцией}. Целая функция называется \emph{трансцендентной}, если бесконечность является ее существенно о.т.
\end{definition}

\begin{definition}[Мероморфная функция]
	Функция, голоморфная в области $D$ всюду, за исключением полюсов, называется \emph{мероморфной} в этой области функцией.
\end{definition}

\begin{theorem}[О мероморфной функции]
	Если $\infty $ является устранимой о.т. мероморфной функции, то данная функция является частным двух многочленов, то есть является рациональной функцией.
\end{theorem}

\begin{proof}
	$\infty $ -- изолированная о.т. (в силу условия), $z_1,\ldots,z_n$ -- конечное число оптимальных точек.

	\[
		f(z) = h(z) + \sum_{k=1}^{m} f_k \left(\frac{1}{z - z_0} \right),
	\]
	\[
		\underset{z \rightarrow \infty }{\lim} f(z) = C_0, \quad \underset{z \rightarrow \infty }{\lim} h(z) = C_0
	\]
	$\implies h = const$ (по теореме Лиувиля) $\implies f(z) = C_0 + \sum_{k=1}^{m} f_k \left(\frac{1}{z-z_k} \right) = \frac{P(z)}{Q(z)} $.
\end{proof}

\lesson{7}{от 28 мар 2024 12:45}{Продолжение}


\section{Основные принципы комплексного анализа}

\subsection{Принцип аргумента и Теорема Руше}

\begin{definition}
	Пусть $f$ голоморфна в некоторой проколотой окружности точки $z_0$, а $z_0$ не хуже, чем полюс, тогда:
	\[
		f(z) = \sum_{n}^{\infty } C_n (z-z_0)^n,
	\]
	\[
		M_f(z_0) \coloneq \inf \{n \in \Z: \ C_n \ne 0\}.
	\]
\end{definition}

\begin{lemma}
  Пусть $z_0$ -- обычная точка или полюс функции $f$. Тогда
  \[
    \underset{z_0}{\Res} \frac{f'}{f} = M_f(z_0).
  \]
\end{lemma}

\begin{note}
  $\frac{f'}{f} = \big(\ln f(z)\big)' $ -- логарифмическая производная функции $f$.
\end{note}

\begin{remark}
  Предположим, что есть многозначная функция $\phi $ и кривая $\gamma $. Если мы можем выделить ветвь функции $\phi $, которая будет непрерывна в окружности $\gamma : [a,b] \rightarrow \C, \ \gamma (a), \gamma (b)$, то вариацией этой функции вдоль кривой $\gamma $
\end{remark}

\lesson{8}{от 4 июн 2024 12:45}{Продолжение}


\begin{theorem}[Принцип взаимно однозначного соответствия]
	Пусть $f$ -- голоморфна в области $D, \ \gamma $ -- простой контур в $D: \ D_{\gamma } \subset D$. Если функция $f$ взаимно однозначна на $\gamma $, то $f$ однолистна в $D_{\gamma } $ и, следовательно, осуществяет конформное отображение области $D_{\gamma } $.
\end{theorem}

\subsection{Принцип компактности}

\begin{definition}[Относительно компактное подмножество]
	Подмножество МП называется \emph{относительно компактным}, если его замыкание компактно.
\end{definition}

\begin{definition}[Секвенциально компактное подмножество]
	Подмножество МП называется \emph{севенциально компактным}, если каждая его последовательность имеет подпоследовательность, сходящуюся к элементу этого подмножества.
\end{definition}

\begin{note}
	В МП севенциальная компактность равносильна компактности.
\end{note}

\begin{lemma}
	Подмножество МП является относительно компактным $\iff \forall $ его последовательность имеет сходящуюся подпоследовательность.
\end{lemma}

\begin{remark}
	Подмножество комплексной плоскости компактно $\iff $ оно замкнуто и ограничено.
\end{remark}

\begin{definition}[Равномерно ограниченное множество функций]
	Множество $\mathcal{F}$ функция из $\C$ в $\C$ называется \emph{равномерно ограниченным} на $A \subset \C$, если
	\[
		\underset{f \in \mathcal{F}}{\sup} \underset{z \in A}{\sup} \big|f(z)\big| < +\infty .
	\]
\end{definition}

\begin{definition}[Равностепенно непрерывное множество функций]
	Множество $\mathcal{F}$ функций из $\C$ в $\C$ называется \emph{равностепенно непрерывным} на $A \subset \C$, если $\forall z \in A \ \forall \epsilon > 0 \ \exists \delta > 0 : \ \forall f \in \mathcal{F} \ \forall z ' \in A$ из $\abs{z ' - z} \subset \delta $ следует, что $\big|f(z ') - f(z)\big| < \epsilon $.
\end{definition}

\begin{remark}
	Пусть $K$ -- компакт в $\C$,
	\[
		C(K) \ d (f_1,f_2) \coloneq \underset{z \in K}{\sup} \big|f_1(z) - f_2(z)\big|.
	\]
\end{remark}

\begin{theorem}[Арцела-Асколи]
	Пусть $K$ -- компакт в $\C$. Множество $\mathcal{F}\subset C(k)$ относительно компактно $\iff $ оно равномерно ограничено на $K$ и равностепенно непрерывно на $K$.
\end{theorem}

\begin{definition}[Относительно компактное множество]
	$\mathcal{F} \subset C(D)$ называется \emph{относительно компактным} в $D$, если для $\forall \{f_n\}_{n \in N} \subset \mathcal{F} \ \exists $ ее подпоследовательность $\{f_{n_k} \}_{k \in N} : \ \forall K \Subset D \ f_{n_k} \rightrightarrows f$ на $K$.
\end{definition}

\begin{lemma}
	Пусть $D$ -- область в $\C, \ \mathcal{F}\subset C(D)$ и $\mathcal{F}$ относительно компактно в $C(K) \ \forall K \Subset D$. Тогда $\mathcal{F}$ относительно компактно в $D$.
\end{lemma}

\begin{proof}
	$K_n \eqcolon \{z \in D : \dist (z,\partial D)\geqslant \frac{1}{n} \& \abs{z} \leqslant n\}, \ n \in \N $.

	$K_n \subset K_{n+1} $ и $\underset{n \in \N}{\bigcup} K_n = D$ (стандартная последовательность).

	$\exists \{f_{n_k} \}_{n\in\N} $ -- подпоследовательность $\{f_n\}$.

	\[
		\{f_{n}^{n} \}_{n\in\N} \quad f_{n}^{n} \rightrightarrows \text{ на } K_m \ \forall m \in \N,
	\]

	$\exists m \in \N : \ \frac{1}{m} \subset \dist \implies  K \subset K_m$.
\end{proof}

\begin{definition}[Равномерно ограниченное отображение]
	$f \subset C(D)$ называется \emph{равномерно ограниченным} в $D$, если это множество равномерно ограничено на каждом компакте.
\end{definition}

\begin{lemma}
	Пусть $D$ -- область в $\C$, $K$ -- компакт в $D$ и $V \in O_p \C : \ K \subset V \subset \cl V \subset D :$ если $\mathcal{F} \subset \mathcal{H}(D): \ \mathcal{F}_1 \eqcolon \{f ' : f \in \mathcal{F}\}$ равномерно ограничен на $V$, то $\mathcal{F}$ равностепенно непрерывен на $K$.
\end{lemma}

\begin{proof}
	$z \in K \ \exists \delta _1 > 0: \ B_{\delta _1} [z]\subset V$,
	\[
		z ' \in V : \ \abs{z ' - z } < \delta _1,
	\]
	\[
		\big|f(z ') - f(z)\big| = \left|\int_{z}^{z '}f '(\xi )\d \xi \right| \leqslant \underset{\xi  \in V}{\sup} \big|f '(\xi )\big| \cdot \abs{z ' - z} < M \cdot \delta _1 \leqslant \epsilon .
	\]

	$\delta \eqcolon \min \{\delta _1 ; \frac{\epsilon }{M + 1} \}$.
\end{proof}

\begin{theorem}[Принцип компактности, теорема Ментеля]
	Если $\mathcal{F}\subset A(D)$ равномерно ограничен в $D$, то $\mathcal{F}$ относительно компактно в $A(D)$.
\end{theorem}

\begin{proof}
	$K \subset D \ \exists \gamma : \ K \subset D_{\gamma } \ \& \ \cl D_{\gamma } \subset D$,
	\[
		\delta = \dist(K,\gamma ) > 0.
	\]

	$f '(z) = \frac{1}{2\pi} \int_{\gamma } \frac{f(\xi )}{(\xi - z)^2} \d \xi \ \forall z \in K $.

	$\forall z \in K \ \big|f '(z)\big| \leqslant \frac{M \cdot l (\gamma )}{2\pi \delta ^2} $, где $M = \underset{z \in \gamma }{\sup} \big|f(z)\big| < +\infty $, $l (\gamma )$ -- длина $\gamma \implies $ по пред. лемме, множество равномерно ограниченно $\implies $ по теореме Ацела-Аскяли все доказано.
\end{proof}

\subsection{Принцип непрерывности}

\begin{theorem}[Принцип непрерывности]
	Пусть $f$ непрерывно в $D$ и голоморфна в $D \setminus \gamma  $, где $\gamma $ -- ломанная в $D$, состоящая из конечного числа дуг окружностей. Тогда $f$ голоморфна в $D$.
\end{theorem}

\begin{proof}
	$\gamma = I \coloneq [z_1,z_2]$, возьмем $\triangle \subset D, \ \partial \triangle$,
	\[
		\cl \triangle \cap I = \varnothing , \quad \int_{\partial \triangle}f(z)\d z = 0,
	\]
	\[
		\cl \triangle \cap I \ne \varnothing .
	\]
	$\int_{\partial \triangle} = \int_{\gamma _1}  + \int_{\gamma _2} =0 $ (по обобщенной теореме Коши) каждый из интегралов равен нулю.
\end{proof}

\begin{remark}
	Утверждение теоремы остается в силе, если $\gamma $ будет спрямляемой кривой Жордана.
\end{remark}

\subsection{Принцип симметрии}

\begin{theorem}[Принцип симметрии]
	Пусть $D$ -- область в $\C$ и часть $\gamma $ ее границы является дугой окрестности.

	Если функция $f$ голоморфна в $D$ и непрерывна вплоть до $\gamma $, то функция $\widetilde{f}$, определенная в области $D^{*} $, симметричной области $D$ относительно $\gamma $, равенство $\widetilde{f}(z^{*} ) = (f^{(z)} )^{*} $, будет голоморфно в области $D \cup \gamma \cup P^{*} $.
\end{theorem}

\begin{proof}
	$z^{*} = \overline{z}  $.

	\[
		\widetilde{f}(\overline{z} ) = \overline{f(z)} = \overline{\sum_{n=0}^{\infty } C_n(z-z_0)} = \sum_{n=0}^{\infty } C_n (\overline{z}  - \overline{z_0} )^n
	\]
	раскладывается в ряд Тейлора, $z_0 \in D, \ z_0^{*} = \overline{z_0} \implies $ голоморфна в $D$, на отрезке $\gamma $ непрерывна $\implies $ по теореме $f$ голоморфно в объединении.
\end{proof}

\section{Конформные отображения}

\begin{theorem}[Лемма Шварца]
	$\mathbb{D} \coloneq \{z \in \C: \ \abs{z} \subset 1\}$.

	Пусть $f$ голоморфна в $\mathbb{D}$ и $f(0) = 0, \ \abs{f(z)} \leqslant 1$ в $\mathbb{D}$. Тогда $\big|f '(0)\big| \leqslant 1$ и $\big|f(z)\big| \leqslant \abs{z} \ \forall z \in \mathbb{D}$.
\end{theorem}

\begin{remark}
	Если в одном из неравенств имеет место равенство, то $\exists $ число $\lambda \in \C: \ \abs{\lambda } =1 : \forall z \in \mathbb{D}$
	\[
		f(z) = \lambda z.
	\]
\end{remark}

\begin{remark}
	Если $\phi $ -- конформное отображение $\mathbb{D}$ на себя и $\phi (0) \coloneq 0$, то $\forall  z \in \mathbb{D} \ \phi (z) = \lambda z$, где $\lambda = const, \ \abs{\lambda }  = 1$.
\end{remark}

\begin{theorem}[Римана]
	Если $D$ -- односвязная область, отличная от $\C$, то $\forall z_0 \in D \ \exists $ единственное конформное отображение $\phi : D \rightarrow \mathbb{D}: \ \phi (z_0)= 0$ и $\phi ' (z_0) > 0$.
\end{theorem}


\addcontentsline{toc}{section}{Список используемой литературы}
\begin{thebibliography}{}
	\bibitem{litlink1}  Шабат                         -- «Введение в комплексный анализ, 1976» (том 1)
	\bibitem{litlink2}  Привалов                      -- «Введение в ТФКП, 1967»
	\bibitem{litlink3}  Бицадзе                       -- «Основы теории аналитических функций комплексного переменного, 1984»
	\bibitem{litlink4}  Волковыский, Лунц, Араманович -- «Сборник задач по ТФКП», 1975»
	\bibitem{litlink5}  Гилев В.М.                    -- «Основы комплексного анализа. Ч.1», 2000»
	\bibitem{litlink6}  Исапенко К.А.                 -- «Комплексный анализ в примерах и упражнениях (Ч.1, 2017, Ч.2, 2018)»
	\bibitem{litlink7}  Мещеряков Е.А., Чемеркин А.А. -- «Комплексный анализ. Практикум»
	\bibitem{litlink8}  Боярчук А.К.                  -- «Справочное пособие по высшей математике» (том 4)
\end{thebibliography}
\end{document}
