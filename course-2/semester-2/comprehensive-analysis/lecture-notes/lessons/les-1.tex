\chapter{Голоморфные функции}

\lesson{1}{от 15 фев 2024 12:45}{Начало}


\section{Комплексная плоскость}

\subsection{Комплексные числа}

$\R^2 \coloneqq \R \times \R,$
\begin{align*}
	(x_1,y_1) + (x_2,y_2)     & \coloneqq (x_1+x_2,y_1+y_2),                 \\
	(x_1,y_1) \cdot (x_2,y_2) & \coloneqq (x_1x_2 - y_1y_2,x_1y_2 + x_2y_1).
\end{align*}

\[
	z = (x,y)=x+iy, \ x,y\in\R
\]
\begin{multicols}{2}
	\begin{align*}
		(1,0) & \eqqcolon 1, \\
		(0,1) & \eqqcolon i, \\
		(0,0) & \eqqcolon 0
	\end{align*}
	\begin{align*}
		x & \eqqcolon \Re z \\
		y & \eqqcolon \Im z
	\end{align*}

	\begin{figure}[H]
		\centering
		\incfig{fig-1}
		\label{fig:fig-1}
	\end{figure}
	\begin{align*}
		x & = r \cdot \cos \phi \\
		y & = r \cdot \sin \phi
	\end{align*}
\end{multicols}

\[
	r = \sqrt{x^2+y^2} \eqqcolon \abs{z},
\]
\[
	\phi \eqqcolon \arg z, \qquad \underbrace{0 \leqslant \arg z < 2\pi}_{\text{главное значение аргумента}}
\]

\[
	\Arg z \coloneqq \arg z + 2\pi k, \ k \in \Z, \qquad \overline{z} = x - iy
\]

\[
	\begin{array}{ll}
		\text{Формула Эйлера:}                  & e^{i \phi} = \cos \phi + i\sin \phi, \quad \forall \phi \in \R \\
		\text{Тригонометрическая форма записи:} & z = \abs{z}\cdot (\cos\arg z + i\sin\arg z)                    \\
		\text{Показательная форма записи:}      & z = \abs{z}e^{i\arg z}                                         \\
		\text{Формула Муавра:}                  & z^n = r^n(\cos n \phi + i\sin n \phi)
	\end{array}
\]

\begin{multicols}{2}
	$e^z = e^{x+iy} = e^x \cdot e^{iy}$

	$e^{z_1+z_2} = e^{z_1}\cdot e^{z_2}$

	$z^n = \abs{z}^n e^{in\arg z}$

	$z = re^{ir}, \quad z^n = z_0$
\end{multicols}

\[
	\sqrt[n]{z_0} = \sqrt[n]{\abs{z_0}} \cdot e^{i \frac{\arg z_0 + 2\pi k}{n}}, \quad 0 \leqslant k \leqslant n-1.
\]

\begin{theorem}
	$\forall z,z_1,z_2 \in \C$ справедливы равенства:
	\begin{multicols}{2}
		\begin{enumerate}
			\item $z \cdot \overline{z} = \abs{z}^2$ \\
			\item $\overline{(z_1 + z_2)} = \overline{z_1} + \overline{z_2}$ \\
			\item $\overline{z_1 \cdot z_2} = \overline{z_1}\cdot \overline{z_2}$ \\
			\item $\overline{\overline{z}} = z$ \\
			\item $\overline{z} = z \iff z \in \R$ \\
			\item $\abs{z_1 \cdot z_2} = \abs{z_1} \cdot \abs{z_2}$ \\
			\item $\abs{z_1 + z_2} \leqslant \abs{z_1} + \abs{z_2}$ \\
			\item $\big|\abs{z_1} - \abs{z_2}\big| \leqslant \abs{z_1 - z_2}$ \\
			\item $\arg(z_1 \cdot z_2) = \arg z_1 + \arg z_2 \ (\mod 2\pi)$ \\
			\item $\arg \left(\frac{z_1}{z_2}\right) = \arg z_1 - \arg z_2 \ (\mod 2\pi)$
		\end{enumerate}
	\end{multicols}
\end{theorem}

\begin{figure}[H]
	\centering
	\incfig{fig-2}
	\caption{Сфера Римана}
	\label{fig:fig-2}
\end{figure}

\[
	\xi^2 + \eta^2 + \zeta - \zeta = 0, \quad \left\{\begin{array}{l}
		\xi = \frac{x}{1 + \abs{z}^2}  \\
		\eta = \frac{y}{1 + \abs{z}^2} \\
		\zeta = \frac{\abs{z}^2}{1 + \abs{z}^2}
	\end{array}\right..
\]

\[
	P: \C \overset{\text{на}}{\rightarrow } S \setminus \{N\}, \quad P(z) = (\xi,\eta,\zeta).
\]

\[
	\underset{\text{общее уравнение окружности}}{A(x^2 + y^2) + Bx + Cy + D = 0, \ A,B,C,D \in \R}
\]
\[
	\begin{array}{rl}
		\gamma    & \text{ -- окружность на }\C, \\
		P(\gamma) & \text{ -- окружность на }S.
	\end{array}
\]
\[
	\abs{z}^2 = x^2 + y^2 = \frac{\zeta}{1-\zeta}, \quad \left\{\begin{array}{l}
		x = \frac{\xi}{1 - \zeta} \\
		y = \frac{\eta}{1 - \zeta}
	\end{array}\right..
\]

\[
	A \zeta + B \xi + C \eta + D(1 - \zeta) = 0, \quad \begin{array}{rl}
		\overline{\C} & \coloneqq \C \cup \{\infty \} \\
		P(\infty )    & \coloneqq N
	\end{array}.
\]

\subsection{Топология комплексной плоскости}

\[
	\alpha - \beta = \frac{12}{43}.
\]

$M_1,M_2 \in \R^3$,

\[
	\dist(M_1,M_2)\coloneqq \sqrt{(\xi_1 - \xi_2)^2 + (\eta_1 - \eta_2)^2 + (\zeta_1 - \zeta_2)^2},
\]

\begin{multicols}{2}
	\[
		\d(z_1,z_2)\coloneqq \abs{z_1-z_2}, \ z_1,z_2 \in \C,
	\]

	\[
		\rho(z_1,z_2) \coloneqq \dist \big(P(z_1),P(z_2)\big),
	\]

	\[
		B_{\epsilon}(z_0)\coloneqq \big\{z \in \C: \ \abs{z - z_0} < \epsilon\big\},
	\]

	\begin{figure}[H]
		\centering
		\incfig{fig-3}
		\label{fig:fig-3}
	\end{figure}

	\[
		P: \C \overset{\text{на}}{\rightarrow }S \setminus \{N\}.
	\]
\end{multicols}

\begin{definition}[Окрестность точки]
	Множество называется \emph{окрестностью точки}, если оно содержит некоторый шарик с центром в этой точке.
	\begin{notation}
		\[
			O_z, \quad z \in \overline{\C}.
		\]
	\end{notation}
\end{definition}
