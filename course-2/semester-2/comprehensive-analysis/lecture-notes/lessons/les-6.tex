\newpage

\lesson{6}{от 21 мар 2024 12:45}{Продолжение}


\section{Ряды Тейлора и Лорана. Элементы теории вычетов}

\subsection{Разложение голоморфной функции в ряд Тейлора}

\begin{theorem}
	Пусть $f \in \mathcal{H}(D)$. Тогда $\forall z_0 \in D \ \exists r > 0$: при $\abs{z-z_0} < r$
	\[
		f(x) = \sum_{n=0}^{\infty } \frac{f^{(n)} (z_0)}{n!} (z - z_n)^n.
	\]
\end{theorem}

\begin{corollary}
	$\mathcal{H}(D) = \mathcal{A}(D)$.
\end{corollary}

\begin{theorem}
	Пусть $f$ голоморфна в $B_r(z_0) \ \forall z \in B_r(z_0) \ f(z) = \sum_{n=0}^{\infty } C_n (z-z_0)^n$.

	Тогда $\forall n \in \overline{\N}$
	\[
		C_n = \frac{1}{2\pi i} \int_{\abs{\xi -z_0} = \rho} \frac{f(z)}{(\xi - z_0)^{n+1} } \d \xi \quad \forall \rho \in (0,r).
	\]

	То есть любой степенной ряд является рядом Тейлора для своей суммы.
\end{theorem}

\begin{proof}
	Радиус сходимости $\geqslant r, \ \rho \in (0 ; r)$.

	$\abs{z-z_0} = \rho \implies $ ряд сходится, рассмотрим:
	\begin{multline*}
		f(z) = \frac{1}{2\pi i} \int_{\abs{\xi -z_0}= \rho } \frac{f(\xi )}{(\xi -z_0)^{k+1} } \d \xi = \\
		= \frac{1}{2\pi i} \int_{\abs{\xi -z_0} = \rho } \frac{\sum_{n=0}^{\infty } C_n(\xi -z_0)^n}{(\xi -z_0)^{k+1} } \d \xi = \\
		= -\frac{k!}{2\pi i} \cdot C_k \cdot 2\pi i = C_n \cdot k!,
	\end{multline*}
	\[
		C_k = \frac{f^{(k)} (z_0)}{k!} .
	\]
\end{proof}

\begin{theorem}[Неравенство Коши]
	Пусть $f$ голоморфна в $D$ и $B_r[z_0] \subset D, \ f(z) = \sum_{n=0}^{\infty } C_n(z-z_0)^n$.

	Пусть $M \coloneq \underset{\abs{z-z_0} \leqslant r}{\sup} \big|f(z)\big|$. Тогда $\forall n \in \overline{\N} \ \abs{C_n}  \leqslant \frac{M}{r^n} $.
\end{theorem}

\begin{definition}[Предельная точка]
	Точка называется \emph{предельной точкой множества}, если в любой ее окрестности есть точки множества, отличные от данной.
\end{definition}

\begin{corollary}
	Любые две аналитические в области функции, совпадающие на множестве, имеющем в этом множестве предельную точку, тождественно равны.
\end{corollary}

\subsection{Ряды Лорана}

\begin{definition}[Ряд Лорана]
	\emph{Рядом Лорана} называется степенной ряд вида $\sum_{n=-\infty }^{\infty } C_n(z-z_0)^n$. Ряд Лорана раскладывается на сумму двух рядов:
	\[
		\sum_{n=-\infty }^{\infty } C_n (z-z_0)^n \coloneq \sum_{n=0}^{\infty } C_n (z-z_0)^n + \sum_{n=1}^{\infty } C_{-n} (z-z_0)^{-n} .
	\]

	Ряд Лорана сходится $\iff $ сходятся обе его составляющие.

	Область сходимости ряда Лорана: $0 \leqslant r < \abs{z-z_0} < R \leqslant +\infty $.
\end{definition}

\begin{theorem}[О ряде Лорана]
	Если функция $f$ голоморфна в кольце $r < \abs{z-z_0} < R$, то в этом кольце она разлагается в ряд Лорана.

	$f(z) = \sum_{n=-\infty }^{\infty } C_n(z-z_0)^n$ с коэфициентами $C_n$, определяемыми формулами:
	\[
		C_n = \frac{1}{2\pi i} \int_{\abs{\xi -z_0} = \rho} \frac{f(\xi )}{(\xi -z_0)^{n+1} } \d \xi \quad \forall \rho \in (r,R).
	\]
\end{theorem}

\subsection{Классификация изолированных особых точек}

\begin{definition}[Правильная точка]
	Точка $z_0 \in \dom f$ называется \emph{правильной} точкой функции $f$, если $f$ определена в некоторой области и непрерывна в самой функции.
\end{definition}

\begin{definition}[Особая точка]
	\emph{Особой} точкой функции называется предельная точка ее области определения, этой области не принадлежащая.
\end{definition}

\begin{definition}[Изолированная особая точка]
	Особая точка называется \emph{изолированной} особой точкой, если в некоторой ее окрестности других особых точек нет.
\end{definition}

\begin{remark}
	Особая точка функции называется изолированной, если в проколотой окрестности этой точки функция голоморфна.
\end{remark}

\begin{eg}
	\[
		f(z) = \frac{1}{\sin \frac{1}{z} },
	\]

	$z_0 = 0$ -- особая точка,

	$\sin \frac{1}{z} = 0 \implies  \frac{1}{z} = \pi k, \ k \in Z$,

	$z_k = \frac{1}{\pi k} , \ k \in \Z$ -- особые точки,

	\[
		\frac{1}{\pi(k+1)} < \frac{1}{\pi k} < \frac{1}{\pi (k-1)}.
	\]
\end{eg}

\begin{theorem}[О путях и полюсах]
	Изолированная особая точка $z_0$ функции $f$ является полюсом порядка $m$ функции $f \iff $ она является путем $m$-го порядка функции $\rho (z)= \frac{1}{f(x)} $.
\end{theorem}

\begin{proof}
	Самостоятельно.
\end{proof}

\begin{theorem}[Сохоцкий]
	Изолированная особая точка функции является существенно особой точкой $\iff $ в любой ее окрестности функция принимает значения сколь угодно близкие к любому числу $a \in \overline{\C} $.
\end{theorem}

\begin{definition}[$A$-точка]
	Пусть $A \in \C$, точка $z$ называется \emph{$A$-точкой} функции $f$, если $f(z) = A$.
\end{definition}

\begin{theorem}[Большая теорема Пикара]
	В окрестности существенно особой точки $z_0$ голоморфной функции $f \ \forall A \in \C$, за исключением быть может одного, существует последовательность $A$-точек функции $f$, сходящаяся к точке $z_0$.
\end{theorem}

\subsection{Вычеты}

\begin{definition}[Вычет функции относительно точки]
	Если $z_0$ -- изолированная особая точка функции $f$, то \emph{вычетом} $f$ относительно $z_0$ называется интеграл $\frac{1}{2\pi i} \int_{\gamma } f(z)\d z$, где $\gamma $ -- произволный контур, ограничивающий область $D$: $f$ непрерывна в $\cl D \setminus \{z_0\}$ и голоморфна в $D \setminus \{z_0\}$, то есть в качестве $\gamma $ можно брать любую окрестность сколь угодно малого радиуса с центром в точке $z_0$.
	\begin{notation}
		$\Res f\big|_{z=z_0}\coloneq \frac{1}{2\pi i} \int_{\gamma } f(z)\d z$.
	\end{notation}
\end{definition}

\begin{theorem}[Основная теорема теории вычетов]
	Пусть $\gamma $ -- замкнутый контур, ограничивающий односвязную область $D$, функция $f$ непрерывна на $\cl D = D \cup \gamma $ и голоморфна внутри $D$, за исключением конечного числа точек. Тогда:
	\[
		\int_{\gamma } f(z)\d z = 2\pi i \sum_{k=1}^{m} \underset{z_k}{\Res} f.
	\]
\end{theorem}

\begin{proof}
	$m = 3$,
	\begin{figure}[H]
		\centering
		\incfig[0.7]{fig-16}
		\label{fig:fig-16}
	\end{figure}
	\[
		\Gamma = \gamma \cup \gamma_{1}^{-} \cup \gamma_{2}^{-} \cup \gamma_{3}^{-},
	\]
	\[
		\int_{\Gamma } f(z)\d z = 0 = \int_{\gamma } f(z)\d z - \int_{\gamma _1} f(z)\d z - \int_{\gamma _2} f(z)\d z - \int_{\gamma _3} f(z)\d z.
	\]
\end{proof}

\begin{theorem}[О сумме вычетов]
	Если функция голоморфна в $\overline{\C} $, за исключением конечного числа изолированных о.т., то
	\[
		\sum_{k=0}^{m} \underset{z_k}{\Res} f = 0.
	\]
\end{theorem}

\subsection{Вычисление интегралов}

\begin{definition}
	Главным значнием по Коши интеграла $\int_{-\infty }^{+\infty } f(x)dx$ называется
	\[
		\underset{R \rightarrow \infty }{\lim} \int_{-R}^{R} f(x)\d x \eqcolon Vp \int_{-\infty }^{\infty } f(x)\d x.
	\]
\end{definition}

\begin{remark}
	Если несобственный интеграл $\int_{-\infty }^{\infty } f(x)\d x$ сходится, то его значение совпадает с его главным значением по Коши, Обратно неверно.
\end{remark}

\begin{lemma}
	Пусть
	\begin{enumerate}
		\item Для некоторого $R_0 > 0$ функция $f$ непрерывна при $\abs{z} > R_0$ и $\Im z \geqslant 0$.
		\item $\underset{R \rightarrow \infty }{\lim} \underset{z \in \gamma _R}{\sup} \big|zf(z)\big| = 0$.
	\end{enumerate}

	Тогда $\underset{R \rightarrow \infty }{\lim} \int_{\gamma_R } f(z)\d z = 0$.
\end{lemma}

\begin{lemma}[Жордана]
	Пусть $\alpha >0$,
	\begin{enumerate}
		\item Для некоторого $R_0 > 0$ функция $f$ непрерывна при $\abs{z} > R_0$ и $\Im z \geqslant 0$.
		\item $\underset{R \rightarrow \infty }{\lim} \underset{z \in \gamma _A}{\sup} \big|f(z)\big| = 0$.
	\end{enumerate}

	Тогда $\underset{R \rightarrow \infty }{\lim} \int_{\gamma R} e^{i \alpha z} f(z)\d z = 0$.
\end{lemma}

\newpage

\subsection{Гармонические функции}

\begin{definition}[Гармоническая функция]
	Определенная в односвязной области $D \subset \R^2$ функция $u(x,y)$ называется \emph{гармонической функцией}, если $u \in C^2(D)$ и
	\[
		\triangle u \coloneq \frac{\partial^2 u}{\partial x^2} + \frac{\partial^2 u}{\partial y^2} \equiv 0,
	\]
	где $\triangle$ -- оператор Лапласа.
\end{definition}

\begin{theorem}
	Если функция $f$ голоморфна в односвязной области $D \subset \C$, то ее вещественная и мнимая части являются гармоническими функциями в этой области.
\end{theorem}

\begin{proof}
	\[
		f(z) = f(x + iy) = u(x,y) + iv(x,y),
	\]
	\[
		\frac{\partial u}{\partial x} = \frac{\partial v}{\partial y} , \ \frac{partial u}{\partial y} = - \frac{\partial v}{\partial x},
	\]
	\[
		\frac{partial^2 u}{\partial y^2}  = \frac{\partial}{\partial y} \left(\frac{\partial u}{\partial y} \right) = \frac{\partial}{\partial y} \left(-\frac{\partial v}{\partial x} \right) = - \frac{\partial^2 v}{\partial y \partial x} .
	\]

	Получаем, что смеш. производные непрерывны, зачит они равны $\implies \frac{\partial^2u}{\partial x^2} + \frac{\partial^2u}{\partial y^2} \equiv 0 \implies $ вещественная и мнимая части являются гармоническими.
\end{proof}

\subsection{Целые и мероморфные функции}

\begin{definition}[Целая функция]
	Голоморфная в $\C$ функция называется \emph{целой функцией}. Целая функция называется \emph{трансцендентной}, если бесконечность является ее существенно о.т.
\end{definition}

\begin{definition}[Мероморфная функция]
	Функция, голоморфная в области $D$ всюду, за исключением полюсов, называется \emph{мероморфной} в этой области функцией.
\end{definition}

\begin{theorem}[О мероморфной функции]
	Если $\infty $ является устранимой о.т. мероморфной функции, то данная функция является частным двух многочленов, то есть является рациональной функцией.
\end{theorem}

\begin{proof}
	$\infty $ -- изолированная о.т. (в силу условия), $z_1,\ldots,z_n$ -- конечное число оптимальных точек.

	\[
		f(z) = h(z) + \sum_{k=1}^{m} f_k \left(\frac{1}{z - z_0} \right),
	\]
	\[
		\underset{z \rightarrow \infty }{\lim} f(z) = C_0, \quad \underset{z \rightarrow \infty }{\lim} h(z) = C_0
	\]
	$\implies h = const$ (по теореме Лиувиля) $\implies f(z) = C_0 + \sum_{k=1}^{m} f_k \left(\frac{1}{z-z_k} \right) = \frac{P(z)}{Q(z)} $.
\end{proof}
