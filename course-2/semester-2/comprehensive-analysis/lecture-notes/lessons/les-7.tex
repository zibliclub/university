\lesson{7}{от 28 мар 2024 12:45}{Продолжение}


\section{Основные принципы комплексного анализа}

\subsection{Принцип аргумента и Теорема Руше}

\begin{definition}
	Пусть $f$ голоморфна в некоторой проколотой окружности точки $z_0$, а $z_0$ не хуже, чем полюс, тогда:
	\[
		f(z) = \sum_{n}^{\infty } C_n (z-z_0)^n,
	\]
	\[
		M_f(z_0) \coloneq \inf \{n \in \Z: \ C_n \ne 0\}.
	\]
\end{definition}

\begin{lemma}
  Пусть $z_0$ -- обычная точка или полюс функции $f$. Тогда
  \[
    \underset{z_0}{\Res} \frac{f'}{f} = M_f(z_0).
  \]
\end{lemma}

\begin{note}
  $\frac{f'}{f} = \big(\ln f(z)\big)' $ -- логарифмическая производная функции $f$.
\end{note}

\begin{remark}
  Предположим, что есть многозначная функция $\phi $ и кривая $\gamma $. Если мы можем выделить ветвь функции $\phi $, которая будет непрерывна в окружности $\gamma : [a,b] \rightarrow \C, \ \gamma (a), \gamma (b)$, то вариацией этой функции вдоль кривой $\gamma $
\end{remark}
