\lesson{3}{от 29 фев 2024 12:45}{Продолжение}


\begin{definition}[Связное множество]
	$ A \subset \overline{\C} $ называется \emph{связным}, если $ \nexists U,V \in Op \overline{\C}: \ U \cap A \ne \varnothing, \ U \cap V = \varnothing $.
	\begin{notation}
		\[
			Op \overline{\C} \text{ -- совокупность всех открытых множеств}
		\]
	\end{notation}
\end{definition}

\begin{eg}
	Несвязно:
	\begin{figure}[H]
		\centering
		\incfig[0.5]{fig-9}
		\label{fig:fig-9}
	\end{figure}
\end{eg}

\begin{definition}[Линейно связное множество]
	Множество называется \emph{линейно связным}, если любые две его точки можно соединить путем, значения которого лежат в этом множестве.
\end{definition}

\begin{remark}
	В пространстве $ \R^n $, и в частности $ \overline{\C} $, любое открытое множество связно $ \iff $ оно линейно связно.
\end{remark}

\begin{definition}[Область]
	\emph{Областью} в $ \overline{\C} \ (\C) $ называется любое непустое открытое связное множество.
\end{definition}

\begin{definition}[Замкнутая область]
	\emph{Замкнутой областью} будем называть замыкание области.
\end{definition}

\section{Функции комплексного переменного}

\subsection{Структура функции комплексного переменного}

$ f : \C \rightarrow \C $
\[
	\begin{array}{l}
		\dom f \text{ -- область определения функции} \\
		\im f \text{ -- область значения функции}
	\end{array}
\]

\begin{definition}[Предел отображения]
	$ D \subset \dom f, \ z_0 \in \overline{\C} $ -- предельная точка $ D $. Тогда $ w_0 \in \overline{\C} $ называется \emph{пределом отображения} $ f $,
	\[
		w_0 \coloneqq \underset{D \circ z \rightarrow z_0}{\lim}f(z) \text{, если }\forall V \in O_{w_0} \ \exists U \in O_{z_0}: \ f(\mathring{U}\cap D)\subset V,
	\]
	\[
		U \in O_{z_0}, \quad \mathring{U} = U\setminus\{z_0\}.
	\]
\end{definition}

\begin{note}
	В случае, когда $ z_0,w_0 \in\C $ следует, что $ \forall \epsilon > 0 \ \exists \delta > 0 : \ \forall z \in D $
	\[
		0 < | z - z_0 | < \delta \implies \big| f(z) - w_0 \big| < \epsilon.
	\]
\end{note}

\begin{definition}[Непрерывная функция в точке]
	Функция $ f $ называется \emph{непрерывной в точке} $ z_0 \in \C $, если:
	\begin{enumerate}
		\item $ z_0 \in \dom f $.
		\item $ \forall \epsilon > 0 \ \exists \delta > 0: \ \forall z \in D $
		      \[
			      0 < | z - z_0 | < \delta \implies \big| f(z) - w_0 \big| < \epsilon.
		      \]
	\end{enumerate}
\end{definition}

\begin{definition}[Непрерывная функция на множестве]
	Функция $ f : \C \rightarrow \C $ непрерывна на $ D \subset \C $, если
	\begin{enumerate}
		\item $ D \subset \dom f $.
		\item $ \forall z_0 \in D \ \forall \epsilon > 0 \ \exists \delta > 0 \ \forall z \in D $
		      \[
			      | z - z_0 | < \delta \implies \big|f(z) - f(z_0)\big| < \epsilon.
		      \]
	\end{enumerate}
\end{definition}

\begin{note}[Функция Дирихле]
	\[
		D(x) = \left\{\begin{array}{ll}
			1, & x \in \Q              \\
			0, & x \in \R \setminus \Q
		\end{array}\right.,
	\]
	непрерывна на $ \Q $, непрерывна на $ \R \setminus \Q $.
\end{note}

\begin{remark}
	Если множество является открытым или совпадает с областью определения функции, то непрерывность функции на этом множестве равносильно ее непрерывности в каждой точке.
	\[
		f_n: \C \rightarrow \C (n \in \N), \quad D \coloneqq \underset{n \in \N}{\bigcap} \dom f_n.
	\]
\end{remark}

\begin{definition}
	$ A \subset D, \ f : A \rightarrow \C, \ f_n \rightrightarrows f $ на $ A $, если $ \forall \epsilon > 0 \ \exists n_0 \in \N : \ \forall z \in A \ \forall n \geqslant n_0 $
	\[
		\big|f_n(z) - f(z)\big| < \epsilon.
	\]

	\Big($ \forall \epsilon > 0 \ \exists n_0 \in \N:  \forall n \geqslant n_0 \quad \underset{z \in A}{\sup} \big| f_n(z) - f(z) \big| < \epsilon, \ | z - z_0 | < \delta \implies \big| f(z) - f(z_0) \big| < \epsilon $\Big).
\end{definition}

\begin{theorem}[Вейерштрасса]
	Если $ \{f_n\}_{n\in\N} \subset C(A), \ f_n \rightrightarrows f $, то $ f \in C(A) $.
\end{theorem}

\begin{definition}[Функциональный ряд]
	\emph{Функциональным рядом} называется формальная сумма членов последовательности функции.
	\begin{notation}
		$\sum_{n=1}^{\infty}f_n$.
	\end{notation}
\end{definition}

\begin{definition}[Числовой ряд]
	$ \forall z \in D \ \sum_{n=1}^{\infty}f_n(z) $ называется \emph{числовым рядом} $ \big\{f_n(z)\big\}_{n \in \N} $.
	\[
		S_n \coloneqq \sum_{k=1}^{n}f_k \text{ -- частичная сумма}.
	\]
\end{definition}

\begin{theorem}[Признак Вейерштрасса]
	$ \sum_{n=1}^{\infty}f_n $ таков, что $ \forall n \in \N \ \forall z \in A \ | f_n | \leqslant c_n $, причем $ \sum_{n=1}^{\infty} c_n $ сходится. Тогда ряд $ \sum_{n=1}^{\infty} f_n $ равномерно абсолютно сходится на $ A $.
\end{theorem}

\begin{theorem}[Критерий Коши (равномерная сходимость)]
	$ \{f_n\}_{n\in\N} $ равномерно сходится на $ A \iff \forall \epsilon > 0 \ \exists n_0 \in \N: \forall n,m \geqslant n_0 $
	\[
		\underset{z \in A}{\sup}\big|f_n(z) - f_n(z_0)\big| < \epsilon.
	\]
\end{theorem}

\begin{definition}[Линейная функция]
	Функция $ f : \C \rightarrow \C $ называется \emph{линейной}, если $ \forall \alpha, \beta \in \C \ \forall z_1,z_2 \in \C $
	\[
		f(\alpha z_1 + \beta z_2) = \alpha f(z_1) + \beta f(z_2).
	\]
\end{definition}

\begin{remark}
	Функция $ f: \C \rightarrow \C $ является линейной $ \iff \exists a \in \C : \forall z \in \C $
	\[
		f(z) = az.
	\]
\end{remark}

\subsection{Степенные ряды}

\[
	\sum_{n=0}^{\infty}a_n(z-z_0)^n, \text{ где } \{a_n\}_{n\in\N} \subset \C, \ z,z_0 \in \C.
\]

\begin{theorem}[1-я теорема Абеля]
	Если ряд $ \sum_{n=0}^{\infty}a_n(z-z_0)^n $ сходится в точке $ \nequalto{z_1}{z_0} \in \C $, то он абсолютно сходится при $ | z - z_0 | < | z_1 - z_0 | $.

	А если ряд $ \sum_{n=0}^{\infty}a_n(z - z_0)^n $ расходится в точке $ \nequalto{z_1}{z_0} \in \C $, то он расходится и при $ | z - z_0 | > | z_1 - z_0 | $.
\end{theorem}

\begin{proof}\leavevmode
	\begin{enumerate}
		\item $ \sum_{n=0}^{\infty}a_n(z_1 - z_0)^n $ сходится $ \implies \big|a_n(z_1 - z_0)^n\big| \xrightarrow[n \rightarrow\infty]{} 0 $.
		      \[
			      c \coloneqq \underset{n\in\N}{\sup} \big|a_n(z_1 - z_0)^n\big| < +\infty, \quad | z - z_0 | < | z_1 - z_0 |.
		      \]

		      Рассмотрим
		      \[
			      \sum_{n=0}^{\infty}\big|a_n(z-z_0)^n\big| = \sum_{n=0}^{\infty}\big|a_n(z_1 - z_0)^n\big|\cdot \left|\frac{z-z_0}{z_1-z_0}\right|^n \leqslant c \cdot \sum_{n=0}^{\infty}\left|\frac{z-z_0}{z_1-z_0}\right|^n < + \infty.
		      \]
		\item \textbf{добавить}
	\end{enumerate}
\end{proof}

\begin{definition}[Радиус сходимости]
	Элемент $ R \in [0;+\infty] $ называется \emph{радиусом сходимости} ряда $ \sum_{n=0}^{\infty}a_n(z-z_0)^n $, если при $ | z - z_0 | < R $ исходный ряд абсолютно сходится, а при $ | z-z_0 | > R $ исходный ряд расходится.
\end{definition}

\begin{theorem}[Коши-Адамара]
	Для степенного ряда $ \sum_{n=0}^{\infty}a_n(z-z_0)^n $ положим $ l \coloneqq \underset{n \rightarrow\infty}{\overline{\lim}} \sqrt[n]{| a_n |} $. Тогда:
	\begin{enumerate}
		\item Если $ l=0 $, то исходный ряд сходится $ \forall z \in \C $.
		\item Если $ l=\infty $, то исходный ряд сходится только в точке $ z_0 $.
		\item Если $ l\in(0;+\infty) $, то при $ | z-z_0 | < \frac{1}{l} $, а при $ | z-z_0 | > \frac{1}{l} $ исходный ряд расходится.
	\end{enumerate}
\end{theorem}

\begin{proof}\leavevmode
	\begin{enumerate}
		\item $ \underset{n \rightarrow\infty}{\overline{\lim}} \sqrt[n]{| a_n |} = \underset{n \rightarrow\infty}{\lim} \sqrt[n]{| a_n |} = 0 $,
		      \[
			      z \in \C, \ \sum_{n=0}^{\infty}\big| a_n(z-z_0)^n \big|.
		      \]
		      $ \underset{n \rightarrow \infty}{\lim} \sqrt[n]{\big| a_n(z-z_0)^n \big|} = \underset{n \rightarrow \infty}{\lim} \sqrt[n]{\big| a_n \big|} \cdot | z - z_0 | = 0 \implies $ ряд сходится.
		\item $ \underset{n \rightarrow\infty}{\overline{\lim}} \sqrt[n]{| a_n |} = \infty $,
		      \[
			      \exists\{a_{n_k}\}_{k\in\N} \subset \{a_n\}_{n\in\N}, \quad \sqrt[n_k]{| a_{n_k} |} \rightarrow + \infty.
		      \]
		      $ \sqrt[n_k]{| a_{n_k} |} \cdot | z - z_0 | \rightarrow +\infty \implies | a_{n_k} | $.
		\item $ | z - z_0 | < \frac{1}{l} \implies l | z - z_0 | < 1 $.

		      \textbf{Дописать}.
	\end{enumerate}
\end{proof}
