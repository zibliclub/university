\lesson{3}{от 19 мар 2024 10:30}{Продолжение}


\section{Устойчивость по первому приближению}

\begin{theorem}[Ляпунова]
	$(3) \ \dot{\overline{x} } = \overline{f} (\overline{x} ,t)$. Пусть $\overline{f} (\overline{x} ,t) = A \overline{x} + \overline{\phi } (\overline{x} ,t)$. \\
	$(4) \ \big|\overline{\phi } (\overline{x} ,t)\big|< j(\overline{x} ) \cdot |\overline{x} |, \ j(\overline{x} ) \rightarrow 0$ при $\abs{\overline{x} } \rightarrow 0$.

	\[
		\|\cdot \| \sim |\cdot |, \ \abs{\overline{x} } = \sqrt{x_{1}^{2} + x_{2}^{2} + \ldots + x_{n}^{2} }.
	\]

	Тогда:
	\begin{enumerate}
		\item Если все корни $\lambda _1, \ldots ,\lambda _n$ уравнения $(3)$ имеют отриц. действ. части ($\Re \lambda _k < 0, \ k = \overline{1,n} $), то нулевое решение системы $(3)$ асимптотически устойчиво.
		\item Если $\exists$ хотя бы один корень хар-го уравнения $(2)$ с полож. действ. частью ($\exists m: \ \Re \lambda _m > 0$), то нулевое решение системы $(3)$ неустойчиво.
	\end{enumerate}
\end{theorem}

\section{Исследование отрицательности действительных частей корней хар-го уравнения}

$(5) \ a_0 \lambda ^n + a_1 \lambda^{n-1} + \ldots + a_{n-1} + a_n = 0$ -- хар. уравнение, $a_k \in R, \ h = \overline{0,n} , \ a_0 \ne 0$.

Составим матрицу Гурвица:
\[
	\Gamma = \left(\begin{array}{cccccc}
			a_1    & a_0    & 0      & 0      & \cdots & 0      \\
			a_3    & a_2    & a_1    & a_0    & \cdots & 0      \\
			a_5    & a_4    & a_3    & a_2    & \cdots & 0      \\
			\vdots & \vdots & \vdots & \vdots & \ddots & \vdots \\
			0      & 0      & 0      & 0      & \cdots & a_n
		\end{array}\right)
\]

\begin{theorem}[Необходимое условие отрицательности действительных частей всех корней]
	$\lambda _i, \ i = \overline{1,n} $ является $a_k > 0, \ k = \overline{0,n} $.

	Для $n \leqslant 2$ это условие является и достаточным.
\end{theorem}

\begin{theorem}[Достаточное условие отрицательности всех действительных частей корней характеристического уравнения $(5)$]
	Если все главные миноры матрицы Гурвица больше $0$, то все действительные части корней хар. ур. $(5)$ отрицательны.
\end{theorem}

\begin{crit}[Рауса-Гурвица]
	Пусть:
	\begin{enumerate}
		\item $a_k > 0, \ k = \overline{0,n} $ в уравнении $(5)$.
		\item Все главные миноры матрицы Гурвица положительные, то есть $\triangle_1 = a_1 > 0, \ \triangle_2 = \left|\begin{array}{cc}
				      a_1 & a_0 \\ a_3 & a_2
			      \end{array}\right| > 0, \ \triangle_3 = \left|\begin{array}{ccc}
				      a_1 & a_0 & 0   \\
				      a_3 & a_2 & a_1 \\
				      a_5 & a_4 & a_3
			      \end{array}\right| > 0, \ \ldots, \ \triangle_n = a_n \cdot \triangle_{n-1} > 0$,

		      $\Re \lambda _k < 0, \ k = \overline{1,n} $.
	\end{enumerate}
\end{crit}

\section{Фазовый портрет линейной автономной системы на плоскости}

Рассмотрим систему на плоскости:
\[
	\dot{\overline{x} } = \overline{f} (\overline{x} ), \ \overline{x} = \left(\begin{array}{c}
			x_1 \\ x_2 \\ \vdots \\ x_n
		\end{array}\right),
\]

\[
	(6) \ \left\{\begin{array}{l}
		\dot{x} = f_1(x,y) \\
		\dot{y} = f_2(x,y)
	\end{array}\right. \text{ -- нелинейная},
\]
\[
	(7) \ \left\{\begin{array}{l}
		\dot{x} = ax + by \\
		\dot{y} = cx + dy
	\end{array}\right. \text{ -- линейная}, \ \left|\begin{array}{cc}
		a & b \\ c & d
	\end{array}\right|
\]

\begin{definition}[Траектория на дуговой плоскости]
	\emph{Траекторией (или дуговой кривой) на дуговой плоскости} (плоскость переменных $x,y$) называется график решения, состоящий из точек $(x,y)$, где $x = x(t), \ y = y(t)$ в момент времени $t$.
\end{definition}

\begin{definition}[Фазовый портрет]
	\emph{Фазовый портрет} -- совокупность всех фазовых кривых.
\end{definition}

Свойства траекторий:
\begin{enumerate}
	\item Не пересекаются.
	\item Различным решениям системы может соответствовать одна и та же траектория.
	\item \emph{Особой точкой} системы $\left\{\begin{array}{l}
			      \dot{x} = f_1(x,y) \\
			      \dot{y} = f_2(x,y)
		      \end{array}\right.$ называется точка, координаты которой удовлетворяют уравнению $\left\{\begin{array}{l}
			      f_1(x,y) = 0 \\
			      f_2(x,y) = 0
		      \end{array}\right.$

	      \emph{Особая точка} -- траектория системы (так как является решением).

	      \begin{remark}
		      Траектории могут неограниченно приближаться к особой точке, никогда не входя в нее.
	      \end{remark}

	\item Пусть $\big(x_1(t), y_1(t)\big)$ и $\big(x_2(t),y_2(t)\big)$ -- две траектории, и $\exists t_1,t_2$:
	      \[
		      \left\{\begin{array}{l}
			      x_1(t_1) = x_2(t_2) \\
			      y_1(t_1) = y_2(t_2)
		      \end{array}\right.,
	      \]
	      тогда эти траектории совпадают.

	\item Если $\exists t_1,t_2$:
	      \[
		      \left\{\begin{array}{l}
			      x(t_1) = x(t_2) \\
			      y(t_1) = y(t_2)
		      \end{array}\right.,
	      \]
	      то траектория $\big(x(t),y(t)\big)$ -- замкнутая кривая или периодическая.
\end{enumerate}

\begin{theorem}
	Траектории автономной системы либо точка, либо период. кривая, либо кривая без самопересечений.
\end{theorem}
