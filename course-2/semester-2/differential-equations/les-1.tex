\chapter{Системы дифференциальных уравнений 1-го порядка}

\lesson{7}{от 20 фев 2024 10:30}{Начало}

\begin{note}
    Рассмотрим систему 1-го порядка из $ m $ уравнений с $ n $ неизвестными:
    \begin{equation}\label{eq:3.0.1}
        \left\{\begin{array}{l}
            F_1(t,x_1,\ldots,x_n,\dot{x_1},\ldots,\dot{x_n}) = 0 \\
            \vdots                                               \\
            F_m(t,x_1,\ldots,x_n,\dot{x_1},\ldots,\dot{x_n}) = 0
        \end{array}\right.
    \end{equation}

    Далее, $ m = n $.

    \begin{center}
        \begin{tabular}{ c|c }
            1 уравнение & система                                                               \\
            \hline
            $ x $       & $ t $ (нез. переменная)                                               \\
            $ y $       & $ \overline{x} = (x_1,\ldots,x_n)^T $                                 \\
            $ | y |  $  & $ \|\overline{x}\| = \left(\sum_{i=1}^{n}x_i^2\right)^{\frac{1}{2}} $ \\
        \end{tabular}
    \end{center}
\end{note}

\begin{definition}[Нормальная система ДУ 1-го порядка]
    Система ДУ 1-го порядка называется \emph{нормальной}, если она имеет вид:
    \begin{equation}\label{eq:3.0.2}
        \left\{\begin{array}{l}
            \dot{x_1} = f_1(t,x_1,\ldots,x_n) \\
            \vdots                            \\
            \dot{x_n} = f_n(t,x_1,\ldots,x_n)
        \end{array}\right.
    \end{equation}
    \begin{center}
        или
    \end{center}
    \begin{equation}\label{eq:3.0.3}
        \dot{\overline{x}} = \overline{f}(t,\overline{x})\text{, где }\dot{\overline{x}} = \left(\begin{matrix}
                \frac{dx_1}{dt} \\ \vdots \\ \frac{dx_n}{dt}
            \end{matrix}\right)
    \end{equation}
\end{definition}

\begin{definition}[Решение системы ДУ 1-го порядка]
    \emph{Решением системы ДУ 1-го порядка} $ \overline{x}(t) = \phi(t) = \left(\begin{matrix}
                \phi_1(t) \\ \vdots \\ \phi_n(t)
            \end{matrix}\right) $ называется набор дифференцируемых функций, обращающих уравнения системы в тождества.
\end{definition}

\begin{definition}[Задача Коши для системы ДУ 1-го порядка]
    \emph{Задачей Коши для системы ДУ 1-го порядка} называется задача отыскания решения системы \ref{eq:3.0.2} и \ref{eq:3.0.3}, удовлетворяющего начальным условиям:
    \begin{equation}\label{eq:3.0.4}
        \left\{\begin{array}{l}
            x_1(t_0) = x_1^0 \\
            \vdots           \\
            x_n(t_0) = x_n^0
        \end{array}\right.
    \end{equation}
    \begin{center}
        или
    \end{center}
    \begin{equation}\label{eq:3.0.5}
        \overline{x}(t_0)=\overline{x}^0.
    \end{equation}
\end{definition}

\begin{theorem}[$ \exists $ и $ ! $ решения задачи Коши для системы]
    Пусть $ \overline{f}(t,\overline{x}) $ определена и непрерывна в области $ D \subset \R^{n+1} $ и \emph{удовлетворяет условию Липница по переменным $ x_1,\ldots,x_n $} или более сильное условие: частные производные $ \frac{\partial f_i}{\partial x_j} $ непрерывны в области $ D, \ i,j = \overline{1,n} $.

    Тогда решение задачи \ref{eq:3.0.2},\ref{eq:3.0.4} или \ref{eq:3.0.3},\ref{eq:3.0.5} существует и единствунно в интервале $ (t_0 - h;t_0+h) $, где $ h = \frac{r}{\sqrt{M^2 + 1}} $, где $ r $ -- радиус шара $ B_r $ с центром $ (t_0,\overline{x}^0) $, целиком лежищим в области $ D $,
    \[
        M = \underset{B_r}{\sup}\|\overline{f}(t,\overline{x})\|.
    \]
\end{theorem}

\begin{definition}[Линейная система ДУ 1-го порядка]
    Система ДУ 1-го порядка называется \emph{линейной}, если она имеет вид:
    \begin{equation}\label{eq:3.0.6}
        \left\{\begin{array}{l}
            \dot{x_1} = a_{11}(t)\cdot x_1 + \ldots + a_{1n}(t)\cdot x_n + g_1(t) \\
            \vdots                                                                \\
            \dot{x_n} = a_{n1}(t)\cdot x_1 + \ldots + a_{nn}(t)\cdot x_n + g_n(t)
        \end{array}\right.
    \end{equation}
    \begin{center}
        или
    \end{center}
    \begin{equation}\label{eq:3.0.7}
        \dot{\overline{x}} = A(t) \cdot \overline{x} + \overline{g}(t).
    \end{equation}

    \[
        A(t) = \left(\begin{matrix}
                a_{11}(t) & \cdots & a_{1n}(t) \\
                \vdots    & \ddots & \vdots    \\
                a_{n1}(t) & \cdots & a_{nn}(t)
            \end{matrix}\right), \quad \overline{g}(t) = \left(\begin{matrix}
                g_1(t) \\ \vdots \\ g_n(t)
            \end{matrix}\right).
    \]
\end{definition}

\newpage

\begin{definition}[Однородные системы \ref{eq:3.0.6},\ref{eq:3.0.7}]
    Система \ref{eq:3.0.6} или \ref{eq:3.0.7} называется \emph{однородной}, если $ \overline{g}(t) = 0 $, то есть:
    \begin{equation}\label{eq:3.0.8}
        \dot{\overline{x}} = A(t) \cdot \overline{x}.
    \end{equation}

    Если матрица $ A(t) $ имеет постоянные элементы, то $ A(t) = A $.
\end{definition}

\begin{note}[$ (\alpha,\beta) $]
    \[
        -\infty \leqslant \alpha < \beta \leqslant + \infty.
    \]
\end{note}

\begin{theorem}[О продолжаемости решения системы на интервал]
    Пусть $ a_{ij}(t) $ и $ g_j(t) $ непрерывны на $ (\alpha,\beta), \ i,j = \overline{1,n} $.

    Тогда решение задачи Коши \ref{eq:3.0.5},\ref{eq:3.0.7} существует, единственно и продолжаемо на $ (\alpha,\beta) $.
\end{theorem}

\begin{definition}[Линейно зависимая система функций]
    Система функций $ \overline{x}^1,\ldots,\overline{x}^n $ называется \emph{линейно зависимой} на $ (\alpha,\beta) $, если сущесвует набор действительных чисел $ c_1,\ldots,c_n $, на всех равных нулю, такой, что:
    \begin{equation}\label{eq:3.0.9}
        c_1 \overline{x}^1 + \ldots + c_n \overline{x}^n = 0 \quad \text{на }(\alpha,\beta).
    \end{equation}
\end{definition}

\begin{definition}[Линейно независимая система функций]
    Если в равенстве \ref{eq:3.0.9} $ c_1 = c_2 = \ldots = c_n = 0 $, то система функций $ \overline{x}^1,\ldots,\overline{x}^n $ -- \emph{линейно независима}.
\end{definition}

\begin{definition}[Фундаментальная система решений (ФСР)]
    Любая система линейно независимых решений \ref{eq:3.0.8} $ \overline{x}^1,\ldots,\overline{x}^n $ называется \emph{фундаментальной системой решений (ФСР)}.
\end{definition}

\begin{definition}[Фундаментальная матрица]
    Матрица, столбцы которой являются ФСР, называется \emph{фундаментальной матрицей системы},
    \[
        \Phi(t) = \left(\begin{matrix}
                x_1^1  & \cdots & x_1^n  \\
                \vdots & \ddots & \vdots \\
                x_n^1  & \cdots & x_n^n
            \end{matrix}\right).
    \]
\end{definition}

\begin{definition}[Определитель Вронского системы]
    \emph{Определителем Вронского системы} называется определитель фундаментальной матрицы,
    \[
        W(t) = \det \Phi(t).
    \]
\end{definition}

\begin{theorem}
    Если система функций $ \overline{x}^1,\ldots,\overline{x}^n $ линейно зависима, то $ W(t) = 0 $.
\end{theorem}

\begin{corollary}
    Если $ W_{\overline{x}^1,\ldots,\overline{x}^n}(t)\ne 0 $, то $ \overline{x}^1,\ldots,\overline{x}^n $ -- линейно независимая система функций.
\end{corollary}

\begin{theorem}
    Пусть $ \exists t_0 \in (\alpha,\beta): \ W(t_0) = 0 $ и $ a_{ij}(t) $ из \ref{eq:3.0.8} непрерывны на $ (\alpha,\beta) $.

    Тогда $ W(t) = 0 $ на $ (\alpha,\beta) $ и $ \overline{x}^1,\ldots,\overline{x}^n $ линейно зависимые на $ (\alpha,\beta) $.
\end{theorem}

\begin{proof}
    Рассмотрим $ \dot{\overline{x}} = A \overline{x} $,
    \[
        \boxed{\begin{array}{l}
                \overline{x}(t)\equiv 0 \text{ -- решение} \\
                \overline{x}(t_0) = 0
            \end{array}}
    \]

    Пусть $ W(t_0) = 0 \implies \exists c_1,\ldots,c_n :$
    \begin{multline*}
        c_1 \overline{x}^1(t_0) + \ldots + c_n \overline{x}^n(t_0) = 0 \implies \\
        \implies \boxed{\left\{\begin{array}{l}
                \overline{x}(t) \equiv c_1 \overline{x}^1(t) + \ldots + c_n \overline{x}^n(t) \\
                \overline{x}(t_0) = 0
            \end{array}\right.} \implies \overline{x}(t) = c_1 \overline{x}^1(t) + \ldots + c_n x^n(t) = 0 \implies \\
        \implies \overline{x}^1,\ldots,\overline{x}^n \text{ -- линейно зависимая} \implies W(t) = 0.
    \end{multline*}
\end{proof}

\begin{theorem}[Формула Лиувилля-Остроградского]
    Определитель Вронского для матрицы, составленный из решений \ref{eq:3.0.8}, находится по Л-О:
    \[
        W(t) = W(t_0)e^{\int_{t_0}^{t}Tr A(s)dx},
    \]
    где $ trace $ -- след, $ Tr A(t) = a_{11}(t) + \ldots + a_{nn}(t) $.
\end{theorem}

\begin{proof}
    \[
        \dot{W(t)} = \sum_{i=1}^{n}\left|\begin{matrix}
            x_1^1     & \cdots & x_1^n     \\
            \vdots    & \ddots & \vdots    \\
            x_{i-1}^1 & \cdots & x_{i-1}^n \\
            x_i^1     & \cdots & x_i^n     \\
            x_{i+1}^1 & \cdots & x_{i+1}^n \\
            \vdots    & \ddots & \vdots    \\
            x_n^1     & \cdots & x_n^n
        \end{matrix}\right| \circled{=}
    \]

    Для произвольного $ x^i $:
    \[
        \left(\begin{matrix}
                \dot{x}_1^j \\ \vdots \\ \dot{x}_i^j \\ \vdots \dot{x}_n^j
            \end{matrix}\right) = \left(\begin{matrix}
                a_{11}(t) & \cdots & a_{1n}(t) \\
                \vdots    & \ddots & \vdots    \\
                a_{i1}(t) & \cdots & a_{in}(t) \\
                \vdots    & \ddots & \vdots    \\
                a_{n1}(t) & \cdots & a_{nn}(t)
            \end{matrix}\right)\left(\begin{matrix}
                x_1^j \\ \vdots \\ x_i^j \\ \vdots \\ x_n^j
            \end{matrix}\right),
    \]
    \[
        \dot{x}_i^j = a_{i1}(t)x_1^j + a_{i2}(t)x_2^j + \ldots + a_{in}(t)x_n^j = \sum_{k=1}^{n}a_{ik}x_k^j, \quad i = \overline{1,n}, \ j = \overline{1,n}.
    \]

    \[
        \circled{=} \sum_{i=1}^{n}\left|\begin{matrix}
            x_1^1                     & \cdots & x_1^n                     \\
            \vdots                    & \ddots & \vdots                    \\
            \sum_{k=1}^{n}a_{ik}x_k^1 & \cdots & \sum_{k=1}^{n}a_{ik}x_k^n \\
            \vdots                    & \ddots & \vdots                    \\
            x_n^1                     & \cdots & x_n^n
        \end{matrix}\right| = \left|\begin{matrix}
            a_{ii}x_i^1 & a_{ii}x_i^2 & \cdots & a_{ii}x_i^n
        \end{matrix}\right|,
    \]
    \[
        \sum_{k=1}^{n}a_{ik}x_k^1 - a_{i1}x_1^1 - a_{i2}x_2^1 - \ldots - a_{ii-1}x_{i-1}^1 - a_{ii+1}x_{i+1}^1 - \ldots - a_{in}x_n^1 = \circled{$ a_{ii}x_i^1 $}.
    \]
\end{proof}