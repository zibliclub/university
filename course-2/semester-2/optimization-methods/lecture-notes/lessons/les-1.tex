\chapter*{Введение}

\lesson{1}{от 9 фев 2024 8:45}{Начало}

\emph{Методы оптимизации} -- раздел прикладной математики, предметом изучения которого является теория и методы оптимизации.

\begin{definition}[Оптимизационная задача]
	\emph{Оптимизационная задача} -- задача выбора из множества возможных вариантов, наилучших в некотором смысле,
	\[
		\left\{\begin{array}{l}
			f(x) \rightarrow \min \ (\max) \\
			x \in D
		\end{array}\right.,
	\]
	где
	\[
		\begin{array}{rl}
			D       & \text{-- множество допустимых решений},           \\
			x \in D & \text{-- допустимое решение},                     \\
			f(x)    & \text{-- целевая функция (критерий оптимизации)}.
		\end{array}
	\]
\end{definition}

\section*{Задачи математического программирования (МП) и их классификация}

$\begin{array}{ll}
		\text{1939 г.}  & \text{Л.В. Конторович}                                        \\
		\text{1947 г.}  & \text{Д. Данциг}                                              \\
		\text{с 50 гг.} & \text{бурное развитие}                                        \\
		\text{1975 г.}  & \text{Нобелевская премия по экономике Конторовичу и Купмаксу}
	\end{array}$

\begin{definition}[Задача математического программирования]
	\[
		\begin{array}{ll}
			(1) & f(x) \rightarrow \max (\min)                                                     \\
			(2) & g_i(x) \# 0, \ i = \overline{1,m}, \ \# \in \{\leqslant ,\geqslant ,=\}          \\
			(3) & x_j \in \R, \ j = \overline{1,n} \ \big(x \in \R^n, \ x = (x_1,\ldots ,x_n)\big)
		\end{array}
	\]

	Множество точек $x$, удовлетворяющих условиям $(2)$--$(3)$, называется \emph{множеством $D$ доп. решений}.
\end{definition}

\begin{definition}[Оптимальное решение]
	$x^* \in D$ называется \emph{оптимальным решением} задачи $(1)$--$(3)$, если $\forall x \in D \ f(x^*)\geqslant f(x)$ для задачи на $\max$ и $\forall x \in D \ f(x^*)\leqslant f(x)$ для задачи на $\min$.

	$x^*$ является \emph{глобальным экстремумом}.
\end{definition}

\begin{definition}[Разрешимая задача]
	Задача $(1)$--$(3)$, которая обладает оптимальным решением, называется \emph{разрешимой}, и \emph{неразрешимой} в противном случае.

	$D = \R^n$ -- задача \emph{безусловной оптимизации}, в противном случае -- \emph{задача условной оптимизации}.
\end{definition}

\begin{note}[Классификация]\leavevmode
	\begin{enumerate}
		\item Если $f,g_i$ являются линейными, то задача является задачей \emph{линейного программирования (ЛП)}.
		\item Если хотя бы одна из функций $f$ и $g_i$ нелинейная, то задача \emph{нелинейного програмирования}.
		\item $f,g_i$ -- выпуклые, то \emph{выпукголого программирования}.
	\end{enumerate}
\end{note}

\chapter{Линейное программирование}

\section{Постановка задачи, теорема эквивалентности}

\begin{definition}[Общая задача ЛП (ЗЛП)]
	\[
		\begin{array}{ll}
			f(x) = & c_0 + \sum_{j=1}^{n} c_jx_j \rightarrow \max \ (\min)                                         \\
			       & \sum_{j=1}^{n} a_{ij} x_j \# b_i, \ i = \overline{1,m} , \ \# \in \{\leqslant ,\geqslant ,=\} \\
			       & x_j \geqslant 0, \ j \in \mathcal{J} \leqslant \{1,\ldots ,n\}
		\end{array}
	\]

	\[
		\underbrace{A = \left(\begin{array}{cccc}
				a_{11} & a_{12} & \cdots & a_{1n} \\
				\vdots & \vdots & \ddots & \vdots \\
				a_{m1} & a_{m2} & \cdots & a_{mn}
			\end{array}\right)^C \quad b = \left(\begin{array}{c}
				b_1 \\ \vdots \\ b_m
			\end{array}\right)}_{\text{дано}} \quad x = \left(\begin{array}{c}
				x_1 \\ x_2 \\ \vdots \\ x_n
			\end{array}\right)
	\]
\end{definition}

\begin{remark}
	Б.о.о. далее полагаем $c_0 = 0$, так как добавление константы не влияет на процесс нахожденя оптимального решения.
\end{remark}

\begin{note}[Матричная задача]
	\[
		\begin{array}{ll}
			f(x) = & (c,x) \rightarrow \max \ (\min)                                \\
			       & Ax \# b                                                        \\
			       & x_j \geqslant 0, \ j \in \mathcal{J} \subseteq \{1,\ldots ,n\}
		\end{array}
	\]
\end{note}

\begin{note}[Каноническая ЗЛП (КЗЛП)]
	\[
		\begin{array}{ll}
			f(x) = & (c,x) \rightarrow \max                                             \\
			       & Ax = b                                                             \\
			       & x \geqslant \overline{0} \ \big(\overline{0}  = (0,\ldots ,0)\big)
		\end{array}
	\]
\end{note}

\begin{note}[Симметричная ЗЛП]
	\[
		\begin{array}{ll}
			f(x) = & (c,x) \rightarrow \max   \\
			       & Ax < b                   \\
			       & x \geqslant \overline{0}
		\end{array} \quad \text{или} \quad \begin{array}{ll}
			f(x) = & (c,x) \rightarrow \min   \\
			       & Ax \geqslant b           \\
			       & x \geqslant \overline{0}
		\end{array}
	\]
\end{note}

\subsection*{Примеры моделей ЛП}

\begin{eg}
	Задача о составлении оптимального плана пространства

	\[
		\begin{array}{cll}
			m & \text{ресурсов}        & i = \overline{1,m} \\
			n & \text{видов продукции} & j = \overline{1,n}
		\end{array}
	\]

	Известно: $\begin{array}{l}
			b_i \text{ -- запас }i \text{-го ресурса, }i = \overline{1,m}                  \\
			a_{ij} \text{ -- кол-во рес. }i \text{, требуемое для пр-ва 1ед. прод. вида }j \\
			c_j \text{ -- прибыль от продажи 1ед. }j \text{го продукта}
		\end{array}$

	Необходимо составить план производства, максимизирующий суммарную прибыль.

	Переменные: $x_j$ ед. продукции вида $j$ производства ($j = \overline{1,n} $),
	\[
		\begin{array}{l}
			\sum_{j=1}^{n} c_jx_j \rightarrow \max                        \\
			\sum_{j=1}^{n} a_{ij} x_j \leqslant b_i, \ i = \overline{1,m} \\
			x_j \geqslant 0, \ j = \overline{1,n}
		\end{array}
	\]
\end{eg}

\begin{eg}
	О максимальном потоке

	\[
		\begin{array}{l}
			G = (V,E) \text{ ориент. взвешенный}                              \\
			c: E \rightarrow \R \text{ -- веса дуг -- пропускная способность} \\
			s \text{ -- источник}                                             \\
			t \text{ -- сток}
		\end{array}
	\]

	Пусть $x_{ij} $ -- поток по дуге $(i,j) \in E$,
	\[
		\left\{\begin{array}{ll}
			f = & \sum_{j:(s,j) \in E} x_{sj} \rightarrow \max                                          \\
			    & \sum_{j:(j,i)\in E} x_{ji} = \sum_{k:(i,k)\in E} x_{ik} , \ i \in V \setminus \{s,t\} \\
			    & 0 \leqslant x_{ij} \leqslant c_{ij} , \ (i,j) \in E
		\end{array}\right.
	\]
\end{eg}

\newpage

\subsection*{Теорема эквивалентности задач ЛП}

\begin{definition}[Эквивалентные задачи]
	Две задачи МП
	\[
		\overset{(\RomanNumeralCaps{1})}{\left\{\begin{array}{l}
				f(x) \rightarrow opt \\
				x \in D
			\end{array}\right.} \quad \overset{(\overline{\RomanNumeralCaps{1}})}{\left\{\begin{array}{l}
				\overline{f} (\overline{x} ) \rightarrow \overline{opt} \\
				\overline{x} \rightarrow \overline{D}
			\end{array}\right.} \qquad \begin{array}{l}
			D \xrightarrow[]{\phi } \overline{D} \\
			\overline{D} \xrightarrow[]{\overline{\phi } } D
		\end{array}
	\]
	называются \emph{эквивалентными}, если любому допустимому решению каждой из них по некоторому правилу соответсвует допустимое решение другой задачи, причем оптимальному решению соответсвует оптимальное.
\end{definition}

\begin{theorem}[Первая теорема эквивалентности]
	Для любой задачи ЛП $\exists $ эквивалентная ей каноническая ЗЛП.
\end{theorem}

\begin{note}[Идея доказательства]
	$n=2, \ m = 3$

	\[
		\begin{array}{l}
			f = c_1x_1 + c_2x_2 \rightarrow \min   \\
			a_{11} x_1 + a_{12} x_2 = b_1          \\
			a_{21} x_1 + a_{22} x_2 \leqslant b_2  \\
			a_{31} x_1 + a_{32}  x_2 \geqslant b_3 \\
			x_1 \geqslant 0                        \\
			x_2 \in \R
		\end{array} \qquad \begin{array}{l}
			\overline{f} = -c_1x_1 - c_2x_2 \rightarrow \max \\
			a_{11} x_1 + a_{12} x_2 = b_1                    \\
			a_{21} x_1 + a_{22} x_2 + x_3 = b_2              \\
			a_{31} x_1 + a_{32} x_2 - x_4 = b_3              \\
			x_1,x_3,x_4 \geqslant 0                          \\
			x_2 = x_{2}' - x_{2}'', \ x_{2}' \geqslant 0, \ x_{2}'' \geqslant 0
		\end{array}
	\]

	\[
		\text{КЗЛП} \qquad \begin{array}{l}
			\overline{f}  = -c_1x_1 - c_2x_2 ' + c_2 x_2 '' \rightarrow \max \\
			a_{11} x_1 + a_{12} x_2 ' - a_{12} x_2 '' = b_1                  \\
			a_{21} x_1 + a_{22} x_2 ' - a_{22} x_2 '' + x_3 = b_2            \\
			a_{31} x_1 + a_{32} x_2 ' - a_{32} x_2 '' - x_4 = b_3            \\
			x_1,x_2 ', x_2 '', x_3,x_4 \geqslant 0
		\end{array}
	\]

	Неоднозначность -- разность, $\forall x \in D \ f(x) = -f(\overline{x} ), \ \overline{x}  \in \overline{D} $
	\[
		\overline{x} = \phi (x).
	\]

	Очевидно, что оптимальность также сохраняется при таких преобразованиях.
\end{note}

\begin{theorem}[Вторая теорема эквивалентности]
	Для любой задачи ЛП $\exists $ эквивалентная ей симметричная задача ЛП.
\end{theorem}

\begin{note}[Идея доказательства]
	\[
		\alpha = \beta \iff \left\{\begin{array}{l}
			\alpha \leqslant \beta \\
			\alpha \geqslant \beta
		\end{array}\right. \qquad \begin{array}{l}
			(c,x)\rightarrow \max \\
			Ax \leqslant b        \\
			x \geqslant 0
		\end{array} \Bigg| \begin{array}{l}
			(c,x) \rightarrow \min \\
			Ax \geqslant b         \\
			x \geqslant 0
		\end{array}
	\]
\end{note}

\begin{remark}
	Смысл теоремы 1 в том, чтобы свести решение ЗЛП к КЗЛП.
\end{remark}

\begin{note}[Геометрическая интерпретация]
	$n=2$,
	\[
		\begin{array}{l}
			f = c_1x_1 + c_2x_2 \rightarrow \max \\
			a_{i1} x_1 + a_{i2} x_2 \leqslant b_i, \ i = \overline{1,m}
		\end{array}
	\]

	\begin{figure}[H]
		\centering
		\incfig[0.7]{fig-1}
		\label{fig:fig-1}
	\end{figure}

	Линии уровня целевой функции
	\[
		\begin{array}{l}
			c_1x_1 + c_2x_2 = const \\
			\perp \nabla f = (c_1,c_2)
		\end{array}
	\]
	\[
		\begin{array}{ccl}
			                     &          & \exists ! x^* \text{ -- опт. решение} \\
			                     & \nearrow &                                       \\
			\text{ЗЛП разрешима} &          &                                       \\
			                     & \searrow &                                       \\
			                     &          & \text{беск. много опт. реш.}
		\end{array}
	\]
	\[
		\begin{array}{ccl}
			                       &          & f \rightarrow +\infty \text{ на мн-ве }D \text{ (ф-я неогр. сверху на }D \text{)} \\
			                       & \nearrow &                                                                                   \\
			\text{ЗЛП неразрешима} &          &                                                                                   \\
			                       & \searrow &                                                                                   \\
			                       &          & D = \varnothing \text{ нет доп. реш.}
		\end{array}
	\]
\end{note}

\section{Базисные решения КЗЛП}

\[
	\text{КЗЛП} \quad \begin{array}{l}
		(1) \ f =  (c,x) \rightarrow \max \\
		\left.\begin{array}{ll}
			      (2) & Ax = b                   \\
			      (3) & x \geqslant \overline{0}
		      \end{array}\right\} D
	\end{array}
\]

\[
	A_{m \times n} = (A^1,A^2,\ldots ,A^n) \quad A^j = \left(\begin{array}{c}
			a_{1j} \\ a_{2j} \\ \vdots \\ a_{mj}
		\end{array}\right) \text{ -- }j \text{-ый столбец матрицы }A
\]

\begin{definition}[Базисное решение системы $(2)$]
	Пусть $\overline{x} $ -- решение системы $(2)$. Вектор $\overline{x}$ называется \emph{базисным решением системы $(2)$}, если система векторных столбцов матрицы $A$, соответствующая ненулевым компонентам вектора $\overline{x} $, линейно независима.
\end{definition}

\begin{remark}
	В случае однородной системы ($b=0$), решение $x=0$ является базисным.
\end{remark}

\begin{definition}[Базисное решение КЗЛП]
	Неотрицательное базисное решение системы $(2)$ называется \emph{базисным (опорным) решением КЗЛП}.
\end{definition}

\begin{eg}
	\[
		\begin{array}{l}
			3x_1 - 4x_2 + x_3 \rightarrow \max \\
			\left\{\begin{array}{l}
				       2x_1 + 2x_2 + 3x_3 - x_4 + x_5 = 1 \\
				       2x_1 + 4x_2 + x_4 + 2x_5 = 2       \\
				       x_j \geqslant 0, \ j = \overline{1,5}
			       \end{array}\right.
		\end{array}
	\]
	\[
		A = \left(\begin{array}{ccccc}
				2 & 2 & \textbf{3} & \textbf{-1} & 1 \\
				2 & 4 & \textbf{0} & \textbf{1}  & 2
			\end{array}\right)
	\]

	$x^1 = (0,0,1,2,0)$ -- базисное решение системы, так как $\left|\begin{array}{cc}
			3 & -1 \\ 0 & 1
		\end{array}\right| \ne 0$ соответствует базису $\{A^3,A^4\}$.

	\[
		\begin{array}{ll}
			x^1   & \text{и БР КЗЛП}                                               \\
			x^2 = & (1,0,\textbf{$-\frac{1}{3}$ },0,0) \text{ БР СЛАУ, но не КЗЛП} \\
			x^3 = & (0,0,0,0,1) \text{ БР КЗЛП}
		\end{array}
	\]
\end{eg}

\begin{definition}[Вырожденное решение]
	$x$ -- базисное решение КЗЛП называется \emph{вырожденным}, если число ненулевых компонент вектора $x$ меньше ранга матрицы $A$.

\end{definition}

\begin{note}
	$x^3$ -- вырожденное, недост.: соответствует разным наборам баз. столбцов матрицы.

	$x^3$ соответствует $\{A_1,A_5\}, \{A_3,A_5\}, \{A_4,A_5\}$.
\end{note}
