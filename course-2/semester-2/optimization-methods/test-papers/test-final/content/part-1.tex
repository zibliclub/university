\section{Задача линейного программирования (ЛП)}

\begin{definition}
	\emph{Задачей линейного программирования (ЛП)} называется задача поиска $\min/\max$ линейной функции на множестве, описываемом линейными ограничениями.
\end{definition}

\section{Общая задача ЛП}

\begin{definition}
	\emph{Общая задача ЛП} имеет вид:
	\begin{equation}\label{eq:1.1}
		f(x)=c_{0}+\sum_{j=1}^{n}c_jx_j \rightarrow \max \ (\min)
	\end{equation}
	\begin{equation}\label{eq:1.2}
		\sum_{j=1}^{n}a_{ij}x_{j} \leqslant b_{i}, \quad i = \overline{1,k},
	\end{equation}
	\begin{equation}\label{eq:1.3}
		\sum_{j=1}^{n}a_{ij}x_{j} \geqslant b_{i}, \quad i = k + \overline{1,l},
	\end{equation}
	\begin{equation}\label{eq:1.4}
		\sum_{j=1}^{n}a_{ij}x_{j} \geqslant b_{i}, \quad i = l + \overline{1,m},
	\end{equation}
	\begin{equation}\label{eq:1.5}
		x_{j} \geqslant 0, \quad j \in J \subseteq \{1,\ldots,n\},
	\end{equation}
	где $x = (x_1,\ldots,x_n)\in R^n$ -- вектор переменных. Функция $f(x)$ называется \emph{целевой}, а условия (\ref{eq:1.2})-(\ref{eq:1.5}) -- \emph{ограничениями задачи}, причем в одной задаче ЛП не обязаны присутствовать ограничения всех трех типов.
\end{definition}

\section{Допустимое решение задачи ЛП}

\begin{definition}
	Вектор $x \in R^{n}$, удовлетворяющий ограничениям задачи, называется \emph{допустимым решением задачи ЛП}.

	Множество всех допустимых решений будем обозначать через $\mathfrak{D}$.
\end{definition}

\section{Оптимальное решение задачи ЛП}

\begin{definition}
	Вектор $x^{*} \in \mathfrak{D}$ называется \emph{оптимальным решением} задачи ЛП, если $\forall x \in \mathfrak{D} \ f(x^{*}) \geqslant f(x)$ в задаче максимизации или $f(x^{*}) \leqslant f(x)$ в задаче минимизации.
\end{definition}

\newpage

\section{Разрешимая задача ЛП}

\begin{definition}
	Задача (\ref{eq:1.1})-(\ref{eq:1.5}) называется \emph{разрешимой}, если она имеет оптимальное решение, иначе -- \emph{неразрешимой}.
\end{definition}

\section{Неразрешимая задача ЛП}

\begin{definition}
	Задача (\ref{eq:1.1})-(\ref{eq:1.5}) называется \emph{разрешимой}, если она имеет оптимальное решение, иначе -- \emph{неразрешимой}.
\end{definition}

\section{Каноническая задача ЛП (КЗЛП)}

\begin{theorem}
	Для любой задачи ЛП существует эквивалентная ей \emph{каноническая} задача ЛП.
\end{theorem}

\section{Симметричная задача ЛП}

\begin{theorem}
	Для любой задачи ЛП существует эквивалентная ей \emph{стандартная (симметричная)} задача ЛП.
\end{theorem}

\section{Эквивалентность задач ЛП}

\begin{definition}
	Две задачи ЛП $P_1$ и $P_2$ называются \emph{эквивалентными}, если любому допустимому решению задачи $P_1$ соответствует некоторое допустимое решение задачи $P_2$ и наоборот; причем оптимальному решению одной задачи соответствует некоторое оптимальное решение другой задачи.
\end{definition}

\section{Теорема 1 эквивалентности}

\begin{theorem}
	Для любой задачи ЛП существует эквивалентная ей \emph{каноническая} задача ЛП.
\end{theorem}
