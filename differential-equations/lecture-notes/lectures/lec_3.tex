\lesson{3}{от 1 нояб 2023 10:30}{Продолжение}


\section{Интегрирующий множитель}

\begin{note}
    $\equalto{ydx}{P} - \equalto{xdy}{Q} = 0 \ \Big| \ \cdot \frac{1}{y^2}$
    \[
        \begin{array}{ccl}
            \frac{\delta P}{\delta y} & = & y_y' = 1                         \\
            \vertneq                  &   &                                  \\
            \frac{\delta Q}{\delta x} & = & \frac{\delta}{\delta x}(-x) = -1
        \end{array}
    \]
    \[
        \frac{ydx - xdy}{y^2} = 0 \implies d\left(\frac{x}{y}\right) = 0
    \]
    \[
        \frac{x}{y} = C, \quad y = 0
    \]
\end{note}

\begin{definition}[Интегрирующий множитель]
    Пусть
    \begin{equation}\label{eq18}
        M(x,y)dx + N(x,y)dy = 0
    \end{equation}
    не является уравнением в полных дифференциалах, $M,N \in C^2(D), \ D$ -- односвязная область в $\mathbb{R}^2$.

    $M(x,y)$ называется \emph{интегрирующим множителем} уравнения \ref{eq18}, если
    \[
        \mu(x,y)M(x,y)dx + \mu(x,y)N(x,y)dy
    \]
    является полным дифференциалом некоторой функции $\frac{\delta P}{\delta y} = \frac{\delta Q}{\delta x}$.
\end{definition}

\begin{note}
    $\RomanNumeralCaps{1}. \ \mu(x,y) \in C^2(D)$
    \[
        \mu(x,y)\cdot M(x,y) = P(x,y), \quad \mu(x,y)\cdot N(x,y) = Q(x,y)
    \]
    \[
        \frac{\delta \mu(x,y)}{\delta y}M(x,y) + \mu(x,y)\cdot\frac{\delta M(x,y)}{\delta y} = \frac{\delta\mu(x,y)}{\delta x}N(x,y) + \mu(x,y)\cdot\frac{\delta ?}{\delta x}
    \]
    \begin{enumerate}
        \item $\mu = \mu(x)$.
        \item $\mu = \mu(x)$.
        \item $\mu = \mu(\omega(x,y))$.
    \end{enumerate}
\end{note}

\section{Методы построения интегрирующего множителя}

\begin{note}\leavevmode
    \begin{equation}\label{eq19}
        M(x,y)dx + N(x,y)dy = 0
    \end{equation}
    \[
        M,N \in C^2(D), \quad \frac{\delta N}{\delta x} \ne \frac{\delta M}{\delta y}, \quad \mu(x,y)\in C^1(D):
    \]
    \[
        \equalto{\mu(x,y)\cdot M(x,y)dx}{P(x,y)} + \equalto{\mu(x,y) \cdot N(x,y)dy}{Q(x,y)} = 0\text{ -- уравнение в ПД?}
    \]
    \[
        \frac{\delta Q}{\delta x} = \frac{\delta P}{\delta y}
    \]
    \begin{multline*}
        \frac{\delta \mu(x,y)}{\delta x} \cdot N(x,y) + \mu(x,y)\cdot \frac{\delta N(x,y)}{\delta x} = \\
        = \frac{\delta \mu(x,y)}{\delta y} \cdot M(x,y) + \mu(x,y) \cdot \frac{\delta M(x,y)}{\delta y}
    \end{multline*}
    \begin{enumerate}
        \item \begin{multline*}
                  \mu = \mu(x) \implies \frac{\delta \mu}{\delta y} = 0 \implies \\
                  \implies \mu'(x) \cdot N(x,y) + \mu(x)\cdot \frac{\delta N}{\delta x} = \mu(x)\cdot \frac{\delta M}{\delta y}
              \end{multline*}

              \[
                  \underbrace{\frac{\mu'(x)}{\mu(x)}}_{\text{зависит от }x} = \underbrace{\frac{\frac{\delta M}{\delta y} - \frac{\delta N}{\delta x}}{N}}_{\text{зависит от }x} = F(x)
              \]
              \[
                  \left(M\ne0\text{ и }N\ne0\right)
              \]

              \[
                  \int\frac{\mu'(x)}{\mu(x)}dx = \int F(x)dx
              \]

              \[
                  \ln|\mu(x)| = \ln c + \int F(x)dx, \quad \mu(x) = c\cdot e^{\int F(x)dx} \underset{c = 1}{=} e^{\int F(x)dx}
              \]

        \item \begin{multline*}
                  \mu = \mu(y) \implies \frac{\delta \mu}{\delta x} = 0 \implies \\
                  \implies \mu'(y) \cdot M(x,y) + \mu(y)\cdot \frac{\delta M}{\delta y} = \mu(y)\cdot \frac{\delta N}{\delta x}
              \end{multline*}

              \[
                  \underbrace{\frac{\mu'(y)}{\mu(y)}}_{\text{зависит от }y} = \underbrace{\frac{\frac{\delta N}{\delta x} - \frac{\delta M}{\delta y}}{M}}_{\text{зависит от }y} = F(y)
              \]

              \[
                  \int\frac{\mu'(y)}{\mu(y)}dy = \int F(y)dy
              \]

              \[
                  \ln|\mu(y)| = \ln c + \int F(y)dy, \quad \mu(y) = c\cdot e^{\int F(y)dy} \underset{c = 1}{=} e^{\int F(y)dy}
              \]

        \item \[
                  \mu = \mu\big(\omega(x,y)\big)
              \]

              \[
                  \frac{\delta \mu}{\delta \omega} \cdot \frac{\delta \omega}{\delta x} \cdot N + \mu \cdot \frac{\delta N}{\delta x} = \frac{\delta \mu}{\delta \omega} \cdot \frac{\delta \omega}{\delta y} \cdot M + \mu \cdot \frac{\delta M}{\delta y}
              \]

              \begin{eqnarray*}
                  \frac{\frac{\delta \mu}{\delta \omega}}{\mu(\omega)} = \frac{\frac{\delta M}{\delta y} - \frac{\delta N}{\delta x}}{N\cdot \frac{\delta \omega}{\delta x} - M\cdot \frac{\delta \omega}{\delta y}} = F(\omega) & \implies & \\
                  & \implies & \mu(\omega) = e^{\int F(\omega)d\omega}
              \end{eqnarray*}
    \end{enumerate}
\end{note}

\begin{example}
    \[
        (x^2 + y^2 + x)dx + ydy = 0, \quad M = x^2 + y^2 + x, \quad N = y
    \]
    \[
        \begin{array}{ccl}
            \frac{\delta N}{\delta x} & = & 0  \\
            \vertneq                  &   &    \\
            \frac{\delta M}{\delta y} & = & 2y
        \end{array}
    \]
    \begin{multline*}
        \mu = \mu(x) = ?, \quad \mu(x^2 + y^2 + x)dx + \mu y dy = 0, \\
        P = \mu(x^2 + y^2 + x), \quad Q = \mu y, \\
        \frac{\delta P}{\delta y} = \frac{\delta Q}{\delta x}, \quad \frac{\delta M}{\delta y}(x^2 + y^2 + x) + \mu \cdot 2y = \frac{\delta \mu}{\delta x} \cdot y + \mu \cdot 0, \\
        \frac{\mu'(x)}{M} = \frac{2y}{y} = 2, \quad \mu(x) = e^{2x}, \\
        e^{2x}(x^2 + y^2 + x) dx + e^{2x} \cdot y dy = 0, \\
        P = e^{2x}(x^2 + y^2 + x), \quad Q = e^{2x}\cdot y, \\
        \frac{\delta P}{\delta y} = 2e^{2x} \cdot y = \frac{\delta Q}{\delta x}, \\
        \left\{\begin{array}{l}
            \frac{\delta u}{\delta x} = e^{2x}(x^2 + y^2 + x) \\
            \frac{\delta u}{\delta y} = e^{2x}\cdot y
        \end{array}\right. \implies u(x,y) = e^{2x} \cdot \frac{y^2}{2} + c(x), \\
        u_x' = 2e^{2x} \cdot \frac{y^2}{2} + c'(x) = e^{2x}(x^2 + y^2 + x), \quad c'(x) = e^{2x}(x^2 + x)
    \end{multline*}
    \begin{eqnarray*}
        c(x) = \frac{e^{2x}}{2}(x^2 + x) - \int \frac{e^{2x}}{2}(2x + 1)dx & = & \\
        = \frac{e^{2x}}{2}(x^2 + x) - \frac{e^{2x}}{4}(2x + 1) &+& \int \frac{e^{2x}}{4}\cdot 2 dx = \\
        = \frac{e^{2x}}{2}(x^2 + x) &-& \frac{e^{2x}}{4}(2x+1) + \frac{e^{2x}}{4} + C
    \end{eqnarray*}
    \[
        c(x) = \frac{e^{2x}}{2}\cdot x^2 + C
    \]
    \[
        u(x,y) = e^{2x} \cdot \frac{x^2 + y^2}{2} + C = \widetilde{C}
    \]
    \[
        e^{2x} \cdot \frac{x^2 + y^2}{2} = C\text{ -- общий интеграл}
    \]
\end{example}

\begin{note}[Свойства интегрирующего множителя (ИМ)]\leavevmode
    \begin{enumerate}
        \item Если $\mu_0$ -- ИМ, то $\forall c \in \mathbb{R} \quad \mu_1 = C \cdot \mu_0$ -- тоже является ИМ.
        \item Пусть $\mu_0$ -- ИМ уравнения $(1.12)$, $V_0$ -- соответствующий ему интеграл, то есть:
              \[
                  \mu_0 \cdot Mdx + \mu_0 \cdot N dx = d V_0,
              \]
              тогда для произвольной функции $\phi \in C^1(D), \ \phi \ne 0$, $\mu_1 = \mu_0 \cdot \phi(V_0)$ -- так же является ИМ.
              \begin{multline*}
                  Mdx + Ndy = 0, \quad \mu_1 \cdot Mdx + \mu_1 \cdot N dy = \\
                  = \mu_0 \cdot \phi(V_0)\cdot Mdx + \mu_0 \cdot \phi(V_0)\cdot Ndy = \\
                  = \phi(V_0)(\mu_0 \cdot Mdx + \mu_0 \cdot N dy) = \phi(V_0)dV_0 = \\
                  = d\bigg(\int \phi(V_0)dV_0\bigg) = dV_1, \quad \int \phi(V_0)dV_0 = V_1
              \end{multline*}
        \item Если $\mu_1$ и $\mu_2$ -- интегральные множители уравнения \ref{eq19}, тогда:
              \[
                  \mu_2 = \mu_1 \cdot \phi(V_1),
              \]
              где $\phi$ -- произвольная функция класса $C^1$, $V_1$ -- соответствующий интеграл для $\mu_1$.
    \end{enumerate}
\end{note}

\begin{corollary}
    Если $\mu_1$ и $\mu_2$ -- интегральные множители уравнения \ref{eq19} и $\frac{\mu_1}{\mu_2} \ne const$, тогда $\frac{\mu_1}{\mu_2}$ -- является интегралом для уравнения \ref{eq19}.
\end{corollary}

\begin{theorem}
    Если уравнение 1-го порядка имеет общий интеграл $u(x,y) = C$, то оно имеет интегрирующий множитель.
\end{theorem}

\begin{proof}
    \[
        u(x,y) = C\left\{\begin{array}{l}
            Mdx + Ndy = 0 \\
            du \equiv \frac{\delta u}{\delta x}dx + \frac{\delta u}{\delta y}dy = 0
        \end{array}\right.
    \]
    $(dx,dy)$ -- ненулевое решение если определитель равен 0, то есть
    \[
        \left|\begin{array}{cc}
            M                         & N                         \\
            \frac{\delta u}{\delta x} & \frac{\delta u}{\delta y}
        \end{array}\right| = M\cdot \frac{\delta u}{\delta y} - N\cdot \frac{\delta u}{\delta x} = 0
    \]
    \begin{multline*}
        M\cdot \frac{\delta u}{\delta y} = N\cdot \frac{\delta u}{\delta x} \quad \bigg| \ \cdot (MN) \implies \frac{1}{N}\cdot \frac{\delta u}{\delta y} = \frac{1}{M} \cdot \frac{\delta u}{\delta x} \overset{?}{=} \mu, \\
        \mu \cdot Mdx + \mu \cdot Ndy = 0, \quad \frac{1}{M} \frac{\delta u}{\delta x} \cdot Mdx + \frac{1}{N} \cdot \frac{\delta u}{\delta y} \cdot Ndy = 0, \\
        \frac{\delta u}{\delta x}dx + \frac{\delta u}{\delta y}dy = 0 \implies du = 0
    \end{multline*}
\end{proof}

\begin{note}[Еще один способ построения интегрального множителя]
    \[
        \underbrace{M_1dx + N_1dy}_{\RomanNumeralCaps{1}} + \underbrace{M_2dx + N_2dy}_{\RomanNumeralCaps{2}} = 0
    \]

    Пусть $\mu_1$ -- интегральный множитель для $\RomanNumeralCaps{1}$, $V_1$ -- соответствующий ему интеграл, то есть:
    \[
        dV_1 = \mu_1 \cdot M_1 dx + \mu_2 \cdot N_2 dy,
    \]
    $\mu_2$ -- интегральный множитель для $\RomanNumeralCaps{2}$, $V_2$ -- соответствующий ему интеграл, то есть:
    \[
        dV_2 = \mu_2 \cdot M_2 dx + \mu_2 \cdot N_2 dy,
    \]
    тогда $\exists \phi,\psi \in C^1(D): \quad \mu_1 \cdot \phi(V_1) = \mu_2 \cdot \psi(V_2)$ и $\mu = \mu_1 \cdot \phi (V_1)$ или $\mu = \mu_2 \cdot \psi(V_2)$ -- будет интегральным множителем.
\end{note}

\begin{example}
    \[
        (\frac{y}{x} + 3x^2)dx + (1 + \frac{x^3}{y})dy = 0
    \]
    \[
        (\frac{y}{x} + dy) + (3x^2 dx + \frac{x^3}{y}dy) = 0
    \]
    \begin{minipage}{0.4\textwidth}
        $\frac{y}{x}dx + dy = 0$
        \[
            \mu_1 = x
        \]
        $ydx + xdy = 0$ \\
        $d(xy) = 0 \implies xy = C_1$ \\
        \[
            u_1 = xy
        \]
    \end{minipage}
    \hfill
    \begin{minipage}{0.4\textwidth}
        $3x^2dx + \frac{x^3}{y}dy = 0$
        \[
            \mu_2 = y
        \]
        $3x^2ydx + x^3dy = 0$ \\
        $d(x^3y) = 0$ \\
        \[
            u_2 = x^3y
        \]
    \end{minipage}
    \[
        x\phi(xy) = y \psi(x^3y),\quad \phi(t) = t^2, \ \psi(t) = t,
    \]
    \[
        x(x^2y^2) = yx^3y = \mu
    \]
    \[
        x^3y^2\left(\frac{y}{x}+3x^2\right)dx + x^3y^2\left(1 + \frac{x^2}{y}\right)dy = 0
    \]
    \[
        \underbrace{(x^2y^3 + 3x^5y^2)}_{P}dx + \underbrace{(x^3y^2 + x^6y)}_{Q}dy = 0
    \]
    \[
        \begin{array}{ccc}
            \frac{\delta P}{\delta y} & = & 3x^2y^2 + 6x^5y \\
                                      &   & \verteq         \\
            \frac{\delta Q}{\delta x} & = & 3x^2y^2 + 6x^5y
        \end{array} \implies\text{ уравнение в ПД}
    \]
    \[
        \left\{\begin{array}{l}
            \frac{\delta u}{\delta x} = x^2y^3 + 3x^5y^2 \\
            \frac{\delta u}{\delta y} = x^3y^2 + x^6y
        \end{array}\right.
    \]
\end{example}

\section{Теорема существования и единственности}

\begin{note}[Задача Коши]
    \begin{equation}\label{eq20}
        y' = f(x,y),
    \end{equation}
    \begin{equation}\label{eq21}
        y(x_0) = y_0
    \end{equation}
\end{note}

\begin{theorem}[Теорема Пикара]
    Пусть функция $f(x,y)$ определена и непрерывна по совокупности переменных в прямоугольнике $\Pi = \{(x,y): \ |x-x_0| \leqslant a, \ |y-y_0|\leqslant b\}$ и на переменной $y$ удовлетворяет условию Липшица:
    \[
        \big(f(x,y)\in C(\Pi) \cap Lip y(\Pi)\big)
    \]

    Тогда $\exists!$ решение задачи Коши \ref{eq20}, \ref{eq21} в $V_h = (x_0 - h;x_0 + h)$, где $h = \min\left(a,\frac{b}{M},\frac{1}{L}\right), \ M = \underset{(x,y)\in \Pi}{\max}\big|f(x,y)\big|$.
\end{theorem}

\begin{definition}
    $f(x,y)\in Lip y$, если $\exists L > 0: \ \forall (x,y_1), \ (x,y_2)\in \Pi$ имеет место:
    \[
        \big|f(x,y_1) - f(x,y_2)\big| \leqslant L \cdot |y_1 - y_2|
    \]
\end{definition}

\begin{lemma}[Об интегральном уравнении]
    В предположении теоремы $y$ является решением задачи \ref{eq20}, \ref{eq21} $\iff$ оно является решением интегрального уравнения:
    \begin{equation}\label{eq22}
        y(x) = y_0 + \int_{x_0}^{x}f\big(x,y(s)\big)ds
    \end{equation}
\end{lemma}

\begin{definition}[Решение интегрального уравнения \ref{eq22}]
    \emph{Решением интегрального уравнения \ref{eq22}} называется непрерывная функция $y$, обращающая уравнение в тождество.
\end{definition}

\begin{proof}[Доказательство леммы?]\leavevmode
    \begin{description}
        \item[$ \boxed{\Rightarrow} $] $y$ -- решение \ref{eq20}, \ref{eq21}, проинтегрируем \ref{eq20} на $[x_0;x]$:
              \[
                  \int_{x_0}^{x}y'(x)dx = \int_{x_0}^{x}f\big(s,y(s)\big)ds
              \]
              \[
                  y(x) - y(x_0) = \int_{x_0}^{x}f\big(s,y(s)\big)ds\text{, с учетом \ref{eq21}},
              \]
              \[
                  y(x) = y_0 + \int_{x_0}^{x}f\big(s,y(x)\big)dx,\quad y(x)\text{ удовлетворяет уравнению \ref{eq22}}
              \]

              $y(x)$ непрерывна (следует из дифференцируемости).

        \item[$ \boxed{\Leftarrow} $] $y(x)$ -- решение \ref{eq22}, $y(x)$ -- непрерывна, $f(x,y)$ -- непрерывна по условию теоремы $\implies$ интеграл с переменным верхним пределом можно дифференцировать:
              \[
                  y'(x) = f\big(x,y(x)\big) \cdot 1,\quad y(x)\text{ удовлетворяет уравнению \ref{eq20}}
              \]
              \begin{multline*}
                  \frac{d}{d\alpha}\int_{a(\alpha)}^{b(\alpha)}F(x,\alpha)d\alpha = \\
                  = \int_{a(\alpha)}^{b(\alpha)}F_\alpha'(x,\alpha)d\alpha + F(b(\alpha),\alpha)\cdot b'(\alpha) - F(a(\alpha),\alpha)\cdot a'(\alpha)
              \end{multline*}
              \[
                  y(x_0) = y_0 + \int_{x_0}^{x_0}f\big(s,y(s)\big)ds\text{, выполняется условие \ref{eq21}}
              \]
    \end{description}
\end{proof}

\begin{proof}[Доказательство теоремы Пикара]\leavevmode
    \begin{enumerate}
        \item Последовательность приближения Пикара:
              \[
                  \begin{array}{l}
                      y_0(x) = y_0,                                       \\
                      y_1(x) = y_0 + \int_{x_0}^{x}f\big(s,y_0(s)\big)ds, \\
                      y_2(x) = y_0 + \int_{x_0}^{x}f\big(s,y_1(s)\big)ds, \\
                      \vdots                                              \\
                      y_n(x) = y_0 + \int_{x_0}^{x}f\big(s,y_{n-1}(s)\big)ds
                  \end{array}
              \]
        \item $y_n(x)$ -- непрерывна при $|x-x_0| \leqslant h, \ \big(x,y_n(x)\big)\in \Pi$.

              По индукции, $n=1$:
              \[
                  \big|y_1(x) - y_0\big| \overset{?}{\leqslant}b
              \]
              \begin{multline*}
                  \big|y_1(x) - y_0\big| = \left|\int_{x_0}^{x}f\big(s,y_0(s)\big)ds\right| \leqslant \\
                  \leqslant \left|\int_{x_0}^{x}\Big|f\big(s,y_0(s)\big)\Big|dx\right| \leqslant M|x - x_0| \leqslant M\cdot h \leqslant M\cdot \frac{b}{M} = b
              \end{multline*}

              Пусть $\big|y_{n-1}(x) - y_0\big|\leqslant b$, то есть $\big(x,y_{n-1}(x)\big)\in \Pi$. Докажем, что $\big(x,y_n(x)\big)\in\Pi$:
              \begin{multline*}
                  \big|y_n(x) - y_0\big| \leqslant \left|\int_{x_0}^{x}\Big|f\big(s,y_{n_1}(s)\big)\Big|\right|\leqslant \\
                  \leqslant M\cdot |x - x_0| \leqslant M\cdot h \leqslant M\cdot \frac{b}{M} = b
              \end{multline*}
        \item Покажем, что $\big\{y_n(x)\big\}^\infty_{n=1} \rightrightarrows \overline{y}(x)$.

              Составим функциональный ряд:
              \[
                  (\star) \quad \underbrace{\underbrace{\underbrace{\underbrace{y_0(x)}_{S_0(x)}, \ y_1(x) - y_0(x)}_{S_1(x) = y_1(x)}, \ y_2(x) - y_1(x)}_{S_2(x) = y_2(x)}, \ldots, \ y_1(x) - y_{n-1}(x)}_{S_n(x) = y_n(x)}, \ldots
              \]
    \end{enumerate}

    Нужно дописать.
\end{proof}