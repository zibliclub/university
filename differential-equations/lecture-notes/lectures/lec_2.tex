\lesson{2}{от 15 окт 2023 08:47}{Продолжение}


\section{Особые решения}

\begin{note}
    \[
        \left\{\begin{array}{rl}
            y'     & = f(y), \ f \in C(D), \ D \subset\mathbb{R}, \\
            y(x_0) & = y_0
        \end{array}\right.
    \]
    \[
        \left[\begin{array}{rl}
            f(y) = 0                           & \implies y = c\text{ -- ?}       \\
            \left\{\begin{array}{rl}
                       f(y)                & \ne 0,    \\
                       \int\frac{dy}{f(y)} & = \int dx
                   \end{array}\right. & \implies \left[\begin{array}{rl}
                                                           y           & = \phi(x,C), \\
                                                           \psi(y,x,C) & = 0
                                                       \end{array}\right.
        \end{array}\right.
    \]

    Для $\forall$ точки $x \in \{y = C\} \ \exists$ точка $(x_1,y_1)$ и интегральная кривая, проходящая через точку $(x_1,y_1)$, которая пересекает прямую $y = C$ в точке $x \ \big(x \equiv (x,C)\big)$.

    Проинтегрируем на отрезке $[x_1;x]$:
    \[
        \int_{y_1}^{C}\frac{dy}{f(y)} = \int_{x_1}^{x}dx \iff \int_{y_1}^{C}\frac{dy}{f(y)} = x-x_1 \iff \underbrace{x}_{\text{конечная}} = x_1 + \underbrace{\int_{y_1}^{C}\frac{dy}{f(y)}}_{\text{конечный}},
    \]
    \begin{center}
        (несобственный интеграл сходится)
    \end{center}
\end{note}

\begin{note}[Критерий]
    Решение $y = C$ дифференциального уравнения $y' = f(y), \ f\in C(D)$ такое, что $f(C) = 0$ называется \emph{особым} $\iff$
    \[
        \iff \int_{y_1}^{C}\frac{dy}{f(y)} < \infty \quad \text{(несобственный интеграл сходится)}
    \]
\end{note}

\begin{example}
    $y'=3y^{\frac{2}{3}}$
    \begin{enumerate}
        \item Непрерывно.
        \item $f_y' = 2y^{-\frac{1}{3}}$ -- разрывна в точке $0$ (условие Липшица не выполнено?).
    \end{enumerate}
    \[
        \left[\begin{array}{rl}
            y                             & = 0 \ ?                                \\
            \int\frac{dy}{3y^\frac{2}{3}} & = \int dx \implies y^\frac{1}{3} = x+C
        \end{array}\right.
    \]
    \[
        \int_{y_1}^{0}\frac{dy}{3y^\frac{2}{3}} = y^\frac{1}{3}\Big|_{y_1}^0 = 0 - y_1^\frac{1}{3} < + \infty \overset{\text{по критерию}}{\implies} y = 0\text{ -- особое}
    \]
\end{example}

\begin{example}
    Найти особое решение $y' = \left\{\begin{array}{rl}
            y\cdot\ln y, & y > 0 \\
            0,           & y = 0
        \end{array}\right., \ D = [0;+\infty)$:
    \[
        f(y) = \left\{\begin{array}{rl}
            y\cdot\ln y, & y > 0 \\
            0,           & y = 0
        \end{array}\right.
    \]
    \begin{enumerate}
        \item Непрерывно.
              \[
                  \underset{y\rightarrow+0}{\lim}(y\cdot \ln y) = \underset{y\rightarrow+0}{\lim}\frac{\ln y}{\frac{1}{y}} = \underset{y\rightarrow+0}{\lim}\frac{\frac{1}{y}}{\frac{1}{y^2}} = -\underset{y\rightarrow+0}{\lim} = 0
              \]
        \item Условие Липшица:
              \[
                  \big|f(y_1) - f(y_2)\big| \leqslant L \cdot |y_1 - y_2|, \quad \begin{array}{l}
                      y_1 \in (0;+\infty), \\
                      y_2 = 0
                  \end{array}
              \]
              \[
                  \big|f(y_1) - f(y_2)\big| = |y_1\cdot \ln y_1 - 0| \leqslant |y_1|\cdot|\ln y_1| \leqslant |y_1| \cdot L,
              \]
              то есть $|\ln y_1| \leqslant L$.

              Для $\forall L > 0 \ \exists y_1^*$ близкий к $0$ и такой, что $|\ln y_1^*| > L$.
        \item $f(y) = 0 \implies \left[\begin{array}{l}
                      y = 0 \\
                      y = 1
                  \end{array}\right.$
              \begin{enumerate}
                  \item $y = 0$:
                        \[
                            \int_{y_1}^{0}\frac{dy}{y \cdot \ln y} = \int_{y_1}^{0}\frac{d(\ln y)}{\ln y} = \ln |\ln y| \Big|_{y_1}^0 = \infty - \ln|\ln y_1| = \infty \implies
                        \]
                        $\implies y =0$ не является особым.
                  \item $y = 1$:
                        \[
                            \int_{y_1}^{1}\frac{dy}{y\cdot \ln y} = \ln|\ln y| \Big|_{y_1}^1 = -\infty - \ln|\ln y_1| = - \infty \implies
                        \]
                        $\implies y =1$ не является особым.
              \end{enumerate}
    \end{enumerate}
\end{example}

\begin{example}
    Найти особое решение $y' = \left\{\begin{array}{rl}
            y\cdot\ln^2 y, & y > 0 \\
            0,             & y = 0
        \end{array}\right., \ D = [0;+\infty)$:
    \[
        f(y) = \left\{\begin{array}{rl}
            y\cdot\ln^2 y, & y > 0 \\
            0,             & y = 0
        \end{array}\right.
    \]
    \begin{enumerate}
        \item Непрерывно (аналогично).
        \item Условие Липшица (аналогично).
        \item $f(y) = 0 \implies y\cdot\ln^2y=0 \left[\begin{array}{l}
                      y = 0 \\
                      y = 1
                  \end{array}\right.$
              \begin{enumerate}
                  \item $y = 0$:
                        \[
                            \int_{y_1}^{0}\frac{dy}{y \cdot \ln^2 y} = \int_{y_1}^{0}\frac{d(\ln y)}{\ln^2 y} = \frac{1}{\ln y} \Big|_{y_1}^0 = 0 + \frac{1}{\ln y_1} \implies
                        \]
                        $\implies y =0$ -- особое.
                  \item $y = 1$:
                        \[
                            \int_{y_1}^{1}\frac{dy}{y\cdot \ln^2 y} = -\frac{1}{\ln y} \Big|_{y_1}^1 = -\infty + \frac{1}{\ln y_1}\text{ -- расходится } \implies
                        \]
                        $\implies y =1$ не является особым.
              \end{enumerate}
    \end{enumerate}
\end{example}

\chapter{Методы интегрирования дифференциальных уравнений $1$-го порядка}

\section{Однородные уравнения}

\begin{definition}[Однородное уравнение первого порядка]
    \emph{Однородным уравнением первого порядка} называется уравнение вида:
    \begin{equation}\label{eq6}
        y' = f\left(\frac{y}{x}\right),\quad f\in C(D)
    \end{equation}
    или:
    \begin{equation}\label{eq7}
        y'=\frac{P(x,y)}{Q(x,y)},
    \end{equation}
    где $P(x,y)$ и $Q(x,y)$ являются однородными функциями одного и того же порядка.
\end{definition}

\begin{definition}[Однородная функция порядка $k$]
    \emph{Однородной функцией порядка $k$} называется функция:
    \[
        P(\lambda x,\lambda y) = \lambda^k \cdot P(x,y),\quad \lambda \in \mathbb{R}, \ \lambda \ne 0
    \]
\end{definition}

\begin{example}
    $P(x,y) = x^2 - 2xy + 7y^2$
    \[
        P(\lambda x,\lambda y) = (\lambda x)^2 - 2(\lambda x)(\lambda y) + 7(\lambda y)^2 = \lambda^2(x^2 - 2xy + 7y^2)
    \]
\end{example}

\begin{example}
    $x(x^2 + y^2)dy = y(y^2 - xy + x^2)dx$
    \[
        \underbrace{x^3 + xy^2}_{Q(x,y)},\quad \underbrace{y^3 - xy^2 + yx^2}_{P(x,y)}
    \]
\end{example}

\begin{note}[Замена переменной]
    $t = \frac{y}{x} \implies y = t\cdot x$
    \begin{align*}
         & y = t(x)\cdot x \implies y' = t'\cdot x + t\text{ -- подставим в \ref{eq6}}: \\
         & t'\cdot x = f(t) - t\text{ -- уравнение с разделяющей переменной (РП)}
    \end{align*}
    \[
        \frac{dt}{dx}x = f(t) - t
    \]
    \[
        \left[\begin{array}{l}
            f(t) - t = 0 \\
            \left\{\begin{array}{rl}
                       f(t) - t                 & \ne 0               \\
                       \int \frac{dt}{f(t) - t} & = \int \frac{dx}{x}
                   \end{array}\right. \iff \left[\begin{array}{rl}
                                                     t = \phi(x,C)   & \implies y = x \cdot \phi(x,C)     \\
                                                     \psi(t,x,C) = 0 & \implies \psi(\frac{y}{x},x,C) = 0
                                                 \end{array}\right.
        \end{array}\right.
    \]
    $t'x = 0 \implies t'=0 \implies t = C \implies y = Cx$ -- решение при $C: \ f(C) - C = 0$.

    \[
        \text{Изоклины: }\begin{array}{l}
            f(\frac{y}{x}) = const                \\
            \frac{y}{x} = C \implies f(C) = const \\
            y = Cx\text{ -- изоклины уравнения \ref{eq6}}
        \end{array}
    \]

    \begin{enumerate}
        \item \begin{enumerate}
                  \item Уравнение вида $y' = f\left(\frac{a_1x + b_1y + c_1}{a_2x + b_2y + c_2}\right)$ сводится к однородному с помощью замены:
                        \[
                            \left\{\begin{array}{l}
                                x = \xi + \alpha \\
                                y = \eta + \beta
                            \end{array}\right.,\quad (\alpha;\beta)\text{ -- решение системы }\left\{\begin{array}{l}
                                a_1x + b_1y + c_1 = 0 \\
                                a_2x + b_2y + c_2 = 0
                            \end{array}\right.
                        \]
                  \item Прямые параллельны, то есть $a_1x + b_1y = k(a_2x + b_2y)$.

                        Замена переменной $t = a_2x + b_2 y$ и привести к уравнению с РП.
              \end{enumerate}
        \item Замена переменной $y = t^m$. Подставить эту замену в уравнение и из условия однородности выбрать $m$.
    \end{enumerate}
\end{note}

\begin{example}
    $ydx + x(2xy + 1)dy = 0, \quad ydx + (2x^2 + x)dy = 0$
    \[
        y = t^m \implies t^mdx + (2x^2t^m + x)\cdot mt^{m-1}dt = 0
    \]
    \begin{align*}
        m = 2m + 1 = m \\
        m = -1 \implies y = t^{-1} = \frac{1}{t}
    \end{align*}
    \begin{multline*}
        \frac{dx}{t} + \left(\frac{2x^2}{t} + x\right)\left(-\frac{1}{t^2}\right)dt = Q \implies \\
        \implies dx - \frac{x}{t}\left(\frac{2x}{t} + 1\right)dt = 0\text{ -- однородное уравнение}\implies \\
        \implies \frac{dx}{dt}=\frac{x}{t}\left(\frac{2x}{t} + 1\right)\equiv f\left(\frac{x}{t}\right)
    \end{multline*}
    Замена переменной: $u = \frac{x}{t}\implies x = ut \implies dx = udt + tdu$
    \begin{align*}
        udt + tdu - u(2u + t)dt = 0               \\
        tdu - 2u^2dt = 0, \quad t\ne0, \ u^2\ne 0 \\
        \int \frac{du}{2u^2} = \int \frac{dt}{t}  \\
        -\frac{1}{2u} = \ln|t| + C
    \end{align*}
    \[
        \left\{\begin{array}{l}
            u = \frac{x}{t} \\
            t=\frac{1}{y}
        \end{array}\right. \implies \left\{\begin{array}{l}
            u = xy \\ t = \frac{1}{y}
        \end{array}\right., \quad \begin{array}{l}
            -\frac{1}{xy} = \ln \left(\frac{1}{y}\right)^2 + C \\
            -\frac{1}{xy} = \ln y^2 + C                        \\
        \end{array}
    \]
    \[
        \ln y^2 - \frac{1}{xy} = C
    \]
    \[
        u = 0 \implies x\cdot y = 0 \implies \left[\begin{array}{l}
            x = 0\text{ -- решение} \\
            y = 0\text{ -- решение}
        \end{array}\right.
    \]

    Ответ: $\ln y^2 - \frac{1}{xy} = C, \quad x=0, \ y=0$
\end{example}

\section{Линейные уравнения $1$-го порядка}

\begin{definition}[Линейное уравнение $1$-го порядка]
    \emph{Линейным уравнением $1$-го порядка} называется уравнение вида:
    \begin{equation}\label{eq8}
        a_0(x)y' + a_1(x)y = b(x),
    \end{equation}
    \begin{equation}\label{eq9}
        a_0(x)y' + a_1(x)y = 0,
    \end{equation}
    где $b(x), a_0(x),a_1 \in C(\alpha;\beta), \ a_0(x) \ne 0, \quad -\infty \leqslant \alpha \leqslant\beta\leqslant+\infty$.
\end{definition}

\begin{note}[Задача Коши]
    $y(x_0) = y_0$
\end{note}

\begin{theorem}[$\exists$ и $!$]
    \[
        y' = \underbrace{\frac{b(x)}{a_0(x)} - \frac{a_1(x)}{a_0(x)}y}_{f(x,y)}
    \]
    \begin{enumerate}
        \item $f\in C\big((\alpha;\beta) \times (-\infty;+\infty)\big)$.
        \item $f'_y = -\frac{a_1(x)}{a_0(x)}$
    \end{enumerate}

    Однородное уравнение \ref{eq9}.
    \[
        y = 0\text{ -- решение: }y = c\cdot e^{-\int \frac{a_1(x)}{a_0(x)}dx}
    \]
\end{theorem}

\begin{definition}[Линейное уравнение $1$-го порядка]
    \emph{Линейным уравнением $1$-го порядка} называется уравнение вида:
    \begin{equation}\label{eq10}
        y'+p(x)\cdot y = q(x),
    \end{equation}
    \begin{equation}\label{eq11}
        y' + p(x)\cdot y = 0,
    \end{equation}
    где $p(x),q(x) \in C(\alpha;\beta)$.
\end{definition}

\begin{note}[Свойства \ref{eq11}]\leavevmode
    \begin{enumerate}
        \item Пусть $y_1(x)$ -- решение \ref{eq11} $\implies k\cdot y_1$ -- решение \ref{eq11}, $k \in \mathbb{R}(\mathbb{C})$.
        \item Если $y_1,y_2$ -- решения \ref{eq11} $\implies y_1 + y_2$ -- решение \ref{eq11} ($k_1y_1 + k_2y_2$ -- решение).
        \item $y=0$ -- решение \ref{eq11}.
              \begin{statement}
                  Решения \ref{eq11} образуют линейное пространство:
                  \[
                      y' + p(x)y = 0
                  \]
                  \begin{align*}
                       & \frac{dy}{y} = -p(x)dx         \\
                       & \ln|y| = -\int p(x)dx + \ln|C|
                  \end{align*}
                  \[
                      (\star) \ y = C \cdot e^{-\int p(x)dx}
                  \]
                  \begin{align*}
                       & \left\{\begin{array}{rl}
                                    y' + p(x)y & = 0   \\
                                    y(x_0)     & = y_0
                                \end{array}\right., \quad [x_0;x]                   \\
                       & \int_{x_0}^{x}\frac{y'(s)}{y(s)}ds = -\int_{x_0}^{x}p(s)dx
                  \end{align*}
                  \[
                      \int_{y_0}^{y}\frac{dy}{y} = -\int_{x_0}^{x}p(s)dx \implies \ln|y| - \ln|y_0| = - \int_{x_0}^{x}p(x)dx \implies
                  \]
                  \[
                      (\star\star) \ \implies y = y_0\cdot e^{-\int_{x_0}^{x}p(x)dx},
                  \]
                  при $C = y_0$.
              \end{statement}
        \item Если $y$ -- частное решение, то $C \cdot y$ -- общее решение \ref{eq11}.
    \end{enumerate}
\end{note}