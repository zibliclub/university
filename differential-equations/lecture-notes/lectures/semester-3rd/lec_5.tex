\lesson{5}{от 5 дек 2023 10:29}{Продолжение}


\begin{theorem}
    Пусть $y_1(x), \ldots, y_n(x)$ -- система линейно независимых на $(\alpha;\beta)$ решений уравнения $Ly = 0$. Тогда $W(x) \ne 0$ ни в какой точке интервала $(\alpha;\beta)$.
\end{theorem}

\begin{proof}
    От противного. Предположим, что $\exists x_0 \in (\alpha;\beta)$. $W(x_0) = 0$,
    \[
        W(x_0) = \left|\begin{array}{ccc}
            y_1(x_0)         & \cdots & y_n(x_0)         \\
            y_1'(x_0)        & \cdots & y_n'(x_0)        \\
            \vdots           & \ddots & \vdots           \\
            y_1^{(n-1)}(x_0) & \cdots & y_n^{(n-1)}(x_0) \\
        \end{array}\right| = 0, \quad \left(\begin{array}{c}
                c_1    \\
                c_2    \\
                \vdots \\
                c_n
            \end{array}\right)
    \]

    \begin{equation}\label{eq35}
        \left\{\begin{array}{l}
            c_1y_1(x_0) + \ldots + c_ny_n(x_0) = 0                 \\
            c_1y_1'(x_0) + \ldots + c_ny_n'(x_0) = 0               \\
            \vdots                                                 \\
            c_1y_1^{(n-1)}(x_0) + \ldots + c_ny_n^{(n-1)}(x_0) = 0 \\
        \end{array}\right.
    \end{equation}

    Однородная система линейно алгебраических уравнений, $\det = W(x_0) = 0, \implies$ система \ref{eq35} имеет нетривиальное решение: $\overrightarrow{c^0} - (c_1^0,c_2^0,\ldots,c_n^0)$, $y_1(x),\ldots,y_n(x)$ -- линейно зависимая?
    \[
        y(x) = c_1^0 \cdot y_1(x) + \ldots + c_n^0 \cdot y_n(x)
    \]
    \begin{enumerate}
        \item $y(x)$ -- решение $Ly = 0$;
        \item $\left\{\begin{array}{l}
                      y(x_0) = 0                                                        \\
                      y'(x_0) = c_1^0 y_1'(x) + \ldots + c_n^0y_n'(x)\bigg|_{x=x_0} = 0 \\
                      \vdots                                                            \\
                      y^{(n-1)}(x_0) = c_1^0 y_1^{(n-1)}(x) + \ldots + c_n^0y_n^{(n-1)}(x)\bigg|_{x=x_0} = 0
                  \end{array}\right.$
    \end{enumerate}
    \begin{enumerate}
        \item $y\equiv0 \implies Ly = 0 \implies$ из теоремы существования и единственности $\implies y(x) = \sum_{k=1}^{n}c_k^0 y_k(x) \equiv 0 \implies y_1,\ldots,y_n$ -- линейно зависимые $\implies$ противоречие.
    \end{enumerate}
\end{proof}

\begin{theorem}[Лиувилля-Остроградского $\big(W(x), \ W(x_0)\big)$]
    Пусть задано уравнение:
    \[
        a_0(x)\cdot y^{(n)} + a_1(x)\cdot y^{(n-1)} + \ldots + a_{n-1}(x)\cdot y' + a_n(c)\cdot y = 0,
    \]
    $a_0(x) \ne 0, \ a_j(x)\in C(\alpha;\beta), \ j = \overline{0,n}, \ -\infty\leqslant\alpha<\beta\leqslant+\infty$. Тогда:
    \[
        W(x) = W(x_0) \cdot e^{-\int_{x_0}^{x}\frac{a_1(s)}{a_0(s)}ds}, \quad x_0 \in (\alpha;\beta)
    \]
\end{theorem}

\begin{corollary}
    Если $\exists x_0 \in (\alpha;\beta): \ W(x_0) = 0 \implies W(x) = 0 \ \forall x \in (\alpha;\beta)$
\end{corollary}

\section{Построение общего решения уравнения \\ $Ly = 0$}

\begin{definition}[Решение $Ly=0$]
    Функция $y = \phi(x,C_1,C_2,\ldots,C_n)$ называется \emph{решением $Ly = 0$}, если для $\forall$ набора $C_1,C_2,\ldots,C_n$ она является решением $Ly = 0$ и для $\forall$ задачи Коши $y(x_0) = y_0^\circ, \ y'(x_0) = y_1^\circ, \ \ldots, \ y^{(n-1)}(x_0) = y_{n-1}^\circ \ \exists$ набор $C_1^\circ,C_2^\circ,\ldots,C_n^\circ: \ y = \phi(x,C_1^\circ,\ldots,C_n^\circ)$ является решением $Ly = 0$.
\end{definition}

\begin{theorem}[Структура решения однородного уравнения]
    Пусть \\ $y_1(x),\ldots,y_n(x)$ -- линейно независимые решения $Ly=0 \ n$-го порядка. Тогда:
    \[
        y_{\text{ОО}} = C_1y_1(x) + \ldots + C_ny_n(x),
    \]
    где $C_1,\ldots,C_n$ -- произвольные константы.
\end{theorem}

\begin{definition}[Фундаментальная система решений (ФСР)]
    Любые $n$ линейно независимых решений задачи Коши уравнения $Ly=0$ называются \emph{фундаментальной системой решений (ФСР)}.
\end{definition}

\begin{theorem}
    ФСР уравнения $Ly=0$ -- существует.
\end{theorem}

\begin{theorem}
    Любые $(n+1)$ решения задачи Коши для $Ly = 0 n$-го порядка линейно зависимы, то есть $\exists \alpha_1,\ldots,\alpha_{n+1}$:
    \[
        \alpha_1^2 + \ldots + \alpha_{n+1}^2 \ne 0, \quad \alpha_1y_1(x) + \alpha_2y_2(x) + \ldots + \alpha_{n+1}y_{n+1}(x) = 0
    \]
\end{theorem}

\section{Линейные уравнения с переменными коэффициентами}

\begin{note}
    Решения линейного однородного уравнения $n$-го порядка с переменными коэффициентами -- линейное пространство размерности $n$ с базисом ФСР.

    \emph{Нормированная ФСР} -- это задача Коши с начальными условиями:
    \begin{eqnarray*}
        (1,0,\ldots,0),(0,1,\ldots,0),\ldots,(0,0,\ldots,1)
    \end{eqnarray*}

    Если имеем $y_1,y_2,\ldots,y_n$ -- решений $Ly = 0$ и $\exists x_0 \in (\alpha;\beta): \ W(x_0) \ne 0$, то пытаемся восстановить дифференциальное уравнение.
\end{note}

\begin{example}
    $n = 2, \ \{\sin x, \cos x\}, \ w(x) = \left|\begin{array}{cc}
            \sin x & \cos x   \\
            \cos x & - \sin x
        \end{array}\right| = -1 \ne 0$
    \[
        a_0(x)y'' + a_1(x)y' + a_2(x)y = 0, \quad a_0(x) \ne 0
    \]
    \[
        y'' + p(x)y' + q(x) \cdot y = 0
    \]

    \[
        \left\{\begin{array}{l}
            -\sin x + p(x) \cdot \cos x + q(x)\cdot \sin x = 0 \\
            -\cos x - p(x) \cdot \sin x + q(x)\cdot \cos x = 0
        \end{array}\right.
    \]

    \[
        \begin{pmatrix}
            \cos x & \sin x \\ -\sin x & \cos x
        \end{pmatrix} \cdot \begin{pmatrix}
            p(x) \\ q(x)
        \end{pmatrix} = \begin{pmatrix}
            \sin x \\ \cos x
        \end{pmatrix}
    \]
    \[
        \begin{array}{l}
            \Delta = \left|\begin{array}{cc}
                               \cos x & \sin x \\ -\sin x & \cos x
                           \end{array}\right| = \cos^2 x + \sin^2 x = 1 \ne 0 \\
            \Delta_1 = \left|\begin{array}{cc}
                                 \sin x & \sin x \\ \cos x & \cos x
                             \end{array}\right| = 0                \\
            \Delta_2 = \left|\begin{array}{cc}
                                 \cos x & \sin x \\ -\sin x & \cos x
                             \end{array}\right| = \cos^2 x + \sin^2x = 1
        \end{array}
    \]

    \[
        \begin{array}{l}
            p(x) = \frac{\Delta_1}{\Delta} = 0 \\
            q(x) = \frac{\Delta_2}{\Delta} = 1
        \end{array} \implies y'' + y = 0
    \]
\end{example}

\begin{note}[Способы восстановления дифференциального уравнения]\leavevmode
    \begin{enumerate}
        \item Способ первый:
              \[
                  \left\{\begin{array}{l}
                      y^{(n)} + p_{n-1}(x)\cdot y^{(n-1)} + \ldots + p_1(x)\cdot y' + p_0(x)\cdot y = 0         \\
                      y^{(n)}_1 + p_{n-1}(x)\cdot y^{(n-1)}_1 + \ldots + p_1(x)\cdot y'_1 + p_0(x)\cdot y_1 = 0 \\
                      y^{(n)}_2 + p_{n-1}(x)\cdot y^{(n-1)}_2 + \ldots + p_1(x)\cdot y'_2 + p_0(x)\cdot y_2 = 0 \\
                      \vdots                                                                                    \\
                      y^{(n)}_n + p_{n-1}(x)\cdot y^{(n-1)}_n + \ldots + p_1(x)\cdot y'_n + p_0(x)\cdot y_n = 0 \\
                  \end{array}\right.
              \]
              \begin{multline*}
                  \Delta = \left|\begin{array}{cccc}
                      y_1    & y_1'   & \ldots & y_1^{(n-1)} \\
                      y_2    & y_2'   & \ldots & y_2^{(n-1)} \\
                      \vdots & \vdots & \ddots & \vdots      \\
                      y_n    & y_n'   & \ldots & y_n^{(n-1)}
                  \end{array}\right|\begin{array}{l}
                      p_0 \\ p_1 \\ \vdots \\ p_n
                  \end{array} = \\
                  = \left|\begin{array}{cccc}
                      y_1         & y_2         & \ldots & y_n         \\
                      y_1'        & y_2'        & \ldots & y_n'        \\
                      \vdots      & \vdots      & \ddots & \vdots      \\
                      y_1^{(n-1)} & y_2^{(n-1)} & \ldots & y_n^{(n-1)}
                  \end{array}\right| = W(x) \ne 0
              \end{multline*}
              $(\implies y_1,y_2,\ldots,y_n\text{ -- ЛНЗ}) \implies$ система имеет $!$ решение \\ $p_0(x),p_1(x),\ldots,p_{n-1}(x)$, которое выражается через $y_1,y_2,\ldots,y_n$ и их производные.

        \item Способ второй: потерян
    \end{enumerate}
\end{note}

\begin{example}
    По второму способу:

    $y_1 = x, \ y_2 = x^2, \ W(x) = \left|\begin{array}{cc}
            y_1 & y_2 \\ y_1' & y_2'
        \end{array}\right| = \left|\begin{array}{cc}
            x & x^2 \\ 1 & 2x
        \end{array}\right| = 2x^2 - x^2 = x^2 \ne 0, \ \text{при } x\ne 0$.
    \begin{multline*}
        \left|\begin{array}{ccc}
            y_1'' & y_1' & y_1 \\
            y_2'' & y_2' & y_2 \\
            y_1'' & y_1' & y_1 \\
        \end{array}\right| = 0 \iff \left|\begin{array}{ccc}
            0   & 1  & x   \\
            2   & 2x & x^2 \\
            y'' & y' & y
        \end{array}\right| = 0 \iff \\
        \iff y'' \cdot \left|\begin{array}{cc}
            2x & x^2 \\ y' & y
        \end{array}\right| - y' \cdot \left|\begin{array}{cc}
            0 & x \\ 2 & x^2
        \end{array}\right| + y \cdot \left|\begin{array}{cc}
            0 & 1 \\ 2 & 2x
        \end{array}\right| = 0
    \end{multline*}
    \[
        x^2 \cdot y'' - 2x \cdot y' + 2y = 0
    \]
\end{example}

\section{Структура общего решения линейного неоднородного уравнения $Ly = f$}

\begin{note}
    \begin{equation}\label{eq36}
        Ly = a_0(x) \cdot y^{(n)} + a_1(x) \cdot y^{(n-1)} + \ldots + a_{n-1}(x)\cdot y' + a_n(x) \cdot y = f(x),
    \end{equation}
    где $a_0(x)\ne0, \ a_j(x), \ f(x) \in C(\alpha;\beta), \ j = \overline{0,n}, \ -\infty \leqslant \alpha <\beta \leqslant +\infty$
\end{note}

\begin{theorem}
    Все решения уравнения вида \ref{eq36} даются формулой:
    \begin{equation}\label{eq37}
        y_{\text{ОН}} = y_{\text{ОО}} + y_{2\text{ЧН}}
    \end{equation}
\end{theorem}

\begin{proof}
    Пусть $y_{2\text{ЧН}}$ -- произвольное частное решение \ref{eq36}, то есть
    \[
        L(y_{2\text{Н}}) = f(x)
    \]
    \begin{enumerate}
        \item Покажем, что решение \ref{eq37} удовлетворяет \ref{eq36}:
              \[
                  L(y_{\text{ОН}}) = L(y_{\text{ОО}} + y_{2\text{ЧН}}) = L(y_{\text{ОО}}) + L(y_{2\text{ЧН}}) = 0 + f(x) = f(x)
              \]

        \item Покажем, что формула \ref{eq37} покрывает все решения \ref{eq36}:

              $\widetilde{y}$ -- частное решение \ref{eq36}, $L(\widetilde{y}) = f(x)$
              \[
                  \widetilde{y} = (\widetilde{y} - y_{\text{ЧН}}) + y_{\text{ЧН}}
              \]
              \begin{multline*}
                  L(\widetilde{y} - y_{\text{ЧН}}) = L(\widetilde{y}) - L(y_{\text{ЧН}}) = f(x) - f(x) = 0 \implies \\
                  \implies \widetilde{y} = (\widetilde{y} - y_{\text{ЧН}}) + y_{\text{ЧН}} = y_{\text{ОО}} + y_{\text{ЧН}}
              \end{multline*}
    \end{enumerate}
\end{proof}

\subsection*{Построение общего решения неоднородного уравнения \ref{eq36}}

\begin{note}[Метод вариации произвольных постоянных]\leavevmode
    \begin{enumerate}
        \item $y_1,y_2,\ldots,y_n$ -- ФСР уравнения \ref{eq36} $\implies y_{\text{ОО}} = C_1y_1(x) + \ldots + C_ny_n(x)$.
        \item $y = y_{\text{ОН}} = C_1(x)y_1(x) + \ldots + C_n(x)y_n(x)$. Найдем производные до $n$-го порядка:
              \[
                  \begin{array}{rl}
                      a_{n}(x):   & C_1(x)y_1(x) + \ldots + C_n(x)y_n(x)                                                                                                    \\
                      a_{n-1}(x): & C_1(x)y'_1(x) + \ldots + C_n(x)y'_n(x) + \equalto{\underbrace{C_1'(x)y_1(x) + \ldots + C_n'(x)y_n(x)}}{0}                               \\
                      a_{n-2}(x): & C_1(x)y''_1(x) + \ldots + C_n(x)y''_n(x) + \equalto{\underbrace{C_1'(x)y_1'(x) + \ldots + C_n'(x)y_n'(x)}}{0}                           \\
                      \vdots      & \vdots                                                                                                                                  \\
                      a_1(x):     & C_1(x)y^{(n-1)}_1(x) + \ldots + C_n(x)y^{(n-1)}_n(x) + \equalto{\underbrace{C_1'(x)y_1^{(n-2)}(x) + \ldots + C_n'(x)y_n^{(n-2)}(x)}}{0} \\
                      a_0(x):     & C_1(x)y^{(n)}_1(x) + \ldots + C_n(x)y^{(n)}_n(x) + \equalto{\underbrace{C_1'(x)y_1^{(n-1)}(x) + \ldots + C_n'(x)y_n^{(n-1)}(x)}}{0}     \\
                  \end{array}
              \]
              \[
                  C_1(x)\equalto{\underbrace{\big(a_0(x)y_1^{(n)}(x) + a_1(x)y_1^{(n-1)}(x) + \ldots + a_{n_1}(x)y_1' + a_n(x)y_1(x)\big)}}{0}
              \]

              Система $n$ уравенений и $n$ неизвестных $C_1'(x), \ldots, C_n'(x)$, определитель $\Delta = ?$
              \[
                  \Delta = \left|\begin{matrix}
                      y_1         & \cdots & y_n         \\
                      y_1'        & \cdots & y_n'        \\
                      \vdots      & \ddots & \vdots      \\
                      y_1^{(n-1)} & \cdots & y_n^{(n-1)}
                  \end{matrix}\right| = W(x) \ne 0
              \]
    \end{enumerate}
\end{note}

\begin{note}[Метод вариации произвольных постоянных (продолжение?)]
    \[
        y_{\text{ОН}} = C_1(x) \cdot y_1 + \ldots = C_n(x)y_n
    \]
    \[
        \left\{\begin{array}{l}
            C_1'(x)y_1 + \ldots + C_n'(x)y_n = 0                 \\
            C_1'(x)y_1' + \ldots + C_n'(x)y_n' = 0               \\
            \vdots                                               \\
            C_1'(x)y_1^{(n-2)} + \ldots + C_n'(x)y_n^{(n-2)} = 0 \\
            C_1'(x)y_1^{(n-1)} + \ldots + C_n'(x)y_n^{(n-1)} = \frac{f(x)}{a_0(x)}
        \end{array}\right.
    \]
    $\Delta = W(y_1,y_2,\ldots,y_n) \ne 0$, так как $y_1,y_2,\ldots,y_n$ -- ФСР $\implies$ система имеет $!$ решение. Найти это решение по формулам Крамера:
    \begin{equation}\label{eq38}
        C_k'(x) = \frac{W_k(x)}{W(x)}dx + C_k, \quad k = \overline{1,n},
    \end{equation}
    где:
    \[
        W(x) = \left|\begin{matrix}
            y_1         & \cdots & y_n         \\
            y_1'        & \cdots & y_n'        \\
            \vdots      & \ddots & \vdots      \\
            y_1^{(n-1)} & \cdots & y_n^{(n-1)}
        \end{matrix}\right|,
    \]
    \[
        W_k(x) = \left|\begin{matrix}
            y_1         & \cdots & y_n         & 0      & y_{k+1}         & \cdots & y_n         \\
            y_1'        & \cdots & y_n'        & 0      & y_{k+1}'        & \cdots & y_n'        \\
            \vdots      & \ddots & \vdots      & \vdots & \vdots          & \ddots & \vdots      \\
            y_1^{(n-1)} & \cdots & y_n^{(n-1)} & 0      & y_{k+1}^{(n-1)} & \cdots & y_n^{(n-1)}
        \end{matrix}\right|
    \]

    Проинтегрируем \ref{eq38}:

    \begin{equation}
        C_k(x) = \int \frac{W_k(x)}{W(x)}dx + C_k, \quad k = \overline{1,n},
    \end{equation}

    \[
        y(x) = y_{\text{ОН}} = \sum_{k=1}^{n}\left(\frac{W_k(x)}{W(x)}dx + C_k\right)y_k = \equalto{\underbrace{\sum_{k=1}^{n}C_ky_k}}{y_{\text{ОО}}} + \equalto{\underbrace{\sum_{k=1}^{n}y_k\int \frac{W_k(x)}{W(x)}dx}}{y_{\text{ЧН}}}
    \]
\end{note}