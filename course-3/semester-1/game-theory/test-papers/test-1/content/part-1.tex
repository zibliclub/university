\section{Задача о максимальном потоке}

\begin{definition}[Двухполюсная сеть]
	Задан ориентированный граф $S = (V,E)$ с двумя выделенными вершинами: \emph{источником} $s$ и \emph{стоком} $t$. Каждой дуге $e$ графа приписано неотрицательное число $c(e)$ -- \emph{пропускная способность} дуги.

	Такой граф называется \emph{двухполюсной сетью}.
\end{definition}

\begin{definition}[Поток, дуговые потоки, величина потока, максимальный поток]
	\emph{Потоком} из $s$ в $t$ в сети $S$ называется функция
	\[
		f: \ E \rightarrow R,
	\]
	удовлетворяющая условиям:
	\[
		0 \leqslant f(e) \leqslant c(e), \ \forall e \in E,
	\]
	\[
		\sum_{y \in A(x)} f(x,y) - \sum_{y \in B(x)} f(y,x) = \left\{\begin{array}{ll}
			v,  & x = s            \\
			0,  & x \notin \{s,t\} \\
			-v, & x = t
		\end{array}\right.
	\]


	Числа $f(e) \geqslant 0$ называются \emph{дуговыми потоками}. Величина $v = v(f)$ называется \emph{величиной потока} $f$. Поток $f$ называется \emph{максимальным}, если величина $v(f)$ максимальна.
\end{definition}

\begin{definition}[Допустимая дуга, увеличивающий путь]
	Дуга $e$ в сети $S$ называется \emph{допустимой дугой} из $x$ в $y$ относительно потока $f$, если:
	\[
		\text{либо }e = (x,y) \text{ и } f(e) < c(e) \text{ (прямая дуга)},
	\]
	\[
		\text{либо }e = (y,x) \text{ и } f(e) > 0 \text{ (обратная дуга)}.
	\]

	\emph{Увеличивающим путем} для данного потока $f$ и $s$ в $t$ называется такая последовательность вершин $P = (s = x_0,x_1,\ldots ,x_{k-1},x_k = t)$, что $\forall i = 1,\ldots ,k$ либо $e_i = (x_{i-1} ,x_i) \in E$, либо $e_i = (x_i,x_{i-1} )\in E$ и $e_i$ -- допустимая дуга из $x_{i-1} $ в $x_i$ относительно потока $f$.
\end{definition}

\begin{definition}[Разрез]
	\emph{Разрезом} $(X,\overline{X} )$ называется множество дуг $e = (x,y)$ таких, что $x \in X, \ y \in \overline{X} $. Разрез $(X,\overline{X} )$ \emph{разделяет} вершины $s$ и $t$, если $s \in X, \ t \in \overline{X} $. \emph{Пропускная способность} разреза $(X,\overline{X} ): \ c(X,\overline{X} ) = \sum_{e \in (X,\overline{X} )} c(e)$. \emph{Поток через разрез} $(X,\overline{X} )$ определяется:
	\[
		f(X,\overline{X} ) = \sum_{e \in (X,\overline{X} )} f(e).
	\]
\end{definition}

\begin{definition}[Минимальный разрез]
	\emph{Минимальным разрезом} называется разрез, разделяющий $s$ и $t$, с минимальной пропускной способностью среди всех таких разрезов.
\end{definition}

\begin{definition}[Условие баланса]
	Необходимое условие разрешимости задачи являетяс так называемое \emph{условие баланса}:
	\[
		\sum_{i=1}^{m} a_i = \sum_{j=1}^{n} b_j.
	\]
\end{definition}
