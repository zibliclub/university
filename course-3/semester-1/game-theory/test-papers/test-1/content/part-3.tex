\section{Элементы теории игр}

\begin{definition}[Конфликт]
	Под \emph{конфликтом} обычно понимают любое явление, применительно к которому имеет смысл говорить, кто и как в этом явлении участвует, каковы его возможные исходы, как в этих исходах заинтересован и в чем эта заинтересованность состоит.
\end{definition}

\begin{definition}[Теория игр, игрок, игра]
	Математические модели и методы принятия решений в условиях конфликта составляют предмет \emph{теории игр}. Конфликтующие стороны назовем \emph{игроками}, а под \emph{игрой} будем понимать математическую модель конфликта.
\end{definition}

\begin{definition}[Ход, стретегия, ситуация, невозможная ситуация]
	Игра состоит из последовательности \emph{ходов}. \emph{Стратегией} игрока называют систему правил, определяющих его выбор варианта действия при каждом ходе. Комбинация стратегий всех игроков называется \emph{ситуацией}. Некоторые комбинации стратегий могут оказаться несовместимыми, и в этом случае говорят о \emph{невозможной ситуации}.
\end{definition}

\begin{definition}[Игра $n$ лиц в нормальной форме]
	Говорят, что задана \emph{игра $n$ лиц в нормальной форме}, если заданы:
	\begin{enumerate}
		\item Множество игроков $N = \{1,\ldots ,n\}$.
		\item Множества $X_i$ стратегий игроков, $i = 1,\ldots ,n$.
		\item Множество $X \subseteq X_1 \times \ldots \times X_n $ возможных ситуаций.
		\item Вектор-функция \emph{выигрыша} $H: \ X \rightarrow R^n$, ставящая каждой ситуации $x \in X$ вектор выигрышей $H(x) = \big(H_1(x),\ldots ,H_n(x)\big)$.
	\end{enumerate}
\end{definition}

\begin{definition}[Кооперативная игра]
	В случае, когда игроки в процессе игры могут образовывать коалиции и выбирать свои стратегии, преследуя общие цели по договоренности друг с другом, игра называется \emph{кооперативной}.
\end{definition}

\begin{definition}[Антагонистическая игра (с нулевой суммой), платежная функция]
	Игра называется \emph{антагонистической игрой} или \emph{игрой с нулевой суммой}, если $\forall x \in X, \ \forall y \in Y \ H_1(x,y)+H_2(x,y) = 0$. Функция $H(x,y) = H_1(x,y) = -H_2(x,y)$ называется \emph{платежной функцией}.
\end{definition}

\begin{definition}[Седловая точка]
	В случае антагонистической игры ситуация равновесия $(x^*,y^*)$ является \emph{седловой точкой} платежной функции:
	\[
		H(x,y^*) \leqslant H(x^*,y^*) \leqslant H(x^*,y), \ \forall x \in X, \ \forall y \in Y.
	\]
\end{definition}

\begin{definition}[Решение антагонистической игры]
	\emph{Решение антагонистической игры} -- пара оптимальных стратегий $(x^*,y^*)$, образующих седловую точку платежной функции.
\end{definition}

\begin{definition}[Нижняя, верхняя цена игры]
	Величины $u^0$ и $v^0$ называются \emph{нижней} и \emph{верхней ценой игры} соответственно.
\end{definition}

\begin{definition}[Принцип минимакса]
	Приведенные рассуждения носят название \emph{принципа минимакса}, который кратко может быть сформулирован следующим образом: каждый из игроков стремится максимально увеличить свой гарантированный выигрыш.
\end{definition}

\begin{equation}\label{eq:18}
	u(x) \leqslant v(y), \ \forall x \in X, \ \forall y \in Y
\end{equation}
\begin{equation}\label{eq:19}
	u^0 \leqslant v^0
\end{equation}

\begin{definition}[Основные неравенства, цена игры]
	Соотношения (\ref{eq:18})-(\ref{eq:19}) называются \emph{основными неравенствами}. В случае достижения равенства $u^0 = v^0$ эта величина называется \emph{ценой игры}.
\end{definition}

\begin{definition}[Матричная игра, чистые стратегии]
	Конечная антагонистическая игра называется \emph{матричной игрой}. Стратегии каждого игрока в матричной игре можно пронумеровать. Будем считать, что игрок $\RomanNumeralCaps{1}$ имеет стратегии $i = 1,\ldots ,m$, а игрок $\RomanNumeralCaps{2}$ -- стратегии $j=1,\ldots ,n$. В дальнейшем будем называть их \emph{чистыми стратегиями (ч.с.)}.
\end{definition}

\begin{definition}[Оптимальные ч.с., решение игры в ч.с., разрешимая игра в ч.с., цена игры в ч.с.]
	Седлова точка платежной матрицы дает \emph{ситуацию равновесия} $(p,q)$ в матричной игре, когда игроку $\RomanNumeralCaps{1}$ невыгодно отступать от своей максиминной ч.с. $p$, а игроку $\RomanNumeralCaps{2}$ -- от минимаксной ч.с. $q$, поскольку, отклоняясь от этих стратегий, игроки могут разве что уменьшить свой выигрыш.

	Если $(p,q)$ -- седлова точка, то стратегии $p,q$ называются \emph{оптимальными ч.с.} Пара $(p,q)$ оптимальных ч.с. называется \emph{решением игры в ч.с.}, а сама матричная игра -- \emph{разрешимой в ч.с.} Величина $u^0 = v^0$ называется в этом случае \emph{ценой игры в ч.с.}
\end{definition}

\begin{definition}[Смешанная стретегия (с.с.)]
	\emph{Смешанной стратегией (с.с.)} игрока в матричной игре называется вероятностное распределение на множестве его ч.с.
\end{definition}

\begin{definition}[Нижняя, верхняя цены игры в с.с., разрешимая в с.с., оптимальные в с.с., цена игры в с.с.]
	Числа $u^*, v^*$ называются соответственно \emph{нижней} и \emph{верхней ценой игры в с.с.}

	Матричная игра называется \emph{разрешимой в с.с.}, если $u^* = v^*$. Стратегии $x^*,y^*$, для которых $u(x^*) = u^* = v^* = v(y^*)$, называются \emph{оптимальными с.с.} Пара $(x^*,y^*)$ оптимальных с.с. образует \emph{ситуацию равновесия в с.с.}, а величина $u^* = v^*$ -- \emph{цена игры в с.с.} -- равна ожидаемому среднему выигрышу игрока $\RomanNumeralCaps{1}$ (и, соответственно, ожидаемому среднему проигрышу игрока $\RomanNumeralCaps{2}$).
\end{definition}
