\section{Динамическое программирование}

\begin{definition}[Динамическое программирование]
  \emph{Динамическое программирование (ДП)} -- раздел математического программирования, посвященный теории и методам решения многошаговых задач оптимального управления.
\end{definition}

\begin{definition}[Динамическая система, дискретная ДС]
  \emph{Динамическая система} -- это объект, способный развиваться во времени, переходя из одного состояния в другое. Если смена состояний в динамической системе происходит в отдельные дискретные моменты времени, то она называется \emph{дискретной динамической системой (ДС)}.
\end{definition}

\begin{definition}[Марковское свойство]
  Качество $f(u)$ управления $u$, под воздействием которого система переходит из состояния $x$ в состояние $y = \phi (x,u)$, зависит только от $x$ и $y$ и не зависит от того, каким образом ДС пришла в состояние $x$, называется \emph{марковским свойством}. Другими словами, $f(u) = f(x,y)$, где $y = \phi (x,u)$.
\end{definition}

\begin{definition}[Траектория из $x_0$ в $x_k$]
  \emph{Траекторией} из $x_0$ в $x_k$ называется набор состояний $T = (x_0,x_1,\ldots ,x_k)$, где $x_i = \phi (x_{i-1} ,u^i), \ i = \overline{1,k} $.
\end{definition}

\begin{definition}[Показатель качества траектории (вес траектории)]
  В силу марковского свойства мы можем определить \emph{показатель качества траектории} как
  \[
    f(T) = \sum_{i=1}^{k} f(u^i).
  \]
  Поскольку величина $f(T)$ складывается из весов управлений, входящих в траекторию, ее также называют \emph{весом траектории}.
\end{definition}

\begin{definition}[Принцип оптимальности]
  Для оптимальности всей последовательности управлений $(u^1,\ldots ,u^k)$ необходимо, чтобы завершающая часть этой последовательности $(u^i,\ldots ,u^k)$ также была оптимальной $\forall i: \ 1 \leqslant i \leqslant k$.

  Системы (процессы), для которых справедливо это свойство, называются \emph{марковскими}.
\end{definition}

\begin{definition}[Разрешимое, неразрешимое УБ]
  Решить УБ означает определить для любого $x \in X$ величины $L(x)$ и $R(x)$. Если это возможно, то УБ называется \emph{разрешимым}. Если же $\exists x \in X$, такой что $L(X) \ \big(R(x)\big)$ определить невозможно, то УБ называется \emph{неразрешимым}.
\end{definition}

\begin{definition}[Контур]
  \emph{Контуром} в графе переходов называется замкнутый ориентированный маршрут.
\end{definition}

\begin{definition}[Задача о замене оборудования]
  При эксплуатации станков, автомобилей, самолетов и дургих устройств возникает задача определения оптимальных сроков обновления их парка, известная как \emph{задача о замене оборудования}.

  Предприятие собирается использовать автомобиль в течение $n$ лет с возможной его заменой: в конце каждого года имеющийся автомобиль либо остается в эксплуатации, либо производится замена старого автомобиля на новый. Стоимость нового равна $d$, и в течение $n$ лет данная величина остается неизменной. Пусть для автомобиля возраста $t$ заданы $a(t)$ -- доход от использования автомобиля (в год), $b (t)$ -- ежегодные затраты на его эксплуатацию, $c(t)$ -- его ликвидная стоимость, $t = \overline{0,n-1} $. Требуется определить оптимальные сроки замены автомобиля с целью максимизации суммарной прибыли от его использования. 
\end{definition}
