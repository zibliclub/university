\setcounter{section}{3}
\section{Задача коммивояжера}


\begin{definition}[Маршрут коммивояжера]
  $n$ городов, каждому присвоен номер от $1$ до $n$. Известны расстояния $c_{ij} $ между городами $i$ и $j$, $i = \overline{1,n}, \ j = \overline{1,n}$. Если между городами $i$ и $j$, $i \ne j$, нет дороги, то $c_{ij} = \infty $. Вообще говоря, $c_{ij} \ne c_{ji} $. Коммивояжер (бродячий торговец), выезжая из какого-либо города, должен посетить все города, побывав в каждом из них ровно один раз, и вернуться в исходный город. Объезд городов, удовлетворяющий этим требованиям, называется \emph{маршрутом коммивояжера}.
\end{definition}

\begin{definition}[Гамильтонов цикл]
  Цикл, в который каждая вершина графа входит ровно один раз, называется \emph{гамильтоновым циклом}.
\end{definition}

\begin{definition}[Приведенная матрица, приведение матрицы, константы приведения]
  Матрицу $C^2 = \big(c_{ij}^{2} \big)$ называют \emph{приведенной матрицей}, операцию ее построения -- \emph{приведением} матрицы $C$, а величины $\alpha _i, \beta _j$ -- \emph{константами приведения}.
\end{definition}
