\newpage

\section{Задача сетевого планирования и управления}

\begin{definition}[Задача СПУ]
  \emph{Задача сетевого планирования и управления (задача СПУ)} возникает в строительстве, проектировании, на производстве и в других областях практической деятельности человека.

  Имеется проект, состоящий из взаимосвязанных работ. Для каждой работы известно ее время выполнения. Между работами есть логическая взаимосвязь: для каждой работы определен перечень работ, которые должны быть завершены к началу ее выполнения. Необходимо составить расписание выполнения работ, определяющее время начала каждой работы, следуя которому можно завершить весь проект за минимальное время $T_{\min} $.
\end{definition}

\begin{definition}[Событие, сетевая модель]
  \emph{Событием} назовем стадию выполнения проекта, соответствующую началу или окончанию некоторых работ. Тогда всю информацию о проекте, включая заданное отношение частичного порядка на множестве работ, удобно представить в форме ориентированного графа $G = (X,U)$, который назыается \emph{сетевой моделью}.
\end{definition}

\begin{definition}[Свершенное событие]
  Будем говорить, что событие \emph{свершилось}, если завершены все работы, предшествовавшие этому событию, то есть дуги, входящие в вершину-событие. После того, как событие свершилось, могут начинаться работы-дуги, выходящие из соответствующей вершины.
\end{definition}

\begin{definition}[Непротиворечивая сетевая модель]
  В графе $G$ должны присутствовать замкнутые ориентированные последовательности дуг -- контуры. Такая сетевая модель называется \emph{непротиворечивой}.
\end{definition}

\begin{definition}[Расписание, условие допустимости]
  \emph{Расписание} -- это набор чисел $\big\{t(x,y)\big\}$, которые сопоставлены каждой дуге $(x,y) \in U$ и задают время начала работы $(x,y)$, для которых имеет место \emph{условие допустимости}: если работа $(x,y)$ предшествует работе $(y,z)$, то
  \[
    t(y,z) \geqslant t(x,y) + \tau (x,y).
  \]
\end{definition}

\begin{definition}[Ранний момент $T_x^P$]
  \emph{Ранний момент $T_{x}^{P} $} совершения события $x$ -- это минимальный момент времени, в который может свершиться это событие.
\end{definition}

\begin{definition}[Поздний момент $T_x^{\Pi}$]
  \emph{Поздний момент $T_{x}^{\Pi} $} совершения события $x$ -- это максимальный момент времени совершения этого события, не приводящий к увеличению времени выполнения всего проекта сверх $T_{\min} $. 
\end{definition}

\begin{definition}[Критическое событие]
  Событие называется \emph{критическим}, если $T_{x}^{P}= T_{x}^{\Pi} $. Очевидно, что начало проекта $\alpha $ и окончание проекта $\beta $ являются критическими событиями:
  \[
    T_{\alpha }^{P} = T_{\alpha }^{\Pi} = 0, \quad T_{\beta }^{P}  = T_{\beta }^{\Pi}  = T_{\min} . 
  \]
\end{definition}

\begin{definition}[Полный резерв работы, критическая работа]
  \emph{Полным резервом работы} $(x,y)$ называется величина
  \[
    R(x,y) = \big(T_{y}^{\Pi} - \tau (x,y)\big) - T_{x}^{P} .
  \]

  Работа называется \emph{критической}, если $R(x,y) = 0$.
\end{definition}

\begin{definition}[Критический путь]
  \emph{Критический путь} -- это ориентированная последовательность критических работ от начального события $\alpha $ к конечному $\beta $.
\end{definition}

\begin{definition}[Раннее расписание]
  \emph{Раннее расписание} $\big\{t^P(x,y)\big\}$ -- это расписание, в котором каждой работе $(x,y)$ приписан момент ее начала, раньше которого она не может начаться, то есть
  \[
    t^P(x,y) = T_{x}^{P} , \ \forall (x,y) \in U.
  \]
\end{definition}

\begin{definition}[Позднее расписание]
  \emph{Позднее расписание} $\big\{t^\Pi(x,y)\big\}$ -- это расписание, в котором каждой работе $(x,y)$ приписан момент ее начала, позже которого она начаться не должна (так как в противном случае будет нарушен срок окончания проекта $T_{\min} $), то есть
  \[
    t^\Pi(x,y) = T_{y}^{\Pi} - \tau (x,y), \ \forall (x,y) \in U. 
  \]
\end{definition}
