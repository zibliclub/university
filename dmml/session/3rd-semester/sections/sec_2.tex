\section{Логика высказываний}

\subsection{Парадоксы в математике. Парадоксы Г. Кантора и Б. Рассела.}

\begin{statement}[Рассел]
    Множество $M$ будем называть \emph{нормальным}, если оно не принадлежит самому себе как элемент.

    Например, множество кошек нормально, поскольку множество кошек не является кошкой. А вот каталог каталогов по-прежнему остается каталогом, поэтому множество каталогов не является нормальным.

    Рассмотрим теперь множество $B$, составленное из всевозможных нормальных множеств. Формально множество $B$ определяется так:
    \[
        x \in B \iff x \notin x
    \]

    Возникает вопрос: будет ли $B$ принадлежать самому себе как элемент?

    И тут возникает парадокс: дело в том, что если вместо $x$ подставить $B$, то возникнет явное противоречие:
    \[
        B \in B \iff B \notin B
    \]
\end{statement}

\begin{statement}[Кантор]
    Предположим, что множество всех множеств $V = \{x \ \big| x = x\}$ -- существует. В этом случае справедливо
    \[
        \forall x,T \quad x \in T \implies x \in V,
    \]
    то есть всякое множество $T$ является подмножеством $V$. Но из этого следует, что $\forall T \ |T| \leqslant |V|$ -- мощность любого множества не превосходит мощности $V$.

    Но в силу аксиомы множества всех подмножеств, для $V$, как и любого множества, существует множество всех подмножеств $\mathcal{P}(V)$, и по теореме Кантора:
    \[
        |\mathcal{P}| = 2^{|V|} > |V|,
    \]
    что противоречит предыдущему утверждению.

    Следовательно, $V$ не может существовать, что вступает в противоречие с «наивной» гипотезой о том, что
    любое синтаксически корректное логическое условие определяет множество, то есть что
    \[
        \exists y \forall z \quad z \in y \iff A
    \]
    для любой формулы $A$, не содержащей $y$ -- свободно.
\end{statement}

\subsection{Логическое следование в логике высказываний. Проверка логического следования с помощью таблиц истинности и эквивалентных преобразований.}

\begin{definition}[Интерпретация переменных]
    \emph{Интерпретация переменных} -- это отображение вида:
    \[
        \alpha: \{x_1,\ldots,x_n\}\rightarrow \{0,1\}
    \]

    Задать интерпретацию -- приписать $j$-ой переменной значение $0,1$.
\end{definition}

\begin{note}
    Если $\Phi$ -- формула, а $\alpha$ -- интерпретация, то $\Phi^\alpha$ -- значение формулы, когда вместо $x_i$ подставили $\alpha(x_i)$.

    Первый способ определить математическое понятие доказательства -- логическое следование.
\end{note}

\begin{definition}[Логическое следование]
    Пусть $\Gamma$ -- множество формул, $\Phi$ -- формула логики высказываний. Формула $\Phi$ \emph{логически следует} из $\Gamma$, если для любой интерпретации $\alpha_k$ верно, что если истинны все формулы из $\Gamma$ при этой интерпретации, то истинна и $\Phi$:
    \[
        \forall \alpha \quad (\forall \psi \in \Gamma \ \psi^\alpha = 1) \implies \Phi^\alpha = 1
    \]
    \[
        \text{Обозначение: }\Gamma \vDash \Phi
    \]
\end{definition}

\begin{remark}
    Проверять логическое следование можно при помощи таблиц истинности и эквавалентных преобразований, пользуясь свойством:
    \[
        \text{Если }\Gamma = \{\Phi_1,\ldots,\Phi_n\}\text{ -- конечное, то }\Gamma \vDash\Phi \iff \Phi_1\land\ldots\land\Phi_n \rightarrow \Phi\equiv 1
    \]
    (проверить, является ли импликация тождественно истинной функцией или нет).
\end{remark}

\subsection{Понятия прямой теоремы, а также противоположной, обратной и обратной к противоположной теорем.}

\begin{definition}
    Многие математические теоремы имеют структуру, выражаемую формулой $X \rightarrow Y$. Утверждение $X$ называется \emph{условием} теоремы, а утверждение $Y$ -- ее \emph{заключением}.

    Если некоторая теорема имеет форму $X \rightarrow Y$, утверждение $Y \rightarrow X$ называется \emph{обратным} для данной теоремы. Это утверждение может быть справедливым, и тогда оно называется теоремой, \emph{обратной} для теоремы $X \rightarrow Y$, которая, в свою очередь, называется \emph{прямой} теоремой.

    Для теоремы, сформулированной в виде импликации $X \rightarrow Y$, кроме обратного утверждения $Y \rightarrow X$ можно сформулировать \emph{противоположное} утверждение. Им называется утверждение вида $\lnot X \rightarrow \lnot Y$. Утверждение, противоположное данной теореме, может быть также теоремой, то есть быть истинным высказыванием, но может таковым и не быть.

    Теорема, \emph{обратная противоположной}: $\lnot Y \rightarrow \lnot X$ (\emph{контрпозиция}).
\end{definition}

\subsection{Понятия необходимых и достаточных условий.}

\begin{definition}[Необходимое и достаточное условия]
    Если некоторая математическая теорема имеет структуру, выражаемую формулой $X \rightarrow Y$, то высказывание $Y$ называется \emph{необходимым} условием для высказывания $X$ (другими словами, если $X$ -- истинно, то $Y$ с необходимостью также должно быть истинным), а высказывание $X$ называется \emph{достаточным} условием для высказывания $Y$ (другими словами, для того, чтобы $Y$ было истинным, достаточно, чтобы истинным было высказывание $X$).
\end{definition}

\subsection{Формальные системы. Выводы в формальных системах. Свойства выводов.}

\begin{definition}[Формальная система]
    \emph{Формальная система} состоит из четырех элементов:
    \begin{enumerate}
        \item Алфавит (некоторое множество).
        \item Набор формул (множество слов, отобранных с помощью некоторых правил).
        \item Набор аксиом (множество формул, отобранных по некоторым правилам).
        \item Набор правил вывода вида $\frac{\phi_1,\ldots,\phi_n}{\psi}$ (из формул $\phi_1,\ldots,\phi_n$ следует формула $\psi$).
    \end{enumerate}
\end{definition}

\begin{definition}[Вывод]
    \emph{Вывод} формулы $\phi$ из множества формул $\Gamma$ в формальной системе -- это конечная последовательность формул $\phi_1,\ldots,\phi_n = \phi$, в которой каждая $\phi_i$:
    \begin{itemize}
        \item либо аксиома формальной системы;
        \item либо принадлежит множеству $\Gamma$ (является гипотезой);
        \item либо получена из предыдущих формул по одному из правил вывода.
    \end{itemize}
\end{definition}

\begin{definition}[Выводимость]
    Формула $\phi$ выводится из множества формул $\Gamma$, если существует вывод $\phi$ из $\Gamma$.
    \[
        \text{Обозначение: }\Gamma\vdash\phi
    \]
\end{definition}

\begin{statement}[Свойства выводов]\leavevmode
    \begin{enumerate}
        \item Если $\Gamma\vdash\phi$, то существует конечное подмножество $\Gamma_0 \subset \Gamma$ такое, что $\Gamma \vdash\phi$ (выделение конечного подмножества).
        \item Если $\Gamma\vdash\phi$ и $\Gamma\subset\Delta$, то $\Delta\vdash\psi$ (увеличение множества гипотез).
        \item Если $\Gamma\vdash\Delta$, то есть все формулы из $\Delta$ выводятся из $\Gamma$, и $\Delta\vdash\phi$, то и $\Gamma\vdash\phi$ (транзитивность выводимости).
    \end{enumerate}
\end{statement}

\subsection{Исчисление высказываний (ИВ) Гильберта. Примеры выводов.}

\begin{definition}[Исчисление высказываний (ИВ)]
    \emph{Исчисление высказываний} -- конкретная формальная система на базе логики высказываний.
    \begin{enumerate}
        \item Алфавит $ = $ символы переменных, отрицание, импликация, скобки.
        \item Формулы ИВ -- формулы языка ЛВ, использующие только отрицание и импликацию.
        \item Аксиомы ИВ (схемы аксиом):
              \[
                  \begin{array}{rl}
                      A_1 & A \rightarrow (B \rightarrow A),                                                                                 \\
                      A_2 & \big(A \rightarrow (B \rightarrow C)\big) \rightarrow \big((A \rightarrow B) \rightarrow (A \rightarrow C)\big), \\
                      A_3 & (\lnot B \rightarrow \lnot A) \rightarrow \big((\lnot B \rightarrow A) \rightarrow B \big).
                  \end{array}
              \]
        \item Силлогизм (modus ponens):
              \[
                  \frac{A, \ A \rightarrow B}{B}.
              \]
    \end{enumerate}
\end{definition}

\begin{example}
    $ A, \ A \rightarrow B \vdash B $
    \begin{enumerate}
        \item $ A $.
        \item $ A \rightarrow B $.
        \item $ B $ (MP 1,2).
    \end{enumerate}
\end{example}

\begin{example}
    $ A \vdash B \rightarrow A $
    \begin{enumerate}
        \item $ (A_1) \ A \rightarrow (B \rightarrow A) $.
        \item $ A $.
        \item $ B \rightarrow A $ (MP 1,2).
    \end{enumerate}
\end{example}

\begin{remark}
    Если $ \Gamma = \emptyset $, то пишем $ \vdash \phi $ ($ \phi $ доказуема).
\end{remark}

\subsection{Теорема о дедукции для ИВ.}

\begin{theorem}
    $ \Gamma $ -- множество формул, $ A,B $ -- формулы ИВ. Тогда:
    \[
        \Gamma, A \vdash B \iff \Gamma \vdash A \rightarrow B.
    \]
\end{theorem}

\subsection{Теорема о полноте и непротиворечивости ИВ.}

\begin{theorem}[О полноте ИВ]
    $ \vdash A \iff A $ -- тавтология.
\end{theorem}

\begin{theorem}[О непротиворечивости ИВ]
    ИВ Гильберта непротиворечиво.
\end{theorem}

\subsection{Метод резолюций для логики высказываний.}

\begin{note}
    Правило резолюции:
    \[
        \frac{x \lor \phi, \ \lnot x \lor \phi}{\phi \lor \psi}, \quad (x,\lnot x)\text{ -- контарная пара}.
    \]

    Частный случай:
    \[
        \frac{x, \ \lnot x}{\square}\text{ -- пустой дизъюнкт}.
    \]
\end{note}

\begin{theorem}
    $ \Gamma \vdash \phi \iff \Gamma \cup \{\lnot \phi\} $ -- противоречиво.
\end{theorem}

\begin{remark}[Метод резолюций]\leavevmode
    \begin{enumerate}
        \item Формируем множество $ \Gamma $:
              \[
                  \Gamma = \{A_1,\ldots,A_m, \lnot B\}.
              \]
        \item Все ФЛВ из $ \Gamma $ приводим к КНФ. Получится набор дизъюнктов:
              \[
                  D_1,\ldots,D_k.
              \]
        \item Все дизъюнкты $ D_1,\ldots,D_k $ выписываем в столбик.
        \item Будем получать новые дизъюнкты по следующему правилу.

              Пусть имеются два дизъюнкта вида:
              \[
                  \begin{array}{l}
                      x_i \lor P, \\
                      \lnot x_i \lor Q,
                  \end{array}
              \]
              где $ x_i $ -- переменная $ P, \ Q $ -- некоторые выражения. Тогда выписываем дизъюнкт:
              \[
                  P \lor Q
              \]
              (то есть пара $ x_i,\lnot x_i $ уничтожается, а оставшиеся части склеиваются в новом дизъюнкте).

              Дизъюнкт $ P \lor Q $ называется \emph{резольвентой дизъюнктов} $ x_i \lor P, \ \lnot x_i \lor Q $. В частности, если имеется пара дизъюнктов $ x_i, \ \lnot x_i $, то получается так называемый \emph{пустой дизъюнкт}.

        \item Предыдущий шаг алгоритма повторяется до тех пор, пока:
              \begin{itemize}
                  \item появляются новые дизъюнкты,
                  \item пока не появился пустой дизъюнкт.
              \end{itemize}

        \item Если появился пустой дизъюнкт, то логическое следование есть. Если же новые дизъюнкты не появляются, а пустой дизъюнкт так и не был получен, то логического следования -- нет.
    \end{enumerate}
\end{remark}

\begin{remark}
    Правило резолюции помогает убрать только одну контрарную пару.
\end{remark}

\begin{remark}
    Тождественно истинные формулы в выводе $ \square $ не используются.
\end{remark}

\newpage