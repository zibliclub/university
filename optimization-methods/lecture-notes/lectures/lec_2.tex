\lesson{2}{от 16 фев 2024 8:45}{Продолжение}

\begin{example}
    Задача Канторовича

    Производятся различные виды шпона с помощью станков разной производительности в единицу времени.

    Как распределить задание между станками, чтобы получить шпон в нужном ассоритменте в наибольшем количестве?

    \[
        \begin{array}{lll}
            n & \text{видов шпона} & j = 1\ldots n \\
            m & \text{станков}     & i = 1\ldots m
        \end{array}
    \]
    \[
        \begin{array}{ll}
            a_{ij} & \text{ед. шпона }j \text{-го вида, производимое }i \text{-м станков в ед. времени} \\
            t_i    & \text{лимит времени работы }i \text{-го станка}                                    \\
            b_j    & \text{количество ед. шпона }j \text{-го вида в комплекте}
        \end{array}
    \]

    Максимизировать число комплектов.

    Пусть $ z $ -- число комплектов, $ x_{ij} $ -- количество единиц шпона $ j $-го вида, производимого на $ i $-м станке ($ x_{ij} $ -- время $ i $-го станка на пространство $ j $-го продукта).

    \[
        z \rightarrow \max,
    \]
    \[
        \begin{array}{l}
            \sum_{i=1}^{m}x_{ij} \geqslant b_j z, \ j = 1\ldots n \\
            \sum_{j=1}^{n}\frac{x_{ij}}{a_{ij}} \leqslant t_i, \ i = 1\ldots m
        \end{array}
    \]
    \[
        x_{ij} \geqslant 0, \ z \geqslant 0, \ i = 1\ldots m, \ j = 1 \ldots n, \quad z \in \Z.
    \]
\end{example}

\subsection{Теорема эквивалентности задач ЛП}

\begin{definition}[Эквивалентные задачи МП]
    Две задачи МП
    \[
        \overset{(\RomanNumeralCaps{1})}{\left\{\begin{array}{l}
                f(x) \rightarrow opt \\
                x \in D
            \end{array}\right.}, \quad \overset{(\overline{\RomanNumeralCaps{1}})}{\left\{\begin{array}{l}
                \overline{f}(\overline{x}) \rightarrow \overline{opt} \\
                \overline{x} \in \overline{D}
            \end{array}\right.}, \quad \begin{array}{l}
            D \xrightarrow[]{\phi} \overline{D}            \\
            \overline{D} \xrightarrow[]{\overline{\phi}} D \\
        \end{array}
    \]
    называются \emph{эквивалентными}, если любому допустимому решению каждой из них по некоторому правилу соответствует допустимое решение другой задачи, причем оптимальному решению соответствует оптимальное.
\end{definition}

\begin{theorem}[Первая теорема эквивалентности]
    Для любой задачи ЛП $ \exists $ эквивалентная ей каноническая ЗЛП.
\end{theorem}

\newpage

\begin{note}[Идея доказательства]
    $ n = 2, \ m = 3 $
    \begin{multicols}{2}
        $ f = c_1 x_1 + c_2 x_2 \rightarrow \min $
        \[
            \begin{array}{l}
                a_{11}x_1 + a_{12}x_2 = b_1         \\
                a_{21}x_1 + a_{22}x_2 \leqslant b_2 \\
                a_{31}x_1 + a_{32}x_2 \geqslant b_3 \\
                x_1 \leqslant 0                     \\
                x_2 \in \R
            \end{array}
        \]

        $ \overline{f} = -c_1 x_1 - c_2 x_2 \rightarrow \max $
        \[
            \begin{array}{l}
                a_{11}x_1 + a_{12}x_2 = b_1                           \\
                a_{21}x_1 + a_{22}x_2 + x_3 = b_2                     \\
                a_{31}x_1 + a_{32}x_2 - x_4 = b_3                     \\
                x_1 \geqslant 0, \ x_3 \geqslant 0, \ x_4 \geqslant 0 \\
                x_2  = x_2' - x_2'', \ x_2' \geqslant 0, \ x_2'' \geqslant 0
            \end{array}
        \]
    \end{multicols}

    КЗЛП

    $ \overline{f} = - c_1 x_1 - c_2x_2' + c_2x_2'' \rightarrow \max $
    \[
        \begin{array}{l}
            a_{11}x_1 + a_{12}x_2' - a_{12}x_2'' = b_1       \\
            a_{21}x_1 + a_{22}x_2' - a_22x_2'' + x_3 = b_2   \\
            a_{31}x_1 + a_{32}x_2' - a_{32}x_2'' - x_4 = b_3 \\
            x_1 \geqslant 0, \ x_2',x_2'',x_3,x_4 \geqslant 0
        \end{array}
    \]

    Неоднозначность-разность: $ \forall x \in D \ f(x) = -f(\overline{x}), \ \overline{x} \in \overline{D} $
    \[
        \overline{x} = \phi(x).
    \]

    Очевидно, что оптимальность также сохраняется при таких преобразованиях.
\end{note}

\begin{theorem}[Вторая теорема эквивалентности]\label{th:2}
    Для любой задачи ЛП $ \exists $ эквивалентная ей симметричная задача ЛП.
\end{theorem}

\begin{note}[Идея]
    \[
        \alpha = \beta \iff \left\{\begin{array}{l}
            \alpha \leqslant \beta \\
            \alpha \geqslant \beta
        \end{array}\right. \quad \begin{array}{l | l}
            (c,x)\rightarrow\max & (c,x) \rightarrow \min \\
            Ax \leqslant b       & Ax \geqslant b         \\
            x \geqslant 0        & x \geqslant 0
        \end{array}
    \]
\end{note}

\begin{remark}
    Смысл теоремы \ref{th:2} в том, чтобы свести решение ЗЛП к КЗЛП.
\end{remark}

\begin{note}[Геометрическая интерпретация]
    $ n = 2 $

    $ f = c_1x_1 + c_2x_2 \rightarrow \max, \quad a_{i1}x_1 + a_{i2}x_2 \leqslant b_i, \ i = 1\ldots m$

    Линии уровня целевой функции
    \[
        c_1x_1 + c_2x_2 = const, \quad \perp \triangledown f = (c_1,c_2).
    \]
    \[
        \begin{array}{ccc}
                                 &          & \exists ! x^* \text{ -- оптимальное решение} \\
                                 & \nearrow &                                              \\
            \text{ЗЛП разрешима} &          &                                              \\
                                 & \searrow &                                              \\
                                 &          & \text{бесконечно много опт. решений}
        \end{array}
    \]
    \[
        \begin{array}{ccc}
                                   &          & f \rightarrow +\infty \text{ на }D \text{ (неогр. сверху на }D \text{)} \\
                                   & \nearrow &                                                                         \\
            \text{ЗЛП неразрешима} &          &                                                                         \\
                                   & \searrow &                                                                         \\
                                   &          & D = \O \text{ нет дополнительных решений}
        \end{array}
    \]
\end{note}

\section{Базисные решения КЗЛП}

\begin{note}\leavevmode
    \begin{enumerate}
        \item $ f = (c,x) \rightarrow \max $.
        \item\label{item:1.2.2} $ Ax = b $.
        \item $ x \geqslant \overline{0} $.
    \end{enumerate}
    \[
        A_{m\times n} = (A^1,A^2,\ldots,A^n), \quad A^j = \left(\begin{array}{c}
                a_{1j} \\ a_{2j} \\ \vdots \\ a_{mj}
            \end{array}\right) \text{ -- }j \text{-ый столбец матрицы }A.
    \]
\end{note}

\begin{definition}[Базисное решение системы \ref{item:1.2.2}]
    Пусть $ \overline{x} $ -- решение системы \ref{item:1.2.2}. Вектор $ \overline{x} $ называется \emph{базисным решением системы \ref{item:1.2.2}}, если система векторных столбцов матрицы $ A $, соответствующая ненулевым компонентам вектора $ \overline{x} $, линейно независима.
\end{definition}

\begin{remark}
    В случае однородной системы ($ b = 0 $), решение $ x = 0 $ является базисным.
\end{remark}

\begin{definition}[Базисное решение КЗЛП]
    Неотрицательное базисное решение системы \ref{item:1.2.2} называется \emph{базисным (опорным) решением КЗЛП}.
\end{definition}

\begin{example}
    $ 3x_1 - 4x_2 + x_3 \rightarrow \max $
    \[
        \left\{\begin{array}{cccccc}
            2x_1 & + 2x_2 & + 3x_3 & - x_4 & + x_5  & = 1 \\
            2x_1 & + 4x_2 &        & + x_4 & + 2x_5 & = 2
        \end{array}\right.
    \]
    \[
        A = \left(\begin{matrix}
                2 & 2 & \textbf{3} & \textbf{-1} & 1 \\
                2 & 4 & \textbf{0} & \textbf{1}  & 2
            \end{matrix}\right)
    \]
    $ x^1 = (0,0,1,2,0) $ -- базисное решение системы, так как $ \left|\begin{matrix}
            3 & -1 \\ 0 & 1
        \end{matrix}\right| \ne 0 $ соответствует базису $ \{A^3,A^4\} $.
    \[
        \begin{array}{l}
            x^1 \text{ БР КЗЛП}                                                 \\
            x^2 = \left(1,0,-\frac{1}{3},0,0\right) \text{ БР СЛАУ, но не КЗЛП} \\
            x^3 = (0,0,0,0,1) \text{ БР КЗЛП}
        \end{array}
    \]
\end{example}

\begin{definition}[Вырожденное базисное решение]
    $ x $ -- базисное решение КЗЛП называется \emph{вырожденным}, если число ненулевых компонент вектора $ x $ меньше ранга матрицы $ A $.
\end{definition}

\begin{remark}
    $ x^3 $ -- вырожденное. Недостаток: соответствует разным наборам базисных столбцов матрицы.

    $ x^3 $ соответствует $ \{A_1,A_5\}, \{A_3,A_5\}, \{A_4,A_5\} $.
\end{remark}