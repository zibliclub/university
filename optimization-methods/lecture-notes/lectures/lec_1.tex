\lesson{1}{от 9 фев 2024 8:45}{Начало}

\begin{definition}[Методы оптимизации]
    \emph{Методы оптимизации} -- раздел прикладной математикик, предметом изучения которого является теория и методы оптимизационных задач.
\end{definition}

\begin{definition}[Оптимизационная задача]
    \emph{Оптимизационная задача} -- задача выбора из множества возможных ваариантов наилучших в некотором смысле.
\end{definition}

\begin{note}
    \[
        \left\{\begin{array}{l}
            f(x) \rightarrow \min(\max) \\
            x \in D
        \end{array}\right.,
    \]
    где
    \[
        \begin{array}{rl}
            D       & \text{-- множество допустимых решений, }         \\
            x \in D & \text{-- допустимое решение, }                   \\
            f(x)    & \text{-- целевая функция (критерий оптимизации)}
        \end{array}
    \]
\end{note}

\section*{Задачи математического программирования (МП) и их классификация}

\begin{note}
    Немного истории:
    \begin{center}
        \boxed{\begin{array}{l}
                1939\text{г. Л.В. Конторович} \\
                1947\text{г. Д. Данциг}       \\
            \end{array}}
    \end{center}

    С 50-х годов -- бурное развитие
    \begin{center}
        \boxed{$ 1975\text{г. Нобелевская премия по экономике Конторовичу и Купмаксу} $}
    \end{center}
\end{note}

\begin{note}[Задача математического программирования]\leavevmode
    \begin{enumerate}
        \item \label{task:1} $ f(x) \rightarrow \max(\min) $.
        \item \label{task:2} $ g(x) \# 0, \ i = \overline{1,m}, \quad \# \in \{\leqslant,\geqslant,=\} $.
        \item \label{task:3} $ \underset{(x\in R^n)}{x_j \in R}, \ j = \overline{1,n} $.
    \end{enumerate}
    \[
        x = (x_1,\ldots,x_n)
    \]
\end{note}

\begin{definition}[Оптимальное решение, глобальный экстремум]
    $ x^* \in D $ называется \emph{оптимальным решением} задачи \ref{task:1}--\ref{task:3}, если $ \forall x \in D $
    \[
        f(x^*) \geqslant f(x)
    \]
    для задачи на $ \max $ и $ \forall x \in D $
    \[
        f(x^*) \leqslant f(x)
    \]
    для задачи на $ \min $.

    $ x^* $ является \emph{глобальным экстремумом}.
\end{definition}

\begin{definition}[Разделимая, неразделимая задача]
    Задача \ref{task:1}--\ref{task:3}, которая обладает оптимальным решением, называется \emph{разделимой}, и \emph{неразделимой} в противном случае.

    $ D = R^n $ -- задача \emph{безусловной оптимизации}, в противном случае -- задача \emph{условной оптимизации}.
\end{definition}

\begin{note}[Классификация]\leavevmode
    \begin{enumerate}
        \item Если $ f,g_i $ являются линейными, то задача является задачей \emph{линейного программирования} (ЛП).
        \item Если хотя бы одна из функций $ f,g_i $ нелинейная, то задача \emph{нелинейного программирования}.
    \end{enumerate}
    $ f,g_i $ -- выпуклые, то \emph{выпуклого программирования}.
\end{note}

\chapter{Линейное программирование}

\section{Постановка задачи, теорема эквивалентности}

\begin{definition}[Общая задача ЛП (ЗЛП)]
    \[
        f(x) = c_0 + \sum_{j=1}^{n}c_jx_j \longrightarrow \max(\min),
    \]
    \[
        \sum_{j=1}^{n}a_{ij}x_j \# b_i, \quad i = \overline{1,n}, \ \# \in \{\leqslant,\geqslant,=\}
    \]
    \[
        x_j \geqslant 0, \quad j \in \mathfrak{I} \subseteq \{1,\ldots,n\}
    \]
    \[
        A_{m\times n} = \left(\begin{matrix}
                a_{11} & a_{12} & \cdots & a_{1n} \\
                \vdots & \vdots & \ddots & \vdots \\
                a_{m1} & a_{m2} & \cdots & a_{mn}
            \end{matrix}\right), \ b = \left(\begin{matrix}
                b_1 \\ \vdots \\ b_m
            \end{matrix}\right), \ x = \left(\begin{matrix}
                x_1 \\ \vdots \\ x_n
            \end{matrix}\right) \text{ -- }\begin{array}{ll}
            \text{переменные} \\ \text{задачи}
        \end{array}
    \]
\end{definition}

\begin{note}[Матричная задача]
    \[
        f(x) = (c,x) \longrightarrow \max(\min)
    \]
    \[
        Ax \# b
    \]
    \[
        x_i \geqslant 0, \quad j \in \mathfrak{I} \subseteq \{1,\ldots,n\}
    \]
\end{note}

\begin{note}[Каноническая ЗЛП (КЗЛП)]
    \[
        f(x) = (c,x) \longrightarrow \max
    \]
    \[
        Ax = b
    \]
    \[
        x \geqslant \vec{0}, \quad \vec{0} = (0,\ldots,0)
    \]
\end{note}

\begin{note}[Симметричная ЗЛП]
    \[
        \begin{array}{ccc}
            f(x) = (c,x) \longrightarrow \max                 &            & f(x) = (c,x)\longrightarrow \min \\
            Ax \leqslant b                                    & \text{или} & Ax \geqslant b                   \\
            x \geqslant \vec{0}, \quad \vec{0} = (0,\ldots,0) &            & x \geqslant \vec{0}
        \end{array}
    \]
\end{note}

\begin{remark}
    Без ограничения общности далее положим $ c_0 = 0 $, так как добавление константы не влияет на процесс нахождения оптимального решения.
\end{remark}

\subsection{Примеры моделей ЛП}

\begin{example}
    Задача о составлении оптимального плана производства.
    \[
        \begin{array}{ll}
            m \text{ ресурсов},        & i = \overline{1,m} \\
            n \text{ видов продукции}, & j = \overline{1,n}
        \end{array}
    \]

    Известно:
    \[
        \begin{array}{l}
            b_i \text{ -- запас }i\text{-го ресурса, }i = \overline{1,m}                      \\
            a_{ij} \text{ -- }\begin{array}{l}
                                  \text{количество ресурса }i\text{, требуемое для производства } \\
                                  1\text{ единицы продукции вида }j
                              \end{array} \\
            c_j \text{ -- прибыль от продажи }1\text{ единицы }j\text{-го продукта}           \\
        \end{array}
    \]

    Необходимо составить план производства, максимализирующий суммарную прибыль.

    Переменные: $ x_j $ единицы продукции вида $ j $ производства, $ j = \overline{1,n} $,
    \[
        \sum_{j=1}^{n}c_jx_j \longrightarrow \max,
    \]
    \[
        \sum_{j=1}^{n}a_{ij}x_j \leqslant b_i, \ i = \overline{1,m},
    \]
    \[
        x_j \geqslant 0, \quad j = \overline{1,n}.
    \]
\end{example}

\begin{example}
    О максимальном потоке в сети.
    \[
        \begin{array}{ll}
            G = (V,E)          & \text{-- ориентированный взвешенный граф}    \\
            c: E \rightarrow R & \text{-- веса дуг -- пропускная способность}
        \end{array}
    \]
    \[
        \begin{array}{ll}
            s & \text{-- источник} \\
            t & \text{-- сток}
        \end{array}
    \]

    Пусть $ x_{ij} $ -- поток по дуге $ (i,j)\in E $
    \[
        f = \sum_{j: (s,j)\in E} x_{sj} \longrightarrow \max,
    \]
    \[
        \sum_{j:(j,i)\in E}x_{ji} = \sum_{k:(i,k)\in E}x_ik, \quad i \in V \setminus\{s,t\},
    \]
    \[
        0 \leqslant x_{ij} \leqslant c_{ij}, \quad (i,j) \in E.
    \]
\end{example}