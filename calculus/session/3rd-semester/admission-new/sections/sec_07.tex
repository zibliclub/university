\section{Свойства предельной функции}

\setcounter{subsection}{85}

\subsection{Условия коммутирования двух предельных переходов}

\begin{theorem}\label{theorem:6.3}
    Пусть $ X,T $ -- множества, $ \mathfrak{B}_x $ -- база на $ X, \ \mathfrak{B}_T  $ -- база на $ T $, $ Y $ -- полное МП, $ f_t: X \rightarrow Y, \ f: X \rightarrow Y $:
    \begin{itemize}
        \item $ f_t \xrightrightarrows[\mathfrak{B}_T]{} f $ на $ X $,
        \item $ \forall t \in T \ \exists \underset{\mathfrak{B}_X}{\lim} = A_t $,
    \end{itemize}
    тогда существуют и равны два повторных предела:
    \[
        \underset{\mathfrak{B}_T}{\lim}\underset{\mathfrak{B}_X}{\lim} f_t(x) = \underset{\mathfrak{B}_X}{\lim}\underset{\mathfrak{B}_T}{\lim} f_t(x).
    \]

    Запишем условия и утверждение теоремы в форме диаграмы:
    \[
        \begin{matrix}
            f_t(x)                                                      & \xrightrightarrows[\mathfrak{B}_T]{} & f(x)                                               \\
            {\scriptstyle\forall t, \ \mathfrak{B}_X} \xdownarrow[22pt] &                                      & \xdashdownarrow[20pt] {\scriptstyle\mathfrak{B}_X} \\
            A_t                                                         & \xdashrightarrow[\mathfrak{B}_T]{}   & A
        \end{matrix}
    \]
    \[
        \rightarrow \text{ -- дано,}\quad \dashrightarrow \text{ -- утверждение}
    \]
\end{theorem}

\newpage

\subsection{Непрерывность предельной функции}

\begin{theorem}
    Пусть $ X,Y $ -- метрические пространства, $ \mathfrak{B} $ -- база на $ T, \ f_t: X \rightarrow Y, \ f: X \rightarrow Y $:
    \begin{itemize}
        \item $ \forall t \in T $ функция $ f_t $ непрерывна в точке $ x_0 \in X $,
        \item семейство $ f_t \xrightrightarrows[\mathfrak{B}]{} f $ на $ X $,
    \end{itemize}
    тогда функция $ f $ непрерывна в точке $ x_0 $.
\end{theorem}

\subsection{Интегрируемость предельной функции}

\begin{theorem}\label{theorem:6.9.2}
    Пусть $ f_t: [a;b] \rightarrow \R, \ f: [a;b] \rightarrow \R $:
    \begin{itemize}
        \item $ \forall t \in T \ f_t $ интегрируема по Риману на $ [a;b] $,
        \item $ f_t \xrightrightarrows[\mathfrak{B}]{} f $ на $ [a;b] $ ($ \mathfrak{B} $ -- база на $ T $),
    \end{itemize}
    тогда:
    \begin{enumerate}
        \item $ f $ интегрируема по Риману на $ [a;b] $.
        \item \[
                  \int_{a}^{b}f(x)dx = \underset{\mathfrak{B}}{\lim}\int_{a}^{b}f_t(x)dx \iff \underset{\mathfrak{B}}{\lim}\int_{a}^{b}f_t(x)dx = \int_{a}^{b}\underset{\mathfrak{B}}{\lim}f_t(x)dx.
              \]
    \end{enumerate}
\end{theorem}

\subsection{Теорема Дини}

\begin{theorem}
    Пусть $ X $ -- компактное метрическое пространство. Последовательность $ f_n: X \rightarrow\R $ монотонна на $ X $ и $ \forall x \ f_n $ непрерывна на $ X $.

    Если $ f: X \rightarrow \R $ непрерывна на $ X $, то эта сходимость равномерная.
\end{theorem}

\newpage

\subsection{Дифференцируемость предельной функции}

\begin{theorem}\label{theorem:6.9.5}
    Пусть $ -\infty < a < b < +\infty $ ($ a,b $ -- конечны), $ f_t: (a;b)\rightarrow\R, \ f:(a;b) \rightarrow \R $:
    \begin{itemize}
        \item $ \forall t \in T \ f_t $ дифференцируема на $ (a;b) $,
        \item $ \exists \phi: (a;b)\rightarrow\R: \ f_t' \xrightrightarrows[\mathfrak{B}]{} \phi $ на $ (a;b) $,
        \item $ \exists x_0 \in (a;b): \ f_t(x_0) \rightarrow f(x_0) $,
    \end{itemize}
    тогда:
    \begin{enumerate}
        \item $ f_t \xrightrightarrows[\mathfrak{B}]{} f $ на $ (a;b) $.
        \item $ f $ дифференцируема на $ (a;b) $.
        \item $ \forall x \in (a;b) \ f'(x) = \phi(x) $.
    \end{enumerate}
\end{theorem}

\subsection{Следствие из теоремы о непрерывности предельной функции}

\begin{corollary}
    Если
    \begin{itemize}
        \item $\forall n \ f_n(x)$ непрерывна на $(a;b)$,
        \item ряд $\sum_{n = 1}^{\infty} f_n(x)$ равномерно сходится на $(a;b)$,
    \end{itemize}
    то его сумма $f(x) = \sum_{n=1}^{\infty}f_n(x)$ непрерывна на $(a;b)$, то есть $\forall x_0 \in (a;b)$
    \[
        \underset{x\rightarrow x_0}{\lim}\sum_{n=1}^{\infty}f_n(x) = \sum_{n=1}^{\infty}\underset{x\rightarrow x_0}{\lim}f_n(x).
    \]
\end{corollary}

\subsection{Следствие из теоремы об интегрируемости предельной функции}

\begin{corollary}
    Если
    \begin{itemize}
        \item $\forall n \ f_n(x) \in R[a;b]$ (интегрируема на $[a;b]$),
        \item ряд $\sum_{n=1}^{\infty}f_n(x)$ равномерно сходится на $[a;b]$,
    \end{itemize}
    то его сумма интегрируема на $[a;b]$ и
    \[
        \int_{a}^{b}\sum_{n=1}^{\infty}f_n(x)dx = \sum_{n=1}^{\infty}\int_{a}^{b}f_n(x)dx.
    \]
\end{corollary}

\subsection{Следствие из теоремы о дифференцируемости предельной функции}

\begin{corollary}
    Если
    \begin{itemize}
        \item $\forall n \ f_n(x)$ дифференцируема на $(a;b)$,
        \item $\exists x_0 \in [a;b]$: ряд $\sum_{n=1}^{\infty}f_n(x_0)$ сходится,
        \item ряд $\sum_{n=1}^{\infty}f_n'(x)$ сходится равномерно на $(a;b)$,
    \end{itemize}
    то \begin{enumerate}
        \item Ряд сходится на $(a;b)$ равномерно.
        \item Его сумма дифференцируема на $(a;b)$.
        \item $\forall x \in (a;b)$
              \[
                  \bigg(\sum_{n=1}^{\infty}f_n(x)\bigg)' = \sum_{n=1}^{\infty}f_n'(x).
              \]
    \end{enumerate}
\end{corollary}