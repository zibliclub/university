\section{Кратные интегралы. Мера Жордана в $\R^n$}

\setcounter{subsection}{124}

\subsection{Клетка в $\R^n$}

\begin{definition}
    Множество
    \begin{equation}\label{eq:8.1.1}
        \Pi = \big\{(x_1,\ldots,x_n): \ a_i \leqslant x_i < b_i, \ i = \overline{1,n}\big\}
    \end{equation}
    называется \emph{клеткой} в $\R^n$.

    Пустое множество также считается клеткой.
    \[
        \text{клетки: }\begin{array}{ll}
            \text{в }\R   & \text{ -- }[a;b)\text{ полуинтервалы}                \\
            \text{в }\R^2 & \text{ -- }\begin{array}{l}
                                           \text{прямоугольники, у которых удалены} \\
                                           \text{соответствующиеся стороны}
                                       \end{array}  \\
            \text{в }\R^3 & \text{ -- }\begin{array}{l}
                                           \text{параллелепипеды, у которых удалены} \\
                                           \text{соответствующиеся грани}
                                       \end{array} \\
        \end{array}
    \]
\end{definition}

\subsection{Клеточное множество в $\R^n$}

\begin{definition}
    Множество $A \subset \R^n$ называется \emph{клеточным}, если оно является объединением конечного числа попарно непересекающихся клеток.
\end{definition}

\subsection{Свойства клеточных множеств (1-6)}

\begin{itemize}
    \item Свойсто $ \circled{1} $
          \begin{statement}
              Пересечение двух клеток есть клетка.
          \end{statement}

    \item Свойсто $ \circled{2} $
          \begin{statement}
              Объединение конечного числа непересекающихся клеточных множеств является клеточным множеством.
          \end{statement}

    \item Свойсто $ \circled{3} $
          \begin{statement}
              Пересечение двух клеточных множеств есть клеточное множество.
          \end{statement}

    \item Свойсто $ \circled{4} $
          \begin{statement}
              Разность двух клеток есть клеточное множество.
          \end{statement}

    \item Свойсто $ \circled{5} $
          \begin{statement}
              Разность двух клеточных множеств есть клеточное множество.
          \end{statement}

    \item Свойсто $ \circled{6} $
          \begin{statement}
              Объединение конечного числа клеточных множеств есть клеточное множество.
          \end{statement}
\end{itemize}

\subsection{Мера клеточного множества}

\begin{definition}
    \emph{Мерой $m(A)$ клеточного множества $A$}, разбитого на клетки $\Pi_1,\Pi_2,\ldots,\Pi_n$ называется число:
    \begin{equation}\label{eq:8.1.3}
        m(A) = \sum_{i=1}^{n}m(\Pi_i)
    \end{equation}
\end{definition}

\subsection{Лемма о корректности определения меры клеточного множества}

\begin{lemma}
    Мера клеточного множества не зависит от способа разбиения множества на клетки.
\end{lemma}

\subsection{Свойства меры клеточных множеств (1-4)}

\begin{itemize}
    \item Свойсто $ \circled{1} $
          \begin{statement}
              Если клеточные множества $ A_1,\ldots,A_n $ попарно не пересекаются, то
              \begin{equation}\label{eq:8.1.4}
                  m(\overset{n}{\underset{i=1}{\bigcup}}A_i) = \sum_{i=1}^{n}m(A_i)
              \end{equation}
          \end{statement}

    \item Свойство $ \circled{2} $
          \begin{statement}
              Если $ A $ и $ B $ -- клеточные множества и $ A \subset B $, то
              \begin{equation}\label{eq:8.1.5}
                  m(B) = m(A) + m(B \setminus A)
              \end{equation}
              и $ m(A) \leqslant m(B) $.
          \end{statement}

    \item Свойство $ \circled{3} $
          \begin{statement}
              Если $ A_1,\ldots,A_n $ -- клеточные множества, то
              \begin{equation}\label{eq:8.1.6}
                  m(\overset{n}{\underset{i=1}{\bigcup}}A_i) \leqslant \sum_{i=1}^{n}m(A_i)
              \end{equation}
          \end{statement}

    \newpage

    \item Свойство $ \circled{4} $
          \begin{statement}
              Для $ \forall $ клеточного множества $ A $ и $ \forall \epsilon > 0 \ \exists $ клеточное множество
              \[
                  A_\epsilon : \ A_\epsilon \subset \overline{A_\epsilon} \subset A^\circ \subset A,
              \]
              где $ \overline{A_\epsilon} $ -- замыкание множества $ A_\epsilon $, $ A^\circ $ -- совокупность все внутренних точке множества $ A $.
          \end{statement}
\end{itemize}

\subsection{Множество, измеримое по Жордану}

\begin{definition}
    Множество $ Q \subset \R $ называется \emph{измеримым по Жордану}, если $ \forall \epsilon > 0 \ \exists $ клеточные множества $ A $ и $ B $:
    \[
        A \subset \Omega \subset B \quad \text{и} \quad m(B) - m(A) < \epsilon.
    \]
\end{definition}

\subsection{Мера измеримого по Жордану множества}

\begin{definition}
    Если $ \Omega $ -- измеримое по Жордану множество, то его \emph{мерой} $ m(\Omega) $ называется число для $ \forall A $ и $ B $ -- клеточных множеств: $ A \subset \Omega \subset B $ выполнено
    \[
        m(A) \leqslant m(i) \leqslant m(B).
    \]
\end{definition}

\subsection{Лемма о корректности определения меры измеримого по Жордану множества}

\begin{lemma}
    Определение меры измеримого по Жордану множества корректно, число $ m(\Omega) \ \exists $ и $ ! $, причем
    \[
        m(\Omega) = \underset{A\subset\Omega}{\sup}m(A) = \underset{B\supset\Omega}{\inf}m(B).
    \]
\end{lemma}

\subsection{Множество меры нуль}

\begin{statement}
    Если $ E \subset \R^n $ и $ \forall \epsilon > 0 \ \exists B = B_\epsilon: \ E \subset B $ и $ m(B) < \epsilon \implies m(E) = 0 $.
\end{statement}

\begin{definition}
    Множество, удовлетворяющее условию утверждения, называется \emph{множеством меры нуль}.
\end{definition}

\newpage

\subsection{Свойства множества меры нуль (1-3)}

\begin{itemize}
    \item Свойство $ \circled{1} $
          \begin{statement}
              Если $ E \subset \R^n $ и $ \forall \epsilon > 0 \ \exists B = B_\epsilon: \ E \subset B $ и $ m(B) < \epsilon \implies m(E) = 0 $.
          \end{statement}

    \item Свойство $ \circled{2} $
          \begin{statement}
              Объединение конечного числа множеств меры нуль есть множество меры нуль.
          \end{statement}

    \item Свойство $ \circled{3} $
          \begin{statement}
              Подмножество множества меры нуль есть множество меры нуль.
          \end{statement}
\end{itemize}

\subsection{Критерий измеримости множества в $\R^n$}

\begin{theorem}
    Множество $ \Omega \subset \R $ измеримо по Жордану $ \iff \Omega $ -- ограничено и $ m(G\Omega) = 0 $ (его граница меры нуль).
\end{theorem}

\subsection{Свойства множеств, измеримых по Жордану (1-2, без доказательств)}

\begin{itemize}
    \item Свойство $ \circled{1} $
          \begin{statement}
              Если множества $ \Omega_1 $ и $ \Omega_2 $ измеримы по Жордану, то множества $ \Omega_1 \cup \Omega_2, \ \Omega_1 \cap \Omega_2, \ \Omega_1 \setminus \Omega_2 $ также измеримы по Жордану.
          \end{statement}

    \item Свойство $ \circled{2} $
          \begin{statement}
              Если множества $ \Omega_i, \ i = \overline{1,n} $ измеримы по Жордану, то множество $ \overset{n}{\underset{i=1}{\bigcup}}\Omega_i $ измеримо по Жордану и
              \[
                  m\left(\overset{n}{\underset{i=1}{\bigcup}}\Omega_i\right) \leqslant \sum_{i=1}^{n}m(\Omega_i)
              \]
              и более того, если $ \Omega_i $ попарно не пересекаются, то
              \[
                  m\left(\overset{n}{\underset{i=1}{\bigcup}}\Omega_i\right) = \sum_{i=1}^{n}m(\Omega_i).
              \]
          \end{statement}
\end{itemize}