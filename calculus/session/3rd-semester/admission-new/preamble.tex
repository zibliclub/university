\usepackage[utf8]{inputenc}
\usepackage[T1]{fontenc}
\usepackage{textcomp}
\usepackage[russian]{babel}
\usepackage{url}

\usepackage{hyperref}
\hypersetup{
    colorlinks,
    linkcolor={black},
    citecolor={black},
    urlcolor={blue!80!black}
}

\usepackage{amsmath, amsfonts, mathtools, amsthm, amssymb}

\newcommand\N{\ensuremath{\mathbb{N}}}
\newcommand\R{\ensuremath{\mathbb{R}}}
\newcommand\Z{\ensuremath{\mathbb{Z}}}
\renewcommand\O{\ensuremath{\emptyset}}
\newcommand\Q{\ensuremath{\mathbb{Q}}}
\newcommand\Comp{\ensuremath{\mathbb{C}}}
\let\epsilon\varepsilon

% horizontal rule
\newcommand\hr{
    \noindent\rule[0.5ex]{\linewidth}{0.5pt}
}

\usepackage{tikz}
\usepackage{tikz-cd}

% theorems
\usepackage{thmtools}
\usepackage[framemethod=TikZ]{mdframed}
\mdfsetup{skipabove=1em,skipbelow=0em, innertopmargin=5pt, innerbottommargin=6pt}

\theoremstyle{definition}

\makeatletter

\declaretheoremstyle[headfont=\bfseries\sffamily, bodyfont=\normalfont, mdframed={ nobreak } ]{thmgreenbox}
\declaretheoremstyle[headfont=\bfseries\sffamily, bodyfont=\normalfont, mdframed={ nobreak } ]{thmredbox}
\declaretheoremstyle[headfont=\bfseries\sffamily, bodyfont=\normalfont]{thmbluebox}
\declaretheoremstyle[headfont=\bfseries\sffamily, bodyfont=\normalfont]{thmblueline}
\declaretheoremstyle[headfont=\bfseries\sffamily, bodyfont=\normalfont, numbered=no, mdframed={ rightline=false, topline=false, bottomline=false, }, qed=\qedsymbol ]{thmproofbox}
\declaretheoremstyle[headfont=\bfseries\sffamily, bodyfont=\normalfont, numbered=no, mdframed={ nobreak, rightline=false, topline=false, bottomline=false } ]{thmexplanationbox}


\declaretheorem[numbered=no, style=thmgreenbox, name=Определение]{definition}
\declaretheorem[numbered=no, style=thmredbox, name=Следствие]{corollary}
\declaretheorem[numbered=no, style=thmredbox, name=Предположение]{prop}
\declaretheorem[numbered=no, style=thmredbox, name=Теорема]{theorem}
\declaretheorem[numbered=no, style=thmredbox, name=Утверждение]{statement}
\declaretheorem[numbered=no, style=thmredbox, name=Лемма]{lemma}
\declaretheorem[numbered=no, style=thmredbox, name=Задача]{task}



\declaretheorem[numbered=no, style=thmexplanationbox, name=Доказательство]{explanation}
\declaretheorem[numbered=no, style=thmproofbox, name=Доказательство]{replacementproof}
\declaretheorem[style=thmbluebox,  numbered=no, name=Упражнение]{ex}
\declaretheorem[style=thmbluebox,  numbered=no, name=Пример]{eg}
\declaretheorem[style=thmblueline, numbered=no, name=Примечание]{remark}
\declaretheorem[style=thmblueline, numbered=no, name=Замечание]{note}

\renewenvironment{proof}[1][\proofname]{\begin{replacementproof}}{\end{replacementproof}}

\AtEndEnvironment{eg}{\null\hfill$\diamond$}%

\newtheorem*{uovt}{UOVT}
\newtheorem*{notation}{Notation}
\newtheorem*{previouslyseen}{As previously seen}
\newtheorem*{problem}{Problem}
\newtheorem*{observe}{Observe}
\newtheorem*{property}{Property}
\newtheorem*{intuition}{Intuition}


\usepackage{etoolbox}
\AtEndEnvironment{vb}{\null\hfill$\diamond$}%
\AtEndEnvironment{intermezzo}{\null\hfill$\diamond$}%




% http://tex.stackexchange.com/questions/22119/how-can-i-change-the-spacing-before-theorems-with-amsthm
% \def\thm@space@setup{%
%   \thm@preskip=\parskip \thm@postskip=0pt
% }

\usepackage{xifthen}

\def\testdateparts#1{\dateparts#1\relax}
\def\dateparts#1 #2 #3 #4 #5\relax{
    \marginpar{\small\textsf{\mbox{#1 #2 #3 #5}}}
}

\def\@lesson{}%
\newcommand{\lesson}[3]{
    \ifthenelse{\isempty{#3}}{%
        \def\@lesson{Lecture #1}%
    }{%
        \def\@lesson{Lecture #1: #3}%
    }%
    \subsection*{\@lesson}
    \testdateparts{#2}
}

% fancy headers
\usepackage{fancyhdr}
\pagestyle{fancy}

% \fancyhead[LE,RO]{Gilles Castel}
\fancyhead[RO,LE]{\@lesson}
\fancyhead[RE,LO]{}
\fancyfoot[LE,RO]{\thepage}
\fancyfoot[C]{\leftmark}
\renewcommand{\headrulewidth}{0pt}

\makeatother

% figure support (https://castel.dev/post/lecture-notes-2)
\usepackage{import}
\usepackage{xifthen}
\pdfminorversion=7
\usepackage{pdfpages}
\usepackage{transparent}
\newcommand{\incfig}[1]{%
    \def\svgwidth{\columnwidth}
    \import{./figures/}{#1.pdf_tex}
}

% %http://tex.stackexchange.com/questions/76273/multiple-pdfs-with-page-group-included-in-a-single-page-warning
\pdfsuppresswarningpagegroup=1

\DeclareMathOperator{\Ker}{ker}
\DeclareMathOperator{\im}{Im}

\newcommand{\verteq}[0]{\rotatebox{90}{$=$}}
\newcommand{\vertneq}[0]{\rotatebox{90}{$\ne$}}
\newcommand{\equalto}[2]{\underset{\scriptstyle\overset{\mkern4mu\verteq}{#2}}{#1}}
\newcommand{\nequalto}[2]{\underset{\scriptstyle\overset{\mkern4mu\vertneq}{#2}}{#1}}

% for \xrightrightarrows
% из-за другого шрифта (наверное), стрелки съезжают, надо поправить
\makeatletter
\newcommand*{\relrelbarsep}{.340ex}
\newcommand*{\relrelbar}{%
    \mathrel{%
        \mathpalette\@relrelbar\relrelbarsep
    }%
}
\newcommand*{\@relrelbar}[2]{%
    \raise#2\hbox to 0pt{$\m@th#1\relbar$\hss}%
    \lower#2\hbox{$\m@th#1\relbar$}%
}
\providecommand*{\rightrightarrowsfill@}{%
    \arrowfill@\relrelbar\relrelbar\rightrightarrows
}
\providecommand*{\leftleftarrowsfill@}{%
    \arrowfill@\leftleftarrows\relrelbar\relrelbar
}
\providecommand*{\nrightrightarrowsfill@}{%
    \arrowfill@\relrelbar\relrelbar\nrightrightarrows
}
\providecommand*{\nleftleftarrowsfill@}{%
    \arrowfill@\nleftleftarrows\relrelbar\relrelbar
}
\providecommand*{\xrightrightarrows}[2][]{%
    \ext@arrow 0359\rightrightarrowsfill@{#1}{#2}%
}
\providecommand*{\xleftleftarrows}[2][]{%
    \ext@arrow 3095\leftleftarrowsfill@{#1}{#2}%
}
\providecommand*{\xnrightrightarrows}[2][]{%
    \ext@arrow 0359\nrightrightarrowsfill@{#1}{#2}%
}
\providecommand*{\xnleftleftarrows}[2][]{%
    \ext@arrow 3095\nleftleftarrowsfill@{#1}{#2}%
}
\makeatother

\makeatletter
\newcommand*{\da@rightarrow}{\mathchar"0\hexnumber@\symAMSa 4B }
\newcommand*{\da@leftarrow}{\mathchar"0\hexnumber@\symAMSa 4C }
\newcommand*{\xdashrightarrow}[2][]{%
    \mathrel{%
        \mathpalette{\da@xarrow{#1}{#2}{}\da@rightarrow{\,}{}}{}%
    }%
}
\newcommand{\xdashleftarrow}[2][]{%
    \mathrel{%
        \mathpalette{\da@xarrow{#1}{#2}\da@leftarrow{}{}{\,}}{}%
    }%
}
\newcommand*{\da@xarrow}[7]{%
    % #1: below
    % #2: above
    % #3: arrow left
    % #4: arrow right
    % #5: space left 
    % #6: space right
    % #7: math style 
    \sbox0{$\ifx#7\scriptstyle\scriptscriptstyle\else\scriptstyle\fi#5#1#6\m@th$}%
    \sbox2{$\ifx#7\scriptstyle\scriptscriptstyle\else\scriptstyle\fi#5#2#6\m@th$}%
    \sbox4{$#7\dabar@\m@th$}%
    \dimen@=\wd0 %
    \ifdim\wd2 >\dimen@
        \dimen@=\wd2 %   
    \fi
    \count@=2 %
    \def\da@bars{\dabar@\dabar@}%
    \@whiledim\count@\wd4<\dimen@\do{%
        \advance\count@\@ne
        \expandafter\def\expandafter\da@bars\expandafter{%
            \da@bars
            \dabar@
        }%
    }%  
    \mathrel{#3}%
    \mathrel{%   
        \mathop{\da@bars}\limits
        \ifx\\#1\\%
        \else
            _{\copy0}%
        \fi
        \ifx\\#2\\%
        \else
            ^{\copy2}%
        \fi
    }%   
    \mathrel{#4}%
}
\makeatother

\newcommand\xdownarrow[1][2ex]{%
    \mathrel{\rotatebox[origin=c]{-90}{$\xrightarrow{\rule{#1}{0pt}}$}}
}
\newcommand\xdashdownarrow[1][2ex]{%
    \mathrel{\rotatebox[origin=c]{-90}{$\xdashrightarrow{\rule{#1}{0pt}}$}}
}

\newcommand*\circled[1]{\tikz[baseline=(char.base)]{
        \node[shape=circle,draw,inner sep=1pt] (char) {#1};}
}

\newcommand{\RomanNumeralCaps}[1]
    {\MakeUppercase{\romannumeral #1}}

\author{Данил Заблоцкий}