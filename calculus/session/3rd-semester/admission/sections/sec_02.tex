\section{Основные теоремы дифференциального исчисления функций многих переменных}

\setcounter{subsection}{4}

\subsection{Теорема о среднем (аналог теоремы Лагранжа)}

\begin{theorem}[О среднем]
    Пусть $ D $ -- область в $ \R^n, \ x \in D, \ x + h \in D, \ [x, x+h]\subset D, \ f: D \rightarrow\R $ -- дифференцируемо на $ (x,x+h) $ и непрерывно на $ [x,x+h] $. Тогда $ \exists \xi \in (x,x+h): $
    \[
        f(x+h)-f(x) = f'(\xi)\cdot h = \frac{\partial f}{\partial x_1}(\xi)\cdot h^1 + \frac{\partial f}{\partial x_2}(\xi)\cdot h^2 + \ldots + \frac{\partial f}{\partial x_n}(\xi)\cdot h^n,
    \] где $ \{1,2,\ldots,n\} $ над $ h $ -- индексы.
\end{theorem}

\subsection{Следствие теоремы о среднем}

\begin{corollary}
    Пусть $ D $ -- область в $ \R^n, \ f:D \rightarrow \R $ -- дифференцируема на $ D $ и $ \forall x \in D \ d(fx) = 0 $ (то есть $ \forall i \ \frac{\partial f}{\partial x_i} = 0 $). Тогда $ f(x) = const $.
\end{corollary}

\subsection{Достаточное условие дифференцируемости функции}

\begin{theorem}[Достаточное условие дифференцируемости функции]
    Пусть $ D $ -- область в $ \R^n, \ f:D \rightarrow\R, \ f $ имеет непрерывные частные производные в каждой окрестности точки $ x\in D $. Тогда $ f $ -- дифференцируема в точке $ x $.
\end{theorem}

\subsection{Производные высших порядков}

\begin{definition}[Вторая производная функции по переменным]
    Пусть $ f:D \rightarrow\R, \ D $ -- область в $ \R^n $. Производная по переменной $ x^j $ от производной по переменной $ x^i $ называется \emph{второй производной функции $ f $ по переменным} $ x^i,x^j $ и обозначается
    \[
        \frac{\partial^2f}{\partial x^i\partial x^j}(x)\text{ или }f_{x^i,x^j}''(x).
    \]
\end{definition}

\subsection{Теорема о смешанных производных}

\begin{theorem}[О смешанных производных]
    Пусть $ D $ -- область в $ \R^n, \ f: D \rightarrow\R, \ x \in D, \ f $ имеет в $ D $ непрерывные смешанные производные (второго порядка). Тогда эти производные не зависят от порядка дифференцирования.
\end{theorem}

\subsection{Формула Тейлора}

\begin{theorem}[Формула Тейлора]
    Пусть $D$ -- область в $\R^n, \ f:D\rightarrow\R, \ f\in C^{(k)}(D,\R), \ x \in D, \ x + h \in D, \ [x;x+h] \subset D$. Тогда:
    \[
        f(x + h) = f(x) + \sum_{i=1}^{k-1}\frac{1}{i!}\left(\frac{\partial}{\partial x^1}\cdot h^1 + \ldots + \frac{\partial}{\partial x^n}\cdot h^n\right)^i \cdot f(x) + R^k,
    \]
    где $R^k$ -- остаточный член,
    \[
        R^k = \frac{1}{k!}\left(\frac{\partial}{\partial x^1}\cdot h^1 + \ldots + \frac{\partial}{\partial x^n}\cdot h^n\right)^k \cdot f(x + \xi \cdot h),
    \]
    \[
        x = (x^1,\ldots,x^n), \quad h = (h^1,\ldots,h^n).
    \]
\end{theorem}

\subsection{Локальный экстремум функции многих переменных}

\begin{definition}[Точка локального максимума (минимума)]
    Пусть $ X $ -- метрическое пространство (МП), $ f:X \rightarrow\R $. Точка $ x_0 $ называется \emph{точкой локального максимума (минимума)}, если $ \exists U(x_0) \subset X: \ \forall x \in U(x_0) $
    \[
        f(x)\leqslant f(x_0) \quad \big(f(x) \geqslant f(x_0)\big)
    \]
\end{definition}

\subsection{Необходимое условие локального экстремума}

\begin{theorem}[Необходимое условие локального экстремума]
    Пусть $D$ -- область в $ \R^n, \ f:D \rightarrow\R, \ x_0 \in D $ -- точка локального экстремума, тогда в точке $ x_0 \ \forall i = \overline{1,n}$
    \[
        \frac{\partial(x_0)}{\partial x^i} = 0.
    \]
\end{theorem}

\subsection{Критическая точка функции}

\begin{definition}[Критическая точка функции]
    Пусть $ D $ -- область в $ \R^n, \ f:D \rightarrow\R^k $ -- дифференцируемо в точке $ x_0 \in D $. Точка $ x_0 $ называется \emph{критической точкой функции} $ f(x) $, если:
    \[
        rank \mathfrak{I} f(x_0) < \min(n,k),
    \] где $ \mathfrak{I}f(x_0) $ -- матрица Якоби функции $ f(x_0) $.
\end{definition}

\subsection{Достаточное условие локального экстремума}

\begin{theorem}[Достаточное условие локального экстремума]
    Пусть $D$ -- область в $\R^n, \ f: D \rightarrow \R$ дифференцируема в точке $x \in D, \ x$ -- критическая точка для $f, \ f \in C^n(D,\R), \ n = 2$. Тогда, если:
    \begin{enumerate}
        \item $Q(h)$ -- знакоположительна, то в точке $x$ -- локальный минимум.
        \item $Q(h)$ -- знакоотрицательна, то в точке $x$ -- локальный максимум.
        \item $Q(h)$ может принимать различные значения ($>0, < 0$), тогда в точке $x$ нет экстремума.
    \end{enumerate}
\end{theorem}

\subsection{Неявная функция}

\begin{definition}[Наеявно заданная уравнением функция]
    Пусть $D$ -- область в $\R^k, \ \Omega$ -- область в $\R^k, \ F: D \times \Omega \rightarrow \mathbb{R}^k $.

    Пусть функция $f:D \rightarrow\Omega:$
    \[
        y = f(x) \iff F(x,y) = 0.
    \]

    Говорят, что уравнение $F(x,y) = 0$ \emph{неявно задает} функцию $y = f(x)$.
\end{definition}

\subsection{Теорема о неявной функции}

\begin{theorem}[О неявной функции]\label{theorem:1}
    Пусть $ F(x,y) $ отображает окрестность $ U(x_0;y_0) \subset \R^2 $ в $ \R, \ F:U(x_0,y_0)\rightarrow\R $.

    Пусть $ F $ имеет следующие свойства:
    \begin{enumerate}
        \item $ F(x_0,y_0) = 0 $.
        \item $ F(x,y) \in C^P(U,\R), \ p \geqslant 1 $.
        \item $ \frac{\partial F}{\partial y}(x_0,y_0)\ne 0 $.
    \end{enumerate}

    Тогда $ \exists $ открезки $ I_x,I_y: \ f:I_x \rightarrow I_y $:
    \begin{enumerate}
        \item $ I_x \times I_y \subset U(x_0,y_0) $.
        \item $ \forall x \in I_x \ y = f(x) \iff F(x,y) = 0 $.
        \item $ f \in C^P(I_x,I_y) $.
        \item $ \forall x \in I_x \ f'(x) = -\frac{F_x'(x,y)}{F_y'(x,y)} $.
    \end{enumerate}
\end{theorem}