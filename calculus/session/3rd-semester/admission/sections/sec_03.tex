\section{Приложение теоремы о неявной функции}

\setcounter{subsection}{16}

\subsection{Диффиоморфизм, гомеоморфизм}

\begin{definition}[Диффиоморвизм класса $ C^{(p)} $, гомеоморфизм]
    Пусть $ D,G $ -- области в $ \R^n $. Отображение $ f:D \rightarrow G $ называтеся \emph{диффиоморфизмом класса $ C^{(p)} $}, $ p \geqslant 0 $, если:
    \begin{enumerate}
        \item $ f $ -- обратимое.
        \item $ f \in C^{(p)}(D,G) $.
        \item $ f^{-1} \in C^{(p)}(D,G) $.
    \end{enumerate}

    При $ p=0 \ f $ называется \emph{гомеоморфизмом}, то есть $ f $ -- гомеоморфизм, если $ f $ -- взаимно однозначное отображение и $ f,f^{-1} $ -- непрерывны.
\end{definition}

\subsection{Теорема о неявной функции}

\begin{theorem}[О неявной функции, общий случай]
    Пусть $ F:U(x_0,y_0) \rightarrow \R^n $, где $ U(x_0,y_0) \subset \R^{m+n} $ -- окрестность точки $ (x_0,y_0) $ такая, что
    \begin{enumerate}
        \item $ F\in C^{(p)}(U,\R^n), \ p \geqslant 1 $.
        \item $ F(x_0,y_0) = 0 $.
        \item $ F_y'(x_0,y_0) $ -- обратная матрица.
    \end{enumerate}

    Тогда $ \exists (n+m) $-мерный промежуток $ I = I_x^m \times I_y^n \subset U(x_0;y_0) $, где
    \begin{align*}
        I_x^m & = \big\{x\in \R^m \ \big| \ |x - x_0| < \alpha \big\}, \\
        I_x^n & = \big\{y\in \R^n \ \big| \ |y - y_0| < \beta \big\},
    \end{align*} то есть $ f:I_x^m \rightarrow I_y^n $:
    \begin{itemize}
        \item $ \forall (x,y) \in I_x^m \times I_y^n \ F(x,y) = 0 \iff y = f(x) $.
        \item $ f'(x) = -\big[F_y'(x,y)\big]^{-1} \cdot F_x'(x,y) $.
    \end{itemize}
\end{theorem}

\subsection{$ K $-мерная поверхность}

\begin{definition}[$ k $-мерная поверхность]
    Множество $ S \subset \R^n $ называется \emph{$ k $-мерной поверхностью}, если $ \forall x \in S \ \exists U(x)\subset\R^n $ и $ \exists $ диффиоморфизм $ \phi: U(x)\rightarrow I^n $:
    \[
        \phi\big(U(x)\cap S\big) = I^k,
    \] где $ I^n = \big\{x \in \R^n \ \big| \ |x^i| < 1 \big\} $,
    \[
        I^k = \big\{x \in \R^n \ \big| \ x^{k+1} = x^{k+2} = \ldots = x^n = 0\big\}.
    \]
\end{definition}

\subsection{Параметризация $ K $-мерной поверхности}

\begin{definition}[Локальная карта или параметризация поверхности]
    Пусть $S$ -- $k$-мерная поверхность в $\R^n, \ x_0 \in S$ и $\phi: U(x_0) \rightarrow I^n$ -- диффиоморфизм:
    \[
        \phi\big(U(x_0)\cap S\big) = I^k.
    \]

    Ограничение $\phi^{-1}$ на $I^k$ будем называть \emph{локальной картой} или \emph{параметризацией поверхности} $S$ в окрестности точки $x_0$.
\end{definition}

\subsection{Касательная плоскость (касательное пространство) к $ K $-мерной поверхности в $ \R^n $}

\begin{definition}[Касательное пространство]
    Пусть $ S $ -- $ k $-мерная поверхность в $ \R^n, \ x_0 \in S, \ x = x(t):\R^k \rightarrow \R^n $ -- параметризация $ S $ в окрестности точки $ x_0 $, при этом $ x_0 = x(0) $.

    \emph{Касательным пространством} (или плоскостью) к $S$ в точке $x_0$ называется $k$-мерная плоскость, заданная уравнением:
    \begin{equation}\label{eq:8}
        x = x_0 + x'(0) \cdot t,
    \end{equation}
    \begin{align*}
        x_0   & = (x^1_0,x^2_0,\ldots,x^n_0)                                                   \\
        x(t)  & = \left\{\begin{array}{l}
                             x^1(t^1,\ldots,t^k) \\
                             x^2(t^1,\ldots,t^k) \\
                             \vdots              \\
                             x^n(t^1,\ldots,t^k)
                         \end{array}\right.                                                   \\
        x'(t) & = \left(\begin{matrix}
                                \frac{\delta x^1}{\delta t^1} & \ldots & \frac{\delta x^1}{\delta t^k} \\
                                \vdots                        & \ddots & \vdots                        \\
                                \frac{\delta x^n}{\delta t^1} & \ldots & \frac{\delta x^n}{\delta t^k} \\
                            \end{matrix}\right)(t), \quad t = \left(\begin{matrix}
                                                                        t^1 \\ t^2 \\ \vdots \\ t^k
                                                                    \end{matrix}\right)
    \end{align*}

    Таким образом касательное пространство задается системой из \ref{eq:8}:
    \[
        \left\{\begin{array}{l}
            x^1 = x^1_0 + \frac{\delta x^1}{\delta t^1} (0)\cdot t^1 + \ldots + \frac{\delta x^1}{\delta t^k}(0)\cdot t^k \\
            x^2 = x^2_0 + \frac{\delta x^2}{\delta t^1} (0)\cdot t^1 + \ldots + \frac{\delta x^2}{\delta t^k}(0)\cdot t^k \\
            \vdots                                                                                                        \\
            x^n = x^n_0 + \frac{\delta x^n}{\delta t^1} (0)\cdot t^1 + \ldots + \frac{\delta x^n}{\delta t^k}(0)\cdot t^k
        \end{array}\right..
    \]
\end{definition}

\subsection{Теорема о структуре касательного пространства}

\begin{theorem}[О структуре касательного пространства]\label{theorem:2}
    Пусть $S$ -- $k$-мерная поверхность в $\R^n, \ x_0 \in S$. Тогда касательное пространство $TS_{x_0}$ в точке $x_0$ состоит из направляющих векторов касательных к гладким кривым на поверхности $S$, проходящих через точку $x_0$.
\end{theorem}

\subsection{Задача на условный экстремум, условный экстремум}

\begin{task}
    Пусть требуется найти условный экстремум функции $f:D\rightarrow\R, \ D$ -- область в $\R^n$, на поверхности $S$, заданной системой уравнений:
    \[
        \left\{\begin{array}{l}
            F^1(x^1,\ldots,x^n) = 0 \\
            \vdots                  \\
            F^k(x^1,\ldots,x^n) = 0
        \end{array}\right..
    \]

    Составим функцию Лагранжа:
    \begin{multline*}
        L(x,\lambda) = L(x^1,\ldots,x^n,\lambda^1,\ldots,\lambda^k) = \\
        = f(x^1,\ldots,x^n) + \sum_{i=1}^{k}\lambda^i\cdot F^i(x^1,\ldots,x^n),
    \end{multline*}

    $\lambda = (\lambda^1,\ldots,\lambda^k), \ \lambda^i \in \R$ -- коэффициент, в общем случае пока неизвестен.

    Необходимое условие локального экстремума для функции $L:$
    \begin{equation}\label{eq:24}
        \left\{\begin{array}{l}
            \left.\begin{array}{l}
                      \frac{\delta L}{\delta x^1} = \frac{\delta f}{\delta x^1} + \sum_{i = 1}^{k}\lambda^i\cdot\frac{\delta F^i}{\delta x^1} = 0 \\
                      \vdots                                                                                                                      \\
                      \frac{\delta L}{\delta x^n} = \frac{\delta f}{\delta x^n} + \sum_{i = 1}^{k}\lambda^i\cdot\frac{\delta F^i}{\delta x^n} = 0 \\
                  \end{array}\right\}\begin{array}{l}
                                         \text{необходимое условие условного} \\
                                         \text{экстремума функции }f
                                     \end{array} \\
            \left.\begin{array}{l}
                      \frac{\delta L}{\delta \lambda^1} = F^1(x^1,\ldots,x^n) = 0 \\
                      \vdots                                                      \\
                      \frac{\delta L}{\delta \lambda^k} = F^k(x^1,\ldots,x^n) = 0
                  \end{array}\right\}\text{ поверхность }S
        \end{array}\right.
    \end{equation}
\end{task}

\begin{definition}[Условный экстремум]
    Пусть $f:D\rightarrow\R, \ D \subset \R^n$ -- область, $S$ -- поверхность в $D$, \emph{условным экстремумом} функции $f$ называется экстремум функции $f\big|_S$.
\end{definition}

\subsection{Линия уровня}

\begin{definition}[Линия уровня ($c$-уровень)]
    Пусть $f:D\rightarrow\R, \ D \subset \R^n$ -- область. \emph{Линией уровня ($c$-уровнем)} функции $f$ называется множество
    \[
        N_c = \big\{x\in D \ \big| \ f(x) = c\big\}.
    \]
\end{definition}

\subsection{Необходимое условие условного локального экстремума}

\begin{theorem}[Необходимое условие условного локального экстремума]
    Пусть система уровнений
    \begin{equation}\label{eq:20}
        \left\{\begin{array}{l}
            F^1(x^1,\ldots,x^n) = 0 \\
            \vdots                  \\
            F^{n-k}(x^1,\ldots,x^n) = 0
        \end{array}\right.
    \end{equation}
    задает $(n-k)$-мерную гладкую поверхность $S$ в $D \subset \R^n, \ D$ -- область. Функция $f:D\rightarrow\R$ -- гладкая. Если $x_0 \in S$ является точкой условного локального экстремума для функции $f$, то существует такой набор чисел $\lambda_1,\lambda_2,\ldots,\lambda_{n-k} \in \R:$
    \[
        grad f(x_0) = \sum_{i = 1}^{n-k}\lambda_i \cdot grad F^i(x_0).
    \]
\end{theorem}

\subsection{Функция Лагранжа}

\begin{note}
    Пусть $f:D\rightarrow\R, \ D \subset \R^n$ -- область, $f \in C^{(2)} (D,\R), \ S$ -- $(n-k)$-мерная поверхность в $D$, заданная системой уравнений:
    \[
        \left\{\begin{array}{l}
            F^1(x^1,\ldots,x^n) = 0 \\
            \vdots                  \\
            F^k(x^1,\ldots,x^n) = 0
        \end{array}\right..
    \]

    Функция Лагранжа:
    \[
        L(x,\lambda) = f(x^1,\ldots,x^n) + \sum_{i = 0}^{k}\lambda_i \cdot F^i(x^1,\ldots,x^n).
    \]

    Здесь $\lambda_1,\ldots,\lambda_k$ выбираются таким образом, чтобы было выполнено необходимое условие условного экстремума в точке $x_0$ (\ref{eq:24}).
    \[
        \left\{\begin{array}{l}
            \frac{\delta L}{\delta x^i} = 0 \\
            \vdots                          \\
            \frac{\delta L}{\delta \lambda_j} = 0
        \end{array}\right. \implies x_0,\quad \lambda_1,\ldots,\lambda_k.
    \]
\end{note}

\subsection{Метод Лагранжа}

Я запутался.

\subsection{Достаточное условие условного локального экстремума}

\begin{theorem}[Достаточное условие условного экстремума]
    Если при введенных выше условиях квадратичная форма
    \[
        Q(\xi) = \sum_{i,j=1}^{n}\frac{\delta^2 L}{\delta x^i \delta x^j}(x_0)\cdot\xi^i\xi^j,\ \big(\xi=(\xi^1,\ldots,\xi^n)\big)
    \]
    \begin{enumerate}
        \item Знакоопределена на $TS_{x_0}$:
              \begin{itemize}
                  \item если $Q$ знакоположительна, то точка $x_0$ -- точка условного локального $\min$
                  \item если $Q$ знакоотрицательна, то точка $x_0$ -- точка условного локального $\max$
              \end{itemize}
        \item Если $Q$ может принимать значения разных знаков, то в точке $x_0$ условного экстремума не наблюдается.
    \end{enumerate}
\end{theorem}