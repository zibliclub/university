\section{Функциональные свойства несобственного интеграла, зависящего от параметра}

\setcounter{subsection}{115}

\subsection{Теорема о предельном переходе под знаком несобственного интеграла, зависящего от параметра}

\begin{theorem}[О предельном переходе под знаком несобственного интеграла]\label{theorem:7.3.1}
    Если
    \begin{enumerate}
        \item $\forall b \in [a;\omega)$
              \[
                  f(x,y) \xrightrightarrows[\mathfrak{B}_y]{}\phi(x)
              \]
              на $[a;b]$, где $\mathfrak{B}_y$ -- база на $Y$.
        \item $\int_{a}^{\omega}f(x,y)dx$ сходится равномерно на $Y$.
    \end{enumerate}
    Тогда \[
        \underset{\mathfrak{B}_y}{\lim} F(y) = \underset{\mathfrak{B}_y}{\lim} \int_{a}^{\omega}f(x,y)dx = \int_{a}^{\omega}\underset{\mathfrak{B}_y}{\lim} f(x,y)dx = \int_{a}^{\omega}\phi(x)dx.
    \]
\end{theorem}

\begin{proof}
    Имеем $F_b(y) = \int_{a}^{b}f(x,y)dx$
    \[
        \begin{matrix}
            F_b(y) = & \int_{a}^{b}f(x,y)dx                                           & \xrightrightarrows[b\rightarrow\omega]{} & \int_{a}^{\omega} f(x,y)dx                          & = F(y) \\
                     & {\scriptstyle \forall b \ \mathfrak{B}_y}\xdashdownarrow[20pt] &                                          & \xdashdownarrow[20pt] {\scriptstyle \mathfrak{B}_y} &        \\
                     & \int_{a}^{b}\phi(x)dx                                          & \xdashrightarrow[b\rightarrow\omega]{}   & \int_{a}^{\omega}\phi(x)dx                          &
        \end{matrix}
    \]


    Докажем левую вертикальную стрелку. Вспомним теорему \ref{theorem:6.9.2}.
    \begin{enumerate}
        \item $\forall y \in Y \ f(x,y)$ интегрируется на $[a;b]$ (из условия 2 $\implies$)
        \item $f(x,y)\underset{\mathfrak{B}_y}{\rightrightarrows}\phi(x)$ на $[a;b] \implies$
              \[
                  \int_{a}^{b}\phi(x)dx = \underset{\mathfrak{B}_y}{\lim}\int_{a}^{b}f(x,y)dx
              \]
    \end{enumerate}
    $\implies$ используя теорему \ref{theorem:6.9.1}, доказывается утверждение этой теоремы.
\end{proof}

\subsection{Теорема о непрерывности несобственного интеграла, зависящего от параметра}

\begin{corollary}[Непрерывность несобственного интеграла, зависящего от параметров]
    Если \begin{enumerate}
        \item $f(x,y)$ непрерывна на $[a;\omega)\times[c;d]$.
        \item $\int_{a}^{\omega}f(x,y)dx$ равномерно сходится на $[c;d]$.
    \end{enumerate}

    Тогда $F(y) = \int_{a}^{\omega}f(x,y)dx$ непрерывна на $[c;d]$.
\end{corollary}

\begin{proof}
    $ y_0 \in [c;d] $. Докажем, что $ F(y) $ непрерывна в точке $ y_0 $, то есть докажем, что $ \underset{y \rightarrow y_0}{\lim}F(y) = F(y_0) $.

    Имеем:
    \begin{multline*}
        \underset{y \rightarrow y_0}{\lim}F(y) = \underset{y \rightarrow y_0}{\lim}\int_{a}^{\omega}f(x,y)dx \overset{\circled{?}}{=} \\
        \overset{\circled{?}}{=} \int_{a}^{\omega}\underset{y \rightarrow y_0}{\lim}f(x,y)dx \overset{\text{непр. }f(x,y)}{=} \int_{a}^{\omega}f(x,y_0)dx = F(y_0).
    \end{multline*}

    Проверим, выполняются ли условия теоремы \ref{theorem:7.3.1}.

    База: $ y \rightarrow y_0 $. Надо показать, что
    \begin{enumerate}
        \item $ f(x,y)\xrightrightarrows[y \rightarrow y_0]{}\equalto{f(x,y_0)}{\phi(x)} $ на $ [a;b] \ \forall b \in [a;\omega) $.
        \item Дано.
    \end{enumerate}

    Покажем 1.

    Так как $ f(x,y) $ непрерывна на $ [a;\omega) \times [c;d] \implies f(x,y) $ равномерно непрерывна на $ [a;b]\times[c;d] $ (по теореме Кантора) $ \implies \forall(x,y_0) \ \exists U \subset [a;b]\times[c;d]: \ \forall (x,y) \in U $
    \[
        \big|f(x,y) - f(x,y_0)\big| < \epsilon \implies
    \]
    $ \implies \circled{?} $ обоснован.
\end{proof}

\subsection{Теорема о дифференцировании несобственного интеграла, зависящего от параметра}

\begin{theorem}[О дифференцировании несобственного интеграла по параметру]\label{theorem:7.3.2}
    Если
    \begin{enumerate}
        \item $ f(x,y) $ непрерывна на $ [a;\omega)\times[c;d] $ и имеет непрерывную производную по $ y $.
        \item $ \int_{a}^{\omega}f_y'(x,y)dx $ равномерно сходится на $ [c;d] $.
        \item $ \int_{a}^{\omega}f(x,y)dx $ сходится хотя бы в одной точке $ y_0 \in (c;d) $.
    \end{enumerate}

    Тогда
    \begin{enumerate}
        \item $ \int_{a}^{\omega}f(x,y)dx $ сходится равномерно на $ [c';d']\subset (c;d) $.
        \item $ F(y) = \int_{a}^{\omega}f(x,y)dx $ дифференцируема на $ (c;d) $.
        \item $ F'(y) = \left(\int_{a}^{\omega}f(x,y)dx\right)_y' = \int_{a}^{\omega}f_y'(x,y)dx $.
    \end{enumerate}
\end{theorem}

\begin{proof}
    Рассмотрим семейство функций $ F_b(y) = \int_{a}^{b}f(x,y)dx $.

    Имеем $ \forall b \ F_b(y) $ дифференцируема на $ (c;d) $ и
    \[
        F_b'(y)=\int_{a}^{b}f_y'(x,y)dx \ \text{(теорема \ref{theorem:7.1.2})}.
    \]

    Далее, $ F_b'(y) $ сходится равномерно на $ (c;d) $ при $ f \rightarrow\omega $ и $ F_b(y) $ сходится хотя бы в одной точке $ y=y_0 \in (c;d) $ при $ b \rightarrow \omega $.

    Следовательно, по теореме \ref{theorem:6.9.5}, семейство $ F_b(y) $ сходится равномерно на $ [c';d'] $ при $ b \rightarrow\omega $. Предельная функция $ F(y) $ дифференцируема и
    \[
        F'(y) = \underset{b \rightarrow \omega}{\lim}F_b'(y) = \underset{b \rightarrow \omega}{\lim}\int_{a}^{\omega}f_y'(x,y)dx = \int_{a}^{\omega}f_y'(x,y)dx.
    \]
\end{proof}

\subsection{Теорема об интегрировании несобственного интеграла, зависящего от параметра}

\begin{theorem}[Об интегрировании несобственного интеграла по параметру]\label{theorem:7.3.3}
    Если
    \begin{enumerate}
        \item $ f(x,y) $ непрерывна на $ [a;\omega)\times[c;d] $.
        \item $ \int_{a}^{\omega}f(x,y)dx $ равномерно сходится на $ [c;d] $.
    \end{enumerate}

    Тогда функция $ F(y) = \int_{a}^{\omega}f(x,y)dx $ интегрируема по Риману на $ [c;d] $ и
    \[
        \int_{c}^{d}dy \int_{a}^{\omega}f(x,y)dx = \int_{a}^{\omega}dx \int_{c}^{d}f(x,y)dy.
    \]
\end{theorem}

\begin{proof}
    Имеем $ \forall b \in [a;\omega) $
    \[
        \int_{c}^{d}dy \int_{a}^{b}f(x,y)dx = \int_{a}^{b}dx \int_{c}^{d}f(x,y)dy \ \text{(теорема \ref{theorem:7.1.3})}.
    \]
    \begin{multline*}
        \underset{b \rightarrow\omega}{\lim} \int_{c}^{d}dy \int_{a}^{b}f(x,y)dx = \\
        = \underset{b \rightarrow\omega}{\lim} \int_{a}^{b}dx \int_{c}^{d}f(x,y)dy \overset{\text{опр.}}{=} \int_{a}^{\omega}dx \int_{c}^{d}f(x,y)dy,
    \end{multline*}
    \begin{multline*}
        \underset{b \rightarrow\omega}{\lim} \int_{c}^{d}dy \int_{a}^{b}f(x,y)dx = \\
        = \int_{c}^{d}\left(\underset{b \rightarrow\omega}{\lim} \int_{a}^{b}f(x,y)dy\right) \overset{\text{опр.}}{=} \int_{c}^{d}dy \int_{a}^{\omega}f(x,y)dx.
    \end{multline*}

    Покажем правомерность предельного перехода.

    Рассмотрим семейство функций $ F_b'(y) = \int_{a}^{b}f(x,y)dx $. Покажем, что для этого семейства выполняются условия теоремы \ref{theorem:6.9.2}.

    В самом деле,
    \begin{enumerate}
        \item $ \forall b \in [a;\omega) \ F_b(y) $ непрерывна на $ [c;d] $ (теорема \ref{theorem:7.3.1}) $ \implies \forall b \in [a;\omega) \ F_b(y) $ интегрируема по Риману на $ [c;d] $.
        \item Так как $ \int_{a}^{\omega}f(x,y)dx $ равномерно сходится на $ [c;d] $, то множество $ F_b(y) $ сходится равномерно на $ [c;d] $ к $ F(y) $ при $ b \rightarrow \omega $.
    \end{enumerate}
\end{proof}

\subsection{Теорема о перестановке двух несобственных интегралов, зависящих от параметра}

\begin{theorem}[О перестановке несобственного интерграла, зависящего от параметра]
    Пусть
    \begin{enumerate}
        \item $ f(x,y) $ непрерывна на $ [a;\omega)\times[c;\widetilde{\omega}) $.
        \item $ \forall d \in [c;\widetilde{\omega}) \ \int_{a}^{\omega}f(x,y)dx $ сходится равномерно на $ [c;d] $.
        \item $ \forall b \in [a;\omega) \ \int_{c}^{\widetilde{\omega}}f(x,y)dx $ сходится равномерно на $ [a;b] $.
        \item Существует хотя бы одни из интегралов:
              \[
                  \int_{a}^{\omega}dx \int_{c}^{\widetilde{\omega}}\big|f(x,y)\big|dy \quad \text{или}\quad \int_{c}^{\widetilde{\omega}}dy \int_{a}^{\omega}\big|f(x,y)\big|dx.
              \]
    \end{enumerate}

    Тогда существует
    \[
        \int_{a}^{\omega}dx \int_{c}^{\widetilde{\omega}}f(x,y)dy=\int_{c}^{\widetilde{\omega}}dy \int_{a}^{\omega}f(x,y)dx.
    \]
\end{theorem}

\begin{proof}
    $ \forall d \in [c;\widetilde{\omega}) $ верно равенство
    \[
        \int_{c}^{d}dy \int_{a}^{\omega}f(x,y)dx = \int_{a}^{\omega}dx \int_{c}^{d}f(x,y)dy \ \text{(теорема \ref{theorem:7.3.3})}.
    \]
    \begin{multline*}
        \int_{c}^{\widetilde{\omega}}dy \int_{a}^{\omega}f(x,y)dx \overset{\text{опр.}}{=} \underset{d \rightarrow \widetilde{\omega}}{\lim}\int_{c}^{d}dy \int_{a}^{\omega}f(x,y)dx = \\
        = \underset{d \rightarrow \widetilde{\omega}}{\lim} \int_{a}^{\omega}dx \int_{c}^{d}f(x,y)dy \overset{\circled{?}}{=} \int_{a}^{\omega}\left(\underset{d \rightarrow \widetilde{\omega}}{\lim} \int_{c}^{d}f(x,y)dy\right)dx = \\
        = \int_{a}^{\omega}dx \int_{c}^{\widetilde{\omega}}f(x,y)dy.
    \end{multline*}

    Докажем возможность предельного перехода.

    Рассмотрим семейство $ \Phi_d(x) = \int_{c}^{d}f(x,y)dy $. Имеем:
    \begin{enumerate}
        \item $ \forall b \in [a;\omega) \ \Phi_d(x)\xrightrightarrows[d \rightarrow\widetilde{\omega}]{}\Phi(x) = \int_{c}^{\widetilde{\omega}}f(x,y)dy $ на $ [c;d] $ (условие 3.).
        \item $ \int_{a}^{\omega}\Phi_d(x)dx $ равномерно сходится на $ [c;\widetilde{\omega}) $.
    \end{enumerate}

    Покажем пункт 2.

    $ \forall d \in [c;\widetilde{\omega}) $ и $ \forall x \in [a;\omega) $
    \[
        \big|\Phi_d(x)\big| = \left|\int_{c}^{d}f(x,y)dy\right| \leqslant \int_{c}^{d}\big|f(x,y)\big|dy \leqslant \int_{c}^{\widetilde{\omega}}\big| f(x,y) \big|dy \xrightrightarrows[d \rightarrow \widetilde{\omega}]{}\phi(x).
    \]

    Допустим, что $ \exists \int_{a}^{\omega}dx \int_{c}^{\widetilde{\omega}}\big| f(x,y) \big| dy $.

    Тогда получаем, что $ \int_{a}^{\omega}\phi(x)dx $ сходится и не зависит от $ d \implies $ по признаку Вейерштрасса $ \int_{a}^{\omega}\Phi_d(x)dx $ сходится равмномерно на $ [c;\widetilde{\omega}) \implies $ выполняется условие теоремы \ref{theorem:7.3.1} $ \implies $ вопрос о предельном переходе снят.
\end{proof}