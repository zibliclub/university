\section{Определение и свойства кратного интеграла Римана}

\setcounter{subsection}{147}

\subsection{Теорема о сведении двойного интеграла по прямоугольнику к повторному интегралу}

\begin{theorem}[Формула сведения двойного интеграла по прямоугольнику к повторному]
    Пусть
    \begin{enumerate}
        \item Функция $ f(x,y) $ интегрируема на прямоугольнике
              \[
                  \Pi = \big\{(x,y): \ a \leqslant x \leqslant b, \ c \leqslant y \leqslant d\big\}.
              \]

        \item $ \int_{c}^{d}f(x,y)dy \ \exists \ \forall x \in [a;b] $.
    \end{enumerate}

    Тогда функция $ F(x) = \int_{c}^{d}f(x,y)dy $ интегрируема на отрезке $ [a;b] $ и справедлива формула:
    \[
        \boxed{\iint\limits_\Pi f(x,y)dxdy = \int_{a}^{b}dx \int_{c}^{d}f(x)dy}
    \]
\end{theorem}

\begin{proof}
    Возьмем произвольное разбиение отрезков $ [a;b] $ и $ [c;d] $ точками
    \[
        \begin{array}{l}
            a = x_0 < x_1 < \ldots < x_n = b \\
            c = y_0 < y_1 < \ldots < y_m = d
        \end{array}
    \]
    и обозначим $ \Pi_1,\ldots,\Pi_n $ и $ \Pi_1',\ldots,\Pi_m' $ соответсвующие промежутки разбиения.

    Тогда
    \[
        \Pi = \bigcup\limits_{i=1}^{n}\bigcup\limits_{j=1}^{m}\Pi_{ij},
    \]
    где $ \Pi_{ij} = \big\{(x,y): \ x \in \Pi_i, \ y \in \Pi_j'\big\} $.

    Положим
    \[
        M_{ij} = \underset{(x,y)\in\Pi_{ij}}{\sup}f(x,y), \quad m_{ij} = \underset{(x,y)\in\Pi_{ij}}{\inf}f(x,y).
    \]

    Так как $ \int_{c}^{d}f(x,y)dy \ \exists \ \forall x \in [a;b] $, то $ \forall x \in \Pi_i $ справедливо неравенство
    \[
        m_{ij}\cdot\Delta y_i \leqslant \int_{y_{i-1}}^{y_i}f(x,y)dy \leqslant M_{ij}\cdot\Delta y_i,
    \]
    где $ \Delta y_i = y_i - y_{j-1} $.

    Суммируем эти неравенства по индексу $ j $:
    \begin{equation}\label{eq:for_proof7}
        \sum_{j=1}^{m}m_{ij}\Delta y_i \leqslant \int_{c}^{d}f(x,y)dy \leqslant \sum_{j=1}^{m}M_{ij}\Delta y_j
    \end{equation}

    Введем обозначения:
    \[
        F(x) = \int_{c}^{d}f(x,y)dy, \quad M_i = \underset{x\in\Pi_i}{\sup}F(x), \quad m_i = \underset{x\in\Pi_i}{\inf}F(x).
    \]

    Тогда из \ref{eq:for_proof7} $ \implies $
    \[
        \sum_{j=1}^{m}m_{ij}\Delta y_j \leqslant m_i \leqslant M_i \leqslant \sum_{j=1}^{m}M_{ij}\Delta y_j \implies
    \]
    \begin{equation}\label{eq:for_proof8}
        \implies 0 \leqslant M_i - m_i \leqslant \sum_{j=1}^{m}(M_{ij} - m_{ij})\Delta y_j
    \end{equation}

    Домножим на $ \Delta x_i $ неравенство \ref{eq:for_proof8} и просуммируем по $ i $:
    \begin{multline*}
        0 \leqslant \sum_{i=1}^{n}(M_i - m_i)\Delta x_i \leqslant \sum_{i=1}^{n}\sum_{j=1}^{m}(M_{ij} - m_{ij})m(\Pi_{ij}) = \\
        = \overline{S}(f,\Pi) - \underline{S}(f,\Pi) \rightarrow 0 \text{ при }l(T)\rightarrow 0,
    \end{multline*}
    так как $ f(x,y) $ интегрируема на прямоугольнике $ \implies \sum_{i=1}^{n}(M_i - m_i)\Delta x_i \rightarrow 0 $ при $ \max|\Delta x_i| \rightarrow 0 \implies F(x) $ интегрируема на $ [a;b] \implies \exists $
    \[
        \int_{a}^{b}F(x)dx = \int_{a}^{b}dx \int_{c}^{d}f(x,y)dy.
    \]

    Покажем, что он равен двойному.

    Интегрируем неравенство \ref{eq:for_proof7}
    \[
        \sum_{j=1}^{m}m_{ij}\Delta y_j\Delta x_i \leqslant \int_{x_{i-1}}^{x_i}dx \int_{c}^{d}f(x,y)dy \leqslant \sum_{j=1}^{m}M_{ij}\Delta y_j \Delta x_i.
    \]

    Суммируем по $ i $:
    \[
        \sum_{i=1}^{n}\sum_{j=1}^{m}m_{ij} m(\Pi_{ij}) \leqslant \int_{a}^{b}dx \int_{c}^{d} f(x,y)dy \leqslant \sum_{i=1}^{n}\sum_{j=1}^{m}M_{ij}m(\Pi_{ij}).
    \]
    \[
        \underline{S}(f,\Pi) \leqslant \int_{a}^{b}dx \int_{c}^{d}f(x,y)dy \leqslant \overline{S}(f,\Pi).
    \]

    С другой стороны, из условий следует, что
    \[
        \underline{S} \leqslant \iint\limits_\Pi f(x,y)dxdy \leqslant \overline{S}.
    \]

    Разность $ \overline{S} - \underline{S} $ может быть сколь угодно малой $ \implies $
    \[
        \implies \iint\limits_\Pi f(x,y)dxdy = \int_{a}^{b}dx \int_{c}^{d}f(x,y)dy.
    \]
\end{proof}

\subsection{Теорема о сведении двойного интеграла по элементарной области к повторному интегралу}

\begin{theorem}[Сведение двойного интеграла по элементарной области к повторному]
    Пусть $ \Omega $ -- элементарная относительно оси $ Oy $ область, функция $ f(x,y) $ интегрируема на $ \overline{\Omega} = \Omega \cup G\Omega $ и $ \forall x \in [a;b] \ \exists \ \int f(x,y)dx $.

    Тогда справедлива следующая формула:
    \begin{equation}\label{eq:for_proof9}
        \iint\limits_\Omega f(x,y)dxdy = \int_{a}^{b}dx \int_{\phi(x)}^{\psi(x)}f(x,y)dy.
    \end{equation}
\end{theorem}

\begin{proof}
    Есть на фотографиях (надо дописать).
\end{proof}