\section{Свойства предельной функции}

\setcounter{subsection}{85}

\subsection{Условия коммутирования двух предельных переходов}

\begin{theorem}[Условия коммутирования двупредельных переходов]\label{theorem:6.3}
    Пусть $ X,T $ -- множества, $ \mathfrak{B}_x $ -- база на $ X, \ \mathfrak{B}_T  $ -- база на $ T $, $ Y $ -- полное МП, $ f_t: X \rightarrow Y, \ f: X \rightarrow Y $:
    \begin{itemize}
        \item $ f_t \xrightrightarrows[\mathfrak{B}_T]{} f $ на $ X $,
        \item $ \forall t \in T \ \exists \underset{\mathfrak{B}_X}{\lim} = A_t $,
    \end{itemize}
    тогда существуют и равны два повторных предела:
    \[
        \underset{\mathfrak{B}_T}{\lim}\underset{\mathfrak{B}_X}{\lim} f_t(x) = \underset{\mathfrak{B}_X}{\lim}\underset{\mathfrak{B}_T}{\lim} f_t(x).
    \]

    Запишем условия и утверждение теоремы в форме диаграмы:
    \[
        \begin{matrix}
            f_t(x)                                                      & \xrightrightarrows[\mathfrak{B}_T]{} & f(x)                                               \\
            {\scriptstyle\forall t, \ \mathfrak{B}_X} \xdownarrow[22pt] &                                      & \xdashdownarrow[20pt] {\scriptstyle\mathfrak{B}_X} \\
            A_t                                                         & \xdashrightarrow[\mathfrak{B}_T]{}   & A
        \end{matrix}
    \]
    \[
        \rightarrow \text{ -- дано,}\quad \dashrightarrow \text{ -- утверждение}
    \]
\end{theorem}

\begin{proof}
    Докажем наличие нижней стрелки, то есть покажем, что
    \[
        \exists \underset{\mathfrak{B}_T}{\lim} = A.
    \]

    Пусть $ \epsilon > 0 $ задано. Выберем элемент $ B_t \in \mathfrak{B}_T \ \forall t_1,t_2 \in B_t $ и $ \forall x \in X $
    \[
        \rho\big(f_{t_1}(x),f_{t_2}(x)\big) < \frac{\epsilon}{2},
    \]
    это можно сделать, так как $ \exists $ равномерная сходимость $ f_t $ к $ f $ по $ \mathfrak{B}_T $ на $ X $.

    Зафиксируем $ t_1 $ и $ t_2 $ и перейдем к пределу по базе $ \mathfrak{B}_X $ в неравенстве
    \[
        \rho(A_{t_1},A_{t_2}) < \frac{\epsilon}{2} < \epsilon.
    \]

    Таким образом для функции $ A_t: T \rightarrow Y $ выполняются условия критерия Коши $ \exists $-ия предела функции по базе $ \mathfrak{B}_T \implies \exists \underset{\mathfrak{B}_T}{\lim}A_t = A $.

    Покажем, что $ \underset{\mathfrak{B}_X}{\lim}f(x) = A $. Рассмотрим
    \[
        \rho\big(f(x),A\big) \leqslant \rho\big(f(x),f_t(x)\big) + \rho\big(f_{t_2}(x),A_t\big) + \rho_t(A_t,A).
    \]

    Пусть $ \epsilon > 0 $ задано. Выберем $ B_t' \in \mathfrak{B}_T: \ \forall t \in B_t' $ и $ \forall x \in X $
    \[
        \rho\big(f(x),f_t(x)\big) < \frac{\epsilon}{3}.
    \]

    Затем выберем $ B_t'' \in \mathfrak{B}_T: \ \forall t \in B_t'' $
    \[
        \rho(A_t,A) < \frac{\epsilon}{3}.
    \]

    Зафиксируем $ t \in B_t' \cap B_t''$.

    Выберем $ B_x \in \mathfrak{B}_X: \ \forall x \in B_x $
    \[
        \rho\big(f_t(x),A_t\big) < \frac{\epsilon}{3} \quad (f_t \rightarrow A_t).
    \]

    Тогда $ \forall x \in B_x $
    \[
        \rho\big(f(x),A\big) < \frac{\epsilon}{3} + \frac{\epsilon}{3} + \frac{\epsilon}{3} = \epsilon.
    \]
\end{proof}

\subsection{Непрерывность предельной функции}

\begin{theorem}[Непрерывность предельной функции]
    Пусть $ X,Y $ -- метрические пространства, $ \mathfrak{B} $ -- база на $ T, \ f_t: X \rightarrow Y, \ f: X \rightarrow Y $:
    \begin{itemize}
        \item $ \forall t \in T $ функция $ f_t $ непрерывна в точке $ x_0 \in X $,
        \item семейство $ f_t \xrightrightarrows[\mathfrak{B}]{} f $ на $ X $,
    \end{itemize}
    тогда функция $ f $ непрерывна в точке $ x_0 $.
\end{theorem}

\begin{proof}
    Имеем
    \[
        \begin{matrix}
            f_t(x)                                            & \xrightrightarrows[\mathfrak{B}]{} & f(x)                  &          \\
            \begin{array}{c}
                {\scriptstyle\forall t \text{ при }} \\
                {\scriptstyle x \rightarrow x_0}
            \end{array}\xdownarrow[22pt] &                                    & \xdashdownarrow[20pt] &                               \\
            f_t(x_0)                                          & \xdashrightarrow[\mathfrak{B}]{}   & A                     & = f(x_0)
        \end{matrix}
    \]
\end{proof}

\subsection{Интегрируемость предельной функции}

\begin{theorem}[Интегрируемость предельной функции]\label{theorem:6.9.2}
    Пусть $ f_t: [a;b] \rightarrow \R, \ f: [a;b] \rightarrow \R $:
    \begin{itemize}
        \item $ \forall t \in T \ f_t $ интегрируема по Риману на $ [a;b] $,
        \item $ f_t \xrightrightarrows[\mathfrak{B}]{} f $ на $ [a;b] $ ($ \mathfrak{B} $ -- база на $ T $),
    \end{itemize}
    тогда:
    \begin{enumerate}
        \item $ f $ интегрируема по Риману на $ [a;b] $.
        \item \[
                  \int_{a}^{b}f(x)dx = \underset{\mathfrak{B}}{\lim}\int_{a}^{b}f_t(x)dx \iff \underset{\mathfrak{B}}{\lim}\int_{a}^{b}f_t(x)dx = \int_{a}^{b}\underset{\mathfrak{B}}{\lim}f_t(x)dx.
              \]
    \end{enumerate}
\end{theorem}

\begin{proof}
    \begin{multline*}
        \int_{a}^{b}f(x)dx = \underset{\lambda(P)\rightarrow0}{\lim}\sum_{i=1}^{n}f(\xi_i)\Delta x_i = \\
        = \underset{\lambda(P)\rightarrow0}{\lim} \sigma \big(f(P,\xi)\big) = \underset{\lambda(P)\rightarrow0}{\lim} \underset{\mathfrak{B}}{\lim}\sigma\big(f_t,(P,\xi)) = \underset{\mathfrak{B}}{\lim}\underset{\lambda(P)\rightarrow0}{\lim}\sigma \big(f_t,(P,\xi)\big) = \\
        = \underset{\mathfrak{B}}{\lim}\int_{a}^{b}f_t(x)dx.
    \end{multline*}

    \begin{equation}\label{eq:for_proof3}
        \begin{matrix}
            \sigma_t = & \sigma\big(f_t,(P,\xi)\big)           & \overset{\xdashrightarrow[]{}}{\xdashrightarrow[\mathfrak{B}]{}} & \sigma\big(f,(P,\xi)\big)                          \\
                       & \begin{array}{r}
                             {\scriptstyle \forall t} \\
                             {\scriptstyle \lambda(P)\rightarrow0}
                         \end{array}\xdownarrow[22pt] &                                                                  & \xdashdownarrow[20pt] {\scriptstyle \lambda(P)\rightarrow0} \\
                       & \int_{a}^{b}f_t(x)dx                  & \xdashrightarrow[\mathfrak{B}]{}                                 & \int_{a}^{b}f(x)dx
        \end{matrix}
    \end{equation}
    \begin{center}
        (я не научился делать утверждение для равномерной сходимости)
    \end{center}

    Пусть $ \mathcal{P} $ -- множество разбиений с отмеченными точками отрезка $ [a;b] $. Тогда функции $ \sigma\big(f_t,(P,\xi)\big) $ и $ \sigma\big(f,(P,\xi)\big) $ функции на $ \mathcal{P} $.

    Покажем, что семейство $ \sigma_t = \sigma\big(f_t,(P,\xi)\big) $ сходится равномерно к функции $ \sigma\big(f,(P,\xi)\big) $:
    \begin{multline*}
        \Big|\sigma\big(f_t,(P,\xi)\big) - \sigma\big(f,(P,\xi)\big)\Big| = \\
        = \left|\sum_{i=1}^{n}f_t(\xi_i)\Delta x_i - \sum_{i=1}^{n}f(\xi_i)\Delta x_i\right| \leqslant \sum_{i=1}^{n}\big|f_t(\xi_i) - f(\xi_i)\big|\Delta x_i.
    \end{multline*}

    Пусть $ \epsilon > 0 $ задано. Выберем элемент $ B \in \mathfrak{B}: \ \forall t \in B, \ \forall x \in [a;b] $
    \[
        \big|f_t(x) - f(x)\big| < \frac{\epsilon}{b - a}.
    \]

    Тогда
    \[
        \sum_{i=1}^{n}\big|f_t(\xi_i) - f(\xi_i)\big|\Delta x_i < \frac{\epsilon}{b - a}\sum_{i=1}^{n}\Delta x_i = \frac{\epsilon}{b-a}\cdot(b-a) = \epsilon.
    \]

    Таким образом $ | \sigma_t - \sigma | < \epsilon \implies \sigma_t \xrightrightarrows[\mathfrak{B}]{} \sigma \implies $ по теореме \ref{theorem:6.3} все стрелки в диаграмме \ref{eq:for_proof3} доказаны $ \implies $ все переходы в равенстве законны.
\end{proof}

\subsection{Теорема Дини}

\begin{theorem}[Дини]
    Пусть $ X $ -- компактное метрическое пространство. Последовательность $ f_n: X \rightarrow\R $ монотонна на $ X $ и $ \forall x \ f_n $ непрерывна на $ X $.

    Если $ f: X \rightarrow \R $ непрерывна на $ X $, то эта сходимость равномерная.
\end{theorem}

\begin{proof}
    Для $ \forall x \in X $ выберем номер $ N_x: \ \forall n > N_x $
    \[
        \big|f_n(x) - f(x)\big| < \epsilon, \quad \text{где }\epsilon>0 \text{ задано}.
    \]

    Так как $ f_{N_x} $ и $ f $ непрерывны, то $ \exists U_x \subset X \ \forall y \in U_x $
    \[
        \big|f_{N_x}(y) - f(y)\big| < \epsilon, \quad \text{(используя непрерывность)}.
    \]

    Таким образом для каждого $ x \in X $ построим такую окружность $ U_x $.

    Семейство таких окрестностей является открытым покрытием пространства $ X $.

    Пусть $ \{U_{X_1},U_{X_2},\ldots,U_{X_k}\} $ -- конечное подпокрытие $ X $.

    Положим $ N = \max\{N_{X_1},N_{X_2},\ldots,N_{X_1=k}\} $. Тогда $ \forall n > N, \ \forall x \in X $
    \[
        \big|f_n(x) - f(x)\big| < \epsilon.
    \]

    Это и есть равномерная сходимость.
\end{proof}

\subsection{Дифференцируемость предельной функции}

\begin{theorem}[Дифференцируемость предельной функции]\label{theorem:6.9.5}
    Пусть $ -\infty < a < b < +\infty $ ($ a,b $ -- конечны), $ f_t: (a;b)\rightarrow\R, \ f:(a;b) \rightarrow \R $:
    \begin{itemize}
        \item $ \forall t \in T \ f_t $ дифференцируема на $ (a;b) $,
        \item $ \exists \phi: (a;b)\rightarrow\R: \ f_t' \xrightrightarrows[\mathfrak{B}]{} \phi $ на $ (a;b) $,
        \item $ \exists x_0 \in (a;b): \ f_t(x_0) \rightarrow f(x_0) $,
    \end{itemize}
    тогда:
    \begin{enumerate}
        \item $ f_t \xrightrightarrows[\mathfrak{B}]{} f $ на $ (a;b) $.
        \item $ f $ дифференцируема на $ (a;b) $.
        \item $ \forall x \in (a;b) \ f'(x) = \phi(x) $.
    \end{enumerate}
\end{theorem}

\begin{proof}
    Докажем, что семейство функций $f_t$ сходится к $f$ равномерно на $(a;b)$:
    \begin{multline*}
        \big|f_{t_1}(x) - f_{t_2}(x)\big| = \\
        = \big|f_{t_1}(x) - f_{t_2}(x) + f_{t_1}(x_0) - f_{t_1}(x_0) + f_{t_2}(x_0) - f_{t_2}(x_0)\big| \leqslant \\
        \leqslant \big|(f_{t_1}(x) - f_{t_1}(x_0)) - (f_{t_2}(x) - f_{t_2}(x_0))\big| + \big|f_{t_1}(x_0) - f_{t_2}(x_0)\big| = \\
        = \big|f_{t_1}'(\xi) - f_{t_2}'(\xi)\big| \cdot \big|x - x_0\big| + \big|f_{t_1}(x_0) - f_{t_2}(x_0)\big|.
    \end{multline*}

    Пусть $\epsilon > 0$ задано. Выберем $B \in \mathfrak{B}$ ($\mathfrak{B}$ -- база на $T$) $\forall t_1,t_2 \in B$
    \[
        \big|f_{t_1}(x_0) - f_{t_2}(x_0)\big| < \frac{\epsilon}{2}
    \]
    и $\forall x \in (a;b)$ и $\forall t_1',t_2' \in B$:
    \[
        \big|f_{t_2'}'(x) - f_{t_2'}'(x)\big| < \frac{\epsilon}{2(b-a)}.
    \]

    Тогда $\forall t_1,t_2 \in B$ и $\forall x \in (a;b)$
    \[
        \big|f_{t_1}(x) - f_{t_2}(x)\big| < \frac{\epsilon}{2(b-a)}\cdot (b-a) + \frac{\epsilon}{2} = \epsilon.
    \]

    Итак, $f_t \xrightrightarrows[\mathfrak{B}]{} f$ на $(a;b)$. Покажем, что предельная функция $f$ дифференцируема на $(a;b)$ и $\forall x \in (a;b) $
    \[
        f'(x)=\phi(x):
    \]
    \begin{multline*}
        f'(x) = \underset{h\rightarrow0}{\lim}\frac{f(x + h) - f(x)}{h} = \\
        = \underset{h\rightarrow0}{\lim}\frac{\underset{\mathfrak{B}}{\lim}f_t(x + h) - \underset{\mathfrak{B}}{\lim}f(x)}{h} = \underset{h\rightarrow 0}{\lim}\underset{\mathfrak{B}}{\lim}\frac{f_t(x + h) - f_t(x)}{h} \overset{(\star)}{=} \\
        \overset{(\star)}{=} \underset{\mathfrak{B}}{\lim}\underset{h\rightarrow0}{\lim}\frac{f_t(x + h) - f_t(x)}{h} = \underset{\mathfrak{B}}{\lim}f_t'(x) = \phi(x).
    \end{multline*}

    Покажем законность перехода $(\star)$. Пусть $x \in (a;b), \ x + h \in (a;b)$. Рассмотрим
    \[
        \begin{matrix}
            F_t(h) = & \frac{f_t(x+h) - f_t(x)}{h} & \overset{\xdashrightarrow[]{}}{\xdashrightarrow[\mathfrak{B}]{}} & \frac{f(x + h) - f(x)}{h} & = F(h)                                                     \\
                     & \xdownarrow[22pt]           &                                                                  & \xdashdownarrow[20pt]                                                                  \\
                     & f_t'(x)                     & \xrightrightarrows[\mathfrak{B}]{}                               & \phi(x)                   & \overset{\xdashrightarrow[]{}}{\xdashrightarrow[]{}} f'(x)
        \end{matrix}
    \]

    Докажем существование двойной верхней стрелки. Имеем:
    \[
        \left.\begin{array}{c}
            f_t(x) \xrightarrow[\mathfrak{B}]{} f(x) \\
            f_t(x + h) \xrightarrow[\mathfrak{B}]{} f(x+h)
        \end{array}\right\} \implies F_t(h) \xrightarrow[\mathfrak{B}]{} F(h),
    \]
    \begin{multline*}
        \big|F_{t_1}(h) - F_{t_2}(h)\big| = \\
        = \bigg| \frac{\overbrace{f_{t_1}(x + h) - f_{t_1}(x)}^{= f_{t_1}'(\xi) \cdot |h|}}{h} - \frac{f_{t_2}(x+h) - f_{t_2}(x)}{h}\bigg| = \\
        = \frac{1}{|h|}\big|f_{t_1}'(\xi) \cdot |h| - f_{t_2}'(\xi)\cdot |h| \big| = \\
        = \big|f_{t_1}'(\xi) - f_{t_2}'(\xi)\big|, \ \xi \in (x;x+h).
    \end{multline*}

    Пусть $\epsilon > 0$ задано. Тогда $\exists B \in \mathfrak{B}: \ \forall t_1,t_2 \in B$
    \[
        \big|f_{t_1}'(\xi) - f_{t_2}'(\xi)\big| < \epsilon.
    \]

    Таким образом семейство $\big\{F_t(h)\big\}$ сходится равномерно на $(a;b)$.

    Правая вертикальная стрелка следует из теоремы \ref{theorem:6.3}.
\end{proof}