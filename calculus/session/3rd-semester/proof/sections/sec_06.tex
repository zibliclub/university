\section{Функциональный ряд}

\setcounter{subsection}{79}

\subsection{Критерий Коши равномерной сходимости ряда}

\begin{definition}[Функциональный ряд]
    Пусть $f_n: X \rightarrow\R, \ X$ -- произвольное множество.

    \emph{Функциональным рядом} называется выражение вида
    \begin{equation}\label{eq:6.14}
        \sum_{n=1}^{\infty}f_n(x)
    \end{equation}
\end{definition}

\begin{theorem}[Критерий Коши равномерной сходимости функциональных рядов]
    Ряд \ref{eq:6.14} равномерно сходится на $X \iff \forall \epsilon > 0 \ \exists N: \ \forall n > N \ \forall p > 0 \ \forall x \in X$
    \[
        \big|f_{n+1}(x) + \ldots + f_{n+p}(x)\big| < \epsilon.
    \]
\end{theorem}

\begin{proof}
    Самостоятельно (круто).
\end{proof}

\subsection{Следствие из критерия Коши равномерной сходимости ряда}

\begin{corollary}
    Пусть $X,Y$ -- метрические пространства, $E \subset X, \ x_0 \in E$ -- предельная точка для $E$. Семейство $f_t: X \rightarrow Y$:
    \begin{enumerate}
        \item $f_t$ сходится на $E$ по базе $\mathfrak{B}$.
        \item $f_t$ расходится в точке $x_0$ по базе $\mathfrak{B}$.
        \item $\forall t \ f_t$ непрерывно в точке $x_0$.
    \end{enumerate}

    Тогда на $E$ семейство $f_t$ сходится неравномерно.
\end{corollary}

\begin{corollary}[Из следствия выше]
    Если $f_t:(a;b] \rightarrow D, \ D$ -- область в $Y$:
    \begin{enumerate}
        \item $\forall t \ f_t$ непрерывно в точке $b$.
        \item $f_t$ сходится на $(a;b)$ по $\mathfrak{B}$.
        \item $f_t$ расходится в точке $b$.
    \end{enumerate}

    Тогда на $(a;b) \ f_t$ сходится неравномерно.
\end{corollary}

\begin{corollary}[Которое нужно доказать]
    Если:
    \begin{enumerate}
        \item Ряд \ref{eq:6.14} сходится на $(a;b)$.
        \item Расходится в точке $b$.
        \item $\forall n \ f_n(x)$ непрерывно в точке $b$.
    \end{enumerate}

    Тогда ряд \ref{eq:6.14} сходится на $(a;b)$ неравномерно.
\end{corollary}

\begin{proof}
    Следует из предыдущих следствий.
\end{proof}

\subsection{Признак сравнения}

Я не нашел.

\subsection{Признак Вейерштрасса}

\begin{corollary}[Мажорантный признак Вейерштрасса]
    Пусть
    \begin{enumerate}
        \item $\forall n \ \exists M_n$:
              \[
                  \big|a_n(x)\big| \leqslant M_n \quad \forall x \in X.
              \]
        \item Ряд $\sum_{n=1}^{\infty} M_n$ сходится.
    \end{enumerate}

    Тогда ряд $\sum_{n=1}^{\infty}a_n(x)$ сходится на $X$ абсолютно и равномерно.
\end{corollary}

\begin{proof}
    А где
\end{proof}

\subsection{Признак Абеля}

\begin{theorem}[Признак Абеля]\label{theorem:6.9.1}
    Пусть функции $a_n(x)$ и $b_n(x)$ удовлетворяют условиям:
    \begin{itemize}
        \item ряд $\sum_{n=1}^{\infty}a_n(x)$ сходится равномерно на $X$,
        \item последовательность $\big\{b_n(x)\big\}$ равномерно ограничена на $X$ и монотонна (то есть $\exists L > 0: \ \forall n \in \N$ и $\forall x \in X \quad \big|b_n(x)\big| \leqslant L$),
    \end{itemize}
    тогда ряд
    \[
        \sum_{n=1}^{\infty}\big(a_n(x) \cdot b_n(x)\big)
    \]
    сходится на $X$ равномерно.

\end{theorem}

\begin{proof}
    Рассмотрим
    \begin{multline*}
        \big|a_{n+1}(x)\cdot b_{n+1}(x) + a_{n+2}(x)\cdot b_{n+2}(x) + \ldots + a_{n+p}(x)\cdot b_{n+p}(x)\big| = \\
        = \Big|(A_{n+1} - A_n)\cdot b_{n+1}(x) + \big((A_{n+2} - A_n) - (A_{n+1} - A_n)\big)\cdot b_{n+2}(x) + \ldots \\
        \ldots + \big((A_{n+p} -A_n) - (A_{n+p-1} - A_n)\big)\cdot b_{n+p}(x)\Big| = \\
        = \big|(A_{n+1} - A_n)\cdot b_{n+1}(x) + (A_{n+2} - A_n)\cdot b_{n+2}(x) - (A_{n+1} - A_n)\cdot b_{n+2}(x) + \ldots \\
        \ldots + (A_{n+p} -A_n)\cdot b_{n+p}(x) - (A_{n+p-1} - A_n)\cdot b_{n+p}(x)\big| = \\
        = \Big|(A_{n+1}-A_n)\cdot\big(b_{n+1}(x)-b_{n+2}(x)\big) + (A_{n+2}-A_n)\cdot\big(b_{n+2}(x)-b_{n+3}(x)\big) + \ldots \\
        \ldots + (A_{n+p-1}-A_n)\cdot\big(b_{n+p-1}(x)-b_{n+p}(x)\big) + (A_{n+p} - A_n)\cdot b_{n+p}(x)\Big| = \\
        = \left|\sum_{k=1}^{p-1}\Bigl((A_{n+k} - A_n)\cdot\big(b_{n+k}(x) - b_{n+k-1}(x)\big)\Bigr) + (A_{n+p} - A_n)\cdot b_{n+p}(x) \right| \leqslant \\
        \leqslant \sum_{k=1}^{p-1}\Bigl(|A_{n+k} - A_n| \cdot \big|b_{n+k}(x) - b_{n+k+1}(x)\big|\Bigr) + | A_{n+p} - A_n | \cdot | b_{n+p}(x) |.
    \end{multline*}

    Если выполнены условия Абеля, то $ \forall \epsilon > 0 $ выберем $ N: \ \forall n > N, \ \forall p > 0 \ \forall x \in X $
    \[
        \big|a_{n+1}(x) + a_{n+2}(x) + \ldots + a_{n+p}(x)\big| < \frac{\epsilon}{3\cdot L}.
    \]

    Тогда
    \begin{multline*}
        \sum_{k=1}^{p-1}\Bigl(|A_{n+k} - A_n| \cdot \big|b_{n+k}(x) - b_{n+k+1}(x)\big|\Bigr) + | A_{n+p} - A_n | \cdot | b_{n+p}(x) | < \\
        < \frac{\epsilon}{3\cdot L}\left(\sum_{k=1}^{p-1}\big|b_{n+k}(x) - b_{n+k+1}(x)\big| + \big|b_{n+p}(x)\big|\right) \leqslant \\
        \leqslant \frac{\epsilon}{3\cdot L}\Bigl(\big|b_{n+1}(x)\big| + 2\big|b_{n+p}(x)\big|\Bigr) < \frac{\epsilon}{3\cdot L} \cdot 3\cdot L = \epsilon \implies
    \end{multline*}
    $ \implies $ по критерию Коши, $ \sum_{n=1}^{\infty}\big(a_n(x)\cdot b_n(x)\big) $ сходится равномерно на $ X $.
\end{proof}

\subsection{Признак Дирихле}

\begin{theorem}[Признак Дирихле]\leavevmode
    \begin{itemize}
        \item частичные суммы ряда $\sum_{n=1}^{\infty}a_n(x)$ равномерно ограничены на $X$ (то есть $\exists M > 0: \ \forall n$ и $\forall x \in X \quad \big|\sum_{k=1}^{n}a_k(x)\big| \leqslant M$),
        \item последовательность $ \big\{b_n(x)\big\} $ монотонна и равномерно на $ X $ стремится к $ 0 $,
    \end{itemize}
    тогда ряд
    \[
        \sum_{n=1}^{\infty}\big(a_n(x) \cdot b_n(x)\big)
    \]
    сходится на $X$ равномерно.
\end{theorem}

\begin{proof}
    Рассмотрим
    \begin{multline*}
        \big|a_{n+1}(x)\cdot b_{n+1}(x) + a_{n+2}(x)\cdot b_{n+2}(x) + \ldots + a_{n+p}(x)\cdot b_{n+p}(x)\big| = \\
        = \Big|(A_{n+1} - A_n)\cdot b_{n+1}(x) + \big((A_{n+2} - A_n) - (A_{n+1} - A_n)\big)\cdot b_{n+2}(x) + \ldots \\
        \ldots + \big((A_{n+p} -A_n) - (A_{n+p-1} - A_n)\big)\cdot b_{n+p}(x)\Big| = \\
        = \big|(A_{n+1} - A_n)\cdot b_{n+1}(x) + (A_{n+2} - A_n)\cdot b_{n+2}(x) - (A_{n+1} - A_n)\cdot b_{n+2}(x) + \ldots \\
        \ldots + (A_{n+p} -A_n)\cdot b_{n+p}(x) - (A_{n+p-1} - A_n)\cdot b_{n+p}(x)\big| = \\
        = \Big|(A_{n+1}-A_n)\cdot\big(b_{n+1}(x)-b_{n+2}(x)\big) + (A_{n+2}-A_n)\cdot\big(b_{n+2}(x)-b_{n+3}(x)\big) + \ldots \\
        \ldots + (A_{n+p-1}-A_n)\cdot\big(b_{n+p-1}(x)-b_{n+p}(x)\big) + (A_{n+p} - A_n)\cdot b_{n+p}(x)\Big| = \\
        = \left|\sum_{k=1}^{p-1}\Bigl((A_{n+k} - A_n)\cdot\big(b_{n+k}(x) - b_{n+k-1}(x)\big)\Bigr) + (A_{n+p} - A_n)\cdot b_{n+p}(x) \right| \leqslant \\
        \leqslant \sum_{k=1}^{p-1}\Bigl(|A_{n+k} - A_n| \cdot \big|b_{n+k}(x) - b_{n+k+1}(x)\big|\Bigr) + | A_{n+p} - A_n | \cdot | b_{n+p}(x) |.
    \end{multline*}

    Пусть выполнены условия Дирихле. Тогда $ \forall \epsilon > 0 $ выберем $ N: \ \forall n > N \ \forall x > X $
    \[
        \big|b_n(x)\big| < \frac{\epsilon}{3\cdot M}.
    \]
    \begin{multline*}
        \sum_{k=1}^{p-1}\Bigl(|A_{n+k} - A_n| \cdot \big|b_{n+k}(x) - b_{n+k+1}(x)\big|\Bigr) + | A_{n+p} - A_n | \cdot | b_{n+p}(x) | \leqslant \\
        \leqslant \frac{\epsilon}{p\cdot M}\left(\sum_{k=1}^{p-1}| A_{n+k} - A_n | + | A_{n+p} - A_n |  \right) = \\
        = \frac{\epsilon}{p\cdot M}\Big(\big| a_{n+1}(x) \big| + \big|a_{n+1}(x) + a_{n+2}(x)\big| + \ldots + \big|a_{n+1}(x) + \ldots + a_{n+p}(x)\big|\Big) \leqslant \\
        \leqslant \frac{\epsilon}{p\cdot M}\cdot (p \cdot M) = \epsilon.
    \end{multline*}
\end{proof}