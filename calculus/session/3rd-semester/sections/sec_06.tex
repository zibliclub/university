\section{Функциональный ряд}

\setcounter{subsection}{76}

\subsection{Функциональный ряд}

\begin{definition}[Функциональный ряд]
    Пусть $f_n: X \rightarrow\R, \ X$ -- произвольное множество.

    \emph{Функциональным рядом} называется выражение вида
    \begin{equation}\label{eq:6.14}
        \sum_{n=1}^{\infty}f_n(x)
    \end{equation}
\end{definition}

\subsection{Поточечная сходимость функциональных рядов}

\begin{note}
    Говорят, что ряд \ref{eq:6.14} сходится на $X$ \emph{поточечно}, если на $X$ сходится поточечно последовательность его частичных сумм.
    
    Ряд \ref{eq:6.14} {равномерно} сходится на $X$, если на $X$ равномерно сходится последовательность его частичных сумм.
\end{note}

\subsection{Равномерная сходимость функциональных рядов}

\begin{theorem}
    Пусть ряды $ (A),(B) $ такие, что:
    \begin{enumerate}
        \item $\forall n$ функции $a_n(x)$ и $b_n(x)$ определены на $X$.
        \item $\exists N: \ \forall n > N$
              \[
                  \big|a_n(x)\big| \leqslant b_n(x) \quad \forall x \in X.
              \]
        \item Ряд $(B)$ сходится на $X$ равномерно.
    \end{enumerate}

    Тогда ряд $(A)$ сходится на $X$ равномерно.
\end{theorem}

\subsection{Критерий Коши равномерной сходимости ряда}

\begin{theorem}[Критерий Коши равномерной сходимости функциональных рядов]
    Ряд \ref{eq:6.14} равномерно сходится на $X \iff \forall \epsilon > 0 \ \exists N: \ \forall n > N \ \forall p > 0 \ \forall x \in X$
    \[
        \big|f_{n+1}(x) + \ldots + f_{n+p}(x)\big| < \epsilon.
    \]
\end{theorem}

\subsection{Следствие из критерия Коши равномерной сходимости ряда}

\begin{corollary}
    Если:
    \begin{enumerate}
        \item Ряд \ref{eq:6.14} сходится на $(a;b)$.
        \item Расходится в точке $b$.
        \item $\forall n \ f_n(x)$ непрерывно в точке $b$.
    \end{enumerate}

    Тогда ряд \ref{eq:6.14} сходится на $(a;b)$ неравномерно.
\end{corollary}

\subsection{Признак сравнения}

Я не нашел.

\subsection{Признак Вейерштрасса}

\begin{corollary}[Мажорантный признак Вейерштрасса]
    Пусть
    \begin{enumerate}
        \item $\forall n \ \exists M_n$:
              \[
                  \big|a_n(x)\big| \leqslant M_n \quad \forall x \in X.
              \]
        \item Ряд $\sum_{n=1}^{\infty} M_n$ сходится.
    \end{enumerate}

    Тогда ряд $\sum_{n=1}^{\infty}a_n(x)$ сходится на $X$ абсолютно и равномерно.
\end{corollary}

\subsection{Признак Абеля}

\begin{theorem}[Признак Абеля]\label{theorem:6.9.1}
    Пусть функции $a_n(x)$ и $b_n(x)$ удовлетворяют условиям:
    \begin{itemize}
        \item ряд $\sum_{n=1}^{\infty}a_n(x)$ сходится равномерно на $X$,
        \item последовательность $\big\{b_n(x)\big\}$ равномерно ограничена на $X$ и монотонна (то есть $\exists L > 0: \ \forall n \in \N$ и $\forall x \in X \quad \big|b_n(x)\big| \leqslant L$),
    \end{itemize}
    тогда ряд
    \[
        \sum_{n=1}^{\infty}\big(a_n(x) \cdot b_n(x)\big)
    \]
    сходится на $X$ равномерно.

\end{theorem}

\subsection{Признак Дирихле}

\begin{theorem}[Признак Дирихле]\leavevmode
    \begin{itemize}
        \item частичные суммы ряда $\sum_{n=1}^{\infty}a_n(x)$ равномерно ограничены на $X$ (то есть $\exists M > 0: \ \forall n$ и $\forall x \in X \quad \big|\sum_{k=1}^{n}a_k(x)\big| \leqslant M$),
        \item последовательность $ \big\{b_n(x)\big\} $ монотонна и равномерно на $ X $ стремится к $ 0 $,
    \end{itemize}
    тогда ряд
    \[
        \sum_{n=1}^{\infty}\big(a_n(x) \cdot b_n(x)\big)
    \]
    сходится на $X$ равномерно.
\end{theorem}