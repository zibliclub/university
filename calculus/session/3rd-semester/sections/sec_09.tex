\section{Интегралы, зависящие от параметра}

\setcounter{subsection}{103}

\subsection{Собственный интеграл, зависящий от параметра}

\begin{definition}[Интеграл, зависящий от параметра]
    \emph{Интегралом, зависящим от параметра} называется функция
    \[
        F(y) = \int_{E_y}f(x,y)dx = \int_{\alpha(y)}^{\beta(y)}f(x,y)dx.
    \]
\end{definition}

\subsection{Теорема о непрерывности собственного интеграла, зависящего от параметра}

Тут не уверен, оно или нет.

\begin{theorem}\label{theorem:7.1.1}
    Если функция $f(x,y)$ непрерывна на $P = [a;b] \times [c;d]$, то функция $F(y) = \int_{a}^{b}f(x,y)dx$ непрерывна на $[c;d]$.
\end{theorem}

\subsection{Теорема о дифференцировании собственного интеграла, зависящего от параметра}

\begin{theorem}[О дифференцировании собственного интеграла, зависящего от параметра]\label{theorem:7.1.2}
    Пусть:
    \begin{itemize}
        \item $\alpha(y), \beta(y)$ -- дифференцируемые на $[c;d]$,
        \item $\forall y \in [c;d] \ a\leqslant \alpha(y) \leqslant b$ и $a \leqslant \beta(y) \leqslant b$,
        \item $f(x,y)$ -- непрерывна на $P = [a;b] \times [c;d]$,
        \item $\frac{\delta f}{\delta y}$ -- непрерывна на $P$,
    \end{itemize}
    тогда $F(y) = \int_{\alpha(y)}^{\beta(y)}f(x,y)dx$ дифференцируема на $[c;d]$ и
    \[
        F'(y) = \int_{\alpha(y)}^{\beta(y)}\frac{\delta f}{\delta y}(x,y)dx + f\big(\beta(y),y\big) \cdot \beta'(y) - f\big(\alpha(y),y\big)\cdot \alpha'(y)
    \]
    \begin{center}
        (формула Лейбница)
    \end{center}
\end{theorem}

\subsection{Теорема об интегрировании собственного интеграла, зависящего от параметра}

\begin{theorem}[Об интегрировании собственного интеграла по параметру]\label{theorem:7.1.3}
    Если $ f(x,y) $ непрерывна на $ P = [a;b] \times [c;d] $, то функция $ F(y) = \int_{a}^{b}f(x,y)dx $ интегрируема на $ [c;d] $ и
    \[
        \int_{c}^{d}F(y)dy = \int_{c}^{d}\left(\int_{a}^{b}f(x,y)dx\right)dy = \int_{a}^{b}\left(\int_{c}^{d}f(x,y)dy\right)dx.
    \]

    Обычно пишут:
    \[
        \int_{c}^{d}dy \int_{a}^{b}f(x,y)dx = \int_{a}^{b}dx \int_{c}^{d}f(x,y)dy.
    \]
\end{theorem}

\subsection{Несобственный интеграл, зависящий от параметра}

\begin{definition}[Несобственный интеграл, зависящий от параметра]
    Пусть $ \forall y \in Y \ \exists \int_{a}^{\omega}f(x,y)dx $.

    \emph{Несобственным интегралом, зависящим от параметра $ y $} называется функция
    \begin{equation}\label{eq:7.2.1}
        F(y) = \int_{a}^{\omega}f(x,y)dx.
    \end{equation}
\end{definition}

\subsection{Равномерная сходимость несобственного интеграла, зависящего от параметра}

\begin{definition}[Равномерно сходящийся интеграл]
    Говорят, что интеграл \ref{eq:7.2.1} сходится на $ Y $ \emph{равномерно}, если $ \forall \epsilon > 0 \ \exists B \in [a;\omega): \ \forall b \in (B;\omega) $
    \[
        \left|\int_{b}^{\omega}f(x,y)dx\right| < \epsilon.
    \]
\end{definition}

\subsection{Утверждение об эквивалентности сходимости несобственного интгерала, зависящего от параметра и семейства функций – интегралов по верхнему пределу, зависящих от параметра}

\begin{note}
    Далее, рассмотрим семейство функций
    \begin{equation}\label{eq:7.2.2}
        F_b(y) = \int_{a}^{b}f(x,y)dx, \ b \in [a;\omega).
    \end{equation}
\end{note}

\begin{statement}
    Интеграл \ref{eq:7.2.1} сходится на $ Y $ равномерно $ \implies $ семейство функций \ref{eq:7.2.2} сходится на $ Y $ равномерно при $ b \rightarrow \omega $.
\end{statement}

\subsection{Критерий Коши равномерной сходимости несобственных интегралов, зависящих от параметра}

\begin{theorem}[Критерий Коши равномерной сходимости несобственного интеграла зависящего от параметра]
    Интеграл $F(y) = \int_{a}^{\omega}f(x,y)dx$ равномерно сходится $\iff \forall \epsilon > 0 \ \exists B \in [a;\omega): \ \forall b_1,b_2 \in (B;\omega) \ \forall y \in Y$
    \[
        \left|\int_{b_1}^{b_2}f(x,y)dx\right| < \epsilon.
    \]
\end{theorem}

\subsection{Следствие критерия Коши равномерной сходимости несобственных интегралов, зависящих от параметра}

\begin{corollary}
    Пусть $f(x,y)$ непрерывна на множестве $[a;\omega)\times[c;d] $, \\ $\int_{a}^{\omega}f(x,y)dx$ сходится на $[c;d)$ и расходится в точке $y = d$.

    Отсюда следует, что $\int_{a}^{\omega}f(x,y)dx$ на $[c;d)$ сходится неравномерно.
\end{corollary}

\subsection{Признак Вейерштрасса и его следствие}

\begin{theorem}[Признак Вейерштраса]
    Пусть \begin{enumerate}
        \item $\forall y \in Y$ и $\forall x \in [a;\omega)$
              \[
                  \big|f(x,y)\big| \leqslant g(x,y).
              \]
        \item $\int_{a}^{\omega}g(x,y)dx$ -- равномерно сходится на $Y$.
    \end{enumerate}

    Тогда $\int_{a}^{\omega}f(x,y)dx$ -- равномерно сходится на $Y$.
\end{theorem}

\begin{corollary}
    Если $\forall y \in Y, \ \forall x \in [a;\omega)$
    \[
        \big|f(x,y)\big| \geqslant g(x),
    \]
    то из сходимости $\int_{a}^{\omega}g(x)dx \implies$ равномерна сходимость
    \[
        \int_{a}^{\omega}f(x,y)dx \text{ на }Y.
    \]
\end{corollary}

\subsection{Признак Абеля}

\begin{theorem}[Признак Абеля]
    Если \begin{enumerate}
        \item $\int_{a}^{\omega}g(x,y)dx$ равномерно сходится на $Y$.
        \item $\forall y \in Y$ функция $f(x,y)$ монотонна по $x$ и равномерно ограничена, то есть $\exists M > 0: \ \forall x \in [a;\omega)$ и $\forall y \in Y$
              \[
                  \left|f(x,y)\right| \leqslant M.
              \]
    \end{enumerate}

    Тогда
    \[
        \int_{a}^{\omega}\big(f(x,y) \cdot g(x,y)\big)dx \text{ -- сходится равномерно на }Y.
    \]
\end{theorem}

\subsection{Признак Дирихле}

\begin{theorem}[Признак Дирихле]
    Если
    \begin{enumerate}
        \item $\int_{a}^{b}g(x,y)dx$ огоаничена в совокупности, то есть $\exists L > 0: \ \forall y \in Y$ и $\forall b \in [a;\omega)$
              \[
                  \left|\int_{a}^{b}g(x,y)dx\right| \leqslant L.
              \]
        \item $\forall y \in Y \ f(x,y)$ монотонна по $x$ и $f(x,y) \rightarrow 0$ равномерно при $x \rightarrow \omega$.
    \end{enumerate}

    Тогда
    \[
        \int_{a}^{\omega}\big(f(x,y)\cdot g(x,y)\big)dx \text{ -- сходится равномерно на } Y.
    \]
\end{theorem}