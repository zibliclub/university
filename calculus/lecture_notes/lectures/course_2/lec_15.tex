\lesson{15}{от 7 нояб 2023 8:43}{Продолжение}


\section{Несобственные интегралы, зависящие от параметра}

\begin{definition}[Несобственный интеграл, зависящий от параметра]
    Пусть $ \forall y \in Y \ \exists \int_{a}^{\omega}f(x,y)dx $.

    \emph{Несобственным интегралом, зависящим от параметра $ y $} называется функция
    \begin{equation}\label{eq:7.2.1}
        F(y) = \int_{a}^{\omega}f(x,y)dx.
    \end{equation}
\end{definition}

\begin{example}
    \[
        \int_{1}^{+\infty} f(x)dx = \underset{b \rightarrow + \infty}{\lim}\int_{1}^{b}f(x)dx.
    \]
\end{example}

\begin{definition}[Равномерно сходящийся интеграл]
    Говорят, что интеграл \ref{eq:7.2.1} сходится на $ Y $ \emph{равномерно}, если $ \forall \epsilon > 0 \ \exists B \in [a;\omega): \ \forall b \in (B;\omega) $
    \[
        \left|\int_{b}^{\omega}f(x,y)dx\right| < \epsilon.
    \]
\end{definition}

\begin{note}
    Далее, рассмотрим семейство функций
    \begin{equation}\label{eq:7.2.2}
        F_b(y) = \int_{a}^{b}f(x,y)dx, \ b \in [a;\omega).
    \end{equation}
\end{note}

\begin{statement}
    Интеграл \ref{eq:7.2.1} сходится на $ Y $ равномерно $ \implies $ семейство функций \ref{eq:7.2.2} сходится на $ Y $ равномерно при $ b \rightarrow \omega $.
\end{statement}

\begin{proof}\leavevmode
    \begin{enumerate}
        \item Интеграл \ref{eq:7.2.1} сходится на $ Y $ равномерно $ \overset{\text{по опр.}}{\iff} \forall \epsilon > 0 \ \exists B \in [a;\omega): \ \forall b \in (B;\omega) $
              \[
                  \left|\int_{b}^{\omega}f(x,y)dx\right| < \epsilon.
              \]
        \item Семейство функций \ref{eq:7.2.2} равномерно сходится на $ Y $ при $ b \rightarrow \omega \overset{\text{по опр.}}{\iff} \forall \epsilon > 0 \ \exists \widetilde{B} \in [a;\omega): \ \forall b \in (\widetilde{B};\omega) $
              \[
                  \big|F_b(y) - F(y)\big| < \epsilon, \ \forall y \in Y.
              \]
              Но
              \begin{multline*}
                  \big|F_b(y) - F(y)\big| = \left|\int_{a}^{b}f(x,y)dx - \int_{a}^{\omega}f(x,y)dx\right| = \\
                  = \left|-\left(\int_{a}^{b}f(x,y)dx + \int_{a}^{\omega}f(x,y)dx\right)\right| = \left|\int_{a}^{\omega}f(x,y)dx\right|.
              \end{multline*}
    \end{enumerate}
\end{proof}

\begin{example}
    $ \int_{0}^{+\infty}e^{-xy}dx $
    \begin{enumerate}
        \item Если $ y \leqslant 0 $, то интеграл расходится.
        \item $ y > 0 $.
    \end{enumerate}

    Пусть $ y_0 > 0 $, тогда $ \forall y \geqslant y_0 $
    \[
        \int_{b}^{+\infty}e^{-xy}dx = - \frac{1}{y}e^{-xy}\Big|_b^{+\infty} = \frac{1}{y} \cdot e^{-by} \leqslant \frac{1}{y^2}\cdot e^{-by_0}.
    \]

    В самом деле,
    \[
        \left(\frac{1}{y}\cdot e^{-by}\right) = -\frac{1}{y^2}\cdot e^{-by} + \frac{1}{y}(-b)\cdot e^{-by} = -\frac{1}{y}\cdot e^{-by}\left(\frac{1}{y} + b\right) < 0.
    \]

    Отсюда $ \forall y \geqslant y_0 $ имеем, если $ \epsilon > 0 $ задано, то взяв $ B > 0: \ \frac{1}{y_0}\cdot e^{-By_0} < \epsilon $ получим $ \forall b > B \ \forall y \geqslant y_0 $
    \[
        \left|\int_{b}^{+\infty}e^{-xy}dx\right| < \epsilon \implies
    \]
    $ \implies $ на множестве $ y \geqslant y_0 \ \int_{0}^{+\infty}e^{-xy}dx $ сходится равномерно.

    Если $y_0 > y > 0$ интеграл расходится.

    В самом деле, если $\epsilon = 1$ и по заданному $B$ выберем $b = B+1$ и для $y > 0: \ \frac{1}{y}\cdot e^{-xy} > 1$. Получим:
    \[
        \left|\int_{0}^{\infty}e^{-xy}dx\right| > 1.
    \]
\end{example}

\begin{theorem}[Критерий Коши равномерной сходимости несобственного интеграла зависящего от параметра]
    Интеграл $F(y) = \int_{a}^{\omega}f(x,y)dx$ равномерно сходится $\iff \forall \epsilon > 0 \ \exists B \in [a;\omega): \ \forall b_1,b_2 \in (B;\omega) \ \forall y \in Y$
    \[
        \left|\int_{b_1}^{b_2}f(x,y)dx\right| < \epsilon.
    \]
\end{theorem}

\begin{proof}
    Для семейства функций $F_b(y) = \int_{a}^{b}f(x,y)dx$ равномерная сходимость на $Y$ при $b\rightarrow\infty$ равносильна утверждению $\forall \epsilon > 0 \ \exists B \in [a;\omega): \ \forall b_1,b_2 \in (B;\omega)$ и $\forall y \in Y$
    \[
        \big|F_{b_1}(y) - F_{b_2}(y)\big| < \epsilon,
    \]
    \begin{multline*}
        \big|F_{b_1}(y) - F_{b_2}(y)\big| = \\
        = \left|\int_{a}^{b_1}f(x,y)dx - \int_{a}^{b_2}f(x,y)dx\right| = \left|-\left(\int_{a}^{b_1}f(x,y)dx + \int_{a}^{b_2}f(x,y)dx\right)\right| = \\
        = \left|\int_{b_1}^{b_2}f(x,y)dx\right| < \epsilon.
    \end{multline*}
\end{proof}

\begin{corollary}
    Пусть $f(x,y)$ непрерывна на множестве $[a;\omega)\times[c;d] $, \\ $\int_{a}^{\omega}f(x,y)dx$ сходится на $[c;d)$ и расходится в точке $y = d$.

    Отсюда следует, что $\int_{a}^{\omega}f(x,y)dx$ на $[c;d)$ сходится неравномерно.
\end{corollary}

\begin{proof}
    Так как при $y = d \ \int_{a}^{\omega}f(x,y)dx$ расходится $\implies \exists \epsilon > 0 \ \forall B \in [a;\omega) \ \exists b_1,b_2 \in (B;\omega):$
                \[
                    \left|\int_{b_1}^{b_2}f(x,d)dx\right| \geqslant \epsilon.
                \]

                Далее, в силу непрерывности функции $f(x,y)$ на $[a;\omega)\times [c;d]$ следует, что $F(y) = \int_{b_1}^{b_2}f(x,y)dx$ непрерывна на $[c;d]$ (смотреть теорему \ref{theorem:7.1.1}).

                        Следовательно, $\exists$ окрестность $(d - \delta;d]: \ \forall y \in (d - \delta;d]$
    \[
        \left|\int_{b_1}^{b_2}f(x,y)dx\right| \geqslant \epsilon.
    \]

    Таким образом, $\exists \epsilon > 0: \ \forall B \in [a;\omega) \ \exists b_1,b_2 \in (B;\omega):$
    \[
        \left|\int_{b_1}^{b_2}f(x,y)dx\right| \geqslant \epsilon
    \]
    $\implies$ по критерию Коши $\int_{a}^{\omega}f(x,y)dx$ сходится на $[c;d)$ неравномерно.
\end{proof}

\begin{example}
    $\int_{1}^{+\infty}\frac{\sin x}{x^y}dx$, если $y > 0$, то интеграл сходится по признаку Дирихле.

    При $y = 0$ интеграл расходится $\implies$ на интервале $y > 0$ интеграл сходится неравномерно.
\end{example}

\begin{theorem}[Признак Вейерштраса]
    Пусть \begin{enumerate}
        \item $\forall y \in Y$ и $\forall x \in [a;\omega)$
              \[
                  \big|f(x,y)\big| \leqslant g(x,y).
              \]
        \item $\int_{a}^{\omega}g(x,y)dx$ -- равномерно сходится на $Y$.
    \end{enumerate}

    Тогда $\int_{a}^{\omega}f(x,y)dx$ -- равномерно сходится на $Y$.
\end{theorem}

\begin{proof}
    Имеем
    \[
        \left|\int_{b_1}^{b_2}f(x,y)dx\right| \leqslant \int_{b_1}^{b_2}\left|f(x,y)\right|dx \leqslant \int_{b_1}^{b_2}g(x,y)dx.
    \]

    Так как $\int_{a}^{\omega}g(x,y)dx$ сходится равномерно на $Y$, то по признаку Коши
    \[
        \left|\int_{b_1}^{b_2}g(x,y)dx\right| < \epsilon
    \]
    $\implies \int_{a}^{\omega}f(x,y)dx$ сходится равномерно на $Y$.
\end{proof}

\begin{corollary}
    Если $\forall y \in Y, \ \forall x \in [a;\omega)$
    \[
        \big|f(x,y)\big| \geqslant g(x),
    \]
    то из сходимости $\int_{a}^{\omega}g(x)dx \implies$ равномерна сходимость
    \[
        \int_{a}^{\omega}f(x,y)dx \text{ на }Y.
    \]
\end{corollary}

\begin{theorem}[Признаки Абеля и Дирихле]\leavevmode
    \begin{enumerate}
        \item \textbf{Признак Абеля}:

              Если \begin{enumerate}
                  \item $\int_{a}^{\omega}g(x,y)dx$ равномерно сходится на $Y$.
                  \item $\forall y \in Y$ функция $f(x,y)$ монотонна по $x$ и равномерно ограничена, то есть $\exists M > 0: \ \forall x \in [a;\omega)$ и $\forall y \in Y$
                        \[
                            \left|f(x,y)\right| \leqslant M.
                        \]
              \end{enumerate}

              Тогда
              \[
                  \int_{a}^{\omega}\big(f(x,y) \cdot g(x,y)\big)dx \text{ -- сходится равномерно на }Y.
              \]

        \item \textbf{Признак Дирихле}:

              Если
              \begin{enumerate}
                  \item $\int_{a}^{b}g(x,y)dx$ ограничена в совокупности, то есть $\exists L > 0: \ \forall y \in Y$ и $\forall b \in [a;\omega)$
                        \[
                            \left|\int_{a}^{b}g(x,y)dx\right| \leqslant L.
                        \]
                  \item $\forall y \in Y \ f(x,y)$ монотонна по $x$ и $f(x,y) \rightarrow 0$ равномерно при $x \rightarrow \omega$.
              \end{enumerate}

              Тогда
              \[
                  \int_{a}^{\omega}\big(f(x,y)\cdot g(x,y)\big)dx \text{ -- сходится равномерно на } Y.
              \]
    \end{enumerate}
\end{theorem}

\begin{proof}\leavevmode
    \begin{multline*}
        \left|\int_{b_1}^{b_2}\big(f(x,y)\cdot g(x,y)\big)dx\right| \overset{\begin{array}{c}
                \text{2-я теорема} \\
                \text{о среднем}
            \end{array}}{=} \\
        = \left|f(b_1,y)\cdot \int_{b_1}^{\xi}g(x,y)dx + f(b_2,y)\cdot \int_{\xi}^{b_2}g(x,y)dx\right| \leqslant \\
        \leqslant \left|f(b_1,y)\right| \cdot \left|\int_{b_1}^{\xi}g(x,y)dx\right| + \left|f(b_2,y)\right| \cdot \left|\int_{\xi}^{b_2}g(x,y)dx\right|.
    \end{multline*}

    \begin{enumerate}
        \item Пусть выполнены (a) и (b) для признака Абеля. Пусть $\epsilon > 0$ задано, тогда
              \begin{multline*}
                  \left|\int_{b_1}^{\xi}g(x,y)dx\right| < \frac{\epsilon}{2\cdot M} \quad\text{и}\quad \left|\int_{\xi}^{b_2}g(x,y)dx\right| < \frac{\epsilon}{2 \cdot M} \implies \\
                  \implies \left|\int_{b_1}^{b_2}\big(f(x,y)\cdot g(x,y)\big)dx \right| < M \cdot \frac{\epsilon}{2 \cdot M} + M \cdot \frac{\epsilon}{2 \cdot M} = \epsilon \implies
              \end{multline*}
              $\implies \int_{a}^{\omega}\big(f(x,y)\cdot g(x,y)\big)dx$ сходится равномерно на $Y$ по критерию Коши.
        \item Пусть выполнены (a) и (b) для признака Дирихле. Пусть $\epsilon > 0$ задано, тогда:
              \[
                  \left|f(x,y)\right| < \frac{\epsilon}{2\cdot L} \implies \left|\int_{b_1}^{b_2}\big(f(x,y)\cdot g(x,y)\big)dx\right| < \frac{\epsilon}{2\cdot L} \cdot L + \frac{\epsilon}{2 \cdot L} \cdot L = \epsilon \implies
              \]
              $\implies \int_{a}^{\omega}\big(f(x,y)\cdot g(x,y)\big)dx$ сходится равномерно на $Y$.
    \end{enumerate}
\end{proof}

\section{Функциональные свойства несобственного интеграла, зависящего от параметра}

\begin{theorem}[О предельном переходе под знаком несобственного интеграла]\label{theorem:7.3.1}
    Если
    \begin{enumerate}
        \item $\forall b \in [a;\omega)$
              \[
                  f(x,y) \xrightrightarrows[\mathfrak{B}_y]{}\phi(x)
              \]
              на $[a;b]$, где $\mathfrak{B}_y$ -- база на $Y$.
        \item $\int_{a}^{\omega}f(x,y)dx$ сходится равномерно на $Y$.
    \end{enumerate}
    Тогда \[
        \underset{\mathfrak{B}_y}{\lim} F(y) = \underset{\mathfrak{B}_y}{\lim} \int_{a}^{\omega}f(x,y)dx = \int_{a}^{\omega}\underset{\mathfrak{B}_y}{\lim} f(x,y)dx = \int_{a}^{\omega}\phi(x)dx.
    \]
\end{theorem}

\begin{proof}
    Имеем $F_b(y) = \int_{a}^{b}f(x,y)dx$
    \[
        \begin{matrix}
            F_b(y) = & \int_{a}^{b}f(x,y)dx                                           & \xrightrightarrows[b\rightarrow\omega]{} & \int_{a}^{\omega} f(x,y)dx                          & = F(y) \\
                     & {\scriptstyle \forall b \ \mathfrak{B}_y}\xdashdownarrow[20pt] &                                          & \xdashdownarrow[20pt] {\scriptstyle \mathfrak{B}_y} &        \\
                     & \int_{a}^{b}\phi(x)dx                                          & \xdashrightarrow[b\rightarrow\omega]{}   & \int_{a}^{\omega}\phi(x)dx                          &
        \end{matrix}
    \]


    Докажем левую вертикальную стрелку. Вспомним теорему \ref{theorem:6.9.2}.
    \begin{enumerate}
        \item $\forall y \in Y \ f(x,y)$ интегрируется на $[a;b]$ (из условия 2 $\implies$)
        \item $f(x,y)\underset{\mathfrak{B}_y}{\rightrightarrows}\phi(x)$ на $[a;b] \implies$
              \[
                  \int_{a}^{b}\phi(x)dx = \underset{\mathfrak{B}_y}{\lim}\int_{a}^{b}f(x,y)dx
              \]
    \end{enumerate}
    $\implies$ используя теорему \ref{theorem:6.9.1}, доказывается утверждение этой теоремы.
\end{proof}

\begin{corollary}[Непрерывность несобственного интеграла, зависящего от параметров]
    Если \begin{enumerate}
        \item $f(x,y)$ непрерывна на $[a;\omega)\times[c;d]$.
        \item $\int_{a}^{\omega}f(x,y)dx$ равномерно сходится на $[c;d]$.
    \end{enumerate}

    Тогда $F(y) = \int_{a}^{\omega}f(x,y)dx$ непрерывна на $[c;d]$.
\end{corollary}

\begin{proof}
    $ y_0 \in [c;d] $. Докажем, что $ F(y) $ непрерывна в точке $ y_0 $, то есть докажем, что $ \underset{y \rightarrow y_0}{\lim}F(y) = F(y_0) $.

    Имеем:
    \begin{multline*}
        \underset{y \rightarrow y_0}{\lim}F(y) = \underset{y \rightarrow y_0}{\lim}\int_{a}^{\omega}f(x,y)dx \overset{\circled{?}}{=} \\
        \overset{\circled{?}}{=} \int_{a}^{\omega}\underset{y \rightarrow y_0}{\lim}f(x,y)dx \overset{\text{непр. }f(x,y)}{=} \int_{a}^{\omega}f(x,y_0)dx = F(y_0).
    \end{multline*}

    Проверим, выполняются ли условия теоремы \ref{theorem:7.3.1}.

    База: $ y \rightarrow y_0 $. Надо показать, что 
    \begin{enumerate}
       \item $ f(x,y)\xrightrightarrows[y \rightarrow y_0]{}\equalto{f(x,y_0)}{\phi(x)} $ на $ [a;b] \ \forall b \in [a;\omega) $.
       \item Дано.
    \end{enumerate}

    Покажем 1.

    Так как $ f(x,y) $ непрерывна на $ [a;\omega) \times [c;d] \implies f(x,y) $ равномерно непрерывна на $ [a;b]\times[c;d] $ (по теореме Кантора) $ \implies \forall(x,y_0) \ \exists U \subset [a;b]\times[c;d]: \ \forall (x,y) \in U $
    \[
       \big|f(x,y) - f(x,y_0)\big| < \epsilon \implies
    \]
    $ \implies \circled{?} $ обоснован.
\end{proof}