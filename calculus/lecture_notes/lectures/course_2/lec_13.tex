\lesson{13}{от 27 окт 2023 10:34}{Продолжение}


\begin{theorem}[Дифференцируемость предельной функции]
    Пусть $ -\infty < a < b < +\infty $ ($ a,b $ -- конечны), $ f_t: (a;b)\rightarrow\R, \ f:(a;b) \rightarrow \R $:
    \begin{itemize}
        \item $ \forall t \in T \ f_t $ дифференцируема на $ (a;b) $,
        \item $ \exists \phi: (a;b)\rightarrow\R: \ f_t' \xrightrightarrows[\mathfrak{B}]{} \phi $ на $ (a;b) $,
        \item $ \exists x_0 \in (a;b): \ f_t(x_0) \rightarrow f(x_0) $,
    \end{itemize}
    тогда:
    \begin{enumerate}
        \item $ f_t \xrightrightarrows[\mathfrak{B}]{} f $ на $ (a;b) $.
        \item $ f $ дифференцируема на $ (a;b) $.
        \item $ \forall x \in (a;b) \ f'(x) = \phi(x) $.
    \end{enumerate}
\end{theorem}

\begin{proof}
    Докажем, что семейство функций $f_t$ сходится к $f$ равномерно на $(a;b)$:
    \begin{multline*}
        \big|f_{t_1}(x) - f_{t_2}(x)\big| = \\
        = \big|f_{t_1}(x) - f_{t_2}(x) + f_{t_1}(x_0) - f_{t_1}(x_0) + f_{t_2}(x_0) - f_{t_2}(x_0)\big| \leqslant \\
        \leqslant \big|(f_{t_1}(x) - f_{t_1}(x_0)) - (f_{t_2}(x) - f_{t_2}(x_0))\big| + \big|f_{t_1}(x_0) - f_{t_2}(x_0)\big| = \\
        = \big|f_{t_1}'(\xi) - f_{t_2}'(\xi)\big| \cdot \big|x - x_0\big| + \big|f_{t_1}(x_0) - f_{t_2}(x_0)\big|.
    \end{multline*}

    Пусть $\epsilon > 0$ задано. Выберем $B \in \mathfrak{B}$ ($\mathfrak{B}$ -- база на $T$) $\forall t_1,t_2 \in B$
    \[
        \big|f_{t_1}(x_0) - f_{t_2}(x_0)\big| < \frac{\epsilon}{2}
    \]
    и $\forall x \in (a;b)$ и $\forall t_1',t_2' \in B$:
    \[
        \big|f_{t_2'}'(x) - f_{t_2'}'(x)\big| < \frac{\epsilon}{2(b-a)}.
    \]

    Тогда $\forall t_1,t_2 \in B$ и $\forall x \in (a;b)$
    \[
        \big|f_{t_1}(x) - f_{t_2}(x)\big| < \frac{\epsilon}{2(b-a)}\cdot (b-a) + \frac{\epsilon}{2} = \epsilon.
    \]

    Итак, $f_t \xrightrightarrows[\mathfrak{B}]{} f$ на $(a;b)$. Покажем, что предельная функция $f$ дифференцируема на $(a;b)$ и $\forall x \in (a;b) $
    \[
        f'(x)=\phi(x):
    \]
    \begin{multline*}
        f'(x) = \underset{h\rightarrow0}{\lim}\frac{f(x + h) - f(x)}{h} = \\
        = \underset{h\rightarrow0}{\lim}\frac{\underset{\mathfrak{B}}{\lim}f_t(x + h) - \underset{\mathfrak{B}}{\lim}f(x)}{h} = \underset{h\rightarrow 0}{\lim}\underset{\mathfrak{B}}{\lim}\frac{f_t(x + h) - f_t(x)}{h} \overset{(\star)}{=} \\
        \overset{(\star)}{=} \underset{\mathfrak{B}}{\lim}\underset{h\rightarrow0}{\lim}\frac{f_t(x + h) - f_t(x)}{h} = \underset{\mathfrak{B}}{\lim}f_t'(x) = \phi(x).
    \end{multline*}

    Покажем законность перехода $(\star)$. Пусть $x \in (a;b), \ x + h \in (a;b)$. Рассмотрим
    \[
        \begin{matrix}
            F_t(h) = & \frac{f_t(x+h) - f_t(x)}{h} & \overset{\xdashrightarrow[]{}}{\xdashrightarrow[\mathfrak{B}]{}} & \frac{f(x + h) - f(x)}{h} & = F(h)            \\
                     & \xdownarrow[22pt]           &                                                                  & \xdashdownarrow[20pt]                         \\
                     & f_t'(x)                     & \xrightrightarrows[\mathfrak{B}]{}                               & \phi(x)                   & \overset{\xdashrightarrow[]{}}{\xdashrightarrow[]{}} f'(x)
        \end{matrix}
    \]

    Докажем существование двойной верхней стрелки. Имеем:
    \[
        \left.\begin{array}{c}
            f_t(x) \xrightarrow[\mathfrak{B}]{} f(x) \\
            f_t(x + h) \xrightarrow[\mathfrak{B}]{} f(x+h)
        \end{array}\right\} \implies F_t(h) \xrightarrow[\mathfrak{B}]{} F(h),
    \]
    \begin{multline*}
        \big|F_{t_1}(h) - F_{t_2}(h)\big| = \\
        = \bigg| \frac{\overbrace{f_{t_1}(x + h) - f_{t_1}(x)}^{= f_{t_1}'(\xi) \cdot |h|}}{h} - \frac{f_{t_2}(x+h) - f_{t_2}(x)}{h}\bigg| = \\
        = \frac{1}{|h|}\big|f_{t_1}'(\xi) \cdot |h| - f_{t_2}'(\xi)\cdot |h| \big| = \\
        = \big|f_{t_1}'(\xi) - f_{t_2}'(\xi)\big|, \ \xi \in (x;x+h).
    \end{multline*}

    Пусть $\epsilon > 0$ задано. Тогда $\exists B \in \mathfrak{B}: \ \forall t_1,t_2 \in B$
    \[
        \big|f_{t_1}'(\xi) - f_{t_2}'(\xi)\big| < \epsilon.
    \]

    Таким образом семейство $\big\{F_t(h)\big\}$ сходится равномерно на $(a;b)$.

    Правая вертикальная стрелка следует из теоремы \ref{theorem:6.3}.
\end{proof}

\begin{corollary}
    Если
    \begin{itemize}
        \item $\forall n \ f_n(x)$ непрерывна на $(a;b)$,
        \item ряд $\sum_{n = 1}^{\infty} f_n(x)$ равномерно сходится на $(a;b)$,
    \end{itemize}
    то его сумма $f(x) = \sum_{n=1}^{\infty}f_n(x)$ непрерывна на $(a;b)$, то есть $\forall x_0 \in (a;b)$
    \[
        \underset{x\rightarrow x_0}{\lim}\sum_{n=1}^{\infty}f_n(x) = \sum_{n=1}^{\infty}\underset{x\rightarrow x_0}{\lim}f_n(x).
    \]
\end{corollary}

\begin{corollary}
    Если
    \begin{itemize}
        \item $\forall n \ f_n(x) \in R[a;b]$ (интегрируема на $[a;b]$),
        \item ряд $\sum_{n=1}^{\infty}f_n(x)$ равномерно сходится на $[a;b]$,
    \end{itemize}
    то его сумма интегрируема на $[a;b]$ и
    \[
        \int_{a}^{b}\sum_{n=1}^{\infty}f_n(x)dx = \sum_{n=1}^{\infty}\int_{a}^{b}f_n(x)dx.
    \]
\end{corollary}

\begin{corollary}
    Если
    \begin{itemize}
        \item $\forall n \ f_n(x)$ дифференцируема на $(a;b)$,
        \item $\exists x_0 \in [a;b]$: ряд $\sum_{n=1}^{\infty}f_n(x_0)$ сходится,
        \item ряд $\sum_{n=1}^{\infty}f_n'(x)$ сходится равномерно на $(a;b)$,
    \end{itemize}
    то \begin{enumerate}
        \item Ряд сходится на $(a;b)$ равномерно.
        \item Его сумма дифференцируема на $(a;b)$.
        \item $\forall x \in (a;b)$
              \[
                \bigg(\sum_{n=1}^{\infty}f_n(x)\bigg)' = \sum_{n=1}^{\infty}f_n'(x).
              \]
    \end{enumerate}
\end{corollary}

% \section{Степенные ряды}

% \begin{definition}[степенной ряд]
%     \emph{Степенным рядом} называется выражение вида
%     \begin{equation*}
%         \sum_{n=0}^{\infty}\big(a_n\cdot (x-x_0)^n\big)
%     \end{equation*}
%     или
%     \begin{equation*}
%         (\star) \quad \sum_{n=0}^{\infty}(a_n \cdot x^n)
%     \end{equation*}
% \end{definition}

% \begin{theorem}[о сходимости степенного ряда]
%     \begin{enumerate}
%         \item Областью сходимости степенного ряда $\sum_{n=0}^{\infty}(a_n \cdot x^n)$ является промежуток $(-R;R)$, где $R \geqslant 0 \ (+ \infty)$;
%         \item $\forall [\alpha;\beta] \subset (-R;R)$ ряд $\sum_{n=0}^{\infty}(a_n \cdot x^n)$ сходится равномерно на $[\alpha;\beta]$;
%         \item Число $R$, называемое радиусом сходимости степенного ряда $\sum_{n=0}^{\infty}(a_n \cdot x^n)$, может быть вычислено:
%               \begin{equation*}
%                   R = \frac{1}{\underset{n\rightarrow\infty}{\overline{\lim}}\sqrt[n]{|a_n|}}
%               \end{equation*}
%     \end{enumerate}
% \end{theorem}

% \begin{proof}
%     Воспользуемся признаком Коши:
%     \begin{equation*}
%         \underset{n\rightarrow\infty}{\overline{\lim}}\sqrt[n]{|a_n|\cdot|x|^n} = |x| \cdot \underset{n\rightarrow\infty}{\overline{\lim}}\sqrt[n]{|a_n|} = k
%     \end{equation*}

%     При $k < 1$ ряд $\sum_{n=0}^{\infty}|a_n \cdot x^n |$ сходится $\implies$ ряд $\sum_{n=1}^{\infty}(a_n \cdot x^n)$ сходится абсолютно.

%     Покажем, что при $k > 1$ ряд $\sum_{n=1}^{\infty}(a_n \cdot x^n)$ расходится для этого покажем, что при $k > 1 \ a_n \cdot x^n \nrightarrow 0$. В самом деле, $\exists$ подпоследовательность номеров $n_k$ и $\exists k: \ \forall k > K$
%     \begin{equation*}
%         |a_{n_k} \cdot x^{n_k}| > \bigg(\frac{1 + k}{2}\bigg)^{n_k} > 1 \implies a_n \cdot x^n \underset{n\rightarrow\infty}{\nrightarrow} 0
%     \end{equation*}

%     Таким образом, $|x|\cdot \underset{n\rightarrow\infty}{\overline{\lim}}\sqrt[n]{|a_n|} < 1$,
%     \begin{equation*}
%         |x| < \frac{1}{\underset{n\rightarrow\infty}{\overline{\lim}}\sqrt[n]{|a_n|}} = R \implies x \in (-R;R),
%     \end{equation*}
%     где $(-R;R)$ -- область сходимости $\sum_{n=0}^{\infty}(a_n \cdot x^n)$.

%     При $k = 1$ ряд $\sum_{n=0}^{\infty}(a_n \cdot x^n)$ может как сходиться, так и расходиться. Таким образом, доказали пункты 1. и 3.. Докажем пункт 2.:

%     Пусть $[\alpha;\beta]\subset(-R;R)$. Возьмем $x_0$:
%     \begin{equation*}
%         -R < -x_0 < \alpha < \beta < x_0 < R
%     \end{equation*}

%     Тогда $\forall x \in [\alpha;\beta]$
%     \begin{equation*}
%         |a_n \cdot x^n| < |a_n \cdot x_0^n|
%     \end{equation*}

%     Заметим, что так как $x_0 \in (-R;R)$, то ряд $\sum_{n=1}^{\infty}|a_n\cdot x_0^n|$ сходится $\implies$ по признаку Вейерштрасса ряд $\sum_{n=0}^{\infty}(a_n \cdot x^n)$ сходится равномерно на $[\alpha;\beta]$.
% \end{proof}