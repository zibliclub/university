\lesson{19}{от 1 дек 2023 10:32}{Продолжение}


\section{Замена переменной в кратном интеграле}

\begin{note}
    Пусть $ G $ -- ограниченная область в $ \R^n, \ F: G \rightarrow \R^n $ -- взаимнооднозначное и непрерывно дифференцируемое отображение, которое аналитически задается при помощи непрерывно дифференцируемых функций:
    \[
        x_1 = \phi_1(u_1,\ldots,u_n),\ldots,x_n = \phi_n(u_1,\ldots,u_n).
    \]

    Будем считать, что для функций $ \phi_i, \ i = \overline{1,n} $ выполнены следующие условия:
    \begin{enumerate}
        \item Производные $ \frac{\delta \phi_i}{\delta u_j} $ ограничены в $ G $.
        \item Производные $ \frac{\delta \phi_i}{\delta u_j} $ равномерно непрерывны в $ G $.
        \item Якобиан отображения удовлетворяет при $ u = (u_1,\ldots,u_n) \in G $ условно:
              \[
                  \big|\mathfrak{I}(u)\big| \geqslant \alpha > 0
              \]
              \begin{center}
                  (якобиан -- определитель матрицы Якоби $ \|\frac{\delta \phi_i}{\delta u_j}\| $)
              \end{center}
    \end{enumerate}
\end{note}

\begin{theorem}[Формула замены переменных в кратном интеграле]
    Пусть отображение $ F: \Omega \rightarrow \R^n $ ($ \Omega \subset \R^n $ -- открытое множество) является взаимнооднозначным и удовлетворяет условиям 1.-3.

    Пусть $ G $ -- измеримый компат: $ G \subset \Omega $. Тогда, если функция $ f(x) = f(x_1,\ldots,x_n) $ непрерывна на множестве $ G'=F(G) $, то справедлива следующая формула замены переменных в кратном интеграле:
    \begin{multline*}
        \int\limits_{G'}f(x)dx = \underset{G'}{\int\ldots\int}f(x_1,\ldots,x_n)dx_1dx_2\ldots dx_n = \\
        = \int\limits_G f\big(\phi_1(u),\ldots,\phi_n(u)\big)\big|\mathfrak{I}(u)\big|du,
    \end{multline*}
    \[
        u = (u_1,\ldots,u_n), \quad du = du_1du_2\ldots du_n.
    \]
\end{theorem}

\chapter{Криволинейные интегралы}

\begin{note}
    Пусть на некотором множестве, содержащем кривую $ \Gamma $ задано непрерывная функция $ R(x,y,z) $.

    Если гладкая кривая $ \Gamma $ задана уравнением
    \[
        \overline{r} = \overline{r}(A) \quad \text{или} \quad \left\{\begin{array}{l}
            x = x(t) \\
            y = y(t) \\
            z = z(t)
        \end{array}\right., \ \alpha \leqslant t \leqslant\beta,
    \]
    то определенный интеграл
    \begin{multline*}
        \int_{\alpha}^{\beta}R\big(x(t),y(t),z(t)\big)\cdot \big|\overline{r}'(t)\big|dt = \\
        = \int_{\alpha}^{\beta}R\bigl(x(t),y(t),z(t)\bigr)\cdot\sqrt{x^{'2}(t) + y^{'2}(t) + z^{'2}(t)}dt.
    \end{multline*}

    Будем называть \emph{криволинейным интегралом \RomanNumeralCaps{1}-го рода} от функции $ R(x,y,z) $ по кривой $ \Gamma $ и обозначать:
    \[
        \boxed{\int\limits_\Gamma R(x,y,z)ds}
    \]

    То есть
    \[
        \int\limits_\Gamma R(x,y,z)ds = \int_{\alpha}^{\beta}R\big(x(t),y(t),z(t)\big)\big|\overline{r}'(t)\big|dt.
    \]
\end{note}

\subsection{Свойства криволинейного интеграла \RomanNumeralCaps{1}-го рода}

\begin{itemize}
    \item Свойство $ \circled{1} $
          \begin{statement}
              Криволинейный интеграл \RomanNumeralCaps{1}-го рода не зависит от параметризации кривой.
          \end{statement}

          \newpage

    \item Свойство $ \circled{2} $
          \begin{statement}
              Криволинейный интеграл \RomanNumeralCaps{1}-го рода не зависит от ориентации кривой, то есть
              \[
                  \int\limits_\Gamma R(x,y,z)ds = \int\limits_{\overline{\Gamma}}\overline{R}(x,y,z)ds.
              \]
          \end{statement}

    \item Свойство $ \circled{3} $
          \begin{statement}
              Криволинейный интеграл \RomanNumeralCaps{1}-го рода аддитивен относительно кривой, если $ \Gamma = \bigcup\limits_{i=1}^N \Gamma_i $, то
              \[
                  \int\limits_\Gamma R(x,y,z)ds = \sum_{i=1}^{N}\int\limits_{\Gamma_i}R(x,y,z)ds.
              \]
          \end{statement}
\end{itemize}

\begin{note}
    Пусть $ \Omega \subset \R^3 $ -- область, в каждой точке которой задан вектор. Тогда говорят, что в области $ \Omega $ \emph{задано векторное поле}.

    Если фиксирована декартова прямоугольная система координат, то векторное поле можно задать при помощи трех скалярных функций:
    \[
        \overline{F}(x,y,z) = \big\{P(x,y,z),Q(x,y,z),R(x,y,z)\}.
    \]

    Если функции $ P,Q,R $ непрерывны в области $ \Omega $, то и поле $ \overline{F}(x,y,z) $ называется \emph{непрерывным} в области $ \Omega $.

    Если $ P,Q,R $ непрерывно дифференцируемы в $ \Omega $, то и векторное поле $ \overline{F} $ называется непрерывно дифференцируемым в $ \Omega $.

    Если можно так подобрать ДСК, что $ R \equiv 0 $, а $ P $ и $ Q $ не зависят от $ z $, то векторное поле $ \overline{F} $ называется \emph{плоским}:
    \[
        \overline{F}(x,y) = \big\{P(x,y),Q(x,y)\big\}.
    \]

    Пусть в области $ \Omega \subset \R^3 $ определено непрерывное векторное поле $ \overline{F}(x,y,z) = \big\{P(x,y,z),Q(x,y,z),R(x,y,z)\}, \ \overline{r} = \overline{r}(t), \ \alpha \leqslant t \leqslant \beta $, уравнение гладкой (кусочно гладкой) кривой $ \Gamma \subset \Omega $.

    Тогда
    \begin{multline*}
        \int_{\alpha}^{\beta}\overline{F}\big(x(t),y(t),z(t)\big)\cdot \overline{r}'(t)dt = \int_{\alpha}^{\beta}\Big(P\big(x(t),y(t),z(t)\big)\cdot x'(t) + \\
        + Q\big(x(t),y(t),z(t)\big)\cdot y'(t) + R\big(x(t),y(t),z(t)\big)\cdot z'(t)\Big)dt
    \end{multline*}
    называется \emph{криволинейным интегралом \RomanNumeralCaps{2}-го рода} от векторного поля $ \overline{F} $ на кривой $ \Gamma \subset \Omega $.

    Тогда по определению
    \[
       \int\limits_\Gamma(\overline{F},d \overline{r}) = \int_{\alpha}^{\beta}F\big(x(t),y(t),z(t)\big)\overline{r}'(t)dt.
    \]
\end{note}