\lesson{18}{от 24 нояб 2023 10:32}{Продолжение}


\section{Кратный интеграл Римана}

\begin{definition}[Разбиение совокупности измеримых по Жордану множеств]
    Пусть множество $ G \subset \R^n $ измеримо по Жордану.

    Совокупность измеримых по Жордану множеств $ G_i \subset \R^n, \ i = \overline{1,N} $, попарно пересекающихся $ G = \overset{n}{\underset{i=1}{\bigcup}}G_i $ называются \emph{разбиением} множества $ G $.
    \[
        \text{Обозначение:}\quad T = \{G_i\}
    \]
\end{definition}

\begin{definition}[Мелкость разбиения]
    Число $ l(T) = \max d(G_i) $ называется \emph{мелкостью разбиения} $ T $.
\end{definition}

\begin{definition}[Интегральная сумма Римана от функции на множестве]
    Пусть функция $ f(x) = f(x_1,\ldots,x_n): \ G \rightarrow \R $ определена на измеримом по Жордану множестве $ G \subset \R^n $, $T = \{G_{ij}\}$ -- разбиение множества $ G $.

    Возьмем $ \xi_i \in G_i, \ i = \overline{1,N} $.

    Выражение
    \[
        \sigma_T = \sigma_T(f,\xi,G) = \sum_{i=1}^{N}f(\xi_i)m(G_i)
    \]
    называется \emph{интегральной суммой Римана} от функции \\ $ f(x) = f(x_1,\ldots,x_n) $ на множестве, соответствующей разбиению $ T $ и выборке $ \xi = (\xi_1,\ldots,\xi_N) $.
\end{definition}

\begin{definition}[Предел интегральной суммы]
    Число $ I $ называется \emph{пределом интегральной суммы} $ \sigma_T $ при мелкости разбиения $ l(T)\rightarrow0 $, если $ \forall T: \ l(T) < \delta $ и $ \forall \xi = (\xi_1,\ldots,\xi_n) $ верно неравенство
    \[
        \big| I - \sigma_T(f,\xi,G) \big| < \epsilon.
    \]
    \[
        \text{Обозначение: }\quad I = \underset{l(T)\rightarrow0}{\lim}\sigma_T.
    \]
\end{definition}

\begin{note}
    Число $ I $ будем называть \emph{кратным интегралом Римана} от функции $ f(x) $ по множеству $ G $, а функцию $ f(x) $ -- \emph{интегрируемой} на множестве $ G $.
    \[
        \begin{array}{ccccc}
                                      &          & \begin{array}{cc}
                                                       \text{Обозн.} \\
                                                       \text{инт. Римана}
                                                   \end{array} &          &                                                                                                    \\
                                      & \swarrow &                    & \searrow &                                                                                             \\
            \text{кратное}            &          &                    &          & \text{развернутое}                                                                          \\
            \underset{G}{\int} f(x)dx &          &                    &          & \underbrace{\underset{G}{\int\ldots\int}}_{n \text{ раз}}f(x_1,\ldots,x_n)dx_1,\ldots,d x_n
        \end{array}
    \]

    При $ n=2 $ кратный интеграл Римана называется \emph{двойным} и обозначается
    \[
        \iint\limits_G f(x,y)dxdy.
    \]

    При $ n=3 $ -- \emph{тройным} и обозначается
    \[
        \iiint\limits_G f(x,y,z)dxdydz.
    \]
\end{note}

\begin{theorem}[Критерий интегрируемости]
    Ограниченная формула $ f(x) $ интегрируема на измеримом по Нордану множестве $ G \subset \R^n \iff \forall \epsilon > 0 \ \exists\delta > 0: \ \forall T \ l(T)<\delta$
    \[
        \overline{S_T} - S_T < \epsilon
    \]
    \begin{center}
        (то есть $ \overline{S_T} - S_T \rightarrow 0 $ при $ l(T) \rightarrow 0 $)
    \end{center}
\end{theorem}

\begin{proof}
    Смотреть доказательство соответствующей теоремы для $ \int_{a}^{b}f(x)dx $.
\end{proof}

\begin{theorem}[Критерий интегрируемости, более сильная]
    Ограниченная функция $ f(x) $ интегрируема на измеримом по Жордану множестве $ G \subset \R^n \iff \forall \epsilon > 0 \ \exists T $ множества $ G $:
    \[
        \overline{S_T} - S_T < \epsilon.
    \]
\end{theorem}

\begin{proof}
    Без доказательства.
\end{proof}

\subsection{Классы интегрируемых функций}

\begin{theorem}
    Непрерывная на измеримом по Жордану компактном множестве $ G $, функция $ f(x) $ интегрируема на этом множестве.
\end{theorem}

\begin{proof}
    Доказательство аналогично доказательству соответствующей теореме для $ \int_{a}^{b}f(x)dx $.
\end{proof}

\begin{theorem}
    Пусть функция $ f(x) $ ограничена на измеримом компакте $ G \subset \R^n $ и множество разрыва $ f(x) $ имеет Жорданову меру нуль.

    Тогда $ f(x) $ интегрируема на $ G $.
\end{theorem}

\begin{proof}
    Без доказательства.
\end{proof}

\subsection{Свойства кратного интеграла Римана}

\begin{itemize}
    \item Свойство $ \circled{1} $
          \begin{statement}
              Справедливо равенство
              \[
                  \underset{G}{\int}1 dx = m(G).
              \]
          \end{statement}

    \item Свойство $ \circled{2} $
          \begin{statement}
              Если $ f(x) > 0 $ и $ f(x) $ -- интегрируемая на измеримом по Жордану множестве $ G $ функция, то
              \[
                  \underset{G}{\int}f(x)dx \geqslant 0.
              \]
          \end{statement}

    \item Свойство $ \circled{3} $
          \begin{statement}
              Если $ f_1(x) $ и $ f_2(x) = f_2(x_1,\ldots,x_n) $ -- интегрируемые на измеримом по Жордану множестве $ G $ функции, $ \alpha,\beta \in \R $, то и функция $ \alpha \cdot f_1(x) + \beta \cdot f_2(x) $ интегрируема на $ G $ и
              \[
                  \underset{G}{\int}\big(\alpha\cdot f_1(x) + \beta \cdot f_2(x)\big)dx = \alpha \underset{G}{\int}f_1(x)dx + \beta \underset{G}{\int}f_2(x)dx.
              \]
          \end{statement}

    \item Свойство $ \circled{4} $
          \begin{statement}
              Если $ f_1(x) $ и $ f_2(x) $ -- интегралы на измеримом по Жордану множестве $ G $ и $ \forall x \in G \ f_1(x) \leqslant f_2(x) $, то
              \[
                  \underset{G}{\int} f_1(x)dx \leqslant \underset{G}{\int}f_2(x)dx.
              \]
          \end{statement}

          \newpage

    \item Свойство $ \circled{5} $
          \begin{statement}
              Если функция $ f(x) $ непрерывна на измеримом связном компакте $ G $, то $ \exists \xi \in G $:
              \[
                  \underset{G}{\int}f(x)dx = f(\xi)m(G).
              \]
          \end{statement}

    \item Свойство $ \circled{6} $
          \begin{statement}
              Если $ G_k, \ k = \overline{1,m} $ если разбиение множества $ G $, то функция $ f(x) $ интегрируема на $ G \iff f(x) $ интегрируема на $ G_k, \ k = \overline{1,m} $, при этом
              \[
                  \underset{G}{\int}f(x)dx = \sum_{k=1}^{m}\underset{G_k}{\int}f(x)dx.
              \]
          \end{statement}

    \item Свойство $ \circled{7} $
          \begin{statement}
              Произведение интегрируемых на измеримом множестве $ G $ функцией является интегрируемой на $ G $ функцией.
          \end{statement}

    \item Свойство $ \circled{8} $
          \begin{statement}
              Если $ f(x) $ интегрируема на множестве $ G $ функция, то функция $ \big| f(x) \big| $ также интегрируема
              \[
                  \left|\underset{G}{\int}f(x)dx\right| \leqslant \underset{G}{\int}\big|f(x)\big|dx.
              \]
          \end{statement}
\end{itemize}

\begin{lemma}
    Пусть функция $ f(x) $ ограничена на измеримом по Жордану множестве $ G $, а $ E $ есть множество меры нуль.

    Если $ \forall T = \{G_k\}, \ k = \overline{1,m} $ отбрасывать в интегральной сумме $ \sigma_T $ слагаемые, соответствующие тем множествам $ G_i $, которые имеют непустое пересечение с $ E $, то это не повлияет ни на существование предела интегральной суммы при $ l(T)\rightarrow0 $, ни на величину этого предела.
\end{lemma}

\begin{theorem}
    Пусть $ G $ -- измеримое множество в $ \R^n $ и функция $ f(x) $ интегрируема на $ G $. Тогда график функции $ f(x) $ имеет в $ \R^{n+1} $ Жорданову меру нуль.
\end{theorem}

\newpage

\section{Сведение кратных интегралов к повторам}

\begin{theorem}[Формула сведения двойного интеграла по прямоугольнику к повторному]
    Пусть
    \begin{enumerate}
        \item Функция $ f(x,y) $ интегрируема на прямоугольнике
              \[
                  \Pi = \big\{(x,y): \ a \leqslant x \leqslant b, \ c \leqslant y \leqslant d\big\}.
              \]

        \item $ \int_{c}^{d}f(x,y)dy \ \exists \ \forall x \in [a;b] $.
    \end{enumerate}

    Тогда функция $ F(x) = \int_{c}^{d}f(x,y)dy $ интегрируема на отрезке $ [a;b] $ и справедлива формула:
    \[
        \boxed{\iint\limits_\Pi f(x,y)dxdy = \int_{a}^{b}dx \int_{c}^{d}f(x)dy}
    \]
\end{theorem}

\begin{proof}
    Возьмем произвольное разбиение отрезков $ [a;b] $ и $ [c;d] $ точками
    \[
        \begin{array}{l}
            a = x_0 < x_1 < \ldots < x_n = b \\
            c = y_0 < y_1 < \ldots < y_m = d
        \end{array}
    \]
    и обозначим $ \Pi_1,\ldots,\Pi_n $ и $ \Pi_1',\ldots,\Pi_m' $ соответсвующие промежутки разбиения.

    Тогда
    \[
        \Pi = \bigcup\limits_{i=1}^{n}\bigcup\limits_{j=1}^{m}\Pi_{ij},
    \]
    где $ \Pi_{ij} = \big\{(x,y): \ x \in \Pi_i, \ y \in \Pi_j'\big\} $.

    Положим
    \[
        M_{ij} = \underset{(x,y)\in\Pi_{ij}}{\sup}f(x,y), \quad m_{ij} = \underset{(x,y)\in\Pi_{ij}}{\inf}f(x,y).
    \]

    Так как $ \int_{c}^{d}f(x,y)dy \ \exists \ \forall x \in [a;b] $, то $ \forall x \in \Pi_i $ справедливо неравенство
    \[
        m_{ij}\cdot\Delta y_i \leqslant \int_{y_{i-1}}^{y_i}f(x,y)dy \leqslant M_{ij}\cdot\Delta y_i,
    \]
    где $ \Delta y_i = y_i - y_{j-1} $.

    Суммируем эти неравенства по индексу $ j $:
    \begin{equation}\label{eq:for_proof7}
        \sum_{j=1}^{m}m_{ij}\Delta y_i \leqslant \int_{c}^{d}f(x,y)dy \leqslant \sum_{j=1}^{m}M_{ij}\Delta y_j
    \end{equation}

    Введем обозначения:
    \[
        F(x) = \int_{c}^{d}f(x,y)dy, \quad M_i = \underset{x\in\Pi_i}{\sup}F(x), \quad m_i = \underset{x\in\Pi_i}{\inf}F(x).
    \]

    Тогда из \ref{eq:for_proof7} $ \implies $
    \[
        \sum_{j=1}^{m}m_{ij}\Delta y_j \leqslant m_i \leqslant M_i \leqslant \sum_{j=1}^{m}M_{ij}\Delta y_j \implies
    \]
    \begin{equation}\label{eq:for_proof8}
        \implies 0 \leqslant M_i - m_i \leqslant \sum_{j=1}^{m}(M_{ij} - m_{ij})\Delta y_j
    \end{equation}

    Домножим на $ \Delta x_i $ неравенство \ref{eq:for_proof8} и просуммируем по $ i $:
    \begin{multline*}
        0 \leqslant \sum_{i=1}^{n}(M_i - m_i)\Delta x_i \leqslant \sum_{i=1}^{n}\sum_{j=1}^{m}(M_{ij} - m_{ij})m(\Pi_{ij}) = \\
        = \overline{S}(f,\Pi) - \underline{S}(f,\Pi) \rightarrow 0 \text{ при }l(T)\rightarrow 0,
    \end{multline*}
    так как $ f(x,y) $ интегрируема на прямоугольнике $ \implies \sum_{i=1}^{n}(M_i - m_i)\Delta x_i \rightarrow 0 $ при $ \max|\Delta x_i| \rightarrow 0 \implies F(x) $ интегрируема на $ [a;b] \implies \exists $
    \[
        \int_{a}^{b}F(x)dx = \int_{a}^{b}dx \int_{c}^{d}f(x,y)dy.
    \]

    Покажем, что он равен двойному.

    Интегрируем неравенство \ref{eq:for_proof7}
    \[
        \sum_{j=1}^{m}m_{ij}\Delta y_j\Delta x_i \leqslant \int_{x_{i-1}}^{x_i}dx \int_{c}^{d}f(x,y)dy \leqslant \sum_{j=1}^{m}M_{ij}\Delta y_j \Delta x_i.
    \]

    Суммируем по $ i $:
    \[
        \sum_{i=1}^{n}\sum_{j=1}^{m}m_{ij} m(\Pi_{ij}) \leqslant \int_{a}^{b}dx \int_{c}^{d} f(x,y)dy \leqslant \sum_{i=1}^{n}\sum_{j=1}^{m}M_{ij}m(\Pi_{ij}).
    \]
    \[
        \underline{S}(f,\Pi) \leqslant \int_{a}^{b}dx \int_{c}^{d}f(x,y)dy \leqslant \overline{S}(f,\Pi).
    \]

    С другой стороны, из условий следует, что
    \[
        \underline{S} \leqslant \iint\limits_\Pi f(x,y)dxdy \leqslant \overline{S}.
    \]

    Разность $ \overline{S} - \underline{S} $ может быть сколь угодно малой $ \implies $
    \[
        \implies \iint\limits_\Pi f(x,y)dxdy = \int_{a}^{b}dx \int_{c}^{d}f(x,y)dy.
    \]
\end{proof}

\begin{corollary}
    Пусть $ \exists \iint\limits_\Pi f(x,y)dxdy $ и $ \forall x \in [a;b] \ \exists \int_{c}^{d}f(x,y)dy $ и $ \forall y \in [c;d] \ \exists \int_{a}^{b}f(x,y)dx $.

    Тогда
    \[
        \iint\limits_\Pi f(x,y)dxdy = \int_{a}^{b}dx \int_{c}^{d}f(x,y)dy = \int_{c}^{d}dy \int_{a}^{b}f(x,y)dx.
    \]
\end{corollary}

\begin{definition}[Элементарная область относительно оси $ Oy $]
    Пусть $ \phi(x) $ и $ \psi(x) $ непрерывные на отрезке $ [a;b] $ функции и $ \phi(x) < \psi(x) \ \forall x \in[a;b] $.

    Область $ \Omega = \big\{(x,y): \ \phi(x) < y < \psi(x), \ a < x < b \big\} $ называется \emph{элементарной областью относительно оси $ Oy $}.
\end{definition}

\begin{theorem}[Сведение двойного интеграла по элементарной области к повторному]
    Пусть $ \Omega $ -- элементарная относительно оси $ Oy $ область, функция $ f(x,y) $ интегрируема на $ \overline{\Omega} = \Omega \cup G\Omega $ и $ \forall x \in [a;b] \ \exists \ \int f(x,y)dx $.

    Тогда справедлива следующая формула:
    \begin{equation}\label{eq:for_proof9}
        \iint\limits_\Omega f(x,y)dxdy = \int_{a}^{b}dx \int_{\phi(x)}^{\psi(x)}f(x,y)dy.
    \end{equation}
\end{theorem}

\begin{proof}
    Есть на фотографиях.
\end{proof}

\begin{corollary}
    Для функции, непрерывной на $ \overline{\Omega} $, справедлива формула \ref{eq:for_proof9}.
\end{corollary}