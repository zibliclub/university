\lesson{7}{от 2 окт 2023 10:32}{Продолжение}


\begin{definition}[$ m $-ый остатный ряд]
    Пусть дан ряд \ref{eq:6.1}. Ряд вида
    \begin{equation}\label{eq:6.3}
        \sum_{n=m+1}^{\infty}a_n
    \end{equation}
    называется \emph{$ m $-ым остатным ряда} \ref{eq:6.1}.
\end{definition}

\begin{theorem}[Об остатке ряда]
    Следующие условия эквивалентны:
    \begin{enumerate}
        \item Ряд \ref{eq:6.1} сходится.
        \item Любой его остаток сходится.
        \item Некоторый его остаток \ref{eq:6.3} сходится.
    \end{enumerate}
\end{theorem}

\begin{proof}\leavevmode
    \begin{itemize}
        \item Докажем, что из 1. $ \implies $ 2.

              Пусть ряд \ref{eq:6.1} сходится и его сумма равна $ A $.

              Пусть $ A_k^* = \sum_{n=m+1}^{m+k}a_n $ -- $ k $-тая частичная сумма ряда \ref{eq:6.3}.

              Ряд \ref{eq:6.3} сходится, если $ \exists \underset{k \rightarrow\infty}{\lim} A_k^*$.
              \[
                  A_k^* = \underbrace{A_{m+k}}_{\begin{array}{c}
                          \text{частичная сумма} \\
                          \text{ряда \ref{eq:6.1}}
                      \end{array}} - A_m.
              \]
              \begin{eqnarray*}
                  \underset{k \rightarrow\infty}{\lim}A_k^* & = & \underset{k \rightarrow\infty}{\lim}(A_{m+k} - A_m) = \\
                  = \underset{k \rightarrow\infty}{\lim}A_{m+k} - \equalto{\underset{k \rightarrow\infty}{\lim}A_m}{const = A_m} &=& A - A_m.
              \end{eqnarray*}

        \item Доказательство того, что из 2. $ \implies $ 3. -- очевидно.

        \item Докажем, что из 3. $ \implies $ 1.

              Пусть ряд \ref{eq:6.3} -- сходится.

              Тогда при $ n > m $:
              \[
                  A_n = A_m + \overbrace{\equalto{A_{n-m}^*}{\sum_{k=m+1}^{m+(n-m)}a_k}}^{\begin{array}{c}
                          m\text{-тая частичная} \\
                          \text{сумма ряда \ref{eq:6.3}}
                      \end{array}}
              \]
              \[
                  A_n = \underbrace{a_1 + a_2 + \ldots + a_m}_{A_m} + \overbrace{\underbrace{a_{m+1} + \ldots + a_n}_{A_{n-m}^*}}^{k \text{ штук, }k=n-m}.
              \]

              Ряд \ref{eq:6.1} сходится $ \underset{\text{по опр.}}{\iff} \exists \underset{n \rightarrow\infty}{\lim}A_n $.

              Рассмотрим:
              \[
                  \underset{n \rightarrow\infty}{\lim}A_n = \underset{n \rightarrow\infty}{\lim}(\equalto{A_m}{const} + A_{n-m}^*) = \equalto{\underset{n \rightarrow\infty}{\lim}A_m}{A_m = const} + \underbrace{\underset{n \rightarrow\infty}{\lim}A_{n-m}^*}_{\begin{array}{c}
                      \exists \text{, так как ряд} \\
                      \text{\ref{eq:6.3} сходится}
                  \end{array}} \implies
              \]
              $ \implies \exists \underset{n \rightarrow\infty}{\lim}A_n \implies $ \ref{eq:6.1} сходится.
    \end{itemize}
\end{proof}

\begin{note}
    Обозначим $\alpha_m$ -- сумма $m$-того остатка ряда  $=$ сумме ряда \ref{eq:6.3}:
    \[
        \alpha_m = \sum_{n=m+1}^{\infty}a_n
    \]
    \begin{center}
        (\ref{eq:6.3} сходится в этом случае)
    \end{center}
\end{note}

\begin{corollary}
    Ряд \ref{eq:6.1} сходится $\iff \underset{m\rightarrow\infty}{\lim} \alpha_m = 0$.
\end{corollary}

\begin{proof}
    Самостоятельно.
\end{proof}

\begin{definition}[Сумма рядов]
    Пусть даны ряды
    \[
        (A) \ \sum_{n=1}^{\infty}a_n, \quad (B) \ \sum_{n=1}^{\infty}b_n.
    \]

    \emph{Суммой рядов} $ A, B $ называется ряд:
    \[
        (A+B) \ \sum_{n=1}^{\infty}(a_n + b_n).
    \]
\end{definition}

\begin{theorem}
    Если ряды $ (A),(B) $ сходятся, то:
    \begin{enumerate}
        \item $ \forall a \in \R $ ряд $ \sum_{n=1}^{\infty}\alpha a_n $ -- сходится и его сумма равна $ \alpha \cdot A $, где $ A = \sum_{n=1}^{\infty}a_n $.

        \item Ряд $ (A+B) $ сходится и его сумма равна $ A^* + B^* $, где $ A^* = \sum_{n=1}^{\infty}a_n,$ $ B^* = \sum_{n=1}^{\infty}b_n $.
    \end{enumerate}
\end{theorem}

\begin{proof}
    \begin{enumerate}
        \item Пусть ряд $ (A) $ сходится.

              Рассмотрим ряд $ \sum_{n=1}^{\infty}\alpha \cdot a_n $:
              \[
                  A_n' = \sum_{k=1}^{n}\alpha \cdot a_k,
              \]
              \[
                  \underset{n \rightarrow\infty}{\lim}A_n' = \underset{n \rightarrow\infty}{\lim}\sum_{k=1}^{n} \alpha \cdot a_k = \alpha \cdot \underset{n \rightarrow\infty}{\lim}\sum_{k=1}^{n}a_k = \alpha \cdot A
              \]

        \item Самостоятельно.
    \end{enumerate}
\end{proof}

\section{Сходимость положительных рядов}

\begin{definition}[Положительный ряд]
    Ряд $ (A) $ называется \emph{положительным}, если $ \forall n \ a_n>0 $.
\end{definition}

\begin{theorem}\label{theorem:2.1}
    Положительный ряд $ (A) $ сходится $ \iff $ его частичные суммы ограничены, то есть $ \exists M > 0: \ \forall n \ A_n < M $.
\end{theorem}

\begin{proof}
    Заметим, что последовательность частичных сумм $ A_n $ возрастает, то есть $ \forall n \ A_{n+1} > A_n $.

    По теореме Вейерштрасса, возрастающая последовательность $ A_n $ имеет предел $ \iff $ она ограничена, то есть $ \exists M>0: \ \forall n \ A_n < M $.
\end{proof}

\begin{theorem}[1-ый признак сравнения]\label{theorem:6.2}
    Пусть даны ряды $ (A),(B) $, причем $a_n > 0, \ b_n > 0 \ \forall n$.

    Если $\exists N \in \mathbb{N}: \ \forall n > N \ a_n \leqslant b_n$, то:
    \begin{enumerate}
        \item Из сходимости ряда $(B) \implies$ сходимость ряда $(A)$.
        \item Из расходимости ряда $(A) \implies$ расходимость ряда $(B)$.
    \end{enumerate}
\end{theorem}

\begin{proof}\leavevmode
    \begin{enumerate}
        \item Пусть ряд $(B)$ -- сходится $\implies$ по теореме \ref{theorem:2.1} его частичные суммы ограничены $\implies$ по неравенству $a_n\leqslant b_n$ частичные суммы ряда $(A)$ также ограничены $\implies$ по \ref{theorem:2.1} ряд $(A)$ сходится.
        \item Аналогично.
    \end{enumerate}
\end{proof}

\begin{theorem}[2-ой признак сравнения]
    Пусть даны ряды $ (A),(B) $, причем $a_n > 0, \ b_n > 0 \ \forall n$.

    Если $\underset{n\rightarrow\infty}{\lim}\frac{a_n}{b_n}=k, \ k\in [0;\infty]$, то:
    \begin{enumerate}
        \item При $k=\infty$ из сходимости $(A) \implies$ сходимость ряда $(B)$.
        \item При $k=0$ из сходимости ряда $(B) \implies$ сходимость ряда $(A)$.
        \item При $0<\equalto{k}{const \ne 0}<\infty$ ряды $(A)$ и $(B)$ ведут себя одинаково.
    \end{enumerate}
\end{theorem}

\begin{proof}
    Переписать доказательство для несобственных интегралов, заменив слово "интеграл" на слово "ряд".
\end{proof}

\begin{theorem}[3-й признак сравнения]
    Пусть даны ряды $ (A),(B) $, причем $a_n > 0, \ b_n > 0 \ \forall n$.

    Если $\exists N \in \N \cup \{0\}: \ \forall n > N \ \frac{a_{n+1}}{a_n}\leqslant\frac{b_{n+1}}{b_n} $, то:
    \begin{enumerate}
        \item Из сходимости ряда $(B) \implies$ сходимость ряда $(A)$.
        \item Из расходимости ряда $(A) \implies$ расходимость ряда $(B)$.
    \end{enumerate}
\end{theorem}

\begin{proof}
    Можно считать, что $N = 0$. Тогда $\forall n > N$ имеем:
    \[
        \frac{a_2}{a_1}\leqslant\frac{b_2}{b_1}; \quad \frac{a_3}{a_2}\leqslant\frac{b_3}{b_2}; \quad \frac{a_4}{a_3}\leqslant\frac{b_4}{b_3}; \quad \ldots; \quad \frac{a_{n+1}}{a_n} \leqslant \frac{b_{n+1}}{b_n}.
    \]

    Перемножим левые и правые части:
    \[
        \frac{a_2 \cdot a_3 \cdot a_4 \cdot \ldots \cdot a_{n+1}}{a_1 \cdot a_2 \cdot a_3 \cdot \ldots \cdot a_n} \leqslant \frac{b_2 \cdot b_3 \cdot b_4 \cdot \ldots \cdot b_{n+1}}{b_1 \cdot b_2 \cdot b_3 \cdot \ldots \cdot b_n},
    \]
    \[
        \frac{a_{n+1}}{a_1} \leqslant \frac{b_{n+1}}{b_n} \implies a_{n+1} \leqslant \frac{a_1}{b_1}\cdot b_{n+1}\text{ (по теореме \ref{theorem:6.2})}.
    \]
    \begin{enumerate}
        \item Если ряд $(B)$ сходится $\implies$ сходится ряд $\sum_{n=1}^{\infty}\frac{a_1}{b_1}\cdot b_{n+1} \implies$ сходится ряд $\sum_{n=1}^{\infty}a_{n+1} \implies $ сходится $ (A) $.
        \item Аналогично.
    \end{enumerate}
\end{proof}

\begin{theorem}[Интегральный признак Коши-Маклорена]
    Пусть дан положительный ряд $ (A) $.

    Пусть функция $f(x)$ удовлетворяет следующим условиям:
    \begin{enumerate}
        \item $f(x): \ [1;+\infty) \rightarrow\R$.
        \item $f(x)$ -- непрерывна.
        \item $f(x)$ -- монотонна.
        \item $f(x) = a_n, \ \forall n \in \N$.
    \end{enumerate}

    Тогда ряд $(A)$ и интеграл $\int_{1}^{\infty}f(x)dx$ ведут себя одинаково.
\end{theorem}

\begin{proof}
    Ограничимся случаем, когда $f(x)$ монотонно убывает.

    Рассмотрим функцию $\phi(x) = a_n$ при $n \leqslant x < n+1$ и $\psi(x) = a_{n+1}$ при $n\leqslant x <n + 1$. Тогда $\forall x \in [1;+\infty)$:
    \[
        \psi(x) \leqslant f(x) \leqslant\phi(x).
    \]

    Отсюда
    \begin{multline*}
        \int_{1}^{N}\psi(x)dx \leqslant \int_{1}^{N}f(x)dx \leqslant\int_{1}^{N}\phi(x)dx \implies \\
        \implies \underbrace{\sum_{n=1}^{N}a_{n+1}}_{\begin{array}{c}
            \text{частичная сумма} \\
            \text{ряда }(A)
        \end{array}} \leqslant \int_{1}^{N}f(x)dx \leqslant \underbrace{\sum_{n=1}^{N}a_n}_{\begin{array}{c}
                \text{частичная сумма} \\
                \text{ряда }(A)
            \end{array}}
    \end{multline*}
    \begin{itemize}
        \item Если интеграл сходится, то частичная сумма $\sum_{n=1}^{N}a_{n+1}$ ограничена $\implies$ ряд $(A)$ сходится.

        \item Если интеграл расходится, то частичная сумма $\sum_{n=1}^{N}a_n$ непрерывна $\implies$ ряд $(A)$ -- расходится.

        \item Если ряд $(A)$ сходится, то $\sum_{n=1}^{N}a_n$ -- ограничена $\implies$ $\int_{1}^{N}f(x)dx$ -- ограничен $\implies \int_{1}^{\infty}f(x)dx$ -- сходится.

        \item Если ряд $(A)$ расходится $\implies$ частичная сумма $\sum_{n=1}^{N}a_{n+1}$ неограничена $\implies$ интеграл расходится.
    \end{itemize}
\end{proof}

\begin{example}
    $\sum_{n=1}^{\infty}\frac{1}{n^p}$.

    Рассмотрим $f(x) = \frac{1}{x^p}$ на $[1;+\infty)$ -- неограниченно монотонно $\searrow$,
    \[
        f(n) = \frac{1}{n^p}.
    \]

    $\sum_{n=1}^{\infty}\frac{1}{n^p}$ ведет себя одинаково с интегралом $\int_{1}^{\infty}\frac{dx}{x^p}$ -- сходится при $p>1$ и расходится при $p\leqslant1 \implies$ ряд $\sum_{n=1}^{\infty}\frac{1}{n^p}$ сходится при $p>1$ и расходится при $p\leqslant1$.
\end{example}

\begin{example}
    $\sum_{n=1}^{\infty}\frac{1}{n\cdot \ln n}$.

    $f(x) = \frac{1}{x\ln x}, \ x \in [e;+\infty), \ \nearrow$, непрерывна.
    \begin{multline*}
        \int_{e}^{\infty}\frac{dx}{x\ln x} = \underset{b\rightarrow\infty}{\lim}\int_{e}^{b}\frac{d(\ln x)}{\ln x} =\\
        = \underset{b\rightarrow\infty}{\lim}\big(\ln(\ln x)\big)\Big|_e^b = \underset{b\rightarrow\infty}{\lim}\ln (\ln b) = \infty \implies
    \end{multline*}
    $\implies$ ряд $\sum_{n=1}^{\infty}\frac{1}{n\ln n}$ расходится (по интегралу Коши-Маклорена).
\end{example}

\begin{theorem}[Радикальный признак Коши]
    Пусть ряд $(A)$ положительный и $\underset{n\rightarrow\infty}{\overline{\lim}}\sqrt[n]{a_n} = q$. Тогда:
    \begin{enumerate}
        \item При $q < 1$ ряд $(A)$ сходится.
        \item При $q > 1$ ряд $(A)$ расходится.
        \item При $q = 1$ может как сходиться, так и расходиться.
    \end{enumerate}
\end{theorem}

\begin{proof}\leavevmode
    \begin{enumerate}
        \item Пусть $q < 1$. Возьмем число $r: \ q < r < 1$. Тогда $\exists N: \ \forall n > N$
              \[
                  \sqrt[n]{a_n} < r \implies a_n < r^n.
              \]

              $0 < r < 1 \implies \sum_{n=1}^{\infty}r^n$ -- сходится $\implies$ по 1-му признаку сравнения сходится ряд $(A)$.

        \item Пусть $q > 1$, тогда существует подпоследовательность $\sqrt[n_i]{a_{n_i}} \rightarrow q$ при $i\rightarrow\infty \implies a_{n_i}\rightarrow q^{n_i} > 1 \implies a_n \nrightarrow  0 \implies $ необходимое условие сходимости не выполняется $ \implies $ ряд $(A)$ расходится.

        \item Рассмотрим ряды $\underset{\text{расходится}}{\sum_{n=1}^{\infty}\frac{1}{n}}$ и $\underset{\text{сходится}}{\sum_{n=1}^{\infty}\frac{1}{n^2}}$:
              \[
                  \underset{n\rightarrow\infty}{\lim}\sqrt[n]{\frac{1}{n}} = \underset{n\rightarrow\infty}{\lim}\sqrt[n]{\frac{1}{n^2}} = 1.
              \]
    \end{enumerate}
\end{proof}

\begin{theorem}[Признак Даламбера]
    Пусть ряд $(A)$ положительный и  \\ $ \underset{n\rightarrow\infty}{\lim}\frac{a_{n+1}}{a_n} = d $. Тогда:
    \begin{enumerate}
        \item При $d < 1$ ряд $(A)$ сходится.
        \item При $d > 1$ ряд $(A)$ расходится.
        \item При $d = 1$ может как сходиться, так и расходиться.
    \end{enumerate}
\end{theorem}

\begin{proof}\leavevmode
    \begin{enumerate}
        \item Пусть $d < 1$. Возьмем $d < r < 1 \implies \exists N: \ \forall n > N \ \frac{a_{n+1}}{a_n}<r $,
              \[
                  b_1 = \frac{a_2}{a_1}; \quad b_2 = \frac{a_3}{a_2}; \quad b_3 = \frac{a_4}{a_3}; \quad \ldots; \quad b_n = \frac{a_{n+1}}{a_n}; \quad \ldots.
              \]
              Можно считать, что $N=0$, тогда $\forall n > N$:
              \[
                  \begin{array}{l}
                      a_2 < r \cdot a_1                 \\
                      a_3 < r \cdot a_2 < r^2 \cdot a_1 \\
                      a_4 < r \cdot a_3 < r^3 \cdot a_1 \\
                      \vdots                            \\
                      a_{n+1} < r^n \cdot a_1
                  \end{array}.
              \]

              Так как $0 < r < 1$, то $\sum_{n=1}^{\infty} r^n \cdot a_1$ сходится $\implies$ сходится ряд $(A)$ по 1 признаку сравнения.
        \item Самостоятельно.
        \item $\sum_{n=1}^{\infty}\frac{1}{n}, \quad \sum_{n=1}^{\infty}\frac{1}{n^2}$.
              \[
                  \underset{n\rightarrow\infty}{\lim}\frac{a_{n+1}}{a_n} = \underset{n\rightarrow\infty}{\lim}\frac{\frac{1}{n+1}}{\frac{1}{n}} = \underset{n\rightarrow\infty}{\lim}\frac{n}{n+1} = 1,
              \]
              \[
                  \underset{n\rightarrow\infty}{\lim}\frac{\frac{1}{(n+1)^2}}{\frac{1}{n^2}} = \underset{n\rightarrow\infty}{\lim}\frac{n^2}{(n+1)^2} = 1.
              \]
    \end{enumerate}
\end{proof}