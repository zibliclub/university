\lesson{14}{от 2 нояб 2023 10:34}{Продолжение}


\section{Ряд Тейлора}

\begin{definition}[Ряд Тейлора]
    Пусть $f(x)$ бесконечно дифференцируема в окрестности точки $x_0$. \emph{Рядом Тейлора} функции $f(x)$ в этой окрестности называется ряд:
    \[
        f(x)\approx f(x_0) + \frac{f'(x_0)}{1!}\cdot (x-x_0) + \frac{f''(x_0)}{2!}\cdot (x - x_0)^2 + \ldots + \frac{f^{(n)}(x_0)}{n!}\cdot (x-x_0)^n + \ldots
    \]
\end{definition}

\begin{statement}
    Если функция $f(x)$ в окрестности точки $x_0$ является суммой степенного ряда $\sum_{n=0}^{\infty}\big(a_n \cdot (x-x_0)^n\big)$, то этот ряд является ее рядом Тейлора.
\end{statement}

\begin{proof}
    Имеем, $\forall x \in U(x_0) = (x_0 - \epsilon;x_0 + \epsilon)$:
    \begin{equation}\label{eq:for_proof4}
        f(x) = \sum_{n=0}^{\infty}\big(a_n\cdot (x-x_0)^n\big).
    \end{equation}

    Положим, что $x=x_0$, тогда $f(x_0) = a_0$. Продифференцируем выражение \ref{eq:for_proof4} и вычислим производную в точке $x=x_0$ (и далее по аналогии):
    \begin{eqnarray*}
        f'(x_0) &=& 1 \cdot a_1; \\
        f''(x_0) &=& 2 \cdot 1 \cdot a_2 \implies a_2 = \frac{f''(x_0)}{1 \cdot 2} = \frac{f''(x_0)}{2!}; \\
        f'''(x_0) &=& 3 \cdot 2 \cdot 1 \cdot a_3 \implies a_3 = \frac{f'''(x_0)}{1 \cdot 2 \cdot 3} = \frac{f'''(x_0)}{3!}; \\
        &\vdots& \\
        f^{(n)}(x_0) &=& n \cdot (n-1) \cdot \ldots \cdot 1 \cdot a_n \implies a_n = \frac{f^{(n)}(x_0)}{1 \cdot 2 \cdot \ldots \cdot n} = \frac{f^{(n)}(x_0)}{n!}.
    \end{eqnarray*}
\end{proof}

\newpage

\section{Разложение элементарных функций в степенной ряд}

\begin{lemma}
    Если $f(x)$ -- $\infty$-но дифференцируемая функция на $[0;H]$ и $\exists L > 0: \ \forall n \in \N$ и $\forall x \in [0;H]$
    \[
        \big|f^{(n)}(x)\big| \leqslant L,
    \]
    то на $[0;H]$ функция $f$ может быть разложена в степенной ряд (ряд Тейлора).
\end{lemma}

\begin{proof}
    Имеем:
    \[
        \big|f(x) - F_n(x)\big| = \left|f(x) - \sum_{k=0}^{n}\left(\frac{f^{(k)}(0)}{k!}\cdot x^k\right)\right| = \big|R_n(x)\big|,
    \]
    где $ F_n(x) $ -- частичная сумма ряда Тейлора (степенной ряд), $ R_n(x) $ -- остаточный член в формуле Тейлора.

    Так как $f(x)$ есть сумма ряда $\sum_{n=0}^{\infty}\frac{f^{(n)}(0)}{n!}\cdot x^n \iff R_n(x)$ должен $\rightarrow$ к $0$ при $n\rightarrow\infty$.

    Рассмотрим
    \[
        R_n(x) = \frac{f^{(n+1)}(\xi)}{(n+1)!} \cdot x^{n+1}, \quad 0 < \xi < x.
    \]

    Если выполнены условия леммы, то
    \[
        \big|R_n(x)\big| = \left|\frac{f^{(n+1)}(\xi)}{(n+1)!} \cdot x^{n+1}\right| = \frac{\big|f^{(n+1)}(\xi)\big|}{(n+1)!} \cdot |x^{n+1}| \leqslant \frac{L \cdot H^{n+1}}{(n+1)!},
    \]
    \[
        \underset{n\rightarrow\infty}{\lim}\frac{L\cdot H^{n+1}}{(n+1)!} = 0 \implies
    \]
    $ \implies $ (упражнение: доказать) $\implies R_n(x) \rightarrow 0$ при $n \rightarrow\infty$.
\end{proof}

\begin{note}\leavevmode
    \begin{enumerate}
        \item $f(x) = e^x$

              Тогда $\forall n \ f^{(n)}(x) = e^x, \ f^{(n)}(0) = 1$. Так как $\forall x \in [0;H], \ \forall n \in \N$
              \[
                  0 < f^{(n)}(x) \leqslant e^H = L.
              \]

              В силу произвольности $H$ ряд Тейлора для функции $f(x) = e^x$ сходится на $(-\infty;+\infty)$
              \[
                  e^x = 1 + x + \frac{x^2}{2!} + \frac{x^3}{3!} + \ldots + \frac{x^n}{n!} + \ldots
              \]

        \item $f(x) = \sin x, \ f(x) = \cos x$
              \begin{multline*}
                  f^{(n)}(x) = (\sin x )^{(n)} = \sin \left(x + n \cdot \frac{\pi}{2}\right) \implies \\
                  \implies \left|f^{(n)}(x)\right| = \left|\sin \left(x + n \cdot \frac{\pi}{2}\right)\right| \leqslant 1 = L \implies
              \end{multline*}
              $\implies$ ряд Тейлора для $\sin x$ сходится на $(-\infty;+\infty)$ и имеет своей суммой $\sin x$,
              \[
                  \begin{array}{cl}
                      \sin x                          & = x - \frac{x^3}{3!} + \frac{x^5}{5!} + \ldots + (-1)^{n-1}\cdot \frac{x^{2n - 1}}{(2n-1)!} + \ldots \\
                      \overset{(\sin x)'}{\Downarrow} &                                                                                                      \\
                      \cos x                          & = 1 - \frac{x^2}{2!} + \frac{x^4}{4!} - \ldots + (-1)^{n-1}\cdot \frac{x^{2n}}{(2n)!} + \ldots
                  \end{array}
              \]

        \item $f(x) = \ln(1 + x)$

              \[
                  \ln(1+x) \sim x - \frac{x^2}{2} + \frac{x^3}{3} + \ldots + (-1)^{n-1} \cdot \frac{x^n}{n} + \ldots
              \]
              \[
                  R = \frac{1}{\underset{n\rightarrow \infty}{\overline{\lim}}\sqrt[n]{|a_n|}} = \frac{1}{\underset{n\rightarrow \infty}{\overline{\lim}}\sqrt[n]{\frac{1}{n}}} = 1 \implies
              \]
              $\implies$ интервал сходимости $x \in (-1;1)$. Проверим границы точки $x = -1 \implies \sum_{n=1}^{\infty}\frac{(-1)^{n-1}\cdot (-1)^n}{n} = (-1)\cdot \sum_{n=1}^{\infty}\frac{1}{n}$ -- расходится.

              $x = -1 \implies \sum_{n=1}^{\infty}\frac{(-1)^{n-1}}{n}$ -- сходится по признаку Лейбница $\implies$ для данного степенного ряда интервал сходимости: $x \in (-1;1]$.

              $f^{(n)}(x):$
              \[
                  \begin{array}{l}
                      f'(x) = (\ln(1+x))' = \frac{1}{1 + x} \\
                      f''(x) = -\frac{1}{(1+x)^2}           \\
                      f'''(x) = \frac{2}{(1+x)^3}           \\
                      \vdots                                \\
                      f^{(n)}(x) = \frac{(-1)^{n-1}\cdot (n-1)!}{(1+x)^n}
                  \end{array}.
              \]

              Рассмотрим остаточный член ряда в форме Коши:
              \[
                  R_n(x) = \frac{f^{(n+1)}(\xi x)}{n!}\cdot \big((1-\xi) \cdot x\big)^n \cdot x, \quad 0 < \xi < 1.
              \]

              Тогда
              \[
                  \big|R_n(x)\big| = \left|\frac{(-1)^n \cdot n!}{(1 + \xi x)^{n+1} \cdot n!} \cdot ((1-\xi)\cdot x)^n \cdot x\right| = \frac{(1 - \xi)^n}{(1 + \xi x)^n} \cdot \frac{|x|^{n+1}}{(1 + \xi x)}.
              \]

              Что бы показать, что $R_n(x)\rightarrow 0$ при $n\rightarrow \infty$, нужно доказать, что $\frac{1 - \xi}{1 + \xi x}<1$.
              \begin{enumerate}
                  \item $0<x<1$

                        \begin{figure}[H]
                            \centering
                            \incfig{fig_02}
                            \label{fig:fig_02}
                        \end{figure}

                  \item $-1<x<0$

                        \begin{figure}[H]
                            \centering
                            \incfig{fig_03}
                            \label{fig:fig_32}
                        \end{figure}
              \end{enumerate}

              Из рисунков видно, что $\forall \xi \in (0;1)$ и $\forall x \in (-1;1] \ \frac{1-\xi}{1 + \xi x} < 1$. Таким образом, $R_n(x) \rightarrow 0$ при $n \rightarrow \infty \implies$
              \[
                  \implies ln(1+x) = x - \frac{x^2}{2} + \frac{x^3}{3} - \ldots \quad x \in (-1;1]
              \]

        \item Биномиальный ряд
              \begin{multline*}
                  (1 + x)^m \sim \\
                  \sim 1 + m \cdot x + \frac{m\cdot(m-1)}{2!}\cdot x^2 + \ldots + \frac{m\cdot(m-1)\cdot \ldots \cdot(m-n+1)\cdot x^n}{n!} + \ldots
              \end{multline*}
              \[
                  f^{(n)}(x) = \big[(1+x)^m\big]^{(n)} = m \cdot (m-1) \cdot \ldots \cdot \big(m-(n-1)\big) \cdot (1 + x)^{m - n},
              \]
              \begin{multline*}
                  R_n(x) = \frac{f^{(n+1)}(\xi x)}{n!}\cdot \big((1-\xi) \cdot x\big)^n \cdot x = \\
                  = \frac{m \cdot (m-1) \cdot \ldots \cdot (m-n) \cdot (1 + \xi x)^{m - n + 1}}{n!} \cdot ((1-\xi) \cdot x)^n \cdot x, \ \xi \in (0;1),
              \end{multline*}
              \[
                  \left|R_n(x)\right| = \underbrace{\frac{m\cdot (m-1) \cdot \ldots \cdot (m-n)}{n!}}_{\rightarrow0}\cdot \underbrace{\left|\frac{1-\xi}{1 + \xi x}\right|^n}_{\rightarrow0} \cdot (1 + \xi x)^{m-1} \cdot \underbrace{|x|^{n+1}}_{\rightarrow 0} \rightarrow 0
              \]
              при $n \rightarrow \infty$.

              Тогда
              \begin{multline*}
                  (1 + x)^m = \\
                  = 1 + m\cdot x + \frac{m(m-1)x^2}{2!} + \ldots + \frac{m(m-1)\ldots(m-n+1)}{n!}x^n + \ldots, \ x \in (-1;1).
              \end{multline*}
    \end{enumerate}
\end{note}

\begin{note}[Упражнение]
    Доказать, что область сходимости степенного ряда: $ x \in (-1;1) $.
\end{note}

\chapter{Интегралы, зависящие от параметра}

\begin{definition}[Интеграл, зависящий от параметра]
    \emph{Интегралом, зависящим от параметра} называется функция
    \[
        F(y) = \int_{E_y}f(x,y)dx = \int_{\alpha(y)}^{\beta(y)}f(x,y)dx.
    \]
\end{definition}

\section{Собственные интегралы, зависящие от параметра}

\begin{theorem}\label{theorem:7.1.1}
    Если функция $f(x,y)$ непрерывна на $P = [a;b] \times [c;d]$, то функция $F(y) = \int_{a}^{b}f(x,y)dx$ непрерывна на $[c;d]$.
\end{theorem}

\begin{proof}
    Пусть $y_0 \in [c;d]$. Покажем, что $F(y)$ непрерывна в точке $y_0$.
    \begin{multline*}
        \big|F(y) - F(y_0)\big| = \\
        = \left|\int_{a}^{b}f(x,y)dx - \int_{a}^{b}f(x,y_0)dx\right| = \left|\int_{a}^{b}\big(f(x,y) - f(x,y_0)\big)dx\right| \leqslant \\
        \leqslant \int_{a}^{b}\big|f(x,y) - f(x,y_0)\big|dx.
    \end{multline*}

    Так как $f(x,y)$ непрерывна на $P$ и $P$ -- компактное, то $f(x,y)$ -- равномерно непрерывна на $P \implies \forall \epsilon > 0 \ \exists \delta > 0: \ \forall \equalto{M_1}{(x_1,y_1)},\equalto{M_2}{(x_2,y_2)} \in P$:
    \begin{multline*}
        \rho\big((x_1,y_1),(x_2,y_2)\big) = \sqrt{(x_2 - x_1)^2 + (y_2 - y_1)^2} < \delta \implies \\
        \implies \big|f(x_1,y_1) - f(x_2;y_2)\big| < \epsilon.
    \end{multline*}

    Пусть $\epsilon > 0$ задано. Выберем $\delta > 0: \forall M_1,M_2 \in P$:
    \[
        \rho(M_1,M_2) < \delta \implies \big|f(x_1,y_1) - f(x_2,y_2)\big| < \frac{\epsilon}{b - a}.
    \]

    Тогда
    \begin{multline*}
        \big|F(y) - F(y_0)\big| \leqslant \\
        \leqslant \int_{a}^{b}\big|f(x,y) - f(x,y_0)\big|dx < \int_{a}^{b}\frac{\epsilon}{b-a}dx = \\
        = \frac{\epsilon}{b - a} \cdot \int_{a}^{b}dx = \frac{\epsilon}{b - a} \cdot (b-a) = \epsilon
    \end{multline*}
    $\implies F(y)$ непрерывна в точке $y_0$, где точка $y_0$ -- прозвольная.
\end{proof}

\begin{remark}
    Заметим, что равномерная непрерывность $f(x,y)$ на $P$ влечет за собой то, что $f(x,y)$ равномерно сходится к $f(x,y_0)$ при $y \rightarrow y_0$, то есть $f(x,y) \xrightrightarrows[y \rightarrow y_0]{} f(x,y_0)$. Следовательно,
    \begin{multline*}
        \underset{y\rightarrow y_0}{\lim}F(y) = \\
        = \underset{y\rightarrow y_0}{\lim}\int_{a}^{b}f(x,y)dx = \int_{a}^{b}\underset{y \rightarrow y_0}{\lim}f(x,y)dx = \int_{a}^{b}f(x,y_0)dx = \\
        = F(y_0).
    \end{multline*}
\end{remark}

\begin{lemma}\label{lemma:7.1.1}
    Если:
    \begin{itemize}
        \item $f(x,y)$ непрерывна на $P$,
        \item $\frac{\partial f}{\partial y}(x,y)$ непрерывна на $P$,
    \end{itemize}
    то $F(y) = \int_{a}^{b}f(x,y)dx$ дифференцируема на $[c;d]$ и
    \[
        F'(y) = \int_{a}^{b}\frac{\partial f}{\partial y}(x,y)dx.
    \]
\end{lemma}

\begin{proof}
    \begin{multline*}
        F'(y) = \\
        = \underset{h\rightarrow 0}{\lim}\left(\frac{1}{h} \cdot \big(F(y+h) - F(y)\big)\right) =\underset{h\rightarrow0}{\lim}\left(\int_{a}^{b}f(x,y+h)dx - \int_{a}^{b}f(x,y)dx\right) = \\
        =\underset{h\rightarrow0}{\lim}\left(\frac{1}{h}\cdot \left(\int_{a}^{b}\big(f(x,y+h) - f(x,y)\big)\right)dx\right) = \left|\begin{array}{c}
            \text{по теореме} \\
            \text{Лагранжа}
        \end{array}\right| = \\
        = \underset{h\rightarrow0}{\lim}\left(\frac{1}{h}\cdot \int_{a}^{b}\frac{\partial f}{\partial y}(x,y+\theta h) \cdot h\right)dx = \underset{h\rightarrow 0}{\lim}\int_{a}^{b}\frac{\partial f}{\partial y}(x,y + \theta \cdot h)dx = \\
        = \left|\begin{array}{c}
            \text{используя непрерывность} \\
            \text{производной } \frac{\partial f}{\partial y}
        \end{array}\right| =\int_{a}^{b}\underset{h\rightarrow 0}{\lim}\frac{\partial f}{\partial y}(x,y+\theta\cdot h)dx = \\
        = \int_{a}^{b}\frac{\partial f}{\partial y}(x,y)dx.
    \end{multline*}
\end{proof}

\begin{theorem}[О дифференцировании собственного интеграла, зависящего от параметра]\label{theorem:7.1.2}
    Пусть:
    \begin{itemize}
        \item $\alpha(y), \beta(y)$ -- дифференцируемые на $[c;d]$,
        \item $\forall y \in [c;d] \ a\leqslant \alpha(y) \leqslant b$ и $a \leqslant \beta(y) \leqslant b$,
        \item $f(x,y)$ -- непрерывна на $P = [a;b] \times [c;d]$,
        \item $\frac{\partial f}{\partial y}$ -- непрерывна на $P$,
    \end{itemize}
    тогда $F(y) = \int_{\alpha(y)}^{\beta(y)}f(x,y)dx$ дифференцируема на $[c;d]$ и
    \[
        F'(y) = \int_{\alpha(y)}^{\beta(y)}\frac{\partial f}{\partial y}(x,y)dx + f\big(\beta(y),y\big) \cdot \beta'(y) - f\big(\alpha(y),y\big)\cdot \alpha'(y)
    \]
    \begin{center}
        (формула Лейбница)
    \end{center}
\end{theorem}

\begin{proof}
    Используя лемму \ref{lemma:7.1.1}, рассмотрим функцию
    \[
        \Phi\big(y,\alpha(y),\beta(y)\big) = \int_{\alpha(y)}^{\beta(y)}f(x,y)dx:
    \]
    \begin{eqnarray*}
        \Phi_y' & = & \Phi_y' \cdot y_y' + \Phi_\alpha'\cdot\alpha_y' + \Phi_\beta' \cdot \beta_y'                                       \\
        \verteq &   &                                                                                                                    \\
        F_y'    & = & \int_{\alpha(y)}^{\beta(y)}\frac{\partial f}{\partial y}(x,y)dx + f(\beta,y)\cdot \beta_y' - f(\alpha,y)\cdot\alpha_y'
    \end{eqnarray*}
\end{proof}

\begin{theorem}[Об интегрировании собственного интеграла по параметру]\label{theorem:7.1.3}
    Если $ f(x,y) $ непрерывна на $ P = [a;b] \times [c;d] $, то функция $ F(y) = \int_{a}^{b}f(x,y)dx $ интегрируема на $ [c;d] $ и
    \[
        \int_{c}^{d}F(y)dy = \int_{c}^{d}\left(\int_{a}^{b}f(x,y)dx\right)dy = \int_{a}^{b}\left(\int_{c}^{d}f(x,y)dy\right)dx.
    \]

    Обычно пишут:
    \[
        \int_{c}^{d}dy \int_{a}^{b}f(x,y)dx = \int_{a}^{b}dx \int_{c}^{d}f(x,y)dy.
    \]
\end{theorem}

\begin{proof}
    Рассмотрим функции
    \begin{eqnarray*}
        \phi(u) &=& \int_{c}^{u}\left(\int_{a}^{b}f(x,y)dx\right)dy, \\
        \psi(u) &=& \int_{a}^{u}\left(\int_{c}^{d}f(x,y)dy\right)dx.
    \end{eqnarray*}

    $ \phi(u) $ и $ \psi(u) $ непрерывны и дифференцируемы на $ [a;b] $.

    В самом деле, $ F(y) = \int_{a}^{b}f(x,y)dx $ непрерывна на $ [c;d] $ (так как $ f(x,y) $ непрерывна на $ P $ и по теореме \ref{theorem:7.1.1}).

    А функция $ \phi(u) = \int_{c}^{u}F(y)dy $ -- непрерывна и дифференцируема на $ [c;d] $ (по теореме \ref{theorem:7.1.1}).

    При этом 
    \[
       \phi'(y) = F(u) = \int_{a}^{b}f(x,u)fx.
    \]

    Далее, функция $ \Phi(x,u) = \int_{c}^{u}f(x,y)dy $. $ \Phi(x,u) $ -- дифференцируема по $ u $ и $ \Phi_u'(x,u) = f(x,u) $.
    \[
       \psi'(u) = \int_{a}^{b}\Phi_u'(x,u)dx = \int_{a}^{b}f(x,u)fx.
    \]

    Имеем, что $ \phi'(u) = \psi'(u) \ \forall u \in [c;d] \implies $
    \[
       \implies \phi(u) - \psi(u) = const \ \forall u \in [c;d].
    \]

    Заметим, что $ \phi(c) - \psi(c) = 0 - 0 = 0 \implies \forall u \in [c;d] \ \phi(u) - \psi(u) = 0 \implies \phi(u) = \psi(u) \implies $
    \[
       \implies \int_{c}^{d}\left(\int_{a}^{b}f(x,y)dx\right)dx = \int_{a}^{b}\left(\int_{c}^{d}f(x,y)dy\right)dx.
    \]
\end{proof}

\newpage