\lesson{10}{от 13 окт 2023 8:46}{Продолжение}


\section{Двойные и повторные ряды}

\begin{note}
    Рассмотрим таблицу:
    \begin{center}
        $(\star) \quad$ \begin{tabular}{c | c | c | c | c | c}
            $a_{11}$ & $a_{12}$ & $a_{13}$ & $\cdots$ & $a_{1k}$ & $\cdots$ \\
            \hline
            $a_{21}$ & $a_{22}$ & $a_{23}$ & $\cdots$ & $a_{2k}$ & $\cdots$ \\
            \hline
            $\vdots$ & $\vdots$ & $\vdots$ & $\ddots$ & $\vdots$ & $\ddots$ \\
            \hline
            $a_{n1}$ & $a_{n2}$ & $a_{n3}$ & $\cdots$ & $a_{nk}$ & $\cdots$ \\
            \hline
            $\vdots$ & $\vdots$ & $\vdots$ & $\ddots$ & $\vdots$ & $\ddots$ \\
        \end{tabular}
    \end{center}
\end{note}

\begin{definition}[Повторный ряд]
    \emph{Повторным рядом} называются выражения
    \begin{equation}\label{eq:6.6.1}
        \sum_{n=1}^{\infty}\sum_{k=1}^{\infty}a_{nk},
    \end{equation}
    \begin{center}
        и
    \end{center}
    \begin{equation}\label{eq:6.6.2}
        \sum_{k=1}^{\infty}\sum_{n=1}^{\infty}a_{nk}.
    \end{equation}

    Говорят, что ряд \ref{eq:6.6.1} сходится, если сходятся все ряды $(A_n)$ по строкам $(\sum_{k=1}^{\infty}a_{n_k} = A_n)$ и сходится ряд $ \sum_{n=1}^{\infty}A_n $.
\end{definition}

\begin{definition}[Двойной ряд]
    \emph{Двойным рядом} называется выражение:
    \begin{equation}\label{eq:6.6.3}
        \sum_{n,k = 1}^{\infty} a_{nk}
    \end{equation}

    Говорят, что ряд \ref{eq:6.6.3} сходится, если:
    \[
        \exists A = \underset{N\rightarrow\infty}{\underset{K\rightarrow\infty}{\lim}}A_{NK} = \underset{N\rightarrow\infty}{\underset{K\rightarrow\infty}{\lim}}\sum_{n=1}^{N}\sum_{k=1}^{K}a_{nk}.
    \]

    То есть $\forall \epsilon > 0 \ \exists N_0$ и $K_0: \ \forall N > N_0$ и $\forall k > K_0$
    \[
        \bigg|\underbrace{\sum_{n=1}^{N}\sum_{k=1}^{K}a_{nk}}_{A_{NK}} - A\bigg| < \epsilon.
    \]
\end{definition}

\begin{definition}[Простой ряд]
    Пусть ряд
    \begin{equation}\label{eq:6.6.4}
        \sum_{r=1}^{\infty}U_r
    \end{equation}
    построен из элементов таблицы, взятых в произвольном порядке. Такой ряд будем называть \emph{простым}, связанным с данной таблицей.
\end{definition}

\begin{theorem}[О связи сходимости простого и повторного рядов]\leavevmode
    \begin{enumerate}
        \item Если ряд \ref{eq:6.6.4} абсолютно сходится, то ряд \ref{eq:6.6.1} сходится и его сумма равна $U$.

        \item Если после замены элементов таблицы $(\star)$ их модулями ряд \ref{eq:6.6.1}$ ^* $ ходится, то ряд \ref{eq:6.6.4} сходится абсолютно и суммы рядов \ref{eq:6.6.1} (без модулей) и \ref{eq:6.6.4} совпадают.
    \end{enumerate}
\end{theorem}

\begin{proof}\leavevmode
    \begin{enumerate}
        \item Пусть \ref{eq:6.6.1}$ ^* $ сходится. Покажем, что все ряды по строкам сходятся:
              \[
                  (A_n) \ \sum_{k=1}^{\infty}a_{nk} \quad (\forall n \in \N)
              \]
              и сходится ряд $ \sum_{n=1}^{\infty}A_n $.

              Рассмотрим
              \[
                  |a_{n1}| + |a_{n2}| + \ldots + |a_{nk}| \leqslant |u_1| + |u_2| + \ldots + |u_r|,
              \]
              где $r$ выбран таким образом, чтобы среди $|u_i|$ были все слагаемые $|a_{n1}, \ldots, a_{nk}|$.

              Таким образом,
              \[
                  \underbrace{|a_{n1}| + \ldots + |a_{nk}|}_{A_{nk}^*} \leqslant U^* \implies \exists \underset{k\rightarrow\infty}{\lim}A_{nk}^* = A_n^* \implies
              \]
              $\implies$ ряд $\sum_{k=1}^{\infty}a_{nk} \ \forall n \in \mathbb{N}$ сходится абсолютно $\implies$ он сходится.

              Далее, пусть $\epsilon > 0$ задано. Выберем номер $r_0: \ \forall r > r_0$
              \[
                  \sum_{i=1}^{\infty}|u_{r+i}| < \frac{\epsilon}{3}.
              \]

              Тогда
              \[
                  \bigg|\sum_{i=1}^{r} u_i - U\bigg| = \bigg| \sum_{i=1}^{\infty}u_{r + i}\bigg| \leqslant \sum_{i=1}^{\infty}|u_{r + i}| < \frac{\epsilon}{3}
              \]

              Так как ряды по строкам сходятся, то $\forall n$ выберем $m(n)$:
              \[
                  \bigg|\sum_{k=1}^{m(n)}a_{n_k} - A_n\bigg| < \frac{\epsilon}{3}.
              \]

              Наконец, выберем номер $N_0$ такой, что все числа $u_1, u_2,\ldots, u_{r_0}$ содержались бы в первых $N_0$ строках:
              \begin{multline*}
                  \bigg|\sum_{n=1}^{N_0}A_n - U\bigg| = \\
                  = \bigg|\sum_{n=1}^{N_0}A_n - \sum_{n=1}^{N_0}\sum_{k=1}^{m(n)}a_{n_k} + \sum_{n=1}^{N_0}\sum_{k=1}^{m(n)}a_{n_k} - \sum_{i=1}^{r_0}u_i + \sum_{i=1}^{r_0}u_i - U\bigg| \leqslant \\
                  \leqslant \sum_{n=1}^{N_0}\bigg|A_n - \sum_{k=1}^{m(n)}a_{n_k}\bigg| + \bigg|\sum_{n=1}^{N_0}\sum_{k=1}^{m(n)}a_{n_k} - \sum_{i=1}^{r_0}u_i\bigg| + \underbrace{\bigg|\sum_{i=1}^{r_0}u_i - U\bigg|}_{<\frac{\epsilon}{3}} < \\
                  < \frac{\epsilon}{3} + \sum_{i=r_0 + 1}^{\infty}(u_i) + \frac{\epsilon}{3} < \frac{\epsilon}{3} \cdot 3 = \epsilon.
              \end{multline*}

        \item Пусть ряд $\sum_{n=1}^{\infty}\sum_{k=1}^{\infty}|a_{n_k}| = A^*$ сходится.

              Тогда $\forall r \ \exists N,K$ такие, что числа $u_1,\ldots,u_r$ содержатся в $N$ первых строчках и $K$ первых столбцах таблицы:
              \[
                  \sum_{i=1}^{r}|u_i| \leqslant \sum_{n=1}^{N}\sum_{k=1}^{K}|a_{n_k}| \leqslant A^* \implies
              \]
              $\implies |u_r|\nearrow$ и ограничен $\implies$ ряд \ref{eq:6.6.4} сходится абсолютно $\implies$ по пункту 1. суммы рядов \ref{eq:6.6.4} и \ref{eq:6.6.1} равны.
    \end{enumerate}
\end{proof}

\begin{theorem}[Свойства двойных рядов]\leavevmode
    \begin{enumerate}
        \item Если ряд \ref{eq:6.6.3} сходится, то
              \[
                  \underset{k\rightarrow\infty}{\underset{n\rightarrow\infty}{\lim}}a_{nk} = 0.
              \]

        \item (Критерий Коши)
              Ряд \ref{eq:6.6.3} сходится $\iff \forall \epsilon > 0 \ \exists N_0,K_0: \ \forall n > N_0, \ \forall k > K_0, \ \forall p > 0, \ \forall q > 0$
              \[
                  \bigg|\sum_{n=1}^{p}\sum_{k=1}^{q}a_{(N_0 + n)(K_0 + k)}\bigg| < \epsilon.
              \]

        \item Если ряд \ref{eq:6.6.3} сходится, то $\forall c \in \R$ ряд
              \[
                  \sum_{n,k=1}^{\infty}(c\cdot a_{nk})
              \]
              сходится, и его сумма равна $c\cdot A$ (где $A = \sum_{n,k=1}^{\infty}a_{nk}$).

        \item Если ряд \ref{eq:6.6.3} сходится и ряд
              \[
                  \sum_{n,k=1}^{\infty}b_{nk}
              \]
              сходится, то
              \[
                  \sum_{n,k=1}^{\infty}(a_{nk} + b_{nk}) = A + B,
              \]
              а к тому же -- сходится.

        \item Если $\forall n, \ \forall k \ a_{nk} \geqslant 0$, то ряд \ref{eq:6.6.3} сходится $ \iff $ его частичные суммы ограничены в совокупности.
    \end{enumerate}
\end{theorem}

\begin{proof}\leavevmode
    \begin{enumerate}
        \item Пусть ряд \ref{eq:6.6.3} сходится. Заметим, что
              \[
                  A_{nk} = \sum_{i,j=1}^{n,k},
              \]
              \[
                  a_{nk} = A_{nk} - A_{n(k-1)} - A_{(n-k)k} + A_{(n-1)(k-1)}
              \]
              $\implies a_{nk} \rightarrow 0 $.

        \item (Критерий Коши)
              На декартовом произведении $\N\times\N$ введем базу:
              \[
                  B_{nk} = \big\{(n,k): \ n > N_0, \ k > K_0\big\}.
              \]

              Тогда критерий Коши сходимости ряда -- это есть критерий Коши существования предела функции $A_{nk}$ по данной базе.

        \item Самостоятельно.

        \item Самостоятельно.

        \item \begin{itemize}
                  \item $ |\Rightarrow| $ Очевидно.

                  \item $ |\Leftarrow| $ Пусть множество $\{A_{nk}\}$ -- ограничено.

                        Пусть $A = \sup\{A_{nk}\}$. Покажем, что $A$ -- сумма ряда \ref{eq:6.6.3}.

                        Пусть $\epsilon > 0$ задано. Выберем $N_0$ и $K_0$:
                        \[
                            \begin{array}{c}
                                A - A_{N_0 K_0} < \epsilon \\
                                (\text{по опр. }\sup)
                            \end{array}
                        \]

                        Тогда $ \forall n > N_0$ и $ \forall k > K_0 \ A_{nk} \geqslant A_{N_0K_0} \implies 0 < A - A_{nk} \leqslant A - A_{N_0K_0} < \epsilon \implies |A - A_{nk}| < \epsilon $.
              \end{itemize}

              $ \implies  $ ряд \ref{eq:6.6.3} сходится.
    \end{enumerate}
\end{proof}

\begin{theorem}[О связи сходимости двойного ряда и повторного]
    Если
    \begin{itemize}
        \item ряд \ref{eq:6.6.3} сходится (двойной),
        \item все ряды по строкам сходятся,
    \end{itemize}
    тогда повторный ряд $ \sum_{n=1}^{\infty}\sum_{k=1}^{\infty}a_{nk} $ сходится и
    \[
        A = \sum_{n=1}^{\infty}\sum_{k=1}^{\infty}a_{nk} = \sum_{n,k=1}^{\infty}a_{nk}.
    \]
\end{theorem}

\begin{proof}
    Пусть $ \epsilon>0 $ задано. Выберем $ N_0,K_0: \ \forall n > N_0 $ и $ k>K_0 $
    \begin{equation}\label{eq:for_proof1}
        \left|\sum_{i=1}^{n}\sum_{j=1}^{k}a_{ij} - A\right| < \frac{\epsilon}{2}.
    \end{equation}
    \[
        \sum_{i,j=1}^{n,k}a_{ij} = A_{nk}\text{ двойного ряда}.
    \]

    В неравенстве \ref{eq:for_proof1} переходим к пределу при $ k \rightarrow\infty $.

    Тогда $ \forall n > N_0 $
    \[
        \left|\sum_{i=1}^{n}\sum_{j=1}^{\infty}a_{ij} - A\right| = \left|\sum_{i=1}^{n}A_n - A\right| < \frac{\epsilon}{2} < \epsilon
    \]
    $ \implies $ повторный ряд $ \sum_{n=1}^{\infty}\sum_{k=1}^{\infty}a_{nk} = A $.
\end{proof}

\begin{theorem}[О связи сходимости двойного и простого рядов]
    Если ряд \ref{eq:6.6.3}$ ^* $ сходится, то сходится ряд \ref{eq:6.6.4}.

    И наоборот, если сходится ряд \ref{eq:6.6.4}$ ^* $, то сходится ряд \ref{eq:6.6.3}.

    И в обоих случаях суммы рядов равны:
    \[
        \sum_{n,k=1}^{\infty}a_{nk} = \sum_{r=1}^{\infty}u_r
    \]
\end{theorem}

\begin{proof}\leavevmode
    \begin{itemize}
        \item $ |\Rightarrow| $ Пусть двойной ряд сходится абсолютно, то есть сходится ряд $\sum_{n,k=1}^{\infty}|a_{nk}|$.

              Тогда для любого номера $S \ \exists N,K$ такие, что все числа $u_1,\ldots,u_S$ содержатся в первых $N$ строках и первых $K$ столбцах, тогда:
              \[
                |u_1| + |u_2| + \ldots + |u_S| \leqslant \sum_{n=1}^{N}\sum_{k=1}^{K}|a_{nk}| \leqslant A^* = \sum_{n,k=1}^{\infty}|a_{nk}| \implies
              \]
              $\implies$ последовательность $U_i^* \nearrow$ и ограничена $\implies$ ряд $\sum_{r=1}^{\infty}u_r$ сходится абсолютно $\implies$ сходится.

        \item $ |\Leftarrow| $ Пусть ряд $\sum_{r=1}^{\infty}|u_r|$ сходится $\implies \forall N,K \ \exists S:$ все числа $a_{11},a_{12},\ldots,a_{1K},a_{21},\ldots,a_{2K},\ldots,a_{N1},\ldots,a_{NK}$ содержатся среди чисел $u_1,\ldots,u_S$. Тогда
              \[
                A_{NK}^* = \sum_{n=1}^{N}\sum_{k=1}^{K}|a_{nk}| \leqslant\sum_{r=1}^{S}|u_r| \leqslant U^* = \sum_{r=1}^{\infty}|u_r| \implies
              \]
              $\implies$ ряд $\sum_{n,k=1}^{\infty}a_{nk}$ сходится.

              Покажем, что $\sum_{n,k=1}^{\infty}a_{nk} = \sum_{r=1}^{\infty}u_r$.

              Так как ряд $\sum_{r=1}^{\infty}u_r$ сходится абсолютно, то расположим элементы по квадратам:
              \[
                \begin{array}{l}
                    a_{11} = u_{r_1}                                       \\
                    a_{12} + a_{22} + a_{21} = u_{r_2} + u_{r_3} + u_{r_4} \\
                    \vdots                                                 \\
                    A_{nn} = a_{11} + \ldots + a_{nn} = U_n = u_{r_1} + \ldots + u_{r_n}
                \end{array},
              \]
              \[
                A = \underset{n\rightarrow\infty}{\lim}A_{nn} = \underset{n\rightarrow\infty}{\lim}U_n = U.
              \]
    \end{itemize}
\end{proof}

\begin{theorem}[«Главная»]
    Пусть дана таблица $ (\star)(a_{ij}) $ и по ней построены ряды \ref{eq:6.6.1}, \ref{eq:6.6.2}, \ref{eq:6.6.3}, \ref{eq:6.6.4}.

    Если после замены элементов таблицы их модулями хотя бы один из 4-х рядов становится сходящимся, то сходятся остальные и их суммы равны.
\end{theorem}

\begin{proof}
    Из четырех предыдущих теорем $\implies$ «Главная» теорема.
\end{proof}